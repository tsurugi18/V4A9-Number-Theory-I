\documentclass{article}

\usepackage{amsmath}
\usepackage{amssymb}
\usepackage{amsthm}
\usepackage{ mathrsfs }
\usepackage{ mathtools}
\usepackage{enumerate}
\usepackage{bbm}
\usepackage{lipsum}
\usepackage{fancyhdr}
\usepackage{tikz-cd} 
\usepackage{nicematrix}
\usetikzlibrary{arrows}
\newcommand{\midarrow}{\tikz \draw[-triangle 90] (0,0) -- +(.1,0);}

\newtheorem{theorem}{Theorem} [section] 
\newtheorem{proposition}{Proposition}[section] 
\newtheorem{defi}{Definition}[section] 
\newtheorem{lemma}{Lemma}[section] 
\newtheorem{notation}{Notation}[section] 
\newtheorem{remark}{Remark}[section] 
\newtheorem{corollary}{Corollary} [section] 
\newtheorem{terminology}{Terminology}[section] 
\newtheorem{fact}{Fact}[section] 
\newtheorem{example}{Example}[section] 
\DeclareMathOperator{\ev}{ev}
\DeclareMathOperator{\st}{s.t.}
\DeclareMathOperator{\triv}{triv}
\DeclareMathOperator{\Aut}{Aut}
\DeclareMathOperator{\Stab}{Stab}
\DeclareMathOperator{\Ind}{Ind}
\DeclareMathOperator{\GL}{GL}
\DeclareMathOperator{\SL}{SL}
\DeclareMathOperator{\SO}{SO}
\DeclareMathOperator{\Sp}{Sp}
\DeclareMathOperator{\Rep}{Rep}
\DeclareMathOperator{\esssup}{esssup}
\DeclareMathOperator{\diam}{diam}
\DeclareMathOperator{\rank}{rank}
\DeclareMathOperator{\Hom}{Hom}
\DeclareMathOperator{\End}{End}
\DeclareMathOperator{\Image}{Im}
\DeclareMathOperator{\Ker}{Ker}
\DeclareMathOperator{\Dom}{Dom}
\DeclareMathOperator{\grad}{grad}
\DeclareMathOperator{\Span}{Span}
\DeclareMathOperator{\interior}{int}
\DeclareMathOperator{\supp}{supp}
\DeclareMathOperator{\id}{id}
\DeclareMathOperator{\sgn}{sgn}
\DeclareMathOperator{\Mat}{Mat}
%\newcommand*{\name}[\num_arguments][default values]{{\color{#1}\Large #2}}
\newcommand{\defeq}{\vcentcolon=}
\newcommand{\norm}[1]{\Vert #1 \Vert}
\newcommand{\opNorm}[2]{\norm{#1}_{#2\to#2}}
\newcommand{\normL}[3]{\norm{#1}_{L^{#2}(#3)}}
\newcommand{\N}[0]{\mathbb{N}}
\newcommand{\R}[0]{\mathbb{R}}
\newcommand{\Z}[0]{\mathbb{Z}}
\newcommand{\C}[0]{\mathbb{C}}
\newcommand{\Q}[0]{\mathbb{Q}}
\newcommand{\F}[0]{\mathbb{F}}
\newcommand{\G}[0]{\mathbb{G}}
\newcommand{\Para}[0]{\mathbb{P}}
\newcommand{\M}[0]{\mathbb{M}}
\newcommand{\torus}[0]{\mathbb{T}}

\newcommand{\fib}[1]{%
  \mathbin{\mathop{\times}\limits_{#1}}%
}
\newcommand{\tens}[1]{%
  \mathbin{\mathop{\otimes}\displaylimits_{#1}}%
}

\title{V4A9 Homework 1}
\author{So Murata}
\date{WiSe 25/26, University of Bonn Number Theory 1}

\begin{document}
\maketitle
\par\textbf{(1)} Let $\varphi:H\times V\to V$ be such that $\varphi(g,v) = \pi(g)(v)$.\\
\par\textbf{(a)$\Rightarrow$(b)} For any $v\in V$, $\Stab_H(v)$ is open, thus it contains a basis consists of locally compact open subgroups. Pick one such subgroup $K$, we have $v\in V^K$.\\
\par\textbf{(b)$\Rightarrow$(c)} For any $v\in V$, there is $K$ such that $K$ fixes $v$. Clearly $K\times\{v\}\subseteq\varphi^{-1}(v)$. For arbitrary $(g,w)\in\varphi^{-1}(v)$, note that $G$ is a topological group thus $gK\times\{w\}\subseteq\varphi^{-1}(v)$ is an open subset. Thus each basis element of the topology on $V$ has an open preimage.\\
\par\textbf{(c)$\Rightarrow$(a)} For any $v\in V$, we have $H\times\{v\}$ is open by definition of Product topology and $\varphi^{-1}(v)$ is also open by assumption. Observe that $\Stab_H(v) = H\times\{v\}\cap\varphi^{-1}(v)$ which is an intersection of two open sets which is open.\\
\par\textbf{(2)}
Using Iwasawa decomposition, we have,
\begin{equation*}
  \SL_2(F) = BK,\quad(K = \SL_2(\mathcal{O}_F)).
\end{equation*}
By basic group theory, we have,
\begin{equation*}
  B\backslash \SL_2(F)/K = 
\end{equation*}
\par\textbf{(3)} The direction $\Leftarrow$ is trivial. Suppose $G$ has a proper parabolic subgroup. Consider a morphism of $G$-representation $\varphi:\triv\to\Ind^G_P\sigma$ such that $\varphi(1) = \triv$, note that $\triv:G\to \C$ itself is contained in $\Ind_P^G\C$. We can easily check
\begin{equation*}
  \varphi\circ\triv(g)(1) = \Ind_P^G\sigma(g)\circ \varphi(1) \Rightarrow \varphi(1) = \Ind_P^G\sigma(g)\circ \varphi(1). 
\end{equation*}
In other words, $f$ is right $G$ invariant. This is of course injective. We conclude that $(\triv,\C)$ can be embedded into any positive dimensional representation. Thus we have proven the contrapositive of the direction we claimed.
\par\textbf{(4)} We will prove the contraposition. Suppose $f\in\Hom_H(V_1,V_2)\backslash\{0\}$ exists, then by definition we have
\begin{equation*}
\forall z\in Z(H), f\circ\pi_1(z) = \pi_2(z)\circ f.
\end{equation*}
Since $f$ is $F$-linear, we obtain that 
\begin{equation*}
\forall z\in Z(H), \chi_1(z)f = \chi_2(z) f.
\end{equation*}
Thus the central characters coincide. 
\par\textbf{(5)}
\par\textbf{(a)+(b)} For any $k\in K$, we have 
\begin{equation*}
\pi(k)e_K(v) = {\frac 1 {\vert K\vert}}\int_K \pi(k)\pi(g)(v)dg = {\frac 1 {\vert K\vert}}\int_K \pi(g)(v)dg = e_K(v),
\end{equation*}
as $K$ is a subgroup. Thus $e_K:V\to V^K$. For any $v\in V^K$, obviously $e_K(v) = {\frac {\vert K\vert} {\vert K\vert}}v = v$. Thus this defines a projection.\\
\par\textbf{(c)}Clearly $V=V^K\oplus(1-e_K)V$ as a vectorspace. We have 
\begin{equation*}
  \pi(k)e_K(v) = e_K(\pi(k)v)
\end{equation*}
as $K$-representations. Thus we conclude the statement to be true.\\
\par \textbf{(d)} By definition, $K$ is a subgroup thus if $g$ runs through $K$, so does $g^{-1}$. Since $\lambda\in\tilde{V}$, there is a compact open subgroup $K'$ such that $\lambda\in (V^*)^{K'}$. Using $H$ is Hausdorff, we conclude $K\cap K'$ is again a compact open subgrouop. This assures us that $e_K:\tilde{V}\to \tilde{V^K}$ is well-defined. Also using $\lambda$ is linear, we get,
\begin{equation*}
  e_K\lambda(v) = {\frac 1 {\vert K\vert}}\int_K \pi^*(g)\lambda(v)dg = {\frac 1 {\vert K\vert}}\int_K \lambda(\pi(g^{-1})v)dg = \lambda(e_K(v)).
\end{equation*}
\par\textbf{(e)} By (c), there is a bijection between $V^*$ and $(V^K)^*\oplus ((1-e_K)V)^*$. By (d), $e_K$ defines a map from $\tilde{V}$ to $(V^K)^*$. For $\lambda\in\tilde{V}^K$, we have,
\begin{equation*}
  \lambda(v)= e_K\lambda(v) = \lambda(e_K v).
\end{equation*}
This is clearly injective and linear. For any $\lambda\in (V^K)^*$, we take $\nu:V\to k$ such that $\nu(v) = \lambda(e_K(v))$, then $\nu\in \tilde{V}^K$ and $e_K(\nu) = \lambda$.  Therefore, surjective.\\
\par \textbf{(6)} a)$\Rightarrow$b) follows from Exercise 5 (e) together with the fact that a dual space of a finite dimensional vector space is of finite dimension. Note that by definition of dual representation we have,
\begin{equation*}
  \pi^{**}(g)\ev_v(\lambda) = \pi^*(g^{-1})\lambda(v) = \lambda(\pi(g)v) = \ev_{\pi(g)(v)}(f).
\end{equation*}
\par $a)b)\Leftrightarrow c)$, For any vectorspace $V$, we have the following inequality,
\begin{equation*}
  \dim V\leq\dim V^*.
\end{equation*}
In particular, the equality holds if and only if $V$ is finite dimensional. Again using (e) of the previous problem, we have
\begin{equation*}
\tilde{\tilde{V}}^K = (V^K)^{**},
\end{equation*} 
which is of finite dimension. Thus have the same dimension as $V^K$. We conclude $a),b)$ are equivalent to $c)$ by passing the exact sequence 
\begin{center}
  \begin{tikzcd}
0 \arrow[r] & V \arrow[r] & \tilde{\tilde{V}} \arrow[r] & 0
\end{tikzcd}
\end{center}
to the fixed part of arbitrary compact open subgroup $K$ of $H$.
\end{document}