\documentclass{article}

\usepackage{amsmath}
\usepackage{amssymb}
\usepackage{amsthm}
\usepackage{ mathrsfs }
\usepackage{ mathtools}
\usepackage{enumerate}
\usepackage{bbm}
\usepackage{lipsum}
\usepackage{fancyhdr}
\usepackage{tikz-cd} 
\usepackage{nicematrix}
\usepackage{ stmaryrd }
\usetikzlibrary{arrows}
\newcommand{\midarrow}{\tikz \draw[-triangle 90] (0,0) -- +(.1,0);}

\newtheorem{theorem}{Theorem} [section] 
\newtheorem{proposition}{Proposition}[section] 
\newtheorem{defi}{Definition}[section] 
\newtheorem{lemma}{Lemma}[section] 
\newtheorem{notation}{Notation}[section] 
\newtheorem{remark}{Remark}[section] 
\newtheorem{corollary}{Corollary} [section] 
\newtheorem{terminology}{Terminology}[section] 
\newtheorem{fact}{Fact}[section] 
\newtheorem{example}{Example}[section] 
\newtheorem{claim}{Claim}[section] 
\DeclareMathOperator{\ev}{ev}
\DeclareMathOperator{\pr}{pr}
\DeclareMathOperator{\st}{s.t.}
\DeclareMathOperator{\triv}{triv}
\DeclareMathOperator{\Aut}{Aut}
\DeclareMathOperator{\Stab}{Stab}
\DeclareMathOperator{\Ind}{Ind}
\DeclareMathOperator{\cind}{c-Ind}
\DeclareMathOperator{\GL}{GL}
\DeclareMathOperator{\SL}{SL}
\DeclareMathOperator{\SO}{SO}
\DeclareMathOperator{\Sp}{Sp}
\DeclareMathOperator{\Rep}{Rep}
\DeclareMathOperator{\esssup}{esssup}
\DeclareMathOperator{\diam}{diam}
\DeclareMathOperator{\rank}{rank}
\DeclareMathOperator{\Hom}{Hom}
\DeclareMathOperator{\End}{End}
\DeclareMathOperator{\Image}{Im}
\DeclareMathOperator{\Ker}{Ker}
\DeclareMathOperator{\Dom}{Dom}
\DeclareMathOperator{\grad}{grad}
\DeclareMathOperator{\Span}{Span}
\DeclareMathOperator{\interior}{int}
\DeclareMathOperator{\supp}{supp}
\DeclareMathOperator{\id}{id}
\DeclareMathOperator{\ord}{ord}
\DeclareMathOperator{\sgn}{sgn}
\DeclareMathOperator{\Mat}{Mat}
\DeclareMathOperator{\Res}{Res}
\DeclareMathOperator{\diag}{diag}
\DeclareMathOperator{\Mor}{Mor}
\DeclareMathOperator{\Sets}{Sets}
\DeclareMathOperator{\Ab}{Ab}
\DeclareMathOperator{\Grp}{Grp}
\DeclareMathOperator{\image}{im}
\DeclareMathOperator{\nr}{nr}
%\newcommand*{\name}[\num_arguments][default values]{{\color{#1}\Large #2}}
\newcommand{\defeq}{\vcentcolon=}
\newcommand{\norm}[1]{\Vert #1 \Vert}
\newcommand{\opNorm}[2]{\norm{#1}_{#2\to#2}}
\newcommand{\normL}[3]{\norm{#1}_{L^{#2}(#3)}}
\newcommand{\N}[0]{\mathbb{N}}
\newcommand{\R}[0]{\mathbb{R}}
\newcommand{\Z}[0]{\mathbb{Z}}
\newcommand{\C}[0]{\mathbb{C}}
\newcommand{\Q}[0]{\mathbb{Q}}
\newcommand{\F}[0]{\mathbb{F}}
\newcommand{\G}[0]{\mathbb{G}}
\newcommand{\Para}[0]{\mathbb{P}}
\newcommand{\M}[0]{\mathcal{M}}
\newcommand{\torus}[0]{\mathbb{T}}
\newcommand{\Ouv}[0]{\mathcal{O}}
\newcommand{\sep}[0]{\:|\:}
\newcommand{\hecke}[0]{\mathscr{H}}
\newcommand{\zet}[0]{\mathfrak{z}}

\newcommand{\fib}[1]{%
  \mathbin{\mathop{\times}\limits_{#1}}%
}
\newcommand{\tens}[1]{%
  \mathbin{\mathop{\otimes}\displaylimits_{#1}}%
}

\title{V4A9 Homework 10}
\author{So Murata, Heijing Shi}
\date{WiSe 25/26, University of Bonn Number Theory 1}

\begin{document}
\maketitle
\textbf{(1)}
\par\textbf{(a)}
By Bruhats-decomposition, we have $G= B\cup BwB$. By the definition of induced representations, we have 
\begin{equation*}
    f(1) = 0\Rightarrow \forall b\in B, f(b) = \sigma(b)f(1) = 0.
\end{equation*}
Thereore $f$ is supported in $BwB$.
By definition, we have to show that 
\begin{equation*}
    \int_Nf(wn)dn,
\end{equation*}
is well-defined for any $f\in\Ind_B^G(W),f(1)=0$. 
Observe that 
\begin{equation*}
    \begin{pmatrix}
        t_1 & \:\\
        \: & t_2
    \end{pmatrix}
    \begin{pmatrix}
        1 & x\\
        \: & 1
    \end{pmatrix}
    =\begin{pmatrix}
        t_1 & \:\\
        \: & t_2
    \end{pmatrix},
    \begin{pmatrix}
        1 & x\\
        \: & 1
    \end{pmatrix}
    \begin{pmatrix}
        t_1 & xt_1\\
        \: & t_2
    \end{pmatrix}
    =\begin{pmatrix}
        t_1 & xt_2\\
        \: & t_2
    \end{pmatrix},
\end{equation*}
therefore we have $TN = NT$. Furthermore, elements of $T$ and $w$ commute therefore, we have,
\begin{equation*}
    BwB = TNwTN = BwN.
\end{equation*}
As $f$ is a smooth element of a $G$-representation, we have a compact open subgroup $K$ fixing $f$. By Iwahori decomposition, we may assume that,
\begin{equation*}
    K = (K\cap N)(K\cap T)(K\cap \overline{N}).
\end{equation*}
By the assumption on $f$, we conclude that $f$ vanishes on $B(K\cap\overline{N})$. We have,
\begin{equation*}
    \overline{N}w = wN.
\end{equation*}
thus the support of $f$ is contained in $Bw(K\cap N)$ which follows from that parabolic inductions are always compact.
Observe that 
\begin{equation*}
    f\mapsto f_N(1) = \int_{K_N}f(wn)dn = \int_{K_N}\Ind_B^G(\sigma)(n)f(w)dn.
\end{equation*}
Thus this is $0$ if and only if $f\in V(N)$. Setting $f(w) = w_1$ for some $w_1\in W$, we see the map is surjective. Thus the latter statement is proven.
\par\textbf{(b).}
We have,
\begin{equation*}
    (tf)_N(x) = \int_N f(xwnt)dt.
\end{equation*}
Note that 
\begin{equation*}
    xwnt = xwtt^{-1}nt = xwtwwt^{-1}nt.
\end{equation*}
Using the definition 
\begin{equation*}
    \delta_B(t) = \left\vert{\frac {tKt^{-1}} {tKt^{-1}\cap K}}\right\vert\left\vert{\frac {K} {tKt^{-1}\cap K}}\right\vert^{-1},
\end{equation*}
and use $n\to t^{-1}nt$, we see 
\begin{equation*}
    (tf)_N(x) = \delta_B(t^{-1})\int_Nf(x\prescript{w}{}{t}wn)dn=\delta_B^{-1}(t)\int_Nf(x\prescript{w}{}{t}wn)dn.
\end{equation*}
Therefore $f\mapsto f_N(1)$ induces a morphism of $B$-representation $V\mapsto(\prescript{w}{}{\sigma}\otimes \delta_B,W)$.
\par\textbf{(c).}
We have a short exact sequence,
\begin{center}
\begin{tikzcd}
0 \arrow[r] & V(N) \arrow[r] & V \arrow[r] & V_N \arrow[r] & 0
\end{tikzcd}
\end{center}
From part a), we have $V_N\cong W$. $V$ is a subrepresntation of an induced representation and by definition $N$ acts trivially on $V$. Therefore, we obtain the following,
\begin{center}
    \begin{tikzcd}
0 \arrow[r] & \prescript{w}{}{\sigma}\otimes\delta_B \arrow[r] & (\Ind_B^G\sigma) \arrow[r] & \sigma \arrow[r] & 0
\end{tikzcd}
\end{center}
Note that $N$ acts trivially via $\prescript{w}{}{\sigma}\otimes\delta_B,\sigma$ as they are inflated from $T$. Thus, taking the Jacquet functor, we obtain,
\begin{center}
    \begin{tikzcd}
0 \arrow[r] & \prescript{w}{}{\sigma}\otimes\delta_B \arrow[r] & (\Ind_B^G\sigma)_N \arrow[r] & \sigma \arrow[r] & 0
\end{tikzcd}
\end{center}
Setting $\sigma\otimes \delta_B^{{\frac 1 2}}\to\sigma'$, we get, 
\begin{center}
    \begin{tikzcd}
0 \arrow[r] & \prescript{w}{}{\sigma'}\otimes\delta_B^{{\frac 1 2}} \arrow[r] & (\Ind_B^G\sigma\otimes\delta_B^{{\frac 1 2}})_N \arrow[r] & \sigma\otimes\delta_B^{{\frac 1 2}} \arrow[r] & 0
\end{tikzcd}
\end{center}
\begin{equation*}
    0\subseteq \prescript{w}{}{\sigma}\subseteq \rho_N(i_B^G(\sigma)).
\end{equation*}
\par\textbf{2.} $Z(G)G^\circ \subseteq T$ and $G^\circ \ker (\chi),\chi\in X_{\nr}(G)$ Thus we have,
\begin{equation*}
    G^\circ\subseteq T.
\end{equation*}
\begin{align*}
    \chi\otimes\pi\cong\pi&\Rightarrow W_{\chi\otimes\pi}=W_\pi,\\
    &\Rightarrow \chi(z)W_{\pi}(z) = W_\pi(z)\forall z\in Z(G),\\
    &\Rightarrow \chi|_{Z(G)} =1, (\Rightarrow \forall \chi\in X_{\nr}(G), Z(G)\subseteq \ker\chi).
\end{align*}
For part b). take $H=G/Z(G)G^\circ$ is finite abelian ths $H_T= T/Z(G)G^\circ \subseteq H$.
Consider the map 
\begin{equation*}
    \Phi:G\to(\widehat{X_{\nr}(G)_\pi}), g\mapsto(\chi\mapsto\chi(g)).
\end{equation*}
We have $\Ker\Phi = T$ and 
\begin{equation*}
    \left\vert {\frac G T}\right\vert\leq \vert\widehat{(X_{\nr}(T)_\pi)}=\vert X_{\nr}(G)_\pi\vert.
\end{equation*}
We have for all $\chi\in X_{\nr}(G)_\pi,T\subseteq\ker\chi$,
\begin{equation*}
    X_{\nr}(G)_\pi\subseteq\{\nu\in X_{\nr}(G)_{\pi}\sep \nu|_T=1\}.
\end{equation*}
Thus we have,
\begin{equation*}
    \vert X_{\nr}(G)_{\pi}\vert \geq\left\vert{\frac G T}\right\vert.
\end{equation*}
\begin{equation*}
    \{\nu\in X_{\nr}(G)\sep \nu|_T=1\} = \Ker F.
\end{equation*}
where $F:\hat{H}\to\hat{H}_1,\chi\mapsto\chi|_{H}$. 
\begin{equation*}
    \{\nu\in X_{\nr}(G)\sep \nu|_T=1\}=\{\nu:G\to\C^\times \sep \nu|_{G^\circ}=\nu|_T = 1\} = \{\nu:G/Z(G)G^\circ\to\C^\times\}.
\end{equation*}
Furthermore,
\begin{align*}
    \vert \hat{H}/\ker F\vert = \vert \image F\vert , {\frac {\vert G/Z(G)G^\circ\vert} {\vert \ker F\vert}}=\vert T/Z(G)G^\circ\vert \Rightarrow \vert G / T\vert = \vert\ker F\vert.
\end{align*}
\par\textbf{(c)} We have $G/G^\circ \cong \Z^n$ and $S/G^\circ\cong \Z$, 
\par\textbf{3.}Suppose $F$ is faithful then we have an injection $\Hom(X,X)\to\Hom(F(X),F(X)) = \{0\}$. Thus $X=0_{\mathscr{A}}$.
\par For the other direction, note that $F$ induces a morphism of abelian groups $F:\Hom(X,Y)\to\Hom(F(X),F(Y))$.
Observe that $F(\Ker(F)) = 0$, therefore $\Ker(F) = 0$. Since we have an exact sequence,
\begin{center}
    \begin{tikzcd}
0 \arrow[r] & \ker F \arrow[r] & {\Hom(X,Y)} \arrow[r] & \image F \arrow[r] & 0
\end{tikzcd}
\end{center}
Therefore, we conclude $F$ is faithful.
\end{document}