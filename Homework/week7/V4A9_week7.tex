\documentclass{article}

\usepackage{amsmath}
\usepackage{amssymb}
\usepackage{amsthm}
\usepackage{ mathrsfs }
\usepackage{ mathtools}
\usepackage{enumerate}
\usepackage{bbm}
\usepackage{lipsum}
\usepackage{fancyhdr}
\usepackage{tikz-cd} 
\usepackage{nicematrix}
\usepackage{ stmaryrd }
\usetikzlibrary{arrows}
\newcommand{\midarrow}{\tikz \draw[-triangle 90] (0,0) -- +(.1,0);}

\newtheorem{theorem}{Theorem} [section] 
\newtheorem{proposition}{Proposition}[section] 
\newtheorem{defi}{Definition}[section] 
\newtheorem{lemma}{Lemma}[section] 
\newtheorem{notation}{Notation}[section] 
\newtheorem{remark}{Remark}[section] 
\newtheorem{corollary}{Corollary} [section] 
\newtheorem{terminology}{Terminology}[section] 
\newtheorem{fact}{Fact}[section] 
\newtheorem{example}{Example}[section] 
\newtheorem{claim}{Claim}[section] 
\DeclareMathOperator{\ev}{ev}
\DeclareMathOperator{\pr}{pr}
\DeclareMathOperator{\st}{s.t.}
\DeclareMathOperator{\triv}{triv}
\DeclareMathOperator{\Aut}{Aut}
\DeclareMathOperator{\Stab}{Stab}
\DeclareMathOperator{\Ind}{Ind}
\DeclareMathOperator{\cind}{c-Ind}
\DeclareMathOperator{\GL}{GL}
\DeclareMathOperator{\SL}{SL}
\DeclareMathOperator{\SO}{SO}
\DeclareMathOperator{\Sp}{Sp}
\DeclareMathOperator{\Rep}{Rep}
\DeclareMathOperator{\esssup}{esssup}
\DeclareMathOperator{\diam}{diam}
\DeclareMathOperator{\rank}{rank}
\DeclareMathOperator{\Hom}{Hom}
\DeclareMathOperator{\End}{End}
\DeclareMathOperator{\Image}{Im}
\DeclareMathOperator{\Ker}{Ker}
\DeclareMathOperator{\Dom}{Dom}
\DeclareMathOperator{\grad}{grad}
\DeclareMathOperator{\Span}{Span}
\DeclareMathOperator{\interior}{int}
\DeclareMathOperator{\supp}{supp}
\DeclareMathOperator{\id}{id}
\DeclareMathOperator{\sgn}{sgn}
\DeclareMathOperator{\Mat}{Mat}
\DeclareMathOperator{\Res}{Res}
\DeclareMathOperator{\diag}{diag}
\DeclareMathOperator{\Mor}{Mor}
\DeclareMathOperator{\Sets}{Sets}
\DeclareMathOperator{\Ab}{Ab}
\DeclareMathOperator{\Grp}{Grp}
%\newcommand*{\name}[\num_arguments][default values]{{\color{#1}\Large #2}}
\newcommand{\defeq}{\vcentcolon=}
\newcommand{\norm}[1]{\Vert #1 \Vert}
\newcommand{\opNorm}[2]{\norm{#1}_{#2\to#2}}
\newcommand{\normL}[3]{\norm{#1}_{L^{#2}(#3)}}
\newcommand{\N}[0]{\mathbb{N}}
\newcommand{\R}[0]{\mathbb{R}}
\newcommand{\Z}[0]{\mathbb{Z}}
\newcommand{\C}[0]{\mathbb{C}}
\newcommand{\Q}[0]{\mathbb{Q}}
\newcommand{\F}[0]{\mathbb{F}}
\newcommand{\G}[0]{\mathbb{G}}
\newcommand{\Para}[0]{\mathbb{P}}
\newcommand{\M}[0]{\mathbb{M}}
\newcommand{\torus}[0]{\mathbb{T}}
\newcommand{\Ouv}[0]{\mathcal{O}}
\newcommand{\sep}[0]{\:|\:}
\newcommand{\hecke}[0]{\mathscr{H}}
\newcommand{\zet}[0]{\mathfrak{z}}

\newcommand{\fib}[1]{%
  \mathbin{\mathop{\times}\limits_{#1}}%
}
\newcommand{\tens}[1]{%
  \mathbin{\mathop{\otimes}\displaylimits_{#1}}%
}

\title{V4A9 Homework 6}
\author{So Murata, Heijing Shi}
\date{WiSe 25/26, University of Bonn Number Theory 1}

\begin{document}
\maketitle
\par\textbf{(1)}
For each $w\in W$, denote $f_w$ be such that
\begin{equation*}
    f_w(1) = w, f(g) =0\forall g\in H\backslash H'. 
\end{equation*}
Then $f_w$ is compactly supported modulo $H'$ and is a smooth element since $H'$ is open, and any compact open $K\subseteq H'$ is again compact open in $H$. Clearly, the set 
\begin{equation*}
    \{\cind_{H'}^H\sigma(g)f_w\sep g\in H'\backslash H,w\in W\},
\end{equation*}
generates the whole representation. For each $\varphi\in\Hom_H(\sigma,\pi|_{H'})$, define,
\begin{equation*}
    \varphi^*(f_w) = \varphi(w).
\end{equation*}
Then by setting, for $g\in H'\backslash H$, 
\begin{equation*}
    \varphi^*(gf_w) = \pi(g)\varphi(w),
\end{equation*}
we can extend it to an element $\varphi^*\in\Hom_H(\cind_{H'}^H\sigma,\pi)$ by observing for any $f\in\cind_{H'}^HW$,
\begin{equation*}
f = \sum_{g\in H'\backslash H}g^{-1}f_{f(g)}.
\end{equation*}
Similarly, if we are given $\psi\in\Hom_H(\cind_{H'}^H\sigma,\pi)$, we can examine at each $f_w$. Since this is an induced representation which is compatible with actions $\sigma$, this 
induces a morphism $\psi|_{H'}:W\to V|_{H'}$. Orbits of $f_w$-s generate the whole $\cind_{H'}^H W$, we obtain this gives an isomorphism.
\par\textbf{(2)} Since $(\pi_0,W)$ is irreducible it is generated by a single element $w\in W$. Since $G^\circ$ is normal in $G$, $G^\circ \backslash G = G/G^\circ$. 
Let $\{g_1,\cdots,g_n\}$ be the generator of $G/G^\circ$. Define $f\in\cind_{G^\circ}^GW$ to be such that 
\begin{equation*}
    f(1) = w, f(g_i)=0\forall g_i\not\in G^\circ.
\end{equation*}
Then $f$ is indeed compactly supported modulo $G^\circ$. Set 
\begin{equation*}
    \{\cind_{G^\circ}^G(g_i)f\}_{i=1,\cdots,n},
\end{equation*}
this set generates $\cind_{G^\circ}^GW$. Indeed, any $f\in\cind_{G^\circ}^GW$ is uniquely determined by the image of $g_i$ since 
\begin{equation*}
    f(g_iG^\circ) = f(G^\circ g_i) = \sigma(G^\circ)f(g_i).
\end{equation*}
Therefore, the set spans the whole.\\
\par\textbf{(3)} For $\phi\in\hecke(G,\pi^\circ)$ and $f\in\Pi$, define
\begin{equation*}
    t_\phi(f) = \left[G\ni x\mapsto \sum_{g\in G/G^\circ}\phi(g)f(g^{-1}x)\right].
\end{equation*}
This is clearly linear and defines an element in $\End(\Pi)$. We show this is an isomorphism. This preserves the multiplication since,
\begin{align*}
    t_{\phi\ast\psi}(f)(x) & = \sum_{g\in G/G^\circ}(\phi\ast\psi)(g)f(g^{-1}x),\\
    & = \sum_{g\in G/G^\circ}\sum_{h\in G/G^{\circ}}\phi(h)\psi(h^{-1}g)f(g^{-1}x),\quad(\text{set $g' = h^{-1}g$})\\
    & = \sum_{h\in G/G^\circ}\sum_{g'\in G/G^{\circ}}\psi(g')f((g')^{-1}h^{-1}x),\\
    & = t_\phi(t_\psi(f))(x).
\end{align*}
The last equality is justified since each $\phi,\psi$ are supported in finite union of $G^\circ$-cosets. Now define 
\begin{equation*}
    \End(\Pi)\ni t\mapsto [G\ni [W\ni w\mapsto t(f_w)(g)]].
\end{equation*}
Then since $t$ is a $H$-representation morphism, we have,
\begin{equation*}
    t(gx)(f_w) = t(f_w)(gx) = \pi^\circ(g) t(f_w)(x).
\end{equation*}
Then we have,
\begin{equation*}
    t_\phi(f_w)(g) = \phi(g),
\end{equation*}
since $f_w$ is supported in $G^\circ$. Also,
\begin{equation*}
    \sum_{g\in G/G^\circ}t(f_w)(g)f_{v}(g^{-1}x) = t(f_w)(1_G).
\end{equation*}
\par\textbf{(4)}\\
\par\textbf{(a)}
Let us define the addition as the point-wise addition and the multiplication as the composition. Indeed, for $z^1,z^2\in\mathfrak{z}(\mathscr{A})$ and an object $A\in\mathscr{A}$, since $z^1_A,z^2_A$ are in $\Mor_{\mathscr{A}}(A,A)$, they indeed commute. The multiplicative identity $(\id_A)_{A\in\mathscr{A}}$.
\par\textbf{(b)} We have $\Mor_{(\Sets)}(\{\ast\},A) = A$ and $z_{\{\ast\}} = \id_{\{\ast\}}$. Therefore, $z_A = \id_A$. That is to say $\mathfrak{z}(\Sets) = \{\ast\}$.
\par\textbf{(c)} Note that we have for any abelian group $A$, $\Hom(\Z,A)\cong A$. Since $z\in\mathfrak{z}(\Ab)$ is a natural transformation, we have, 
\begin{center}
    \begin{tikzcd}
\Z \arrow[d, "z_\Z"'] \arrow[r] & A \arrow[d, "z_A"] \\
\Z \arrow[r]                    & A                 
\end{tikzcd}
\end{center}
The value of $z_A$ evaluated at each $a\in A$, is $\varphi(z_\Z(1))=z_\Z(1)$. Therefore, we conclude $\mathfrak{z}(\Ab) = \Z$.\\
\par\textbf{(d)} Since for any group $G$ we have $\Z\to G, 1\mapsto g$. Then we have,
\begin{equation*}
z_G(g) = g^{z_\Z(1)}.
\end{equation*}
Consider $G=\langle x,y\rangle$. Then %show that we have a counter example.
\par\textbf{(e)} We have, $\Hom_G(\C[G],V) \cong V$. Therefore, by the similar argument we conclude $\mathfrak{z}(\Rep(G)) = \C[Z(G)]$.
\par\textbf{(f)} We have $\Hom(R,M) \cong M$ as $R$-modules. Thus $z_R$ determines $z$, we have,
\begin{center}
    \begin{tikzcd}
R \arrow[r, "z'_R"] \arrow[d, "z_R"'] & R \arrow[d, "z_R"] \\
R \arrow[r, "z_R'"']                  & R                 
\end{tikzcd}
\end{center}
Therefore the center of $R$-mod is the center of the ring.
\par\textbf{(5)}
\par\textbf{(a)}
\begin{equation*}
    \mathfrak{z}_\pi=[z\mapsto z_\pi\in\End(\pi,V)].
\end{equation*}
\par\textbf{(b)}Consider $(\sigma,V) = \cind_{K}^G(\triv)$. Then we have,
\begin{equation*}
    \Hom_G(V,V) \cong V^K, \chi\mapsto\chi(1).
\end{equation*}
Note that 
\begin{equation*}
    V^K = (\cind_K^G\triv)^K \cong\hecke(G,K).
\end{equation*}
Since we have $f(kgk') = f(g)$ for any $g\in G$ and $k,k'\in K$. Now we want to show that $\hecke(G,K)$ is finitely generated as $Z(\hecke(G,K))$-module. Note that $V$ is generated by $\id_K$. 
\begin{equation*}
    z(\mathfrak{z}(\Rep(G)))\subseteq\End_G(V)
\end{equation*}
is finitely generated as $\C$ algebra. $\Hom_G(V,V)=V^K$ is finitely generated as $\mathfrak{z}(\Rep(G))$-module. Note that $Z(\hecke(G,K))\supseteq \mathfrak{z}(\Rep(G))$ in $\End_G(V)$.
Given by 
\begin{equation*}
    \hecke(G,K)\ni f\mapsto \sigma(f) \in\End_G(V),
\end{equation*}
and induces
\begin{equation*}
    \mathfrak{z}(\Rep(G))\to\mathfrak{z}_\sigma(\mathfrak{z}(\Rep(G)))\subseteq\End_G(V).
\end{equation*}
We have 
\begin{equation*}
    \hecke(G,K)=\End_G(V)
\end{equation*}
as $\C$-algebra. Thus,
\begin{equation*}
    Z(\hecke(G,K)) = Z(\End_G(V)).
\end{equation*}
We have,
\begin{equation*}
\zet_\sigma(\zet(\Rep(G)))\subseteq Z(\End_G(V))\subseteq\hecke(G,K).
\end{equation*}

Note that $\zet_\sigma(\zet(\Rep(G)))$ is finitely generated as $\C$ algebra therefore Noetherian. And $\hecke(G,K)$ is finitely generated as $\zet_\sigma(\zet(\Rep(G)))$-module. 
Thus $Z(\End_G(V))$ is also finitely generated. %check $Z(\hecke(G,K))$ is finitely generatd as $\C$-algebra.

\end{document}