\documentclass{article}

\usepackage{amsmath}
\usepackage{amssymb}
\usepackage{amsthm}
\usepackage{ mathrsfs }
\usepackage{ mathtools}
\usepackage{enumerate}
\usepackage{bbm}
\usepackage{lipsum}
\usepackage{fancyhdr}
\usepackage{tikz-cd} 
\usepackage{nicematrix}
\usepackage{ stmaryrd }
\usetikzlibrary{arrows}
\newcommand{\midarrow}{\tikz \draw[-triangle 90] (0,0) -- +(.1,0);}

\newtheorem{theorem}{Theorem} [section] 
\newtheorem{proposition}{Proposition}[section] 
\newtheorem{defi}{Definition}[section] 
\newtheorem{lemma}{Lemma}[section] 
\newtheorem{notation}{Notation}[section] 
\newtheorem{remark}{Remark}[section] 
\newtheorem{corollary}{Corollary} [section] 
\newtheorem{terminology}{Terminology}[section] 
\newtheorem{fact}{Fact}[section] 
\newtheorem{example}{Example}[section] 
\newtheorem{claim}{Claim}[section] 
\DeclareMathOperator{\ev}{ev}
\DeclareMathOperator{\pr}{pr}
\DeclareMathOperator{\st}{s.t.}
\DeclareMathOperator{\triv}{triv}
\DeclareMathOperator{\Aut}{Aut}
\DeclareMathOperator{\Stab}{Stab}
\DeclareMathOperator{\Ind}{Ind}
\DeclareMathOperator{\GL}{GL}
\DeclareMathOperator{\SL}{SL}
\DeclareMathOperator{\SO}{SO}
\DeclareMathOperator{\Sp}{Sp}
\DeclareMathOperator{\Rep}{Rep}
\DeclareMathOperator{\esssup}{esssup}
\DeclareMathOperator{\diam}{diam}
\DeclareMathOperator{\rank}{rank}
\DeclareMathOperator{\Hom}{Hom}
\DeclareMathOperator{\End}{End}
\DeclareMathOperator{\Image}{Im}
\DeclareMathOperator{\Ker}{Ker}
\DeclareMathOperator{\Dom}{Dom}
\DeclareMathOperator{\grad}{grad}
\DeclareMathOperator{\Span}{Span}
\DeclareMathOperator{\interior}{int}
\DeclareMathOperator{\supp}{supp}
\DeclareMathOperator{\id}{id}
\DeclareMathOperator{\sgn}{sgn}
\DeclareMathOperator{\Mat}{Mat}
\DeclareMathOperator{\Res}{Res}
%\newcommand*{\name}[\num_arguments][default values]{{\color{#1}\Large #2}}
\newcommand{\defeq}{\vcentcolon=}
\newcommand{\norm}[1]{\Vert #1 \Vert}
\newcommand{\opNorm}[2]{\norm{#1}_{#2\to#2}}
\newcommand{\normL}[3]{\norm{#1}_{L^{#2}(#3)}}
\newcommand{\N}[0]{\mathbb{N}}
\newcommand{\R}[0]{\mathbb{R}}
\newcommand{\Z}[0]{\mathbb{Z}}
\newcommand{\C}[0]{\mathbb{C}}
\newcommand{\Q}[0]{\mathbb{Q}}
\newcommand{\F}[0]{\mathbb{F}}
\newcommand{\G}[0]{\mathbb{G}}
\newcommand{\Para}[0]{\mathbb{P}}
\newcommand{\M}[0]{\mathbb{M}}
\newcommand{\torus}[0]{\mathbb{T}}
\newcommand{\Ouv}[0]{\mathcal{O}}

\newcommand{\fib}[1]{%
  \mathbin{\mathop{\times}\limits_{#1}}%
}
\newcommand{\tens}[1]{%
  \mathbin{\mathop{\otimes}\displaylimits_{#1}}%
}

\title{V4A9 Homework 3}
\author{So Murata, Heijing Shi}
\date{WiSe 25/26, University of Bonn Number Theory 1}

\begin{document}
\maketitle
\par\textbf{(1)}\\
\par\textbf{(a)}
Observe that for any $n\in N$, we have,
\begin{equation*}
    v = \nu(n)^{-1}\pi(n)v-\nu(n)^{-1}(\pi(n)v-\nu(n)v).
\end{equation*}
For $\Leftarrow$, we see that 
\begin{align*}
    v = {\frac 1 {\vert N_K\vert }}\int_{N_K}vd\mu_N = \underbrace{{\frac 1 {\vert N_K\vert }}\int_{N_K}\nu(n)^{-1}\pi(n)vd\mu_N}_{=0}-{\frac 1 {\vert N_K\vert }}\int_{N_K}\nu(n)^{-1}(\pi(n)v-\nu(n)v)d\mu_N.
\end{align*}
Thus $v\in V_\nu$.

\par\textbf{(b)}
The functor preserves injectivity since for any $\varphi:(\tau,U)\to (\pi,V)$ a morphism of $N$-representation commutes with $\tau,\pi$ by definition and $\nu$ as it is $\C$-linear.
Since taking a quotient vector space is right-exact, this completes the proof.

\par\textbf{(c)}

Since $(\cdot)_\nu$ is exact, the injectivity is clear. In $V(N)_\nu$, each representative is of the form,
\begin{equation*}
    \nu(n)v-v
\end{equation*}
for some $n\in N$.

We also have,
\begin{center}
    \begin{tikzcd}
0 \arrow[r] & V(N) \arrow[r] \arrow[d] & V \arrow[r] \arrow[d] & V_N \arrow[r] \arrow[d] & 0 \\
0 \arrow[r] & V(N)_\nu \arrow[r]       & V_\nu \arrow[r]       & (V_N)_\nu \arrow[r]     & 0
\end{tikzcd}
\end{center}
as $(\cdot)_\nu$ is exact. Note that $N$ acts trivially on $V_N$, thus elements of $(V_N)_\nu$ is of the form,
\begin{equation*}
    v-\nu(n)v,
\end{equation*}
which is in $V(N)_\nu$. Thus we conclude $V(N)_\nu\to V_\nu$ is an isomorphism.

\par\textbf{(d)}

By considering a trivial character $\nu$, we obtain,
\begin{equation*}
    V(N)= V(\nu).
\end{equation*}

From \textbf{(a)}, we have,

\begin{equation*}
    V(N) = V(\nu) = \{v\in V\:|\: \exists K_N\text{ open compact, subgroup}, e_{K_N}v = 0\}.
\end{equation*}

\par\textbf{(2)}
\par\textbf{(a)}
Let $v\in V\backslash\{0\}$, consider
\begin{equation*}
    W = \Span_\C{\pi(g)v\:|\:g\in N}.
\end{equation*}
Then this is irreducible representation. Thus the subrepresentation $(\pi,W)$ has a central character $\nu$. This extends to $(\pi,V)$. Then the character $-\nu$, $v\not\in V(-\nu)$.
\par\textbf{(b)}
From the first part and considering the exact sequences,
\begin{center}
    \begin{tikzcd}
0 \arrow[r] & V(N) \arrow[r] \arrow[d] & V \arrow[r] \arrow[d] & V_N \arrow[r] \arrow[d] & 0 \\
0 \arrow[r] & V(N)_\nu \arrow[r]       & V_\nu \arrow[r]       & (V_N)_\nu \arrow[r]     & 0
\end{tikzcd}
\end{center}
we have,
\begin{equation*}
    V(N)\simeq V.
\end{equation*}
Note that $N=\G_a$ thus $N=Z(N)$. From this, observe that $V$ has a central character, namely the trivial representation from the exercise 1. But $V_\nu=0$ tells that $V$ also has a central character which is $\nu$. But $\nu\not=\triv$, thus we conclude $V=\{0\}$.
\end{document}