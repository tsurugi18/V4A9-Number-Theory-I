\documentclass{article}

\usepackage{amsmath}
\usepackage{amssymb}
\usepackage{amsthm}
\usepackage{ mathrsfs }
\usepackage{ mathtools}
\usepackage{enumerate}
\usepackage{bbm}
\usepackage{lipsum}
\usepackage{fancyhdr}
\usepackage{tikz-cd} 
\usepackage{nicematrix}
\usetikzlibrary{arrows}
\newcommand{\midarrow}{\tikz \draw[-triangle 90] (0,0) -- +(.1,0);}

\newtheorem{theorem}{Theorem} [section] 
\newtheorem{proposition}{Proposition}[section] 
\newtheorem{defi}{Definition}[section] 
\newtheorem{lemma}{Lemma}[section] 
\newtheorem{notation}{Notation}[section] 
\newtheorem{remark}{Remark}[section] 
\newtheorem{corollary}{Corollary} [section] 
\newtheorem{terminology}{Terminology}[section] 
\newtheorem{fact}{Fact}[section] 
\newtheorem{example}{Example}[section] 
\DeclareMathOperator{\ev}{ev}
\DeclareMathOperator{\st}{s.t.}
\DeclareMathOperator{\triv}{triv}
\DeclareMathOperator{\Aut}{Aut}
\DeclareMathOperator{\Stab}{Stab}
\DeclareMathOperator{\Ind}{Ind}
\DeclareMathOperator{\GL}{GL}
\DeclareMathOperator{\SL}{SL}
\DeclareMathOperator{\SO}{SO}
\DeclareMathOperator{\Sp}{Sp}
\DeclareMathOperator{\Rep}{Rep}
\DeclareMathOperator{\esssup}{esssup}
\DeclareMathOperator{\diam}{diam}
\DeclareMathOperator{\rank}{rank}
\DeclareMathOperator{\Hom}{Hom}
\DeclareMathOperator{\End}{End}
\DeclareMathOperator{\Image}{Im}
\DeclareMathOperator{\Ker}{Ker}
\DeclareMathOperator{\Dom}{Dom}
\DeclareMathOperator{\grad}{grad}
\DeclareMathOperator{\Span}{Span}
\DeclareMathOperator{\interior}{int}
\DeclareMathOperator{\supp}{supp}
\DeclareMathOperator{\id}{id}
\DeclareMathOperator{\sgn}{sgn}
\DeclareMathOperator{\Mat}{Mat}
\DeclareMathOperator{\tr}{tr}
%\newcommand*{\name}[\num_arguments][default values]{{\color{#1}\Large #2}}
\newcommand{\defeq}{\vcentcolon=}
\newcommand{\norm}[1]{\Vert #1 \Vert}
\newcommand{\opNorm}[2]{\norm{#1}_{#2\to#2}}
\newcommand{\normL}[3]{\norm{#1}_{L^{#2}(#3)}}
\newcommand{\N}[0]{\mathbb{N}}
\newcommand{\R}[0]{\mathbb{R}}
\newcommand{\Z}[0]{\mathbb{Z}}
\newcommand{\C}[0]{\mathbb{C}}
\newcommand{\Q}[0]{\mathbb{Q}}
\newcommand{\F}[0]{\mathbb{F}}
\newcommand{\G}[0]{\mathbb{G}}
\newcommand{\Para}[0]{\mathbb{P}}
\newcommand{\M}[0]{\mathbb{M}}
\newcommand{\torus}[0]{\mathbb{T}}
\newcommand{\mes}[1]{\vert #1 \vert}

\newcommand{\fib}[1]{%
  \mathbin{\mathop{\times}\limits_{#1}}%
}
\newcommand{\tens}[1]{%
  \mathbin{\mathop{\otimes}\displaylimits_{#1}}%
}

\title{V4A9 Homework 2}
\author{So Murata}
\date{WiSe 25/26, University of Bonn Number Theory 1}

\begin{document}
\maketitle
\textbf{Exercise (1)}
Let $v\in W^\perp$, then for any $w\in W$ and $g\in H$,,
\begin{equation*}
    (\pi(g)v,w) = (v,\pi(g^{-1})w) = 0,
\end{equation*}
since $W$ is a subrepresentation. Thus $W^\perp$ is a subrepresentation. 
\par Consider a compact open subgroup $K$ and $V^K$. Since this is admissible, we know $V^K$ is finite dimensional and $W^K$ and $W^{\perp,K}$ are subspaces. 
We have 
\begin{equation*}
    V^K = W^K\oplus (W^K)^\perp\cap V^K.
\end{equation*}
and clearly,
\begin{equation*}
    (W^K)^\perp\cap V^K\supseteq (W^\perp)^K.
\end{equation*}
Let $v\in (W^K)^\perp\cap V^K$, then there exists $v'\in (W^K)^\perp$ such that 
\begin{equation*}
    e_Kv' = v.
\end{equation*}
Take an arbitrary element $w\in W$, since a positive definite bilinear form restricted to some subspce is still positive definite, we observe,
\begin{align*}
    \langle v, w\rangle & = \langle e_Kv',w\rangle,\\
    & = \left\langle {\frac 1 {\mes{K}}}\int_K \pi(g)v'dg,w\right\rangle,\\
    & = \langle v',e_Kw\rangle,\\
    & = 0.
\end{align*}
The last part follows from that $e_Kw\in W^K$. Here I used the $H$-invariance and the property of Haar measure, together with $K$ being compact. Thus we conclude $v\in (W^\perp)^K$.
\par We now have $V^K = W^K\oplus (W^\perp)^K$. We conclude $V = W\oplus W^\perp$.
\par\textbf{Exercise (2)}
Recall that $V_1,V_2$ are irreducible, so is $V_1\otimes V_2$. $V_1\otimes V_2$ consists of basis of the form 
\begin{equation*}
    v_1\otimes v_2
\end{equation*}
where $v_1\in V_1,v_2\in V_2$. A subspace fixed by a compact subgroup $K_1\times K_2$ is, for $k_1\in K_1,k_2\in K_2$,
\begin{equation*}
    (\pi_1\otimes\pi_2)(k_1,k_2)v_1\otimes v_2 = (\pi_1(k_1)v_1)\otimes (\pi_2(k_2)v_2).
\end{equation*}
Therefore, we have,
\begin{equation*}
    (V_1\otimes V_2)^{K_1\times K_2} = V_1^{K_1}\otimes V_2^{K_2}.
\end{equation*}
In particular, if $V_1,V_2$ are admissible so is $V_1\otimes V_2$. 
\par Admissible and irreducible representations have their contragredient representations again admissible and irreducible. 
To see this observe, let $W$ be a subrepresentation of $\tilde{V}$ then we have 
\begin{center}
    \begin{tikzcd}
0 \arrow[r] & W \arrow[r] & \tilde{V} \arrow[r] & \tilde{V}/W \arrow[r] & 0
\end{tikzcd}
\end{center}
as vectorspaces. Using $V$ is admissible, and taking contragredient is a contravariant exact functor, we have 
\begin{center}
    \begin{tikzcd}
0 \arrow[r] & \widetilde{\tilde{V}/W} \arrow[r] & \tilde{\tilde{V}} \arrow[r] & \tilde{W} \arrow[r] & 0
\end{tikzcd}
\end{center}
Since $V$ is irreducible, one of the arrows has to be isomorphism.\\
\par For our case, $\widetilde{V_1\otimes V_2}$ is admissible and irreducible since $V_1\otimes V_2$ is.
$\tilde{V_1},\tilde{V_2}$ are admissible and irreducible so is $\tilde{V_1}\otimes\tilde{V_2}$.
\par The map 
\begin{equation*}
    \lambda_1\otimes \lambda_2\mapsto[v_1\otimes v_2\mapsto \lambda_1(v_1)\lambda_2(v_2)]
\end{equation*}
is a morphism of $H$-representations which is not $0$. By Schur's lemma, we have they are isomorphic.\\
\textbf{(3)}\\
\par\textbf{(a)}Since for the trace function in $\Mat_{n\times n}(\C)$, we have for any $A,B\in\Mat_{n\times n}(\C)$, $\tr(AB) = \tr(BA)$.
From this and $\pi(g)\in\GL(V)$ where $V$ is finite dimensional, we have 
\begin{equation*}
    \tr(\pi(g^{-1})\pi(h)\pi(g)) = \tr(\pi(h)\pi(g)\pi(g^{-1})) = \tr(\pi(h)).
\end{equation*}
\par\textbf{(b)}Without loss of generality, we assume $V$ has a basis $\{b_1,\cdots,b_n\}$. Since $V$ is finite dimensional we have $V=\bigoplus_{i=1}^k V_i$ for some irreducible subrepresentations. 
On each subrepresentation, since it is irreducible and finite, there exists a Hermitian matrix $A_i$ which corresponds to a positive-definite, $H$-invariant Hermitian form. 
Let us define a positive definite Hermitian matrix $A$ to be such that 
\begin{equation*}
    A \defeq \begin{pmatrix}
        A_1&O&\cdots&O\\
        O&A_2&\cdots&O\\
        \vdots & \vdots & \ddots &\vdots\\
        O&O&\cdots&A_k
    \end{pmatrix}.
\end{equation*}
Thus we conclude $(\pi,V)$ is unitary. In the proof for the existence of the formal degree of irreducible and finite representations, 
the irreducibility was used to prove the admissibility and the unitariness of the contragredient representations besides that for a bilinear operator $f:V\times \tilde{V}\to\C$, there is $c$ such that 
\begin{equation*}
f(v,\lambda) = c\lambda(v).
\end{equation*}
However, $V$ is of finite dimension, thus $\tilde{V}$ is also finite dimensional. Thus its contragredient is admissible. 
Since we have 
\begin{equation}
    V = \bigoplus_{i=1}^k V_i.
\end{equation}
For any $v\in V$, and $\lambda\in V^*$, we get 
\begin{equation*}
    v = v_1\oplus\cdots\oplus v_k,\quad \lambda = \lambda_1\oplus\cdots\oplus\lambda_k.
\end{equation*}
Therefore,
\begin{equation*}
    m_{\lambda,v}(g) = \sum_{i=1}^k\lambda_i(\pi(g)v_i).
\end{equation*}
By using Schur's orthogonality, we get 
\begin{equation*}
    \int_H m_{\lambda,v}(g)m_{\mu,u}(g^{-1})dg = \sum_{i=1}^k{\frac 1 {\deg(V_i)}}\lambda_i(u_i)\mu_i(v_i).
\end{equation*}
Thus $(\pi,V)$ admits a formal degree.\\
\par Furthermore, $V^*=\tilde{V}$ since there exists a compact open subgroup set $K$ such that $V=V^K$ . This follows from that 
each basis element is contained in some $V^{K_i}$. By taking the intersection of $K_i$ and using $H$ is Hausdorff, we conclude $K=\bigcap_{i} K_i$ is again a compact and open subgroup
which fixes all the basis elements, so does any elements. We see $V^* = (V^*)^K$ for such $K$.\\
\par Take $\lambda\in V^*$ to be such that it maps each basis element to $1$. Thus we have 
\begin{equation*}
    \deg(V)\int_H m_{\lambda,\sum_{i=1}^n b_i}m_{\lambda,b_1}dg = \lambda(b_1+\cdots+b_n)\lambda(b_1)\dim(V).
\end{equation*}
Thus scaling $dg$ by $\int_H m_{\lambda,\sum_{i=1}^n b_i}m_{\lambda,b_1}$, we obtain the statement.
\par \textbf{(4)}
Let $f_1,f_2\in\mathcal{H}$, then 
\begin{equation*}
    \tau(f_1+f_2)w = \tau(f_1)w+\tau(f_2)w,
\end{equation*}
by the linearity of integrals. Now consider $f_1\ast f_2$, indeed, we have 
\begin{align*}
    \int_H (f_1\ast f_2)(h)\tau(h)wdh &= \int_H\int_H f_1(g)f_2(g^{-1}h)\tau(h)wdgdh.
\end{align*}
Substitute $h\to gh$ and use the property of Haar measure, we have,
\begin{align*}
    \int_H (f_1\ast f_2)(h)\tau(h)wdh &= \int_H\int_H f_1(g)f_2(h)\tau(gh)wdgdh,\\
    &= \int_H f_2(h)\tau(h)\int_H f_1(g)\tau(g)wdgdh,\\
    & = \tau(f_2)\tau(f_1)w.
\end{align*}
For the $H\times H$ equivariance, we have using the property of Haar measure again,
\begin{align*}
    \tau((h_1,h_2)f(g))w & = \tau(f(h_1^{-1}gh_2))wdg, \\
    & = \int_H f(h_1^{-1}gh_2)\tau(g)wdg,\\
    & = \int_H f(g)\tau(h_1gh_2^{-1})wdg, \\
    & = \tau(h_1)\tau(f)\tau(h_2^{-1})w.
\end{align*}
\end{document}