\documentclass{article}

\usepackage{amsmath}
\usepackage{amssymb}
\usepackage{amsthm}
\usepackage{ mathrsfs }
\usepackage{ mathtools}
\usepackage{enumerate}
\usepackage{bbm}
\usepackage{lipsum}
\usepackage{fancyhdr}
\usepackage{tikz-cd} 
\usepackage{nicematrix}
\usetikzlibrary{arrows}
\newcommand{\midarrow}{\tikz \draw[-triangle 90] (0,0) -- +(.1,0);}

\newtheorem{theorem}{Theorem} [section] 
\newtheorem{proposition}{Proposition}[section] 
\newtheorem{defi}{Definition}[section] 
\newtheorem{lemma}{Lemma}[section] 
\newtheorem{notation}{Notation}[section] 
\newtheorem{remark}{Remark}[section] 
\newtheorem{corollary}{Corollary} [section] 
\newtheorem{terminology}{Terminology}[section] 
\newtheorem{fact}{Fact}[section] 
\newtheorem{example}{Example}[section] 
\DeclareMathOperator{\ev}{ev}
\DeclareMathOperator{\st}{s.t.}
\DeclareMathOperator{\triv}{triv}
\DeclareMathOperator{\Aut}{Aut}
\DeclareMathOperator{\Stab}{Stab}
\DeclareMathOperator{\Ind}{Ind}
\DeclareMathOperator{\GL}{GL}
\DeclareMathOperator{\SL}{SL}
\DeclareMathOperator{\SO}{SO}
\DeclareMathOperator{\Sp}{Sp}
\DeclareMathOperator{\Rep}{Rep}
\DeclareMathOperator{\esssup}{esssup}
\DeclareMathOperator{\diam}{diam}
\DeclareMathOperator{\rank}{rank}
\DeclareMathOperator{\Hom}{Hom}
\DeclareMathOperator{\End}{End}
\DeclareMathOperator{\Image}{Im}
\DeclareMathOperator{\Ker}{Ker}
\DeclareMathOperator{\Dom}{Dom}
\DeclareMathOperator{\grad}{grad}
\DeclareMathOperator{\Span}{Span}
\DeclareMathOperator{\interior}{int}
\DeclareMathOperator{\supp}{supp}
\DeclareMathOperator{\id}{id}
\DeclareMathOperator{\sgn}{sgn}
\DeclareMathOperator{\Mat}{Mat}
%\newcommand*{\name}[\num_arguments][default values]{{\color{#1}\Large #2}}
\newcommand{\defeq}{\vcentcolon=}
\newcommand{\norm}[1]{\Vert #1 \Vert}
\newcommand{\opNorm}[2]{\norm{#1}_{#2\to#2}}
\newcommand{\normL}[3]{\norm{#1}_{L^{#2}(#3)}}
\newcommand{\N}[0]{\mathbb{N}}
\newcommand{\R}[0]{\mathbb{R}}
\newcommand{\Z}[0]{\mathbb{Z}}
\newcommand{\C}[0]{\mathbb{C}}
\newcommand{\Q}[0]{\mathbb{Q}}
\newcommand{\F}[0]{\mathbb{F}}
\newcommand{\G}[0]{\mathbb{G}}
\newcommand{\Para}[0]{\mathbb{P}}
\newcommand{\M}[0]{\mathbb{M}}
\newcommand{\torus}[0]{\mathbb{T}}

\newcommand{\fib}[1]{%
  \mathbin{\mathop{\times}\limits_{#1}}%
}
\newcommand{\tens}[1]{%
  \mathbin{\mathop{\otimes}\displaylimits_{#1}}%
}

\title{V4A9 Homework 3}
\author{So Murata}
\date{WiSe 25/26, University of Bonn Number Theory 1}

\begin{document}
\maketitle
\par\textbf{(1)}
Let $v\otimes\lambda,u\otimes\mu\in V\otimes\tilde{V}$. Then their composition is 
\begin{equation*}
    [V\ni w\mapsto \mu(w)u]\circ[V\ni w\mapsto \lambda(w)v] = [V\ni w\mapsto \lambda(w)\mu(v)u].
\end{equation*}
Thus the corresponding element in hecke algebra is 
\begin{equation*}
    f_{\mu(v)\lambda, u}(h) = \deg\pi\mu(v)\lambda(\pi(h^{-1})(u)).
\end{equation*}
Consider the convolution,
\begin{equation*}
    f_{u,\mu}\ast f_{v,\lambda}(h) = (\deg\pi)^2\int_{H} \mu(\pi(g^{-1})(u))\lambda(\pi(h^{-1}g)(v))dg.
\end{equation*}
Let us denote $\lambda' = \tilde{\pi}(g)\lambda$. Since $(V,\pi)$ is finite and irreducible, by Schur orthogonality, we obtain 
\begin{equation*}
    f_{u,\mu}\ast f_{v,\lambda}(h) = (\deg\pi)^2\int_H m_{u,\mu}(g^{-1})m_{v,\lambda'}(g)dg = \deg\pi\lambda'(u)\mu(v) = \deg\pi\mu(v)\lambda(\pi(h^{-1})(v)).
\end{equation*}
Thus $\phi$ preserves products. 
\par\textbf{(2)}
\par\textbf{(b)}
\begin{equation*}
    (e^\pi)(\tau(g)(w)) = \tau(e^\pi_{K})(\tau(g)(w)) = \tau(e^\pi_{gKg^{-1}})(w) = \tau((g\times g)e_K^\pi)(w) = g\cdot e^\pi(w).
\end{equation*}
\par\textbf{(c)}
By Lemma 25 from the class, since $K\subseteq K$, therefore,
\begin{equation*}
    e^\pi\circ e^\pi (w)  = \tau(e_K^\pi)\circ\tau(e_K^\pi)(w) = \tau(e_K^\pi\ast e_K^\pi)(w) = \tau(e_K^\pi)(w) = e^\pi(w).
\end{equation*}

\par\textbf{(d)}
Consider a $H$-equivariant morphism $\alpha:W_1\to W_2$. Then for any $w\in W_1$, we have 
\begin{equation*}
    w\in W_1^K\Rightarrow \alpha(w)\in W_2^K.
\end{equation*}
Furthermore, $e^\pi_K$ is compactly supported by the definition, thus this commutes with a linear map $\alpha$, thus,
\begin{equation*}
    \tau(e_K^\pi)(\alpha(w)) = \alpha(\tau(e_K^\pi)(w)).
\end{equation*}

\par\textbf{3.}
Let $W$ be a subrepresentation of $(\pi,V)$ for $V = \bigoplus_{i\in I} V_i$ for irreducible subrepresentations $V_i$. Then 
\begin{equation*}
    V =W\oplus W'
\end{equation*}
for another subrepresentation $W'$. Let $U\subseteq W$ be also a subrepresentation then again,
\begin{equation*}
    V = U\oplus U',
\end{equation*}
Then 
\begin{equation*}
    W = W\cap U\oplus W\cap U' = U\oplus (W\cap U').
\end{equation*}
Thus $W$ is semisimple. Remains to show the quotient is semisimple. 
Let $W = \bigoplus_{j\in J}W_j$ where $W_j$ is an irreducible subrepresentaion. Consider an inclusion $\iota_j:W_j\hookrightarrow V$. Since $W_j$ is irreducible, we have $\iota_j(W_j)$ is an irreducible subrepresentation of $V$ thus there is $i\in I$ such that 
\begin{equation*}
    W_j  = V_i.
\end{equation*}

This measn that 
\begin{equation*}
    V/W = \bigoplus_{\substack{i\in I\\ V_i\not\subseteq W}}V_i.
\end{equation*}
Thus a quotient is semisimple if $V$ is semisimple. We conclude a subquotient is semisimple.


\end{document}