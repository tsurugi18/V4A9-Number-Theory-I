\documentclass{article}

\usepackage{amsmath}
\usepackage{amssymb}
\usepackage{amsthm}
\usepackage{ mathrsfs }
\usepackage{ mathtools}
\usepackage{enumerate}
\usepackage{bbm}
\usepackage{lipsum}
\usepackage{fancyhdr}
\usepackage{tikz-cd} 
\usepackage{nicematrix}
\usepackage{ stmaryrd }
\usetikzlibrary{arrows}
\newcommand{\midarrow}{\tikz \draw[-triangle 90] (0,0) -- +(.1,0);}

\newtheorem{theorem}{Theorem} [section] 
\newtheorem{proposition}{Proposition}[section] 
\newtheorem{defi}{Definition}[section] 
\newtheorem{lemma}{Lemma}[section] 
\newtheorem{notation}{Notation}[section] 
\newtheorem{remark}{Remark}[section] 
\newtheorem{corollary}{Corollary} [section] 
\newtheorem{terminology}{Terminology}[section] 
\newtheorem{fact}{Fact}[section] 
\newtheorem{example}{Example}[section] 
\newtheorem{claim}{Claim}[section] 
\DeclareMathOperator{\ev}{ev}
\DeclareMathOperator{\pr}{pr}
\DeclareMathOperator{\st}{s.t.}
\DeclareMathOperator{\triv}{triv}
\DeclareMathOperator{\Aut}{Aut}
\DeclareMathOperator{\Stab}{Stab}
\DeclareMathOperator{\Ind}{Ind}
\DeclareMathOperator{\cind}{c-ind}
\DeclareMathOperator{\GL}{GL}
\DeclareMathOperator{\SL}{SL}
\DeclareMathOperator{\SO}{SO}
\DeclareMathOperator{\Sp}{Sp}
\DeclareMathOperator{\Rep}{Rep}
\DeclareMathOperator{\val}{val}
\DeclareMathOperator{\esssup}{esssup}
\DeclareMathOperator{\diam}{diam}
\DeclareMathOperator{\rank}{rank}
\DeclareMathOperator{\Hom}{Hom}
\DeclareMathOperator{\End}{End}
\DeclareMathOperator{\Image}{Im}
\DeclareMathOperator{\Ker}{Ker}
\DeclareMathOperator{\Dom}{Dom}
\DeclareMathOperator{\grad}{grad}
\DeclareMathOperator{\Span}{Span}
\DeclareMathOperator{\interior}{int}
\DeclareMathOperator{\supp}{supp}
\DeclareMathOperator{\id}{id}
\DeclareMathOperator{\ord}{ord}
\DeclareMathOperator{\sgn}{sgn}
\DeclareMathOperator{\Mat}{Mat}
\DeclareMathOperator{\Res}{Res}
\DeclareMathOperator{\diag}{diag}
\DeclareMathOperator{\Mor}{Mor}
\DeclareMathOperator{\Sets}{Sets}
\DeclareMathOperator{\Ab}{Ab}
\DeclareMathOperator{\Grp}{Grp}
\DeclareMathOperator{\image}{im}
\DeclareMathOperator{\coker}{coker}
\DeclareMathOperator{\nr}{nr}
%\newcommand*{\name}[\num_arguments][default values]{{\color{#1}\Large #2}}
\newcommand{\defeq}{\vcentcolon=}
\newcommand{\norm}[1]{\Vert #1 \Vert}
\newcommand{\opNorm}[2]{\norm{#1}_{#2\to#2}}
\newcommand{\normL}[3]{\norm{#1}_{L^{#2}(#3)}}
\newcommand{\N}[0]{\mathbb{N}}
\newcommand{\R}[0]{\mathbb{R}}
\newcommand{\Z}[0]{\mathbb{Z}}
\newcommand{\C}[0]{\mathbb{C}}
\newcommand{\Q}[0]{\mathbb{Q}}
\newcommand{\F}[0]{\mathbb{F}}
\newcommand{\G}[0]{\mathbb{G}}
\newcommand{\Para}[0]{\mathbb{P}}
\newcommand{\M}[0]{\mathcal{M}}
\newcommand{\torus}[0]{\mathbb{T}}
\newcommand{\Ouv}[0]{\mathcal{O}}
\newcommand{\sep}[0]{\:|\:}
\newcommand{\hecke}[0]{\mathscr{H}}
\newcommand{\zet}[0]{\mathfrak{z}}
\newcommand{\conj}[2]{\prescript{#1}{}{#2}}

\newcommand{\fib}[1]{%
  \mathbin{\mathop{\times}\limits_{#1}}%
}
\newcommand{\tens}[1]{%
  \mathbin{\mathop{\otimes}\displaylimits_{#1}}%
}

\title{V4A9 Homework 10}
\author{So Murata, Heijing Shi}
\date{WiSe 25/26, University of Bonn Number Theory 1}

\begin{document}
\maketitle

\par\textbf{1.}
Clearly if all $\phi_i\in Z(\hecke(G,G^\circ,\pi^\circ))$ then $\phi\in Z(\hecke(G,G^\circ,\pi^\circ))$. Let $\{\phi_1,\cdots,\phi_n\}$ be elements of $\hecke(G,G^\circ,\pi^\circ)$ such that supported in $x_iG^\circ$ and
\begin{equation*}
    \{\phi_1^{e_1}\ast\cdots\ast\phi_n^{e_n}\sep e_1,\cdots,e_n\in\Z\}
\end{equation*}
forms the basis of the Hecke algebra. By assumption, we have,
\begin{equation*}
    \phi\ast\phi_k = \phi_k\ast\phi,
\end{equation*}
for all $k=1,\cdots,n$. But we have,
\begin{equation*}
    \phi\ast\phi_k = \sum_{i=1}^r \phi_i\ast\phi_k = \sum_{i=1}^r c_{ik}\phi_k\ast\phi_i =\sum_{i=1}^r \phi_k\ast\phi_i.
\end{equation*}
As they are basis we have $c_{ik}=1$. Thus the statement is proven.

\par\textbf{2.}

The map 
\begin{equation*}
    [tG^\circ]\mapsto\varphi_{[t^{-1}]},
\end{equation*}
respects the addition furthermore for $x\in stG^\circ$, $s^{-1}x\in tG^\circ$ therefore,
\begin{equation*}
    [stG^\circ]\mapsto \varphi_{[t^{-1}s^{-1}]}=[x\mapsto\sigma_T(s)\sigma_T(s^{-1}x)] = \varphi_{[s^{-1}]}\ast\varphi_{[t^{-1}]}.
\end{equation*}

Therefore this is an algebra homomorphism. The algebra $\C[T/G^\circ]$ is well-defined by Lemma 95 from the lecture. Remains to show that $\{\varphi_{[t^{-1}]}\}_{t\in T/G^\circ}$. From Lemma 70 and and the construction in the proof, $\{\phi_{[t^{-1}]}\}_{t\in T/G^\circ}$ is a generator of the algebra. Thus we have a one-to-one correspondence between generators, this induces an isomorphism.

\par\textbf{3.}
\par\textbf{(a)}
First we have $(gk)K(gk)^{-1}=gKg^{-1}$. Let $\phi\in\Hom_{K\cap gKg^{-1}}(\prescript{g}{}{\rho},\rho)$ then set 
\begin{equation*}
    \phi'\defeq \phi\circ\rho(k).
\end{equation*}
Then we have,
\begin{align*}
    \phi'(\prescript{gk}{}{\rho}(x)(w)) & = \phi(\prescript{g}{}{\rho}(x)\rho(k)(w)),\\
    & = \rho(x)\phi(\rho(k)(w)),\\
    & = \rho(x)\phi'(w).
\end{align*}
Similarly, for $\phi\in \Hom_{K\cap gKg^{-1}}(\prescript{g}{}{\rho},\rho)$, set
\begin{equation*}
    \phi'\defeq\rho(k)\circ\phi.
\end{equation*}
Then we have,
\begin{align*}
    \phi'(\rho^{kg}(h)(w))& = \rho(k)\circ\phi(\rho^{g}(k^{-1}hk)(w)),\\
    & = \rho(k)\rho(k^{-1}hk)\circ\phi(w),\\
    & = \rho(h)\phi'(w).
\end{align*}
\par\textbf{(b)}
\begin{remark}
    Two characters intertwines if and only if they are the same.
\end{remark}
Suppose $g$ intertwines $\psi$ if and only if there is $f:\C\to\C$ linear such that 
\begin{equation*}
    f(\psi(g^{-1}xg)(v)) = \psi(x)f(v).\Leftrightarrow\psi(g^{-1}xg)=\psi(x)
\end{equation*}
holds on $x\in ZI_1\cap gZI_1g^{-1}$. From previous part, we would like to look at the representatives of $I_1\backslash \SL(F)/I_1$. Note that we have,
\begin{equation*}
    \SL_2(F)=\bigsqcup_{w\in N_G(T)/T\cap I}IwI
\end{equation*}
where 
\begin{align*}
    T &= \left\{\begin{pmatrix}
        a &\:\\
        \: & a^{-1}
    \end{pmatrix}
    \Bigg{|}\:a\in F^{\times}\right\}\\,I\cap T& = \left\{\begin{pmatrix}
        a &\:\\
        \: & a^{-1}
    \end{pmatrix}
    \Bigg{|}\:a\in \mathcal{O}_F^\times\right\}, \\N_G(T)&=\left\{\begin{pmatrix}
        a &\:\\
        \: & a^{-1}
    \end{pmatrix},
    \begin{pmatrix}
        \: & a \\
        -a^{-1} &\:
    \end{pmatrix}
    \Bigg{|}\:a\in F^{\times}\right\},\\\tilde{W} &= \left\{\begin{pmatrix}
        \varpi^k &\:\\
        \: & \varpi^{-k}
    \end{pmatrix},
    \begin{pmatrix}
        \: & \varpi^{k} \\
        -\varpi^{-k} &\:
    \end{pmatrix}
    \Bigg{|}\:k\in\Z\right\}.
\end{align*}
As $I$ is a preimage of $B(k_F)$, we have 
\begin{equation*}
    \SL_2(\mathcal{O}_F)=K\stackrel{r}{\to}\SL_2(k_F),
\end{equation*}
We then have,
\begin{align*}
    I&\defeq r^{-1}(B(k_F)),\\
    I_1&\defeq r^{-1}(U(k_F)),\\
    I/I_1 &\cong B(k_F)/U(k_F)=T(k_F).\\
    T(\mathcal{O}_F)&\to T(k_F).
\end{align*}
Note that we have,
\begin{equation*}
    I = \bigcup_{t,\text{diagonal}}I_1tI_1,IwI = \bigcup_{t}I_1wtI_1.
\end{equation*}
Thus we need to check 
\begin{equation*}
    g = \begin{pmatrix}
        m & \:\\
        \: & m^{-1}
    \end{pmatrix},\begin{pmatrix}
        \: & m\\
        -m^{-1}&\:
    \end{pmatrix}
\end{equation*}
We only show the case for the latter. Consider $x\in g^{-1}ZI_1g$, such that 
\begin{equation*}
    g^{-1}xg = \begin{pmatrix}
    1 & 0\\
    -m^{2}b & 1
    \end{pmatrix}\in ZI_1.
\end{equation*}
This means that $\val(m^{-2}b) = 1$, thus $\val(b)>2\val(m)+1$. We have,
\begin{equation*}
    \varpi(-\varpi^{-1}m^{-2}b).
\end{equation*}
Therefore,
\begin{equation*}
    \psi(g^{-1}xg)=\phi(x)\Leftrightarrow\forall b\in\mathcal{O}_F, \val(m^{-2}b)\geq1\Leftrightarrow\val(\varpi^{-1}m^{-2}b)\geq0\psi(\overline{b}) = \phi(\overline{-\varpi^{-1}m^{-2}b})
\end{equation*}
Suppose $\val(m)=n$ and $2n+1>0$. Thus the valuation of $b$ is positive thus $\psi(\overline{b})=1$. Take $b_0$ be such that $\psi(\overline{b}_0)\not=0$, we have $b=-\varpi m^2b_0$ is a counter example.
\par If $2n+1<0$, then $b\in\mathcal{O}_F$. Therefore,
\begin{equation*}
    \psi(\overline{\varpi^{-1}m^2b})=1.
\end{equation*}
Take $b_0$ be such that $\psi(b_0)\not=1$, we will have a contradiction.
\par\textbf{(c)}
From Homework 6 (b), if $K$ is an ope group compact modulo center of $G$ and $\psi$ is irreducible we have, $\cind_K^G\psi$ is irreducible implies it is supercuspidal. The strategy is to use  the Schur's lemma 
\begin{equation*}
    \End_{G}(\cind_{ZI_1}^G(\psi))=\C.
\end{equation*}
The above isomorphism holds if and only if $\psi$ is irreducible under the assumption it is unitary. From previous arguments, we have,
\begin{equation*}
    \Hom_G(\cind_{ZI_1}^G(\psi),\Ind_{ZI_1}^G(\psi))\cong \prod_{ZI_1\backslash G/ ZI_1}[\Hom_{ZI_1\cap g^{-1}(ZI_1)g}(\prescript{g^{-1}}{}{\psi},\psi)]
\end{equation*}
From previous argument, we have 
\begin{equation*}
    \dim \Hom_{ZI_1\cap g^{-1}(ZI_1)g}(\prescript{g^{-1}}{}{\psi},\psi)=\begin{cases}
        1,\quad(g\in ZI_1),\\
        0,\quad(\text{otherwise}).
    \end{cases}
\end{equation*}
Since $\End_G(\cind_{ZI_1}^G(\psi))$ can be embedded to $\Hom_G(\cind_{ZI_1}^G(\psi),\Ind_{ZI_1}^G(\psi))$, and the latter is one dimensional the former has to be one dimensional.
\begin{remark}
    Let $\pi$ be an irreducible representation of $G$ and $(\pi^\circ,W)\subseteq(\pi|_{G^\circ},V)$. Take,
    \begin{equation*}
        T\defeq\bigcap_{\chi\in X_{\nr}(G)_\pi}\ker(\chi)
    \end{equation*}
    Then we have,
    \begin{equation*}
        \bigcap_{\substack{\chi\in X_{\nr}(G)\\ \chi|_H=1}}\ker\chi=H = N_G(W) \subseteq N_G(\pi^\circ).
    \end{equation*}
    $T\subseteq H$. The key fact is that $\pi=\cind_H^G(\pi^\circ)$ and $\nu\in X_{\nr}(G)$, then,
    \begin{equation*}
        \pi\cdot\nu = \cind_H^G(\pi^\circ)\nu = \cind_H^G(\pi^\circ\nu|_H).
    \end{equation*}
    If $\nu|_H=1$, then $\nu\in X_{\nr}(G)_\pi$.
\end{remark}
\par\textbf{4.}
Suppose $g$ intertwines $\rho$, then pick
\begin{equation*}
    \phi\in\Hom_{K\cap gKg^{-1}}(\prescript{g}{}{\rho},\rho).
\end{equation*}
Set,
\begin{equation*}
    f(h)=\begin{cases}
        \rho(k_1)\circ\phi\rho(k_2)\quad(h=k_1gk_2),\\
        0\quad(\text{otherwise}).
    \end{cases}
\end{equation*}
This is well-defined and has a compact support by the equivalent condition about intertwining. For the other direction, suppose $f(g)=\phi\not=0$ then for $x\in K\cap gKg^{-1}$,
\begin{align*}
\rho(x^{-1})f(g)\rho(g^{-1}xg) & = f(x^{-1}gg^{-1}xg),\\
& = f(g).\\
\Rightarrow \rho(x)\phi & = \phi(\prescript{g}{}{\rho}(x)(w)).
\end{align*}
Thus $g$ intertwines $\rho$.
\end{document}