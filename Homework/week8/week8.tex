\documentclass{article}

\usepackage{amsmath}
\usepackage{amssymb}
\usepackage{amsthm}
\usepackage{ mathrsfs }
\usepackage{ mathtools}
\usepackage{enumerate}
\usepackage{bbm}
\usepackage{lipsum}
\usepackage{fancyhdr}
\usepackage{tikz-cd} 
\usepackage{nicematrix}
\usepackage{ stmaryrd }
\usetikzlibrary{arrows}
\newcommand{\midarrow}{\tikz \draw[-triangle 90] (0,0) -- +(.1,0);}

\newtheorem{theorem}{Theorem} [section] 
\newtheorem{proposition}{Proposition}[section] 
\newtheorem{defi}{Definition}[section] 
\newtheorem{lemma}{Lemma}[section] 
\newtheorem{notation}{Notation}[section] 
\newtheorem{remark}{Remark}[section] 
\newtheorem{corollary}{Corollary} [section] 
\newtheorem{terminology}{Terminology}[section] 
\newtheorem{fact}{Fact}[section] 
\newtheorem{example}{Example}[section] 
\newtheorem{claim}{Claim}[section] 
\DeclareMathOperator{\ev}{ev}
\DeclareMathOperator{\pr}{pr}
\DeclareMathOperator{\st}{s.t.}
\DeclareMathOperator{\triv}{triv}
\DeclareMathOperator{\Aut}{Aut}
\DeclareMathOperator{\Stab}{Stab}
\DeclareMathOperator{\Ind}{Ind}
\DeclareMathOperator{\cind}{c-Ind}
\DeclareMathOperator{\GL}{GL}
\DeclareMathOperator{\SL}{SL}
\DeclareMathOperator{\SO}{SO}
\DeclareMathOperator{\Sp}{Sp}
\DeclareMathOperator{\Rep}{Rep}
\DeclareMathOperator{\esssup}{esssup}
\DeclareMathOperator{\diam}{diam}
\DeclareMathOperator{\rank}{rank}
\DeclareMathOperator{\Hom}{Hom}
\DeclareMathOperator{\End}{End}
\DeclareMathOperator{\Image}{Im}
\DeclareMathOperator{\Ker}{Ker}
\DeclareMathOperator{\Dom}{Dom}
\DeclareMathOperator{\grad}{grad}
\DeclareMathOperator{\Span}{Span}
\DeclareMathOperator{\interior}{int}
\DeclareMathOperator{\supp}{supp}
\DeclareMathOperator{\id}{id}
\DeclareMathOperator{\ord}{ord}
\DeclareMathOperator{\sgn}{sgn}
\DeclareMathOperator{\Mat}{Mat}
\DeclareMathOperator{\Res}{Res}
\DeclareMathOperator{\diag}{diag}
\DeclareMathOperator{\Mor}{Mor}
\DeclareMathOperator{\Sets}{Sets}
\DeclareMathOperator{\Ab}{Ab}
\DeclareMathOperator{\Grp}{Grp}
%\newcommand*{\name}[\num_arguments][default values]{{\color{#1}\Large #2}}
\newcommand{\defeq}{\vcentcolon=}
\newcommand{\norm}[1]{\Vert #1 \Vert}
\newcommand{\opNorm}[2]{\norm{#1}_{#2\to#2}}
\newcommand{\normL}[3]{\norm{#1}_{L^{#2}(#3)}}
\newcommand{\N}[0]{\mathbb{N}}
\newcommand{\R}[0]{\mathbb{R}}
\newcommand{\Z}[0]{\mathbb{Z}}
\newcommand{\C}[0]{\mathbb{C}}
\newcommand{\Q}[0]{\mathbb{Q}}
\newcommand{\F}[0]{\mathbb{F}}
\newcommand{\G}[0]{\mathbb{G}}
\newcommand{\Para}[0]{\mathbb{P}}
\newcommand{\M}[0]{\mathbb{M}}
\newcommand{\torus}[0]{\mathbb{T}}
\newcommand{\Ouv}[0]{\mathcal{O}}
\newcommand{\sep}[0]{\:|\:}
\newcommand{\hecke}[0]{\mathscr{H}}
\newcommand{\zet}[0]{\mathfrak{z}}

\newcommand{\fib}[1]{%
  \mathbin{\mathop{\times}\limits_{#1}}%
}
\newcommand{\tens}[1]{%
  \mathbin{\mathop{\otimes}\displaylimits_{#1}}%
}

\title{V4A9 Homework 8}
\author{So Murata, Heijing Shi}
\date{WiSe 25/26, University of Bonn Number Theory 1}

\begin{document}
\maketitle
\par\textbf{(1)}
We have,
\begin{equation*}
    \hecke(G,G^\circ,\pi^\circ) = \Span_\C\{c\phi_1^{e_1}\ast\cdots\ast\phi_n^{e_n}\sep e_n\in\Z\}.
\end{equation*}
Define 
\begin{equation*}
    \phi_1^{e_1}\ast\cdots\ast\phi_n^{e_n}\geq\phi_1^{e'_1}\ast\cdots\ast\phi_n^{e'_n}
\end{equation*}
in the dictionary order. Let $\psi_1\psi_2\in\hecke(G,G^\circ,\pi^\circ)$ then $\psi_1\ast\psi_2$ then $\psi_1\ast\psi_2\not=0$, 
Then we have 
\begin{equation*}
    (\text{leading term of $\psi_1$})\times (\text{leading term of $\psi_2$}) = (\text{leading term of $\psi_1\ast\psi_2$}).
\end{equation*}
Note that 
\begin{equation*}
    \phi_i\ast\phi_j = c_{ij}\phi_j\ast\phi_i(c_{ij}\not=0).
\end{equation*}
Thus above argument is justified.
\par\textbf{(2)}
Let $P\subsetneq G$ be a proper parabolic subgroup. For $K\subseteq P$ a compact open subgroup, set 
\begin{equation*}
    \delta_P(p) = \vert pKp^{-1}/pKp^{-1}\cap K\vert \vert K/pKp^{-1}\cap K\vert^{-1}.
\end{equation*}
Let $\mu$ be a left Haar measure, over $P$, and the homeomorphism 
\begin{equation*}
    g\mapsto gp^{-1},
\end{equation*}
induces another left Haar measure such that $\delta'(p)\mu = \mu_P$, then 
\begin{equation*}
    \delta'(p)\mu(K) = \mu_P(K) = \mu(Kp^{-1}) = \mu(p^{-1}pKp^{-1}) = \mu(pKp^{-1})
\end{equation*}
Thus $\delta(p) = {\frac {\mu(pKp^{-1})} {\mu(K)}}$. If $K'\subset K$ be compact open subgroup then 
\begin{equation*}
    K = \bigsqcup_{i=1}^n x_iK', n = [K:K'].
\end{equation*}
Thus we have $\mu(K) = n\mu(K')$ thus $[K:K'] = {\frac {\mu(K)} {\mu(K')}}$. THus $\delta'(p) = \delta_P(p)$ which is invariant. Note that left Haar measure is unique up to positive constant.
\par The triviality on a compact open subgroup follows from that we have proven the definition is invariant of the choice of $K$, thus for any compact open subgroup $K$, define $\delta_P$ using such $K$. 
\par It is indeed a character indeed for $K\subseteq P$, compact, 
\begin{equation*}
    \delta_P(pq)\mu(K) = \mu(K(pq)^{-1}) = \delta_P(p)\delta_P(q)\mu(K).
\end{equation*}

\par\textbf{(4)}

We know that 
\begin{equation*}
    B = TN,
\end{equation*}
where $T$ is the diagonal matrices and $N$ is the unipotent subgroup whose diagonals are all $1$. To determine the value of $\delta_B$, it is enough to do it separately on $T$ and $N$. 
    We have $K = (1+\varpi\Mat_{n\times n}(\Ouv_F))\cap B$ is a compact open subgroup of $B$.
\begin{example}[Toy case, $n=2$]
    \begin{equation*}
        \begin{pmatrix}
            1& x\\
            0 & 1
        \end{pmatrix}
        \begin{pmatrix}
            1+a\varpi& b\varpi\\
            c\varpi & 1+d\varpi
        \end{pmatrix}
        \begin{pmatrix}
            1& -x\\
            0 & 1
        \end{pmatrix}
        = 
        \begin{pmatrix}
            1 + a\varpi & dx\varpi+b\varpi-ax\varpi\\
            0 & 1+d\varpi
        \end{pmatrix}
    \end{equation*}
    Also note that $\ord_\varpi(x) = \ord_\varpi(-x)$, thus 
    \begin{equation*}
        pKp = p^{-1}Kp,
    \end{equation*}
    where $p = \begin{pmatrix}
            1& x\\
            0 & 1
        \end{pmatrix}$ thus 
        \begin{equation*}
            \delta_B(p) = \delta_B(p^{-1}).
        \end{equation*}
        Thus $\delta_B(p) = 1$.
\end{example}
Another argument is that 
\begin{remark}
    The unipotent radical is exhausted by compact subgroups. Thus modulus character is trivial on the unipotent radical.
\end{remark}
\par For the diagonal part and let us suppose that $n=\ord_\varpi\left({\frac {t_i} {t_j}}\right)\geq0$.
\begin{equation*}
    \underbrace{\begin{pmatrix}
        t_1 & \:&\:\\
        \: & \ddots & \:\\
        \: & \: & t_n
    \end{pmatrix}}_{=t\in T}
    K
    \begin{pmatrix}
        t_1^{-1} & \:&\:\\
        \: & \ddots & \:\\
        \: & \: & t_n^{-1}
    \end{pmatrix}
     = 
     \begin{pmatrix}
        1+\varpi\Ouv_F & \:&\:\\
        \: & \ddots & \:\\
        \: & \: & 1+\varpi\Ouv_F
    \end{pmatrix}
\end{equation*}
Thus $tKt^{-1}\subset K$, thus the problem amounts to show that 
\begin{equation*}
    \delta_B(t) = \vert tKt^{-1}/K\vert.
\end{equation*}
Let $\lambda\in 1+\underbrace{\varpi\Ouv_F/1+\varpi^{n+1}\Ouv_F}_{q^n}$,
\begin{equation*}
    (I+\lambda E_{ij})tKt^{-1}
\end{equation*}
Thus we conclude,
\begin{equation*}
    \delta_B(t) = \prod_{i<j}\left\vert{\frac {t_i} {t_j}}\right\vert.
\end{equation*}
\par\textbf{(5)}
Recall that we have, the set of roots is 
\begin{equation*}
    \Phi = \{e_i-e_j\sep i\not=j\},\Phi+= \{e_i-e_j\sep i<j\}. 
\end{equation*}
The set of simple roots are ,
\begin{equation*}
    \Delta = \{e_i-e_{i+1}\sep i=1,\cdots,n-1\}, \Delta'\subset\Delta.
\end{equation*}
We claim that $P_{\Delta'}$ is determined by a partition $(n_1,\cdots,n_k)$ such that $n_1+\cdots +n_k = n$. 
\begin{equation*}
    P_{\Delta'}=\left\{
        \begin{pmatrix}
            g_1 & \ast & \ast \\
            O & \ddots & \ast \\
            O & O & g_k
        \end{pmatrix}
        \:\Bigg{|}\: g_i\in\GL_{n_i}
    \right\}.
\end{equation*}
$e_i-e_{i+1}\in\Delta'$ if and only if $i$ and $i+1$ belong to the same block. $P_\Delta = \GL_n$, $P_\emptyset = B$ where $\Delta$ corresponds to $(n)$ and $\emptyset$ corersponds to $(1,\cdots,1)$.
\begin{equation*}
    \delta(p) = \prod_{i=2}^k \vert \det g_i\vert^{-\sum_{j=1}^{i-1}n_j+\sum_{j=i+1}^k n_j}.
\end{equation*} 
if $n_j=$ then 
\begin{equation*}
    \delta(p) = \prod_{p=1}^n \vert t_i\vert^{2i+1-n} = \prod_{i<j}\left\vert {\frac {t_i} {t_j}}\right\vert.
\end{equation*}
From Exercise \textbf{(3)}, we have, 
\begin{equation*}
    \int f(tn) dtdn
\end{equation*}
is a left Haar measure and 
\begin{equation*}
    \int f(nt) dtdn  = \int f(tn)\delta(tn)dtdn
\end{equation*}
is a right Haar measure. We compute the Jacobian of the following variable substitution,
\begin{equation*}
    \underbrace{\begin{pmatrix}
        u_1 & \ast &\ast \\
        \: & \ddots & \ast \\
        \: & \: & u_n
    \end{pmatrix}}_{\text{variable}}
     \underbrace{\begin{pmatrix}
        a_1 & \ast &\ast \\
        \: & \ddots & \ast \\
        \: & \: & a_n
    \end{pmatrix}}_{\text{constant}}
    =
     \begin{pmatrix}
        v_1 & \ast &\ast \\
        \: & \ddots & \ast \\
        \: & \: & v_n
    \end{pmatrix}
\end{equation*}

For the detailed treatment see Goldfreld 14.3.6
\par\textbf{(5)} Dr. Dillery's solution
\par Let $P = MN$ where $M$ is the levi factor and $N$ is the unipotent part. As we have seen, it suffices to calculate 
\begin{equation*}
    \delta_P(m),\forall m\in M.
\end{equation*}

Recall the Cartan decomposition $T_M = Z(M)^\circ(F) = \{t\in T_M\sep \vert \alpha(t)\vert \leq 1\forall\alpha\in\Delta\}$. Use that there is $K_0$ compact open such that 
\begin{equation*}
    G = \coprod_{t\in T_M^+}K_0wt K_0.
\end{equation*}

For the exercise, in $\GL_n$, $w$ can be taken out from the equation. Assuming this we get,

\begin{equation*}
    G = \coprod_{t\in T_M^+}K_0t K_0.
\end{equation*}

To compute $\delta_P(m)$ it suffices to compute $\delta_P(t)$ where $t\in T_m'$. Choose $K$ small enough so that we have,
\begin{equation*}
    t\cdot K\cap P = t(K\cap M)\cdot(K\cap N)t^{-1}.
\end{equation*}

Since $t \in Z(M)(F)$, we have 
\begin{equation*}
    t\cdot K\cap P = (K\cap M)t(K\cap N)t^{-1}.
\end{equation*}
Thus the explicit computation will be 
\begin{equation*}
    \vert t(K\cap N)t^{-1}/t(K\cap N)t^{-1}\cap (K\cap N)\vert \cdot \vert K\cap N/t(K\cap N)t^{-1}\cap (K\cap N)\vert^{-1}.
\end{equation*}
\end{document}