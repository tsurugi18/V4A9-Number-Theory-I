\documentclass{article}

\usepackage{amsmath}
\usepackage{amssymb}
\usepackage{amsthm}
\usepackage{ mathrsfs }
\usepackage{ mathtools}
\usepackage{enumerate}
\usepackage{bbm}
\usepackage{lipsum}
\usepackage{fancyhdr}
\usepackage{tikz-cd} 
\usepackage{nicematrix}
\usepackage{ stmaryrd }
\usetikzlibrary{arrows}
\newcommand{\midarrow}{\tikz \draw[-triangle 90] (0,0) -- +(.1,0);}

\newtheorem{claim}{Claim} 
\newenvironment{claimproof}[1]{\par\noindent\underline{Proof:}\space#1}{\hfill $\blacksquare$}
\newtheorem{theorem}{Theorem} [section] 
\newtheorem{proposition}{Proposition}[section] 
\newtheorem{definition}{Definition}[section] 
\newtheorem{lemma}{Lemma}[section] 
\newtheorem{notation}{Notation}[section] 
\newtheorem{remark}{Remark}[section] 
\newtheorem{corollary}{Corollary} [section] 
\newtheorem{terminology}{Terminology}[section] 
\newtheorem{fact}{Fact}[section] 
\newtheorem{example}{Example}[section] 
\DeclareMathOperator{\st}{s.t.}
\DeclareMathOperator{\rad}{rad}
\DeclareMathOperator{\triv}{triv}
\DeclareMathOperator{\Aut}{Aut}
\DeclareMathOperator{\Stab}{Stab}
\DeclareMathOperator{\Ind}{Ind}
\DeclareMathOperator{\Spec}{Spec}
\DeclareMathOperator{\unitary}{unitary}
\DeclareMathOperator{\cind}{c-Ind}
\DeclareMathOperator{\GL}{GL}
\DeclareMathOperator{\SL}{SL}
\DeclareMathOperator{\SO}{SO}
\DeclareMathOperator{\Sp}{Sp}
\DeclareMathOperator{\Rep}{Rep}
\DeclareMathOperator{\esssup}{esssup}
\DeclareMathOperator{\diam}{diam}
\DeclareMathOperator{\rank}{rank}
\DeclareMathOperator{\Hom}{Hom}
\DeclareMathOperator{\End}{End}
\DeclareMathOperator{\Image}{Im}
\DeclareMathOperator{\Ker}{Ker}
\DeclareMathOperator{\Dom}{Dom}
\DeclareMathOperator{\Mod}{Mod}
\DeclareMathOperator{\grad}{grad}
\DeclareMathOperator{\Span}{Span}
\DeclareMathOperator{\interior}{int}
\DeclareMathOperator{\supp}{supp}
\DeclareMathOperator{\id}{id}
\DeclareMathOperator{\ev}{ev}
\DeclareMathOperator{\gr}{gr}
\DeclareMathOperator{\Lie}{Lie}
\DeclareMathOperator{\sgn}{sgn}
\DeclareMathOperator{\Mat}{Mat}
\DeclareMathOperator{\Vect}{Vect}
\DeclareMathOperator{\supcus}{sc}
\DeclareMathOperator{\nsc}{nsc}
\DeclareMathOperator{\val}{val}
\DeclareMathOperator{\colimsymb}{colim}
\DeclareMathOperator{\Cent}{Cent}
\DeclareMathOperator{\Irr}{Irr}
\DeclareMathOperator{\Res}{Res}
\DeclareMathOperator{\nr}{nr}
\DeclareMathOperator{\nf}{nf}
\DeclareMathOperator{\nn}{nn}
\DeclareMathOperator{\reg}{reg}
\DeclareMathOperator{\res}{res}
\DeclareMathOperator{\Gal}{Gal}
\DeclareMathOperator{\diag}{diag}
\DeclareMathOperator{\Max}{Max}
\DeclareMathOperator{\Ann}{Ann}
\DeclareMathOperator{\St}{St}
%\newcommand*{\name}[\num_arguments][default values]{{\color{#1}\Large #2}}
\newcommand{\defeq}{\vcentcolon=}
\newcommand{\defqe}{=\vcentcolon}
\newcommand{\norm}[1]{\Vert #1 \Vert}
\newcommand{\opNorm}[2]{\norm{#1}_{#2\to#2}}
\newcommand{\normL}[3]{\norm{#1}_{L^{#2}(#3)}}
\newcommand{\Gc}[0]{G^\circ}
\newcommand{\N}[0]{\mathbb{N}}
\newcommand{\A}[0]{\mathbb{A}}
\newcommand{\R}[0]{\mathbb{R}}
\newcommand{\Z}[0]{\mathbb{Z}}
\newcommand{\C}[0]{\mathbb{C}}
\newcommand{\Q}[0]{\mathbb{Q}}
\newcommand{\F}[0]{\mathbb{F}}
\newcommand{\G}[0]{\mathbb{G}}
\newcommand{\Para}[0]{\mathbb{P}}
\newcommand{\M}[0]{\mathbb{M}}
\newcommand{\torus}[0]{\mathbb{T}}
\newcommand{\sep}[0]{\:|\:}
\newcommand{\zero}[0]{\{0\}}
\newcommand{\nonzero}[0]{\backslash\zero}
\newcommand{\hecke}[0]{\mathscr{H}}
\newcommand{\rest}[1]{\bigg{|}_{#1}}
\newcommand{\isomleft}{\stackrel{\sim}{\to}}
\newcommand{\cat}{\mathscr{C}}
\newcommand{\cA}{\mathcal{A}}
\newcommand{\cB}{\mathcal{B}}
\newcommand{\Ouv}{\mathcal{O}}

\newcommand{\fib}[1]{%
  \mathbin{\mathop{\times}\limits_{#1}}%
}
\newcommand{\tens}[1]{%
  \mathbin{\mathop{\otimes}\displaylimits_{#1}}%
}

\newcommand{\colim}[1]{%
  \mathbin{\mathop{\colimsymb}\displaylimits_{#1}}%
}


\title{Representation Theory of $p$-adic Groups}
\author{So Murata}
\date{WiSe 25/26, University of Bonn V4A9 Number Theory 1}

\begin{document}
\maketitle

Throughout the course, we will denote $p$ as a prime number and $F$ as the underlying non-archimeidian local field (ie. $F$ is either a finite extension of $p$-adic numbers $\Q_p$ or $\F_p((t))$. $\mathcal{O}$ as the ring of integers of $F$ and $\varpi\in F^\times$ as a uniformizer (ie. for $F=\Q_p$, we have $\mathcal{O}=\Z_p$, $\overline{\omega}=p$. 
\begin{fact}
We have the following facts about $F$
\begin{enumerate}
\item $F$ has a discrete valuation (namely $p$-adic) thus has a topology.
\item $F$ is totally disconnected with respect to that topology (ie. the only connected components of $F$ are the points).
\item $F^n$ and $\Mat_{n\times n}(F)\cong F^{n^2}$ are equipped with the product topology.
\item $\GL_n(F)$ is equipped with the subspace topology from $\Mat_{n\times n}(F)$.
\item A basis of open neighborhoods of $I_n\in \GL_n(F)$ is given by $\{I_n+ \overline{\omega}^N\cdot\Mat_{n\times n}(\mathcal{O})\}_{N\in\N}$.
\end{enumerate}
\end{fact}

\begin{notation}
We will denote $\G$ as a connected reductive group over $F$. We also write $G=\G(F)$ as the $F$-points of $\G$. In particular, we call them $p$-adic groups.
\end{notation}

\begin{example}
$\GL_n,\SL_n,\SO_n,\Sp_{2n}$ are all $p$-adic groups.
\end{example}

\begin{remark}
There exists $N\in \N$ such that 
\begin{equation*}
G(F)\hookrightarrow \GL_N(F)
\end{equation*}
over $F$. We will equip $\G$ with the subspace topology. (This is independent of the choice of $N$).
\par With this topology $\G$ becomes a topological group which is Hausdorff, totally disconnected, and locally compact. (ie there exists a neighborhood of $g\in\G$ such that it consists of compact open subgroups). 
\end{remark}

\section{Representation Theory}

\subsection{Smooth representations}

\begin{definition}
Let $G$ be a group. A smooth representation is a pair $(\pi,V)$ such that 
\begin{enumerate}[1).]
\item $V$ is a complex vector space.
\item $\pi:G\to \Aut_\C(V)$ is a group homomorphism.
\item $\forall v\in V$, $\Stab_G(v)\defeq\{g\in G\:|\: \pi(g)(v) =v \}$ is open.
\end{enumerate}
\end{definition}

\begin{example}
The trivial representation of $G$ is $(\triv, \C)$ such that for any $g\in G$,
\begin{equation*}
\triv(g) = 1.
\end{equation*}
\end{example}

\begin{proposition}
Let $G$ be a group and $(\pi, V)$ be a representation. Then the following statements are equivalent. 
\begin{enumerate}[1).]
\item $(\pi,V)$ is smooth.
\item $V = \bigcup_{K\subseteq G,\text{ compact open.}}V^K$ where $V^K = \{v\in V\:|\: \forall k\in K, \pi(k)(v) = v\}$.
\item $G\times V\to V,\: (g,v)\mapsto\pi(g)(v)$ is continuous with respect to the discrete topology on $V$.
\end{enumerate}

\label{representation_smooth_equivalence}
\end{proposition}

\begin{proof}
Exercise.
\end{proof}

\begin{definition}
Let $(\pi_1,V_1),(\pi_2,V_2)$ be two smooth representation of $G$. A smooth representation homomorphism between them if a $\C$-linear map $f:V_1\to V_2$ that commutes with $G$-action. In other words, for any $g\in G$, 
\begin{equation*}
f\circ\pi_1(g) =\pi_2(g)\circ f.
\end{equation*}
The space of such homomorphisms is denoted by $\Hom_G(\pi_1,\pi_2)$ or $\Hom_G(V_1,V_2)$. 
\end{definition}

\begin{definition}
We denote $\Rep(G)$ as the category whose objects are smooth representations of $G$ and whose morphisms are smooth representation homomorphisms.
\end{definition}

\begin{definition}
Let $H$ be a closed (with respect to $p$-adic topology) subgroup of $G$. Let $(\sigma, W)$ be a smooth representation of $H$. We define the induced representation
\begin{equation*}
(R,\Ind_H^G(W))
\end{equation*}
where $\Ind_H^G(W)$ is the space of functions $f:G\to W$ such that 
\begin{enumerate}[i).]
\item $\forall h\in H, g\in G$, $f(hg) = \sigma(h)f(g)$. 
\item There is a compact open subgroup $K(f)$ (which depends on $f$) such that $\forall g\in G, k\in K(f), f(gk) = f(g)$. 
\end{enumerate}
The action of $G$ on $\Ind_H^G(W)$ is defined via a right translation (ie $\forall g,x\in G, f\in\Ind_H^G(W)$) we have
\begin{equation*}
R(g)(f)(x) = f(xg).
\end{equation*}
\end{definition}

\begin{notation}
In order to make things explicit, we will denote 
\begin{equation*}
\Ind_H^G\sigma = R.
\end{equation*}
\end{notation}

\begin{definition}[Non-normalized version of Parabolic Induction]
Let $\Para=\M\ltimes\N\subseteq\G$ be a parabolic subgroup with Levi subgroup $\M$ and the unipotent radical $\N$. (We will denote $P = \Para(F),M=\M(F),N=\N(F)$.) Let $(\sigma, W)$ be a smooth representation of $M$. Then the parabolic induction of $(\sigma,W)$ to $G$ is 
\begin{equation*}
\Ind_P^G\sigma = \Ind_{M\ltimes N}^G(\sigma\ltimes\triv)
\end{equation*}
where $\sigma\ltimes\triv:M\ltimes N\to\Aut_\C(W)$ is such that
\begin{equation*}
M\ltimes N\stackrel{\varphi}{\twoheadrightarrow} M\stackrel{\sigma}{\to}\Aut_\C(W)
\end{equation*}
and $\varphi$ is a projection with its kernel $N$.
\end{definition}

\begin{example}
For $\G= \GL_n$, we get
\begin{equation*}
P = \left\{
\begin{pNiceMatrix}[margin]
\Block[borders={left,bottom,right,top}]{1-1}{A} & \ast & \ast & \ast & \ast\\
0 & \Block[borders={left,bottom,right,top}]{3-3}{B} &&& \ast\\
0 & &&&\ast\\
0 & &&&\ast\\
0 & 0 & 0 & 0& \Block[borders={left,bottom,right,top}]{1-1}{C} 
\end{pNiceMatrix}
\:\Bigg{|} A\in\GL_{n_1}(F),B\in\GL_{n_2}(F),C\in\GL_{n_3}(F),n_1+n_2+n_3=n\right\}.
\end{equation*}
and 
\begin{equation*}
M = \left\{
\begin{pNiceMatrix}[margin]
\Block[borders={left,bottom,right,top}]{1-1}{A} & 0 & 0 & 0 & 0\\
0 & \Block[borders={left,bottom,right,top}]{3-3}{B} &&& 0\\
0 & &&&0\\
0 & &&&0\\
0 & 0 & 0 & 0& \Block[borders={left,bottom,right,top}]{1-1}{C} 
\end{pNiceMatrix}
\:\Bigg{|} A\in\GL_{n_1}(F),B\in\GL_{n_2}(F),C\in\GL_{n_3}(F),n_1+n_2+n_3=n\right\}.
\end{equation*}
\begin{equation*}
N = \left\{
\begin{pNiceMatrix}[margin]
\Block[borders={left,bottom,right,top}]{1-1}{I_{n_1}} & \ast & \ast & \ast & \ast\\
0 & \Block[borders={left,bottom,right,top}]{3-3}{I_{n_2}} &&& \ast\\
0 & &&&\ast\\
0 & &&&\ast\\
0 & 0 & 0 & 0& \Block[borders={left,bottom,right,top}]{1-1}{I_{n_3}} 
\end{pNiceMatrix}
\:\Bigg{|} A\in\GL_{n_1}(F),B\in\GL_{n_2}(F),C\in\GL_{n_3}(F),n_1+n_2+n_3=n\right\}.
\end{equation*}
Find examples for other $p$-adic groups $\SL_n,\SO_n\Sp_{2n}$. 
\end{example}
\begin{definition}
Let $(\pi,V)$ be a smooth representation of $G$. A subrepresentation is a subspace $W\subseteq V$ such that $\forall g\in G$, we have
\begin{equation*}
\pi(g)(W)\subseteq W.
\end{equation*}
In other words, the morphism,
\begin{equation*}
G\to\Aut_\C(W)\quad G\ni g\mapsto\pi(g)|_W
\end{equation*}
is well-defined.
\par A representation is irreducible if it contains exactly $2$ subrepresentations, namely $V$ itself and $\{o\}$).
\end{definition}

\begin{example}
A trivial representation is irreducible.
\end{example}

\begin{example}[Exercise]
Let $P$ be a parabolic subgroup of $\SL_2(F)$ of the form
\begin{equation*}
P = \left\{\begin{pmatrix} \ast &\ast\\ 0 &\ast\end{pmatrix}\right\}.
\end{equation*}
Then the parabolic induction of the trivial representation 
\begin{equation*}
\Ind_P^{\SL_2(F)}\triv
\end{equation*}
is not irreducible.
\end{example}

\begin{definition}
An irreducible representation $(\pi,V)$ of $G$ is called supercuspidal if for any proper parabolic subgroup $P=M\ltimes N\subsetneqq G$ and every smooth representation $(\sigma, W)$ of $M$. We have $(\pi,V)$ is not a subrepresentation of $\Ind_P^G\sigma$.
\end{definition}

\begin{example}
The trivial representation $(\triv,\C)$ of $G$ is supercuspidal if and only if $G$ does not contain proper parabolic subgroups. For example $G$ is a torus.
\end{example}

\begin{definition}
A $1$-dimensional representation $G$ is called a character. (ie. $\chi:G\to\C^\times$). A character is called unramified if it is trivial on all compact subgroups of $G$.
\end{definition}

\begin{definition}
Consider two Levi subgroups together with smooth representations $(M,\sigma),(M',\sigma')$. They are called inertially equivalent if there is $g\in G$ and a unramified character $\chi:M'\to C^\times$ such that  
\begin{equation*}
M' = gMg^{-1}
\end{equation*}
and
\begin{equation*}
\sigma' \cong \prescript{g}{}{\sigma}\tens{}\chi
\end{equation*}
where for all $m'\in M'$, $\prescript{g}{}{\sigma}(m') = \sigma(g^{-1}m'g)$.
\par $[M,\sigma]$ denotes an equivalence class of such relation, and we write $I(G)$ to denote the set of inertially equivalent classes.
\end{definition}

\begin{definition}
Let $[M,\sigma]\in I(G)$, we define $\Rep(G)^{[M,\sigma]}$ to be the full subcategory of $\Rep(G)$ such that its objects consist of those smooth representations $(\pi,V)$ of $G$ all of whose irreducible subquotients (quotient of a subrepresentation) embeded into some parabolic induction 
\begin{equation*}
\Ind_{P'=M'\ltimes N'}^G \sigma'
\end{equation*}
for some $(M',\sigma')\in[M,\sigma]$.
\end{definition}

The goal of the course is to prove the following theorem.
\begin{theorem}[Bernstein]
There exists an equivalence of categories between
\begin{equation*}
\Rep(G)\cong\prod_{\{(M,\sigma)\}/\sim}\Rep(G)^{[M,\sigma]}
\end{equation*}
where $M$ is a Levi subgroup of some parabolic (not necessarily proper) subgroup of $G$ and $\sigma$ is a supercuspidal representation of $M$.
\end{theorem}

\begin{fact}
Each $\Rep(G)^{[M,\sigma]}$ is indecomposable (ie. not a product of subcategories). These are called Bernstein blocks and mathematicians understand the structures of many of these blocks quite well as of recently,
\end{fact}

\begin{lemma}
Let $(\pi,V)$ be a smooth irreducible representation of $G$ then $\dim_\C V$ is at most countable. Moreover, if $G$ is compact then $\dim_\C V$ is finite.
\label{smooth_representation_dimension}
\end{lemma}

\begin{proof}
Let $v\in V\backslash\{o\}$. By Proposition \ref{representation_smooth_equivalence}, there is a compact open subgroup $K$ of $G$ such that $v\in V^K$. Since the representation is irreducible, we have
\begin{equation*}
V = \Span\{\pi(g)(v)\:|\: g\in G/K\}.
\end{equation*}
Since $G/K$ is countable, so is the dimension. If $G$ is compact, then it is finite.
\end{proof}

\begin{lemma}[Schur]
Let $H$ be a open subgroup of $G$ and $(\pi,V)$ be an irreducible smooth representation of $H$. We have
\begin{equation*}
\End_H(V) \defeq \Hom_H(V,V) \cong \C.
\end{equation*}
\label{schur_lemma}
\end{lemma}

\begin{proof}
Let $f\in\End_H(V)\backslash\{0\}$. Then the kernel and the image of $f$ are subrepresentations of $V$. However, the representation is irreducible and $f\not=0$ thus
\begin{equation*}
\Ker f=\{0\}, \quad \Image f = V.
\end{equation*}
In particular, $\End_G(V)$ is a division algebra.
\par Let $v\in V\backslash\{0\}$. Then we have
\begin{equation*}
\forall g\in H,\quad f(\pi(g)(v)) = \pi(g)f(v).
\end{equation*}
By the irreducibility, we have
\begin{equation*}
V=\Span\{g\in H\:|\: \pi(g)(v)\}.
\end{equation*}
By Lemma \ref{smooth_representation_dimension}, we have $\End_H(V)$ has at most countable dimension.
\par Suppose $f\not=c\id$ for any $c\in \C$. then 
\begin{equation*}
\{(f-c\id)^{-1}|\: c\in\C\}
\end{equation*}
is linearly independent. %need reasoning
But this set has uncountably many elements. Thus we derived a contradiction. We conclude
\begin{equation*}
\exists c\in\C\:\st\:f=c\id.
\end{equation*}
\end{proof}

\begin{lemma}
Let $(\pi_1,V_1)$ and $(\pi_2,V_2)$ be two irreducible smooth representations of $H$. Then one of the following statements is true.
\begin{enumerate}[1).]
\item They are isomorphic.
\item $\Hom_H(V_1,V_2) \cong\{0\}$.
\end{enumerate}
\label{schur_isomorphism_lemma}
\end{lemma}

\begin{proof}
Exercise.
\end{proof}
\begin{definition}
Let $(\pi,V)$ be a smooth representation of $H$ and $Z(H)$ be the center of $H$. A central character of $(\pi,V)$ is a character $\chi:Z(H)\to\C^\times$ such that
\begin{equation*}
\forall z\in Z(H), \pi(z) = \chi(z)\id.
\end{equation*}
If a smooth representation is said to have a central character if there exists a central character of it.
\end{definition}

\begin{lemma}
  \label{lemma_3}
If $(\pi,V)$ is an irreducible smooth representation of $H$, then $(\pi,V)$ has a central character.
\end{lemma}

\begin{proof}
Let $z\in Z(H)$ then clearly we have
\begin{equation*}
\forall g\in H, \pi(z)\pi(g) =\pi(zg) = \pi(gz) = \pi(g)\pi(z).
\end{equation*}
In particular $\pi(z)\in\Hom_\pi(V,V)$. By Lemma \ref{schur_lemma}, $\pi(z) = c_z\id$ for some $c_z\in\C$. Define a map 
\begin{equation*}
\chi:Z(H)\to \C^\times,\chi(z)=c_z.
\end{equation*}
then it is a central character.
\end{proof}

\begin{lemma}
Let $(\pi_1,V_1),(\pi_2,V_2)$ be two smooth representations of $H$ which have central characters $\chi_1,\chi_2$, respectively. Then $\chi_1\not=\chi_2$ then $\Hom_H(V_1,V_2)=\{0\}$. 
\end{lemma}

\begin{proof}
Exercise.
\end{proof}

\begin{remark}
The converse of the lemma is not true. For example, any commutative groups have the same central character $\chi:\{e\}\to\C^\times$. 
\end{remark}

We fix the notation from now on that $(\pi,V)$ denotes a smooth representation of $H$ and $dg$ denotes a Haar measure on $G$ hence also on $H$ (ie. a left and right invariant measure on $G$).

\begin{remark}
  Over $p$-adic groups, such Haar measure indeed exists for reductive groups.
\end{remark}

\begin{definition}
A topological group is said to be unimodular if it has a left and right invariant Haar measure.
\end{definition}

\begin{definition}
$(\pi,V)$ is called admissible if for each compact open subgroup $K\subseteq H$, the dimension of $V^K$ is finite.
\end{definition}

\begin{definition}
For the dual space $V^*\defeq\Hom_\C(V,\C)$, we define a potentially non-smooth representation $(\pi^*,V^*)$ of $H$ via
\begin{equation*}
\forall g\in H, \forall \lambda\in V^*,(\pi^*(g))(\lambda)(v) = \lambda(\pi(g)^{-1}v).
\end{equation*}
\end{definition}

\begin{remark}
In general, $(\pi^*,V^*)$ is not smooth.
\end{remark}

\begin{definition}
For a representation $(\pi,V)$ of $H$, the maximal smooth subrepresentationt of $V$ is 
\begin{equation*}
W = \bigcup_{\substack{K\subseteq H,\\ \text{\makebox[0pt][c]{compact open}}\\\text{\makebox[0pt][c]{subgroup}}}}V^K.
\end{equation*}
%ask professor
\end{definition}

\begin{definition}
The contragradient representation $(\tilde{\pi},\tilde{V})$ of $(\pi,V)$ is its maximal smooth subrepresentation $(\tilde{\pi^*},\tilde{V^*})$ of it. In other words, $\tilde{\pi}$ is the restriction of $\pi^*$ to $\tilde{V}$ where
\begin{equation*}
\tilde{V} = \bigcup_{\substack{K\subseteq H,\\ \text{\makebox[0pt][c]{compact open}}\\\text{\makebox[0pt][c]{subgroup}}}}(V^*)^K.
\end{equation*}
\end{definition}

\begin{definition}
Let $K$ be a compact open subgroup of $H$. The measure of $K$ is 
\begin{equation*}
\vert K \vert=\int_K 1dg.
\end{equation*}
\end{definition}

\begin{definition}
Let $K$ be a compact open subset of $H$. We define the projector operator $e_K:V\to V$ with respect to $K$ by
\begin{equation*}
e_K(v) = {\frac 1 {\vert K\vert}}\int_K\pi(g)(v)dg.
\end{equation*}
\end{definition}

\begin{remark}
Let $K'$ be some open subgroup of $K$. $K/K'$ contains a finitely many representatives. Let us denote them by $\{g_1,\cdots,g_n\}$. Then the above integral is
\begin{equation*}
\int_K\pi(g)(v)dg = \sum_{i=1}^n \int_{K'}\pi(g_1k)(v)dk = \sum_{i=1}^n \pi(g_i)\int_{K'}\pi(k)(v)dk = \sum_{i=1}^n\vert K'\vert\pi(g_i)(v).
\end{equation*}
\label{projector_finiteness}
\end{remark}

\begin{lemma}
Let $K$ be a compact open subgroup of $K$. We have
\begin{enumerate}[1).]
\item $e_K$ is a projection. In particular $e_K(V) = V^K$. 
\item $V=V^K\oplus(1-e_K)V$ as a $K$-representation.
\item For $v\in V$ and $\lambda\in \tilde{V}$, we have $e_K(\lambda(v)) = \lambda(e_K(V))$.
\item The restriction $V^*\ni \varphi\mapsto \varphi|_{V^K}$ induces an isomorphism $\tilde{V}^K\to(V^K)^*\defeq\Hom_\C(V^K,\C)$. 
\end{enumerate}
\label{projector_operator_properties}
\end{lemma}

\begin{proof}
Exercise.
\end{proof}

\begin{lemma}
The following statements are equivalent.
\begin{enumerate}[1).]
\item $(\pi,V)$ is admissible.
\item $(\tilde{\pi},\tilde{V})$ is admissible.
\item The canonical map $V\ni v\mapsto (V^*\ni\lambda\mapsto\lambda(v)\in V)\in \tilde{V}$ is an isomorphism of $H$-representations.
\end{enumerate}
\label{characteristics_admissiblity}
\end{lemma}

\begin{proof}
Exercise.
\end{proof}

\begin{lemma}
Let $(\pi_1,V_1),(\pi_2,V_2)$ be two smooth representations of $H$. Then we have for any $f\in \Hom_H(V_1,V_2)$ and any compact open subgroup $K\subseteq H$,
\begin{equation*}
f\circ e_K = e_K\circ f.
\end{equation*}
\label{commutativity_projector}
\end{lemma}

\begin{proof}
From Remark \ref{projector_finiteness}, we see
\begin{equation*}
e_K(f(v)) = {\frac {K'} {K}}\sum_{g\in K/K'}\pi(g)(f(v)) = {\frac {K'} {K}}\sum_{g\in K/K'}f(\pi(g)(v)) = f(e_K(v)).
\end{equation*}
\end{proof}

\begin{lemma}
Let $(\pi_i,V_i)$ be smooth representations of $H$ for $i\in\{1,2,3\}$. Then the sequence of smooth $H$-representations 
\begin{center}
\begin{tikzcd}
0 \arrow[r] & V_1 \arrow[r, hook] & V_2 \arrow[r, two heads] & V_3 \arrow[r] & 0
\end{tikzcd}
\end{center}
is exact if and only if for any compact open subgroup $K\subseteq H$, the sequence 
\begin{center}
\begin{tikzcd}
0 \arrow[r] & V_1^K \arrow[r, hook] & V_2^K \arrow[r, two heads] & V_3^K \arrow[r] & 0
\end{tikzcd}
\end{center}
of $\C$-vectorspace is exact.
\label{exactness_representation}
\end{lemma}

\begin{proof}
Let us denote the map $f:V_2\rightarrow V_3$.
\par $\Rightarrow)$: Clearly we have,
\begin{equation*}
V_1\hookrightarrow V_2\Rightarrow V_1\supseteq V_1^K\hookrightarrow V_2^K\subseteq V_2.
\end{equation*}
Also observe that 
\begin{equation*}
\Image(V_1^K\to V_2^K) = \Image(V_1\to V_2)^K = \Ker (V_2\to V_3)^K = \Ker(V_2^K\to V_3^K).
\end{equation*}
Remains to show that $f:V_2\to V_3$ is surjective. Let $v_3\in V_3^K$. Then there exists $v_2\in V_2$ such that $f(v_2)=v_3$. Since $K$ fixes $V_3$. Together with Lemma \ref{commutativity_projector}, we have
\begin{equation*}
v_3 = e_K(v_3) = e_Kf(v_2) = f(e_K(v_2)).
\end{equation*}
\par $\Leftarrow)$ follows from the smoothness. In other words, for each $i\in\{1,2,3\}$,
\begin{equation*}
V_i = \bigcup_{\substack{K\subseteq H\\ \text{compact open subgroup $K$}}}V_i^K.
\end{equation*}
\end{proof}

Recall that dualizing of vector space is a contravariant exact functor. In other words,

\begin{center}
\begin{tikzcd}
0 \arrow[r] & V_1 \arrow[r, hook] & V_2 \arrow[r, two heads] & V_3 \arrow[r] & 0
\end{tikzcd}
\end{center}
is exact, then the following sequence
\begin{tikzcd}
0 \arrow[r] & V_3^* \arrow[r, hook] & V_2^* \arrow[r, two heads] & V_1^* \arrow[r] & 0
\end{tikzcd}

is exact. Together with the lemma above, we derive the following lemma.

\begin{lemma}
The functor $\Rep(H)\to\Rep(H)\quad V\mapsto\tilde{V}$ is exact.
\end{lemma}

\begin{proof}
Suppose we have an exact sequence of smooth $H$-representations
\begin{center}
\begin{tikzcd}
0 \arrow[r] & V_1 \arrow[r, hook] & V_2 \arrow[r, two heads] & V_3 \arrow[r] & 0
\end{tikzcd}
\end{center}
By Lemma \ref{exactness_representation}, we get
\begin{tikzcd}
0 \arrow[r] & V_1^K \arrow[r, hook] & V_2^K \arrow[r, two heads] & V_3^K \arrow[r] & 0
\end{tikzcd}
is exact for any compact open subgroup $K\subseteq H$. By the forth statement of Lemma \ref{projector_operator_properties}, we get
\begin{center}
\begin{tikzcd}
0 \arrow[r] & (V_3)^* \arrow[r, hook] \arrow[d, "\cong"'] & (V_2)^* \arrow[r, two heads] \arrow[d, "\cong"'] & (V_1)^* \arrow[r] \arrow[d, "\cong"'] & 0 \\
0 \arrow[r] & \tilde{V_3}^K \arrow[r, hook]               & \tilde{V_2}^K \arrow[r, two heads]               & \tilde{V_1}^K \arrow[r]               & 0
\end{tikzcd}
\end{center}
\end{proof}

\begin{definition}
Let $v\in V$ and  $\lambda\in\tilde{V}$. The matrix coefficient with respect to them is 
\begin{equation*}
m_{\lambda,v}:H\to\C\quad g\mapsto\lambda(\pi(g)v).
\end{equation*}
\end{definition}

\begin{remark}
Later, we will prove that smooth irreducible representation of $G$ is supercuspidal if and only if its matrix coefficients are compactly supported modulo the center.
\end{remark}

\begin{definition}
A smooth representation $(\pi,V)$ of $H$ is finite if for all $v\in V$ and $\lambda\in\tilde{V}$, the matrix coefficient $m_{\lambda,v}$ are all compactly supported. 
\end{definition}

\begin{definition}
A smooth representation $(\pi,V)$ of $H$ is finitely generated if there exists a finite subset $\{v_1,\cdots,v_n\}\subseteq V$ such that 
\begin{equation*}
V = \Span_\C\{\pi(g)(v_i)\:|\: g\in H,1\leq i\leq n\}.
\end{equation*}
\end{definition}

\begin{lemma}
  \label{admissible_criteria}
  \label{lemma_9}
A smooth representation which is finitely generated and finite is admissible.
\end{lemma}

\begin{proof}
Let $K\subseteq H$ be a compact open subgroup. We need to show that $V^K$ has a finite dimension.
\par Take $v_1,\cdots,v_n\in V$ such that
\begin{equation*}
V = \Span_\C\{\pi(g)(v_i)\:|\: g\in H,1\leq i\leq n\}.
\end{equation*} 
Let $K_1,\cdots,K_n\subseteq H$ be compact open subgroups such that for each $i$ $v_i\in V^{K_i}$. We denote
\begin{equation*}
V_i = \Span_\C\{e_K\pi(g)(v_i)\sep g\in H/K_i \}.
\end{equation*}
Recall that by assumption,
\begin{equation*}
  \tilde{V}^K \cong (V^K)^*.
\end{equation*}
%later 
\end{proof}
\begin{lemma}
  Let $(\pi,V)$ be a smooth representation of $H$. We have, it is finite if and only if for any compact open subgroup $K\subseteq H$, and for any $v\in V$, the function
  \begin{equation*}
    H\ni g\mapsto e_K(\pi(g)v)\in V
  \end{equation*}
  is compactly supported.
\end{lemma}
\begin{proof}$\:$\\
  \par $\Rightarrow$) We need to show that for any map $\lambda\in \tilde{V}$ and for any $v\in V$, we have
  \begin{equation*}
    H\ni g\mapsto \lambda(\pi(g)v)
  \end{equation*}
  is compactly supported. Let $K\subseteq H$ be a compact open subgroup such that $\lambda\in\tilde{V}^K$.
  We have
  \begin{equation*}
    \lambda(\pi(g)v) = (e_K\lambda)(\pi(g)(v)) = \lambda(e_K(\pi(g)(v))).
  \end{equation*}
  Since $g\mapsto e_K(\pi(g)(v))$ is compactly supported. Since $e_K\lambda = \lambda$, we have 
  \begin{equation*}
    g\mapsto\lambda(\pi(g)(v))
  \end{equation*}
  is compactly supported.\\
  \par $\Leftarrow$) Let $v\in V$ and $K\subseteq H$ be a compact open subgroup. Let $V_1$ be the subrepresentation of $V$ generated by $v$. Recall $\tilde{\:}$ is an exact functor.
  Therefore the surjection $V\twoheadrightarrow V_1$ induces a surjection $\tilde{V}\twoheadrightarrow \tilde{V}_1$.
  \par The subrepresentation $(\pi,V_1)$ is finite and finitely generated thus by Lemma \ref{admissible_criteria}, it is admissible.
  Hence $V_1^K$ is finite dimensional.
  \par Let $\{\lambda_1,\cdots,\lambda_n\}$ be a basis of $\tilde{V_1}^K \cong (V_1^K)^*$. Note that for any $v\in V_1$, $v=o$ if and only if for any $i=1,\cdots,n,\lambda_i(v)=0$.
  Thus we get an inclusion
  \begin{equation*}
    \supp(g\mapsto e_K(\pi(g)(v)))\subseteq\bigcup_{i=1}^n\supp(g\mapsto \lambda(e_K(\pi(g)(v)))).
  \end{equation*}
  Since $(\pi,V_1)$ is finite, we get the right hand is a union of compact sets, so is the left hand side.
\end{proof}

\begin{definition}
  A bilinear form $(\:,\:):V\times V\to \C$ of a $\C$-vectorspace $V$ is Hermitian if 
  \begin{enumerate}[i).]
    \item It is linear in the first coordinate.
    \item For any $u,v\in V,(u,v) = \overline{(v,u)}$.
  \end{enumerate}
\end{definition}
\begin{definition}
Let $(\pi,V)$ be a representation of $H$. A bilinear form $(\:,\:):V\times V\to \C$ is $H$-invariant if 
\begin{equation*}
  \forall g\in H,\forall u,v\in V, (\pi(g)u,\pi(g)v) = (u,v).
\end{equation*}
\end{definition}
\begin{definition}
A representation $(\pi,V)$ is said to be unitary (some authors say pre-unitary instead), if there 
is a positive definite $H$-invariant Hermitian form on $V$.
\end{definition}
\begin{lemma}
  An irreducible and finite representation is unitary.
  \label{unitary_sufficent_condition}
\end{lemma}

\begin{proof}
  We will construct a desired Hermitian form for a representation $(\pi,V)$ of $H$. To do so let us fix a functional $\lambda\in\tilde{V}\backslash\{0\}$.
  and define a bilinear form by
  \begin{equation*}
    \forall u,v\in V, (u,v) = \int_H m_{\lambda,u}(g)\overline{m_{\lambda,v}(g)}dg.
  \end{equation*}
  Clearly this is positive semi-definite. $H$-invariance follows from left and right invariance of Haar measure $g$.
  Clearly this is linear in the first coordinate and the complex conjugation property are also clear. Remains to show that this is definite. 
  \par By the irreducibility, for any $v\in V\nonzero$
  \begin{equation*}
    \Span_\C\{\pi(g)v\sep g\in H\} = V.
  \end{equation*}
  Thus there exist $g\in H$ such that 
  \begin{equation*}
    \vert\lambda(\pi(g)v)\vert^2>0.
  \end{equation*}
  The smoothness assures us the existence of a compact open subgroup $K$ such that $v\in V^K$. This means
  \begin{equation*}
    (v,v)\geq \vert K\vert \vert\lambda(\pi(g)(v))\vert >0.
  \end{equation*}
\end{proof}

\begin{lemma}
  Let $(\pi,V)$ be a unitary and admissible representation of $H$. Let $(\:,\:)$ be a positive-definite $H$-invariant Hermitian form on $V$. 
  If $W\subseteq V$ is a subrepresentation of $V$, then the complement 
  \begin{equation*}
    W^\perp \defeq\{v\in V\sep\forall w\in W(v,w)=0\},
  \end{equation*}
  is also a subrepresentation of $V$ and we have 
  \begin{equation*}
    V = W\oplus W^\perp,
  \end{equation*}
  as a $H$-representation.
  \label{orthogonal_decomposition_representation}
\end{lemma}

\begin{proof}
  Exercise.
\end{proof}

\begin{corollary}
  Any unitary and admissible representation is semi-simple.
\end{corollary}

\begin{proof}
  Exercise. Follows from Zorn's lemma and the axiom of choice. 
\end{proof}

\begin{corollary}
  Let $(\pi,V)$ be a unitary admissible representation of $H$ then the following statements are equivalent.
  \begin{enumerate}[1).]
    \item $\End_H(V)\cong \C$.
    \item $(\pi,V)$ is irreducible.
  \end{enumerate}
\end{corollary}

\begin{proof}
  The direction $\Leftarrow)$ is due to Lemma \ref{schur_lemma}. For the other direction, observe that for any representations $V,W$,
  \begin{equation*}
    \End_H(V\oplus W) \cong \End_H(V)\oplus\End_H(W).
  \end{equation*}
  Lemma \ref{orthogonal_decomposition_representation} tells us that any subrepresentation $W$ of $V$,
  we have a decomposition $V=W\oplus W^\perp$. However, by the dimensional argument, we derive one of them to be of dimension $0$.
  Explicitly,
  \begin{equation*}
  \dim \End_H(V) =\dim\End_H(W)+\dim\End_H(W^\perp) = 1. 
  \end{equation*}
\end{proof}

\begin{lemma}
  Let $(\pi,V)$ be a finitely generated representation of $H$.
  Consider an another representation $(\triv,W)$ where $H$ acts trivially on $W$. Then the morphism
  \begin{equation*}
    \End_H(V\tens{\C} W)\ni f\otimes w\mapsto(v\mapsto f(v)\otimes w)\in\Hom_H(V,V\tens{\C}W),
  \end{equation*}
  is an isomorphism.\\
  \par If $(\pi,V)$ is irreducible then the morphism,
  \begin{equation*}
    W\ni w\mapsto(v\to v\otimes w)\in\Hom_H(V,V\tens{\C}W),
  \end{equation*}
  is an isomorphism. In particular, we have a bijection 
  \begin{equation*}
    \{\text{$\C$-vector subspace of $W$}\}\leftrightarrow\{\text{subrepresentations of }V\tens{\C}W\}.
  \end{equation*}
  \label{subrepresentation_bijection}
\end{lemma}

\begin{proof}
  Exercise.
\end{proof}

\begin{lemma}
  Let $H_1,H_2$ be open subgroups of p-adic group $G$ and $(\pi_1,V_1),(\pi_2,V_2)$ be irreducible smooth representations of $H_1,H_2$, respectively.
  Then $(\pi_1\otimes\pi_2,V_1\otimes V_2)$ is an irreducible and smooth representation of $H_1\times H_2$.
\end{lemma}
\begin{proof}
  Let $W\subset V_1\otimes V_2$ be a representation of a group $H_1$ viewed as a subgroup $H_1\times\{e_{H_2}\}\subset H_1\times H_2$. 
  By Lemma \ref{subrepresentation_bijection}, there is a subspace $U\subseteq V_2$, such that $W=V_1\tens{\C}U$.
  \par Note that $W$ is a subrepresentation, it is a representation of $H_2$ viewed as $\{e_{H_1}\times H_2\}$. 
  Using Lemma \ref{subrepresentation_bijection}, we see $U$ is a subrepresentation of $V_2$. By the irreducibility, we conclude $U=\{0\}$ or $V_2$.
  We conclude the statement.
\end{proof}
\begin{definition}
  Let $(\pi_1,V_1),(\pi_2,V_2)$ be representations of $H$. A bilinear map $f:V_1\times V_2\to\C$ is $H$-equivariant if 
  \begin{equation*}
    \forall g\in H,\forall v_1\in V_1,v_2\in V_2, f(\pi_1(g)v_1,\pi_2(g)v_2) = f(v_1,v_2).
  \end{equation*}
\end{definition}
\begin{lemma}
  Let $(\pi,V)$ be an irreducible and admissible represenattion of $H$. If $f:V\times \tilde{V}\to\C$ is bilinear and $H$-equivariant, then there is a complex number 
  $c\in \C$ such that 
  \begin{equation*}
    \forall v\in V,\forall \lambda\in\tilde{V}, f(v,\lambda) = c\lambda(v).
  \end{equation*}
  \label{dimension_bilinear_form}
\end{lemma}

\begin{proof}
  If $f\equiv0$ then take $c=0$. Suppose not, consider the map 
  \begin{equation*}
    V\ni v\stackrel{F}{\mapsto} (\lambda\mapsto f(v,\lambda))\in \tilde{V}.
  \end{equation*} 
  Then this is a morphism of $H$-representation. Indeed,
  \begin{align*}
    \pi(g) &\mapsto (\lambda\mapsto f(\pi(g)v,\lambda))\\
    & = (\lambda\mapsto f(v,\tilde{\pi}(g^{-1})(\lambda)))\\
    & = \tilde{\tilde{\pi}}(g)(\lambda\mapsto f(v,\lambda)).
  \end{align*}
  By the irreducibility of $V$, the kernel of this morphism is $\{0\}$. By the admissibility, we have $V\cong\tilde{\tilde{V}}$.
  In particular, $\tilde{\tilde{V}}$ is also irreducible, thus the image of $F$ is the whole. We conclude $F$ is an isomorphism. 
  \par By Lemma \ref{schur_isomorphism_lemma}, we have 
  \begin{equation*}
    \End_H(V,\tilde{\tilde{V}})\cong\C.
  \end{equation*}
  By the admissibility, this contains
  \begin{equation*}
    v\mapsto(\lambda\mapsto\lambda(v)).
  \end{equation*}
  By combining these two, $F$ is a scalar multiple of this isomorphism. We get the statement.
\end{proof}

\begin{lemma}
  \label{contragredient_irreducibility}
  Let $(\pi,V)$ be an admissible and irreducible representation of $H$. Then its contragredient $(\tilde{\pi},\tilde{V})$ is again irreducible.
\end{lemma}

\begin{proof}
  Let $W\subset\tilde{V}$ be a subrepresentation. Let us define 
  \begin{equation*}
    W^{\perp} = \{v\in V\sep \forall \lambda\in W, \lambda(v) = 0\}.
  \end{equation*}
  Then since $W$ is a subrepresentation,
  \begin{equation*}
    0 = \tilde{\pi}(g)\lambda(v) = \lambda(\pi(g^{-1}(v))).
  \end{equation*}
  Therefore $W^\perp$ is stable under $H$-action. Thus this is a subrepresentation but $V$ is irreducible, thus $W^\perp = \{0\}$ or $V$. 
  \par If $W^\perp=\{0\}$, then for any $\lambda\in W$, there is $v$ such that $\lambda(v) \not = 0$. Thus $W = \tilde{V}$. If $W^\perp = V$, then $W = \{0\}$. 
  We conclude $\tilde{V}$ is irreducible.
\end{proof}

\begin{lemma}
  Let $H_1,H_2$ be two open subgroups of $G$ and $(\pi_1,V_1),(\pi_2,V_2)$ be two representations of $H_1,H_2$, respectively, which are admissible and irreducible. 
  Then the following map is an isomorphism.
  \begin{equation*}
    \tilde{V}_1\otimes\tilde{V}_2\ni\lambda_1\otimes\lambda_2\mapsto((v_1\otimes v_2)\mapsto(\lambda_1(v_1)\otimes\lambda_2(v_2)))\in\widetilde{(V_1\otimes V_2)}.
  \end{equation*}
  \label{tensor_contragredient_commutativity}
\end{lemma}

\begin{proof}
Exercise.
\end{proof}

\begin{definition}
  Let $(\pi,V)$ be an irreducible representation. The formal dgree of $(\pi,V)$ is a positive number $\deg(\pi)$ such that 
  \begin{equation*}
    \forall u,v\in V,\forall \lambda,\mu\in\tilde{V}, \int_H m_{\lambda,u}(g)m_{\mu,v}(g^{-1})dg = {\frac {\lambda(u)\mu(v)} {\deg(\pi)}}.
  \end{equation*}
\end{definition}

\begin{remark}
  The definition clearly tells us that if such positive number exists, then it is unique.
\end{remark}

\begin{proposition}[Schur Orthogonality]
  Let $(\pi_1,V_1),(\pi_2,V_2)$ be finite irreducible representations of $H$. If they are not isomorphic then for any $v_1\in V_1,v_2\in V_2,\lambda_1\in\tilde{V}_1,\lambda_2\in\tilde{V}_2$, we have
  \begin{equation*}
    \int_Hm_{\lambda_1,v_1}(g)m_{\lambda_2,v_2}(g^{-1})dg = 0.
  \end{equation*}
  \label{schur_orthogonality}
\end{proposition}

\begin{proof}
  Any irreducible representation is generated by a single element. Therefore finite and irreducible representation is admissible. 
  Using the isomorphism between $V_2\cong\tilde{\tilde{V_2}}$, we define a morphism $F:V_1\to\tilde{\tilde{V_2}}$ to be such that for fixed $\lambda_1in\tilde{V}_1$ and $v_2\in V_2$,
  \begin{equation*}
    F(u) = (\mu\mapsto\int_Hm_{\lambda_1,u}(g)m_{\mu,v_2}(g^{-1})dg). 
  \end{equation*}
  This is a $H$-homomorphism. Indeed, for any $h\in H$,
  \begin{equation*}
    \tilde{\tilde{\pi}}_2(h)(F(v)) = (\mu\mapsto\int_Hm_{\lambda_1,u}(g)m_{\tilde{\pi}(h^{-1})\mu,v_2}(g^{-1})dg). 
  \end{equation*}
  By the definition of matrix coefficient, we have 
  \begin{equation*}
  m_{\tilde{\pi}(h^{-1})\mu,v_2}(g^{-1}) = \mu(\pi(hg^{-1})v_2).
  \end{equation*}
Using the left and right invariance of Haar measure, and replacing $g$ by $gh$, we get 
\begin{equation*}
    \tilde{\tilde{\pi}}_2(h)(F(v)) = (\mu\mapsto\int_Hm_{\lambda_1,u}(gh)m_{\mu,v_2}(g^{-1})dg) = F(\pi(h)u). 
  \end{equation*}
  Thus $F\in\Hom_H(V_1,V_2)$, but these two representations are not isomorphic. We conclude $F=0$.
\end{proof}

\begin{proposition}
  Let $(\pi,V)$ be a finite and irreducible representation of $H$. Then there is a formal degree of $\pi$. 
\end{proposition}

\begin{proof}
  Again observe that $(\pi,V)$ is admissible as it is finite and irreducible. By Lemma \ref{contragredient_irreducibility}, $\tilde{V}$ is irreducible. By Lemma \ref{tensor_contragredient_commutativity}, we have 
  \begin{equation*}
    \widetilde{(V\tens{\C}\tilde{V})} \cong \tilde{V}\tens{\C}\tilde{\tilde{V}} \cong \tilde{V}\tens{\C}V \cong V\tens{\C}\tilde{V}.
  \end{equation*}
  By Lemma \ref{characteristics_admissiblity}, $V\otimes\tilde{V}$ is admissible. 
  \par Consider a bilinear form $b:(V\otimes \tilde{V})\times \widetilde{(V\otimes \tilde{V})}\to\C$ such that ,
  \begin{equation*}
    b(u\otimes\lambda,\mu\otimes v) = \int_Hm_{\lambda,u}(g)m_{\mu,v}(g^{-1})dg.
  \end{equation*}
  Let $(h,k)\in H\times H$, then
  \begin{align*}
    b((\pi\times\tilde{\pi})(h,k)(u\otimes\lambda),(\tilde{\pi}\times\pi)(h,k)(\mu\otimes v)) & = \int_H \tilde{\pi}(k)(\lambda(\pi(gh)u))\tilde{\pi}(h)(\mu(\pi(g^{-1}k)v))dg,\\
    & = \int_H (\lambda(\pi(k^{-1}gh)u))(\mu(\pi(h^{-1}g^{-1}k)v))dg,\\
    & = \int_H (\lambda(\pi(g)u))(\mu(\pi(g^{-1})v))d(k^{-1}gh),\\
    & = \int_H (\lambda(\pi(g)u))(\mu(\pi(g^{-1})v))dg,\\
    & = b((u\otimes\lambda),(\mu\otimes v)).
  \end{align*}
  We conclude $b$ is $H$-invariant. By Lemma \ref{dimension_bilinear_form}, there is $c\in\C$, such that 
  \begin{equation*}
    b(u\otimes\lambda,\mu\otimes v) = c\mu(u)\lambda(v).
  \end{equation*}
  Remains to show $c>0$. From Lemma \ref{unitary_sufficent_condition}, $(\pi,V)$ is unitary. Fix a positive-definite $H$-invariant Hermitian form $(\:,\:):V\times V\to \C$. 
  For $v\in V\nonzero$, we define $\lambda_v\in\tilde{V}$ to be 
  \begin{equation*}
    \lambda_v(w) = (w,v).
  \end{equation*}
  Then for a compact open subgroup $K\subset H$ such that $v\in V^K$, we have 
  \begin{equation*}
    \tilde{\pi}(K)\lambda_v(w) = \lambda_{\pi(K)v}(w) = \lambda_v(w).
  \end{equation*}
  Therefore, $\lambda_v\in\tilde{V}$. Let us consider the following 
  \begin{align*}
    b(v\otimes\lambda_v,\lambda_v\otimes v) & = c\lambda_v(v)\lambda_v(v),\\
    & = \int_H\lambda_v(\pi(g)v)\lambda_v(\pi(g^{-1})v)dg.
  \end{align*}
  Note that $\lambda_v(\pi(g^{-1})v) = (\pi(g^{-1})v,v) = \overline{(\pi(g)v,v)} = \lambda_v(\pi(g)v)$. Therefore, 
  \begin{align*}
    b(v\otimes\lambda_v,\lambda_v\otimes v) & = \vert \lambda_v(\pi(g)v)\vert^2dg.
  \end{align*}
  \par Take $K$ to be a compact open subgroup such that $v\in V^K$. Then using $(\:,\:)$ is positive definite. We derive 
  \begin{equation*}
    \vert \lambda_v(\pi(g)v)\vert^2dg\geq \vert K\vert\vert\lambda_v(v)\vert^2>0.
  \end{equation*}
  We conclude $c>0$.
\end{proof}

\begin{definition}
  A collection $\{(\pi_i,V_i)\}_{i\in I}$ of irreducible representations of $H$ is said to be independent if it is pairwise non-isomorphic. 
\end{definition}

\begin{corollary}
  Let $\{(\pi_i,V_i)\}_{i\in I}$ be a family of independent irreducible and finite representations of $H$. Then for any $\{v_i\}_{i\in I}$ and $\{\lambda_i\}_{i\in I}$ be such that 
  \begin{equation*}
    \forall i\in I, v_i\in V_i\nonzero,\quad \lambda\in\tilde{V}_i\nonzero.
  \end{equation*}
  Then $\{m_{\lambda_i,v_i}\}_{i\in I}$ is linearly independent. 
\end{corollary}

\begin{proof}
Suppose we have a $\C$-linear combination
\begin{equation*}
  \sum_{i\in I} c_i m_{\lambda_i,v_i} = 0.
\end{equation*}
Let us denote $(\lambda_i')_{i\in I}$ be such that for each $i\in I$, $\lambda_i'(v_i) = 1$. Similarly, we pick $(v'_i)_{i\in I}$ such that for each $i\in I$, $\lambda_i(v_i') = 1$. 
\par Observe that by Lemma \ref{schur_orthogonality}
\begin{align*}
0& = \int_H\left(\sum_{i\in I}c_im_{\lambda_i,v_i}(g)\right)m_{\lambda'_j,v'_j}(g^{-1})dg,\\
& = c_j\int_Hm_{\lambda_j,v_j}(g)m_{\lambda_j',v_j'}(g^{-1})dg,\\
& = c_j{\frac {\lambda_j(v_j')\lambda_j'(v_j)} {\deg(\pi_j)}},\\
& = {\frac {c_j} {\deg(\pi_j)}}.
\end{align*}
We conclude $c_j=0$ for any $j\in I$. 
\end{proof}

\begin{proposition}
  Any irreducible and admissible representation of $H_1\times H_2$ is of the form $(\pi_1\otimes\pi_2,V_1\otimes V_2)$ where 
  $(\pi_1,V_1)$ and $(\pi_2,V_2)$ are $H_1$,$H_2$ representations, respectively. 
\end{proposition}
\begin{proof}
  Exercise.
\end{proof}

\begin{proposition}
  If $(\pi_1,V_1)$ and $(\pi_2,V_2)$ are admissible $H_1$,$H_2$ representations, respectively. Then 
  \begin{equation*}
    \tilde{V}_1\otimes\tilde{V}_2 \cong \widetilde{V_1\otimes V_2}.
  \end{equation*}
  In particular, we can drop the assumption on Lemma \ref{tensor_contragredient_commutativity}.
\end{proposition}

\begin{proof}
  Exercise. Hints : First prove for the case $V_1^{K_1},V_2^{K_2}$ where $K_1\subset H$, $K_2\subset H_2$ are compact open subgroups.
\end{proof}

\subsection{The splitting induced by an irreducible finite representation}

\begin{definition}
  Let $(\pi,V)$ be an irreducible finite representation of $H$. We define the two subcategories of $\Rep(G)$.\\
  \par 1. $\Rep^\pi(H)$ denotes the full subcategory of $\Rep(H)$ whose objects consists of smooth representations $(\tau,W)$ of $H$ such that each irreducible subquotient of $\tau$ is isomorphic to $\pi$.\\
  \par 2. $\Rep_\pi(H)$ denotes the full subcategory of $\Rep(H)$ whose objects consists of smooth representations $(\tau, W)$ of $H$ such that no irreducible subquotient of $\tau$ is isomorphic to $\pi$.
\end{definition}

Our goal of the section is to prove that 
\begin{equation*}
  \Rep(H) \cong\Rep^\pi(H)\times\Rep_\pi(H).
\end{equation*}

More precisely, for each $(\tau,W)\in\Rep(H)$, there exist two $H$-invariant subspaces $W^\pi,W_\pi$ such that 
\begin{equation*}
  (\tau,W^\pi)\in\Rep^\pi(H),\quad (\tau,W_\pi)\in\Rep_\pi(H).
\end{equation*}

such that $W=W^\pi\oplus W_\pi$.

\begin{proposition}
For any $(\tau_1,W_1)\in\Rep^\pi(H)$ and $(\tau_2,W_2)\in\Rep_\pi(H)$. We have 
\begin{equation*}
  \Hom_H(\tau_1,\tau_2) = \zero.
\end{equation*}
Thus the decomposition $W = W^\pi\oplus W_\pi$ turns out to be unique.
\end{proposition}

\begin{proof}
  Exercise.
\end{proof}

\begin{definition}[Hecke Algebra]
  We define an associative $\C$-algebra (not commutative and without unit) $\mathcal{H}=\mathcal{H}(H)$ such that 
  \begin{equation*}
    \mathcal{H} = \mathcal{C}^\infty_C(H) \defeq \{f:H\to\C\sep \text{$f$ is locally constant and compactly supported}\}.
  \end{equation*}
  For $f_1,f_2\in\mathcal{H}$, we define the convolution product $f_1\ast f_2$ to be 
  \begin{equation*}
    (f_1\ast f_2)(h) = \int_H f_1(g)f_2(g^{-1}h)dg.
  \end{equation*}
\end{definition}

\begin{definition}
  Let $K\subseteq H$ be a compact open subgroup. then we define, $e_K:H\to\C$ to be such that 
  \begin{equation*}
    e_K(g) = \begin{cases}
      {\frac 1 {\vert K\vert}} \quad (g\in K)\\
      0\quad (g\not\in K).
    \end{cases}
  \end{equation*}
  This defines an element in $\mathcal{H}(H)$.
\end{definition}

\begin{definition}
  Let $(\tau,W)\in\Rep(H)$. For $f\in\mathcal{H}$, we define $\tau(f)\in\End_\C(W)$ to be such that 
  \begin{equation*}
    \tau(f)w = \int_H f(g)\tau(g)(w)dg.
  \end{equation*}
\end{definition}

Recall we defined a projetor operator $e_K$. This notation is compatible with what we have just defined. Indeed,
\begin{equation*}
  \tau(e_K)(w) = {\frac 1 {\vert K \vert}}\int_K\tau(g)(w)dg = e_K(w).
\end{equation*}

When $\tau$ is clearl from the context, we use these two notions interchangebly.

\begin{remark}
  If $H$ is finite, we get $\hecke =\C[H]$ with the convolution product just a usual group algebra multiplication.
\end{remark}

\begin{lemma}
The map $\hecke\ni f\mapsto\tau(f)\in \End_\C(W)$ defines a $H\times H$-equivariant algebra homomorphism with respect to the following $H\times H$-action on $\hecke$,
\begin{equation*}
  ((h_1,h_2)f)(g) = f(h_1^{-1}gh_2),
\end{equation*}
and on $\End_\C(W)$,
\begin{equation*}
  (h_1\times h_2)\alpha = \tau(h_1)\circ\alpha\circ\tau(h_2^{-1})
\end{equation*}
\label{lemma_hecke_action}
\end{lemma}

\begin{proof}
  Exercise.
\end{proof}

\begin{definition}
  With the $H\times H$-action defined in the previous lemma, we define $\End_\C(W)^\infty$ to be the smooth part of the representation. Thus $\tau:\hecke\to\End_\C(W)^\infty$ defines a smooth representation.
\end{definition}

\begin{remark}
  We can rephrase the above notion by
  \begin{equation*}
    \End_\C(W)^\infty = \bigcup_{\substack{K\subseteq H \\ \text{compact, open}\\\text{subgroup}}} \End_C(W)^{K\times K}.
  \end{equation*}
  Indeed for any compact open subgroups $K_1,K_2\subseteq H$, we have $K_1\cap K_2$ is again a compact open subgroup. Furthermore,
  \begin{equation*}
    \End_\C(W)^{K_1\times K_2}\subseteq \End_\C(W)^{K\times K},
  \end{equation*}
  which justifies the equality in the first equation.
\end{remark}

Before diving into the actual proofs, we will sketch out the procedure.\\
\par Let $(\pi,V)$ be a irreducible finite representation of $H$. We want to show that 
\begin{equation*}
  \Rep(H) = \Rep^\pi(H)\times\Rep_\pi(H).
\end{equation*}

To do so we construct a section $\phi:\End_\C(W)^\infty\to\hecke$ of $\pi:\hecke\to\End_\C(W)^\infty$ (ie. $\pi\circ\phi = \id_{\End_\C(W)^\infty})$.
\par If such $\phi$ exists then we get 
\begin{equation*}
  \hecke = \Ker\pi\times\Image \phi,
\end{equation*}

where $\Ker\pi$ and $\Image\phi$ correspond to $\Rep_\pi(H)$ and $\Rep^\pi(H)$, respectively.

Using $\Image\phi$, we will construct a projection $e^\pi:\Rep(H)\to\Rep^\pi(H)$.

\begin{lemma}
  Consider the map $r:V\otimes\tilde{V}\to\End_\C(V)^\infty$ be such that 
  \begin{equation*}
    v\otimes \lambda\mapsto [w\mapsto \lambda(w)v].
  \end{equation*}
  Then $r$ is a $H\times H$-equivariant isomorphism.
\end{lemma}

\begin{proof}$\:$
  \begin{center}
\begin{tikzcd}
V\otimes\tilde{V} \arrow[d, no head, Rightarrow] \arrow[r, "r"]  & \End_\C(V)^\infty \arrow[d, no head, Rightarrow] \\
\bigcup_{K\subseteq H}(V\otimes \tilde{V})^{K\times K} \arrow[r] & \bigcup_{K\subseteq H}\End_\C(V)^{K\times K}     \\
(V\otimes \tilde{V})^{K\times K} \arrow[u, hook] \arrow[r]       & \End_\C(V)^{K\times K} \arrow[u, hook]          
\end{tikzcd}
  \end{center}
  $H\times H$-equivariance part is assined as an exercise. \\
  \par By Lemma \ref{exactness_representation}, it is enough to check the level of $K\times K$ fixed part where $K$ is a compact open subgroupos. By Lemma \ref{projector_operator_properties}, we have 
  \begin{equation*}
  V \cong \pi(e_K)V\oplus(1-\pi(e_K))(V) = V^K\oplus (1-\pi(e_K))V.
  \end{equation*}
  Let $\alpha\in\End_\C(W)^{K\times K}$. For any $(k_1,k_2)\in K\times K$, we have,
  \begin{equation*}
    \tau(k_1)\circ\alpha\circ\tau(k_2) = \alpha.
  \end{equation*}
  In particular, $\alpha(V)\subset V^K$. Indeed, consider,
  \begin{equation*}
    \alpha\circ\tau(e_K)(v) = {\frac 1 {\vert K\vert }}\int_K\alpha\circ\tau(k)(v)dk.
  \end{equation*}
  Since $1\in K$, we have 
  \begin{equation*}
    {\frac 1 {\vert K\vert }}\int_K\alpha\circ\tau(k)(v)dk = {\frac 1 {\vert K\vert}}\int_K\alpha(v)dk = \alpha(v).
  \end{equation*}
  This also shows us that 
  \begin{equation*}
    \alpha\circ(1-\tau(e_K)) = 0.
  \end{equation*}
  Thus the restriction map, 
  \begin{equation*}
    \End_\C(V)^{K\times K}\ni\alpha\mapsto \alpha|_{V^K}\in\End_\C(V^K),
  \end{equation*}
  is well-defined and an isomorphism. We have the following commutative diagram,
  \begin{center}
\begin{tikzcd}
(V\otimes \tilde{V})^{K\times K} \arrow[d, "\wr"'] \arrow[r, "r"] & \End_\C(V)^{K\times K} \arrow[d, "\wr"] \\
V^K\otimes \tilde{V}^K \arrow[d, "\wr"', no head]                 & \End_\C(V^K)                            \\
V^K\otimes (V^K)^* \arrow[ru, "\varphi"']                         &                                        
\end{tikzcd}
  \end{center}
  where $\varphi$ is given by 
  \begin{equation*}
    \varphi(v\otimes \lambda)=[w\mapsto \lambda(w)v].
  \end{equation*}
  Since $V^K$ is finite dimensional $\varphi$ is an isomorphism (It is an exercise to verify this). Hence $r$ is also an isomorphism.
\end{proof}

\begin{definition}
  For $v\in V$, and $\lambda\in\tilde{V}$, we define 
  \begin{equation*}
    f_{u,\lambda} = \deg(\pi)m_{\lambda,v}(g^{-1}) = \deg(\pi)\lambda(\pi(g^{-1})v).
  \end{equation*}
\end{definition}

\begin{remark}
  If $(\pi,V)$ is finite then $m_{\lambda,v}$ is compactly supported thus $f_{v,\lambda}\in\hecke$.
\end{remark}

\begin{definition}
  Let $\phi:\End_\C(V)^\infty\to\hecke$ be such that 
  \begin{center}
\begin{tikzcd}
\phi:\End_\C(V)^\infty & V\otimes\tilde{V} \arrow[r, "{v\otimes\lambda\mapsto f_{v,\lambda}}"] \arrow[l, "\stackrel{r}{\sim}"'] & \hecke
\end{tikzcd}
  \end{center}
\end{definition}

\begin{lemma}
  $\phi$ is $H\times H$-equivariant and is indeed a section of $\pi$ Explicitly, 
  \begin{equation*}
  \pi\circ\phi=\id_{\End_\C(V)^\infty}.
  \end{equation*}
  \label{lemma_22}
\end{lemma}

\begin{proof}
  Since $r$ is $H\times H$-equivariant, it suffices to show that 
  \begin{equation*}
    v\otimes \lambda\mapsto[w\mapsto\lambda(w)v],
  \end{equation*}
  is $H\times H$-equivariant.
  \par Let $h_1,h_2\in H$. Consider 
  \begin{center}
    \begin{tikzcd}
v\otimes\lambda \arrow[d] \arrow[r, maps to]               & {f_{v,\lambda}}                         \\
\pi(h_1)v\otimes\tilde{\pi}(h_2)\lambda \arrow[r, maps to] & {f_{\pi(h_1)v,\tilde{\pi}(h_2)\lambda}}
\end{tikzcd}
  \end{center}
  By definition we have 
  \begin{align*}
    (h_1\times h_2)(f_{v,\lambda})(x) &= f_{v,\lambda}(h^{-1}_1xh_2),\\
    & = \deg(\pi)\lambda(\pi(h_1^{-1}xh_2)^{-1}(v)),\\
    & = \deg(\pi)\lambda(\pi(h_2^{-1} x^{-1}h_1)(v)),\\
    & = f_{\pi(h_1)(v),\tilde{\pi}(h_2)(\lambda)}(x).
  \end{align*}
  Recall our strategy,
  \begin{center}
  \begin{tikzcd}
\hecke \arrow[r, "\pi"] & \End_\C(V)^\infty                                                                      \\
                        & V\otimes\tilde{V} \arrow[u, "r"'] \arrow[lu, "{v\otimes\lambda\mapsto f_{v,\lambda}}"]
\end{tikzcd}
  \end{center}
  Now we will prove that $\phi$ indeed defines a section of $\pi$. In other words, for any $v\in V$ and $\lambda\in\tilde{V}$, we have 
  \begin{equation*}
    \pi\circ\phi(r(v\otimes\lambda)) = r(v\otimes \lambda).
  \end{equation*}
  Since $r(v\otimes \lambda)\in\End(V)^\infty$, it is equivalent to show that for any $w\in V$, 
  \begin{equation*}
    \pi\circ\phi(r(v\otimes\lambda))w = r(v\otimes\lambda)w = \lambda(w)v.
  \end{equation*}
  By definition, 
  \begin{equation*}
    \pi\circ\phi(r(v\otimes\lambda))w = \pi(f_{v,\lambda})w = \int_Hf_{v,\lambda}(g)\pi(g)(w)dg.
  \end{equation*}
  Since $V$ is admissible we have $V\cong \tilde{\tilde{V}}$. Thus the statement isequivalent to 
  \begin{equation*}
    \forall w\in W,\forall\mu\in\tilde{V},\mu\left(\int_Hf_{w,\lambda(v)}(g)\pi(v)dg\right) = \mu(\lambda(w)v) = \lambda(w)\mu(v).
    \end{equation*}
    Then by definition,
    \begin{align*}
      \mu\left(\int_Hf_{w,\lambda(v)}(g)\pi(v)dg\right) & = \int_H f_{v,\lambda}(g)\mu(\pi(g)w)dg,\\
      & = \deg(\pi)\int_Hm_{\lambda,v}(g^{-1})m_{\mu,w}(g)dg.
    \end{align*}
    Using Lemma \ref{schur_orthogonality}, we get 
    \begin{equation*}
      \deg(\pi)\int_Hm_{\lambda,v}(g^{-1})m_{\mu,w}(g)dg = \lambda(w)\mu(v).
    \end{equation*}
\end{proof}

\begin{corollary}
  \label{corollary_23}
  \begin{equation*}
    \hecke = \Ker\pi\oplus\Image\phi,
  \end{equation*}
  as $H\times H$-representation.
\end{corollary}

\begin{proof}
  From Lemma \ref{lemma_22}, we have a short exact sequence, 
  \begin{center}
    \begin{tikzcd}
0 \arrow[r] & \End_\C(V)^{\infty} \arrow[r] & \hecke \arrow[r] & \Ker\pi \arrow[r] & 0
\end{tikzcd}
  \end{center}
\end{proof}

\begin{lemma}
  For $f\in\hecke$, $v\in V$, and $\lambda\in\tilde{V}$, we have 
  \begin{equation*}
    f\ast f_{v,\lambda} = f_{\pi(f)(v),\lambda},\quad f_{v,\lambda}\ast f = f_{v,\lambda\circ\pi(f)}.
  \end{equation*}
  \label{convolution_anti_commutativity}
\end{lemma}
\begin{proof}
  Exercise.
\end{proof}

\begin{lemma}
  Both $\Ker\pi$ and $\Image\phi$ are two-sided ideals in $\hecke$. Consequently 
  \begin{equation*}
    \hecke = \Ker\pi\oplus\Image\phi,
  \end{equation*}
  as a $\C$-algebra. Moreover, $\phi$ is an algebra homomorphism.
\end{lemma}
\begin{proof}
  Since $\pi$ is an algebra homomorphism, $\Ker\pi$ is a two-sided ideal. From Lemma \ref{convolution_anti_commutativity}, we get $\Image\phi$ is also a two-sided ideal. 
  \par For $\alpha,\beta\in\End_\C(V)^\infty$, we have 
  \begin{equation*}
    \phi(\alpha\circ\beta)-\phi(\alpha)\ast\phi(\beta)\in\Ker\pi\cap\Image\Phi.
  \end{equation*}
  Thus we conclude 
  \begin{equation*}
    \phi(\alpha\circ\beta)=\phi(\alpha)\ast\phi(\beta).
  \end{equation*}
  We conclude $\phi$ is a $\C$-algebra homomorphism.
\end{proof}

\begin{definition}
  Let $K\subseteq H$ be a compact open subgroup and $(\pi,V)$ be a finite and irreducible representation. Recall that we have 
  \begin{equation*}
  V = V^K\oplus(1-\pi(e_K))V.
  \end{equation*}
  We define $1_{V^K}\in\End_\C(V)^\infty$ as follows 
  \begin{equation*}
    1_{V^K}|_{V^K} = \id_{V^K},\quad 1_{V^K}|_{(1-\pi(e_K))V} = 0.
  \end{equation*}
\end{definition}

\begin{remark}
  As a matrix this is 
  \begin{equation*}
    \begin{pmatrix}
      I_{\dim V^K}&O\\
      O&O
    \end{pmatrix}.
  \end{equation*}
  Also taking a basis of $\{e_1,\cdots,e_{\dim V^K}\}$ and a dual basis $\{\lambda_1,\cdots,\lambda_{\dim V^K}\}$ of $V^K,(V^K)^*$, respectively, this realized as the image of 
  \begin{equation*}
    \sum v_i\otimes \lambda_i,
  \end{equation*}
  under $r$.
  \label{remark_thought_e_K}
\end{remark}

\begin{definition}
  We define 
  \begin{equation*}
    e_K^\pi\defeq\phi(1_{V^K})\in\hecke.
  \end{equation*}
\end{definition}


\begin{remark}
  Since $1_{V^K}$ is an idempotent and $\varphi$ is an algebra homomorphism, $e_K^\pi$ is again an idempotent.
  \par Also by Remark \ref{remark_thought_e_K}, we have,
  \begin{equation*}
  e^\pi_K(g) = \deg(\pi)\sum_{i=1}^{\dim V^K}\lambda_i(\pi(g^{-1})v_i).
  \end{equation*}
\end{remark}

\begin{lemma}
  Let $g\in H$ and $K'\subseteq K$ be a compact open subgroup. We have 
  \begin{enumerate}[1).]
    \item $e^\pi_{gKg^{-1}} = (g\times g)e_K^\pi$.
    \item $e_{K'}^\pi\ast e_K^\pi = e_K^\pi$.
  \end{enumerate}
  \label{lemma_25}
\end{lemma}

\begin{proof}$\:$\\
  \par $1).$ The map $V\ni v\mapsto\pi(g)(v)\in V$ restricts to an isomorphism
  \begin{equation*}
    V^K\stackrel{\sim}{\to}V^{gKg^{-1}},
  \end{equation*}
  and 
  \begin{equation*}
    (1-\pi(e_K))V\isomleft (1-\pi(e_{gKg^{-1}}))V.
  \end{equation*}
  Here we have 
  \begin{equation*}
    (g\times g)1_{V^K}\pi(g)\circ 1_{V^K}\circ \pi(g^{-1}) = 1_{V^{gKg^{-1}}}.
  \end{equation*}
  Applying $\phi$ on both sides, we obtain,
  \begin{equation*}
    (g\times g)e_K^\pi = \phi((g\times g)1_K) = \phi(1_{V^{gKg^{-1}}}) = e^\pi_{g Kg^{-1}}.
  \end{equation*}
  For the second part, we claim that 
  \begin{equation*}
    1_{V^{K'}}\circ\pi(e_K) = 1_{V^K}.
  \end{equation*}
  Indeed, since $V^K\subseteq V^{K'}$, we have,
  \begin{equation*}
    1_{V^{K'}}\circ\pi(e_K)\rest{V^K} = \id,
  \end{equation*}
  and 
  \begin{equation*}
    1_{V^{K'}}\circ\pi(e_K)\rest{(1-\pi(e_K))V} = 0.
  \end{equation*}
  Applying $\phi$ to both sides, we get,
  \begin{equation*}
    e_{K'}^\pi\ast(\phi\circ\pi)(e_K) = e_K^\pi.
  \end{equation*}

\end{proof}

\begin{definition}
  Let $(\tau,W)\in\Rep(H)$, then we define 
  \begin{equation*}
    e^\pi:W\to W,
  \end{equation*}
  as follows. For $w\in W$, choose a compact open subgroup $K\subseteq H$ such that 
  \begin{equation*}
    w\in W^K,
  \end{equation*}
  then 
  \begin{equation*}
    e^\pi(w) \defeq \tau(e_K^\pi)(w).
  \end{equation*}
\end{definition}

\begin{lemma}
  Let $(\tau,W)\in\Rep(H)$, then the following statements hold.
  \begin{enumerate}[1).]
    \item $e^\pi(w)$ is independent of the choice of $K$ such that $w\in W^K$.
    \item $e^\pi:W\to W$ is $H$-equivariant.
    \item $e^\pi$ is idempotent.
    \item $e^\pi:\Rep H\to\Rep H$ is functorial. That is for any $H$-equivariant map $\alpha:W_1\to W_2$, we have,
    \begin{center}
      \begin{tikzcd}
W_1 \arrow[r, "e^\pi"] \arrow[d, "\alpha"'] & W_1 \arrow[d, "\alpha"] \\
W_2 \arrow[r, "e^\pi"']                     & W_2                    
\end{tikzcd}
    \end{center}
    is a commutative diagram.
    \item $e^\pi$ is exact.
    \item $e^\pi$ is an identity if $(\tau,W) = (\pi,V)$.
  \end{enumerate}
\end{lemma}

\begin{proof}$\:$\\
  \par 1). Suppose $w\in W^{K_1}\cap W^{K_2}$ for some compact open subgroups $K_1,K_2$. Then we see that $w\in W^{K_1\cap K_2}$. As $K_1\cap K_2$ is again compact and open, 
  it is enough to show that for any compact open subgroup $K'\subseteq K$ such that $w\in W^{K'}\cap W^{K}$, 
  \begin{equation*}
    \tau(e_{K'}^\pi)(w) = \tau(e_{K}^\pi)(w).
  \end{equation*}
  This follows directly from the second assertion of Lemma \ref{lemma_25}.
  \par 2). again using Lemma \ref{lemma_25} and note that if $\tau(g)w\in W^K$ then $w\in W^{g^{-1}Kg}$, we have,
  \begin{equation*}
    e^\pi(\tau(g)(w)) = \tau(e_K^\pi)(\tau(g)(w)) = \tau(e_{g^{-1}Kg})(w) = ge^\pi(w).
  \end{equation*}
  \par 3). Again directly from Lemma \ref{lemma_25}.
  \par 4). by the $H$-equivariance, we have,
  \begin{equation*}
    w\in W_1^K\Rightarrow \alpha(w)\in W_2^K.
  \end{equation*}
  Also note that $e_K^\pi$ is compactly supported therefore, it commutes with $\alpha$.
  \par For 5), use Lemma \ref{exactness_representation} and observe that $e^\pi$ is functorial and an idempotent element.
  \par Lastly, for $v\in V^K$, 
  \begin{equation*}
    e^\pi(v) = \pi(e_K^\pi)(v) = \pi(\phi(1_{V^K}))(v) = 1_{V^K}(v) = v.
  \end{equation*}
\end{proof}

\begin{corollary}
  For a $H$-representation $W$, we have,
  \begin{equation*}
    W = \Image e^\pi\oplus \Ker e^\pi = e^\pi(W)\oplus(1-e^\pi)W.
  \end{equation*}
\end{corollary}

\begin{proof}
  Since any idempotent representation induces a split we have the first equality. The second equality is also due to that it being idempotent.
\end{proof}

\begin{lemma}
  \label{lemma_28}
  The following are equivalent for a representation $(\tau,W)\in\Rep(H)$.
  \begin{enumerate}[1).]
    \item $W$ is a sum of its irreducible subrepresentation.
    \item $W$ is a direct sum of its irreducible subrepresentation.
    \item For any $H$-subrepresentation $W'\subseteq W$, there is a $H$-subrepresentation $W''$ such that $W= W'\oplus W''$.
  \end{enumerate}
  From these, we conclude that any subquotient of a semisimple representation is again semisimple.
\end{lemma}

\begin{proof}$\:$
  \par $1)\Rightarrow 2)$. Let $\{W_i\}_{i\in I}$ be a collection of subrepresentations such that 
  \begin{equation*}
    W = \sum_{i\in I}W_i.
  \end{equation*}
  Let $\mathcal{I}$ be the collection of subsets of $I$ such that for $J\in\mathcal{I}$,
  \begin{equation*}
    \sum_{j\in J}W_j
  \end{equation*}
  is a direct sum. Clearly $\mathcal{I}$ is not empty and is inductively ordered by inclusion. Applying Zorn's lemma, we obtain a maximal element $J_0\in\mathcal{I}$.
  If 
  \begin{equation*}
    \sum_{j\in J_0}W_j = \bigoplus_{j\in J_0}W_j \not=W,
  \end{equation*}
  then there is $W_i$ with $i\not\in J_0$. Since $W_i$ is irreducible we have 
  \begin{equation*}
    W_i\cap\left(\bigoplus_{j\in J_0} W_j\right) = \{0\}.
  \end{equation*}
  Therefore $J_0\cap\{i\}$ is also an element of $\mathcal{I}$ which contradicts to the maximality of $J_0$. 
  \par $2)\Rightarrow 3)$ is trivial.
  \par $3)\Rightarrow 1)$. Let $\mathcal{W}$ be the collection of subrepresentations of $W$ which are sums of irreducible subrepresentations. 
  Similarly as before introduce an order by inclusion to $\mathcal{W}$ and use Zorn's lemma, we obtain a maximal element $W_0$ then we see $W_0 = W$.
  \par From these, we see that any subrepresentation is a direct sum of irreducible subrepresentations. Thus the last statement follows.
\end{proof}

\begin{lemma}
  \label{lemma_29}
  Let $f\in\hecke$, $(\tau,W)\in\Rep(H), w\in W$. Then the subspace of $\hecke$,
  \begin{equation*}
    \hecke\ast f\defeq\{f'\ast f\sep f'\in\hecke\},
  \end{equation*}
  is stable under the action $H\cong H\times\{1\}\subseteq H\times H$. Furthermore, the subspace,
  \begin{equation*}
    (\hecke\ast f)(w) \defeq \{\tau(f')(w)\sep f'\in\hecke\ast f\}\subseteq W,
  \end{equation*}
  is stable under $H$.
  \par There is a surjection 
  \begin{center}
    \begin{tikzcd}
\hecke\ast f \arrow[r, "f'\mapsto \tau(f')(w)", two heads] & (\hecke\ast f)(w)
\end{tikzcd}
  \end{center}
  which is $H$ equivariant under the identification 
  \begin{center}
    \begin{tikzcd}
H\times\{1\} & H \arrow[l, "{(h,1)\mapsfrom h}"']
\end{tikzcd}
  \end{center}
\end{lemma}

\begin{proof}
  Let $h\in H$ and $f'\in\hecke$. Recall from Lemma \ref{lemma_hecke_action} that the action is defined as 
  \begin{equation*}
    ((h_1,h_2)f)(g) = f(h_1^{-1}gh_2).
  \end{equation*}
  Therefore for $x\in H$,
  \begin{align*}
    ((h\times 1)(f'\ast f))(x)  &=(f'\ast f)(h^{-1}x),\\
     &=  \int_H f'(g)f(g^{-1}h^{-1}x)dx,\\
     &\stackrel{g\to h^{-1}g}{=} \int_H f'(h^{-1}g)f(g^{-1}x)dx,\\
     & =\int_H (h\times 1)f'(g)f(g^{-1}x)dg,\\
     & = (((h\times 1)f')\ast f)(x).
  \end{align*}
  Similarly, 
  \begin{align*}
    \tau(h)\tau(f'\ast f)(w) & = \tau(h)\int_H(f'\ast f)(g)\tau(g)(w)dg,\\
    & \stackrel{g\to h^{-1}g}{=} \int_H(f'\ast f)(h^{-1}g)\tau(g)(w)dg,\\
    & = \tau(((h\times 1)f')\ast f)(w).
  \end{align*}
  From this the last assertion is clear.
\end{proof}

\begin{proposition}
  \label{proposition_26}
  Let $(\pi,V)$ be an irreducible, finite, smooth $H$ representation and $(\tau,W)\in\Rep H$. Then we have the following statements.
  \begin{enumerate}[1).]
    \item The space $e^\pi(W)$ is a direct sum of copies of $(\pi,V)$. In particular, $e^\pi(W)\in\Rep^\pi(H)$.
    \item The map $e^\pi:W\to W$ is trivial if and only if $(\tau,W)$ has no irreducible subquotient which is isomorphic to $(\pi,V)$. Hence $(1-e^\pi)(W)\in\Rep_{\pi}(H)$.
    \item The functors 
    \begin{center}
      \begin{tikzcd}
\Rep(H) \arrow[rr, "{(\tau,W)\mapsto(e^\pi(W),(1-e^\pi)(W))}", shift left] &  & \Rep^\pi(H)\times\Rep_\pi(H) \arrow[ll, "{\tau_1\oplus\tau_2\mapsfrom (\tau_1,\tau_2)}", shift left]
\end{tikzcd}
    \end{center}
    give the equivalence of categories. Thus we simply denote $\Rep(H)=\Rep^\pi(H)\times\Rep_\pi(H)$.
  \end{enumerate}
\end{proposition}

\begin{proof}
  Let $w\in W$ and $K\subseteq H$ be a compact open subgroup such that $w\in W^K$. Using Lemma \ref{lemma_29}, we have a surjective $H$-equivariant map,
  \begin{center}
    \begin{tikzcd}
\hecke\ast e_K^\pi \arrow[r, "f\mapsto\tau(f)(w)", two heads] & (\hecke\ast e_K^\pi)(w)
\end{tikzcd}
  \end{center}
  Note that the right hand side contains $e^\pi(w) = \tau(e_K^\pi)(w)$.
  Since $e_K^\pi=\phi(1_{V^K})\in\Image\phi$ and $\Image\phi$ is a two-sided ideal, we have,
  \begin{equation*}
    \hecke\ast e_K^\pi\subseteq\Image\phi = \Image(V\otimes \tilde{V}\stackrel{\stackrel{r}{\sim}}{\to}\End_\C(V)^\infty\stackrel{\phi}{\to}\hecke).
  \end{equation*}
  In particular, we have,
  \begin{center}
    \begin{tikzcd}
V\otimes\tilde{V} \arrow[r, "\phi\circ r"] & \Image\phi \arrow[r, "\supseteq", two heads] & \hecke\ast e_K^\pi \arrow[r, "\text{Lemma\ref{lemma_29}}", two heads] & (\hecke\ast e_K^\pi)(w)
\end{tikzcd}
  \end{center}
  Thus $(\hecke\ast e_K^\pi)(w)$ is a subquotient of the $H\times\{1\}\cong H$-representation $V\otimes \tilde{V}|_{H\times\{1\}}$.
  Using the trivial representation $(\triv,\C)$, we have,
  \begin{equation*}
    V\otimes\tilde{V}|_{H\times\{1\}} = (\pi,V)\otimes\left(\bigoplus\triv\right) = \bigoplus(\pi,V).
  \end{equation*}
  Using Lemma \ref{lemma_28}, $(\hecke\ast e_K^\pi)(w)$ is also a direct sum of copies of $(\pi,V)$. Moreover, since,
  \begin{equation*}
    e^\pi(W) = \sum_{w\in W}(\hecke\ast e_K^\pi)(w),
  \end{equation*}
  again using Lemma \ref{lemma_28}, we obtain $e^\pi(W)$ is a direct sum of copies of $(\pi,V)$ that is $e^\pi(W)\in\Rep^\pi(H)$.
  \par 2). For $\Rightarrow)$, if $e^\pi$ is trivial then by the functoriality of $e^\pi$, $e^\pi=0$ on each irreducible subquotient of the representation $(\tau,W)$ by taking surjective maps $W\twoheadrightarrow W/U$. 
  Since $e^\pi=\id$ on $(\pi,V)$, this means that $(\tau,W)$ has no irreducible subquotient isomorphic to $(\pi,V)$. For the other direction, $\Leftarrow)$, suppose $e^\pi:W\to W$
  is non-zero, then 
  \begin{equation*}
    0\not= e^\pi(W)=\bigoplus(\pi,V)\subseteq W.
  \end{equation*} 
  Thus $(\tau,W)$ has a subrepresentation, in particular a subquotient isomorphic to $(\pi,V)$.
  \par The last statement follows from the first two.
\end{proof}

\begin{definition}
  We denote the set of isomorphism classes of irreducible finite representations of $H$ by $\Irr_f(H)$. 
  \par We also denote $\Rep^f(H)$ to be the full subcategory of $\Rep(H)$ such that the objects consist of finite smooth representations of $H$.
\end{definition}

\begin{corollary}
  \label{corollary_30}
  Any finite representation of $H$ is a direct sum of irreducible finite representations. Hence we have an equivalence of categories,
  \begin{center}
    \begin{tikzcd}
\Rep^f(H) \arrow[rr, "{(\tau,W)\mapsto(e^\pi(W))_{\pi}}", shift left] &  & \prod_{\pi\in\Irr_f(H)}\Rep^\pi(H) \arrow[ll, "\bigoplus W_\pi\mapsfrom (W_\pi)_{\pi}", shift left]
\end{tikzcd}
  \end{center}
\end{corollary}

\begin{proof}
  Let $(\tau, W)\in\Rep^f(H)$. We denote $W^f$ as the sum of all finite irreducible subrepresentation of $W$.
  Suppose $W^f\subsetneqq W$, then $W/W^f$ admits an irreducible finite subquotient of some $\pi$. By the second statement of Proposition \ref{proposition_26}, we have,
  \begin{equation*}
    e^\pi(W/W^f)\not=0.
  \end{equation*}
  Thus we have,
  \begin{equation*}
    e^\pi(W^f)\subsetneqq e^\pi(W).
  \end{equation*}
  On the other hand, using Proposition \ref{proposition_26}, we have,
  \begin{equation*}
    e^\pi(W) = \bigoplus(\pi,V)\subseteq W^f.
  \end{equation*}
  In particular, we have,
  \begin{equation*}
    e^\pi(W)\subseteq e^\pi(W^f).
  \end{equation*}
  We derived a contradiction. By Lemma \ref{lemma_28}, we derived the statement.
\end{proof}

\section{Splitting Off All Finite Irreducible Representations}

We start the section by giving a definition.

\begin{definition}
  $\Rep^{\nf}(H)$ is the full subcategory of $\Rep$ whose objects consist of smooth representations of $H$  such that no irreducible subquotient of it is finite.
\end{definition}

The goal of section is to prove the decomposition 
\begin{equation*}
  \Rep(G^\circ) = \Rep^f(G^\circ)\times \Rep^{\nf}(G^\circ).
\end{equation*}

\subsection{The Group $G^\circ$}

%Revise this part later.

Recall we have the additive valuation,
\begin{equation*}
  \val_F:F\to\R\cup\{\infty\}.
\end{equation*}

\begin{definition}
  An algebraic character $\chi:G\to \G_m$ is a morphism of algebraic groups. It is said to be defined over $F$ if it is fixed under the following action.
  For $\sigma\in\Gal(\overline{F}/F)$, we define,
  \begin{equation*}
    \sigma\chi = \sigma\circ\chi\circ\sigma^{-1}.
  \end{equation*}
  That is $\chi$ is defined over $F$ if 
  \begin{equation*}
    \sigma\chi = \chi.
  \end{equation*}
  The set of all algebraic characters defined over $F$ is denoted by 
  \begin{equation*}
    \Hom_F(G,\G_m).
  \end{equation*}
\end{definition}

\begin{definition}
  We define,
  \begin{equation*}
    G^\circ \defeq \{g\in G\sep \forall \chi\in\Hom_F(G,\G_m), \val_F(\chi(g)) = 0\}.
  \end{equation*}
\end{definition}


\begin{example}
  For $\G=\mathbb{GL}_n$, we then have, 
  \begin{equation*}
    \Hom_F(G,\G_m) = \{det^n\sep n\in\Z\}.
  \end{equation*}
  In that case,
  \begin{equation*}
    G^0 = \{g\in\GL_nF\sep \val_F(\det(g)) = 0\}\supseteq \GL_n(\mathcal{O})\supseteq\SL_n(F).
  \end{equation*}
\end{example}

\begin{lemma}
  \label{lemma_31}
  Such $G^\circ$ satisfies the following conditions.
  \begin{enumerate}[1).]
    \item $G^\circ$ has a compact center.
    \item $G/Z(G)G^\circ$ is finite.
    \item $G/G^\circ$ is free abelian of finite type.
    \item $G^\circ$ contains all compact open subgroups of $G$ including the derived subgroup $[G,G]\subseteq G$.
  \end{enumerate}
\end{lemma}

\begin{proposition}
  Let $K\subseteq H$ be a compact open subgroup. Then the following three conditions are equivalent for $f\in\hecke$.
  \begin{enumerate}[i).]
    \item There is $f'\in\hecke$ such that $f = e_K\ast f' \ast e_K$.
    \item $e_K\ast f = f\ast e_K$.
    \item For any $k_1,k_2\in K$ and $g\in H$, we have $f(k_1gk_2) = f(g)$.
  \end{enumerate}
  \label{proposition_equivalent_def_hecke}
\end{proposition}

\begin{proof}
  $i)\Rightarrow ii)$ is obvious. For $ii)\Rightarrow iii)$, we have,

  \begin{equation*}
    f(k_1g) = (e_K\ast f)(k_1g) = {\frac 1 {\vert K\vert}}\int_K f(k^{-1}k_1g)dk = e_K\ast f(g) = f(g). 
  \end{equation*}
  Similarly for the case $f(gk_2) = f(g)$.
  \par Lastly if $f\in\hecke$ satisfies the condition of 3). then 
  \begin{align*}
    (e_K\ast f\ast e_K)(g) & = {\frac 1 {\vert K\vert^2}}\int_K\int_H f(h)e_K(h^{-1}k^{-1}g)dhdk,\\
    & \stackrel{h\to k^{-1}gh}{=} {\frac 1 {\vert K\vert^2}}\int_K\int_K f(k^{-1}gh)dhdk,\\
    & = {\frac 1 {\vert K\vert^2}}\int_K\int_K f(g)dhdk.
  \end{align*}
\end{proof}

\begin{definition}
  Let $K\subseteq H$ be a compact open subgroup. By Proposition \ref{proposition_equivalent_def_hecke}, we define the following subalgebra $\hecke(H,K)$ of $\hecke(H)$.
  \begin{equation*}
    \hecke(H,K) \defeq e_K\ast \hecke(H)\ast e_K.
  \end{equation*}
\end{definition}

\begin{remark}
  $\hecke(H,K)$ is a $\C$-algebra with unit $e_K$.
\end{remark}

Observe that we have an algebra homomorphism given by,
\begin{equation*}
  \hecke\ni f\mapsto \tau(f)\in\End_\C(W).
\end{equation*}

This restrictin this homomorphism to $\hecke(H,K)$, we get 
\begin{equation*}
  \hecke(H,K)\ni f\mapsto \tau(f)\in\End_\C(W^K).
\end{equation*}

Hence $W^K$ can be regarded as a $\hecke(H,K)$-module.

\begin{definition}
  We define a $H$-representation $(l_{\reg},\hecke)$ of $\hecke(H)$ by 
  \begin{equation*}
    (l_{\reg}(h)f)(x) = f(h^{-1}x).
  \end{equation*}
\end{definition}

\begin{lemma}
  $l_{\reg}$ gives an algebra homomorphism,
  \begin{equation*}
    \hecke\ni f\mapsto l_{\reg}(f)\in\End_\C(\hecke),
  \end{equation*}
  which agrees with the left multiplication. That is 
  \begin{equation*}
    l_{\reg}(f)(f') = f\ast f'.
  \end{equation*}
  \label{lemma_l_reg_multi}
\end{lemma}

\begin{proof}
  \begin{align*}
    l_{\reg}(f)(f')(x) & = \int_H f(g)(l_{\reg}(g)(f'))(x)dg,\\
    & = \int_Hf(g)f'(g^{-1}x),\\
    & = (f\ast f')(x).
  \end{align*}
\end{proof}

\begin{proposition}
  \label{proposition_34}
  The map $W\mapsto W^K$ gives a bijection between,
  \begin{equation*}
    \left\{\substack{\text{Isomorphism classes of irreducible representation}\\\text{$(\tau,W)$ of which $W^K\not=0$.}}\right\}\leftrightarrow \left\{\substack{\text{Isomorphism classes of simple}\\\text{$\hecke(H,K)$-modules}}\right\}.
  \end{equation*}
\end{proposition}

\begin{proof}
  Let $(\tau,W)$ be an irreducible representation of $H$ such that $W^K\not=0$. We will first prove that $W^K$ is a simple $\hecke(H,K)$-module. Let $M\subseteq W^K$ be a non-zero
  $\hecke(H,K)$-submodule. We define a subspace $\hecke\cdot M$ of $W$ by 
  \begin{equation*}
    \hecke\cdot M\defeq \{\tau(f)(m)\sep f\in\hecke,m\in M\}.
  \end{equation*}
  This is clearly $\hecke$-stable. Since $M$ is a subspace of $M$ and $e_K\in\hecke$, this is non-zero. By the irreducibility of $W$, we conclude $\hecke\cdot M = W$.
  That is 
  \begin{align*}
    W^K &=\tau(e_K)(W),\\
    & = \tau(e_K)(\hecke\cdot M),\\
    & = \{\tau(e_K\ast f)(m)\sep f\in\hecke, m\in M\},\\
    & = \{\tau(e_K\ast f\ast e_K)(m)\sep f\in\hecke,m\in M\},\\
    & = M.
  \end{align*}
  This shows that $W^K$ is a simple $\hecke(H,K)$-module.
  \par Let $M$ be a simple $\hecke(H,K)$-module and consider a smooth $H$-representation, 
  \begin{equation*}
    (l_{\reg}\otimes\id,\hecke\tens{\hecke(H,K)}M).
  \end{equation*}
 
By Lemma \ref{lemma_l_reg_multi}, we get,
  \begin{align*}
    (\hecke\tens{\hecke(H,K)}M)^K & = (l_{\reg}\tens{\hecke(H,K)}\id)(e_K)(\hecke\tens{\hecke(H,K)}M),\\
    & = (e_K\ast \hecke )\tens{\hecke(H,K)}M,\\
    & = (e_K\ast \hecke e_K)\tens{\hecke(H,K)}M,\\
    & \cong M,
  \end{align*}
  where the last isomorphism is by,
  \begin{equation*}
    M\ni m\mapsto e_K\otimes m.
  \end{equation*}
  Let $X$ be a a $\hecke$-subspace of $\hecke\tens{\hecke(H,K)} M$ such that 
  \begin{equation*}
    X = \{x\in\hecke\tens{\hecke(H,K)} M\sep \forall h\in H, e_K(h\cdot x) = 0\}.
  \end{equation*}
  Hence $X$ is the unique maximal $\hecke$-subspace of $\hecke\tens{\hecke(H,K)} M$ with the property,
  \begin{equation*}
    X^K = e_K(X) = 0.
  \end{equation*}
  Let us now define 
  \begin{equation*}
    (\hecke\tens{\hecke(H,K)} M)_{e_K}\defeq (\hecke\tens{\hecke(H,K)} M)/X.
  \end{equation*}
  Recall that $e_K$ is an exact functor. Therefore, we have,
  \begin{equation*}
  ((\hecke\tens{\hecke(H,K)}M)_{e_K})^K = (\hecke\tens{\hecke(H,K)}M)^K \cong M.
  \end{equation*}
  We will show that $(\hecke\tens{\hecke(H,K)}M)_{e_K}$ is irreducible. Take $\overline{V}\subseteq (\hecke\tens{\hecke(H,K)}M)_{e_K}$ to be 
  a non-zero $H$-subspace. Then there is $V\subseteq \hecke\tens{\hecke(H,K)}M$ such that 
  \begin{center}
    \begin{tikzcd}
X \arrow[d, two heads] \arrow[r, hook] & V \arrow[d, two heads] \arrow[r, hook] & {\hecke\tens{\hecke(H,K)}M} \arrow[d, two heads] \\
\{0\} \arrow[r, hook]                  & \overline{V} \arrow[r, hook]           & {(\hecke\tens{\hecke(H,K)}M)_{e_K}}             
\end{tikzcd}
  \end{center}
  By the maximality of $X$, we have $V^K\not=0$. That is,
  \begin{equation*}
    V^K = V\cap \underbrace{(\hecke\tens{\hecke(H,M)}M)^K}_{\cong M} \not = 0.
  \end{equation*}
  By the simplicity of $M$, we have, 
  \begin{equation*}
    (\hecke\tens{\hecke(H,M)}M)^K\subseteq V.
  \end{equation*}
  Since $V$ is $\hecke$-stable and $(e_K\ast\hecke)\tens{\hecke(H,K)}M$ generates $\hecke\tens{\hecke(H,K)}M$ ans a $\hecke$-module. We conclude 
  \begin{equation*}
    V = \hecke\tens{\hecke(H,K)}M.
  \end{equation*}
  It remains to show for any irreducible representation $(\tau,W)$ such that $W^K\not=0$, we have,
  \begin{equation*}
    (\hecke\tens{\hecke(H,K)}W^K)_{e_K} \cong W.
  \end{equation*}
  Define a map $\phi:\hecke\tens{\hecke(H,K)}W^K\to W$ by 
  \begin{equation*}
  \phi(f\otimes w) = \tau(f)(w).
  \end{equation*}
  This is a non-zero $H$-equivariant map. We now show that 
  \begin{equation*}
  X\subseteq \Ker\phi.
  \end{equation*}
  Suppose that $\phi(x)\not=0$ for some $x\in X$. Then as $(\tau,W)$ is irreducible we have 
  \begin{equation*}
    \Span_\C\{\tau(h)\phi(x)\sep h\in H\}=W.
  \end{equation*}
  On the other hand, $\phi$ is $H$-equivariant. Therefore, using the definition of $X$, we have,
  \begin{equation*}
    \forall h\in H,\tau(e_K)(\tau(h)(\phi(x))) = 0.
  \end{equation*}
  That is 
  \begin{equation*}
    W^K = \tau(e_K)(W) = 0.
  \end{equation*}
  This is a contradiction. Thus $\phi$ induces a non-zero $H$-equivariant map 
  \begin{equation*}
  \overline{\phi}:(\hecke\tens{\hecke(H,K)}W^K)_{e_K}\to W.
  \end{equation*}
  Since $(\hecke\tens{\hecke(H,K)}W^K)_{e_K}$ is irreducible and $W$ is irreducible we conclude $\overline{\phi}$ is an isomorphism.
\end{proof}

\begin{remark}
  We denote $\Rep^K(H)$ as the full subcategory of $\Rep(H)$ whose objects are representations $(\tau,W)\in\Rep(H)$ such that $W$ is generated by $W^K$.
  Then we have a functor 
  \begin{equation*}
    \Rep^K(H)\ni W\mapsto W^K\in\Mod(\hecke(H,K)).
  \end{equation*}
  This functor turns out to give an equivalence of categories if and only if the subcategory $\Rep^K(H)$ is stable under taking subquotient.
\end{remark}

\subsection{Some Decomposition of $G$}

\begin{definition}
  Let $\Para_0=\M_0\rtimes \N_0$ be the minimal parabolic subgroup of $\G$ and denote $\A_0$ as 
  the maximal split torus of $\M_0$. Also denote $\overline{\Para_0} = \M_0\overline{\N_0}$ as the opposite parabolic subgroup of $\Para_0$.
\end{definition}

\begin{remark}
  $\A_0$ is a maximal split torus of $\G$ and $\M_0 = \Cent_\G(\A_0)$.
\end{remark}

\begin{example}
  For $\G=\GL_n$, we can take 
  \begin{equation*}
    \Para_0 = \left\{\begin{pmatrix}
      \ast & \cdots & \cdots & \ast \\
      \: & \ddots& \ddots & \vdots \\
       \: & \:& \ddots & \vdots \\
      \: & \: & \:& \ast
    \end{pmatrix}\right\}
    = \left\{\begin{pmatrix}
      \ast & & \: \: & \: \\
      \: & \ddots & \: &  \: \\
      \: & \: & \ddots & \: \\
      \: & \: & \: & \ast
    \end{pmatrix}\right\}\rtimes
    \left\{\begin{pmatrix}
      1& \ast & \cdots & \ast \\
      \: & \ddots & \ddots&  \vdots \\
      \: & \: & \ddots & \ast \\
      \: & \: & \: & 1
    \end{pmatrix}\right\}
  \end{equation*}
  In this case we have $\M_0=\A_0$. Furthermore, we have,
  \begin{equation*}
   \overline{\Para_0} = \left\{\begin{pmatrix}
      \ast & \: & \:& \: \\
      \vdots & \ddots& \: & \: \\
       \vdots & \ddots& \ddots & \: \\
      \ast & \cdots & \cdots& \ast
    \end{pmatrix}\right\}
    = \left\{\begin{pmatrix}
      \ast & & \: \: & \: \\
      \: & \ddots & \: &  \: \\
      \: & \: & \ddots & \: \\
      \: & \: & \: & \ast
    \end{pmatrix}\right\}\rtimes
    \left\{\begin{pmatrix}
      1& \: & \:& \: \\
      \ast & \ddots & \:&  \:\\
      \vdots &\ddots & \ddots & \: \\
      \ast & \cdots& \ast & 1
    \end{pmatrix}\right\}
  \end{equation*}
\end{example}

\begin{remark}
  There exists a maximal compact open subgroup $K_0\subseteq G$ such that 
  \begin{align*}
    K_0\cap M_0 & = (M_0)^\circ,\\
    & = \{m\in M_0\sep \forall \chi\in\Hom_F(\M_0,\G_m), \val_F(\chi(m))\}.
  \end{align*}
\end{remark}

\begin{example}
  For $\G = \GL_2$. We have,
  \begin{equation*}
    K = \GL_2(\mathcal{O}) = \left\{\begin{pmatrix}
    a & b \\ c& d
    \end{pmatrix}
    \:\bigg{|}\: a,d\in\mathcal{O}^\times,b\in\mathcal{O}, c\in\mathfrak{p} = (\varpi)\right\}.
  \end{equation*}
\end{example}

\begin{example}
  \begin{equation*}
    \GL_n(\mathcal{O})\cap\left\{\begin{pmatrix}
    \ast & \: & \: \\
    \: & \ddots & \:\\
    \: & \: & \ast
    \end{pmatrix}\right\} = \left\{\begin{pmatrix}
    t_1  & \: & \:\\
    \: & \ddots & \:\\
    \: & \: &  t_n
  \end{pmatrix}\:\Bigg{|}\: t_i\in\mathcal{O}^\times\right\}.
  \end{equation*}
  And $\Hom_F(\M_0,\G_m)\cong \Z^n$ which is given by 
  \begin{equation*}
    \left[\begin{pmatrix}
    t_1  & \cdots & \:\\
    \: & \ddots & \:\\
    \: & \: &  t_n
  \end{pmatrix}\mapsto \prod_{i=1}^nt_i^{a_i}\right]\leftrightarrow (a_1,\cdots,a_n).
  \end{equation*}
  Thus 
  \begin{align*}
    (M_0)^0 & \left\{\begin{pmatrix}
    t_1  & \: & \:\\
    \: & \ddots & \:\\
    \: & \: &  t_n
  \end{pmatrix}\:\Bigg{|}\:\forall (a_i)_{i=1,\cdots,n}\in\Z^n,\val_F\left(\prod_{i=1}^n t_i^{a_i}\right)=0\right\},\\
  & = \left\{\begin{pmatrix}
    t_1  & \: & \:\\
    \: & \ddots & \:\\
    \: & \: &  t_n
  \end{pmatrix}\:\Bigg{|}\:\forall i=1,\cdots,n, \val_F(t_i)=0\right\},\\
  & = = \left\{\begin{pmatrix}
    t_1  & \: & \:\\
    \: & \ddots & \:\\
    \: & \: &  t_n
  \end{pmatrix}\:\Bigg{|}\:\forall i=1,\cdots,n, t_i\in\mathcal{O}^\times\right\}.\\
  \end{align*}
\end{example}

\begin{proposition}[Iwasawa Decomposition]
  We have 
  \begin{equation*}
    \GL_n(F) = P_0\cdot K_0 = K_0\cdot P_0,
  \end{equation*}
  where $P_0$ is the subgroup of upper triangular matrices and 
  \begin{equation*}
    K_0 = \GL_n(\mathcal{O}).
  \end{equation*}
\end{proposition}

\begin{proof}
  First observe that any matrix expressing transposition belongs to $K_0$. Also we can realize the column operation of adding $a\in\mathcal{O}$ multiple of $i$-th column to $j$-th column as 
  \begin{equation*}
    (g_{ij}(a))_{kl} = \begin{cases}
      1,\quad (k=l),\\
      a, \quad (k=i,l=j),\\
      0, \quad(\text{otherwise}).
    \end{cases}
  \end{equation*}
  From elementary linear algebra, we obtain the first equality. For the second equality, since all three of them are groups, simply by taking inverses, we derived the statement.
\end{proof}

\begin{proposition}[Cartan Decomposition]
  Let $K_0$ be the same as before. Define,
  \begin{align*}
    M_0^+ & = \{\diag(t_1,\cdots,t_n)\sep \forall 1\leq i<j\leq n, \val_F(t_it_j^{-1})\geq 0\},\\
    & = \{\diag(t_1,\cdots,t_n)\sep \forall 1\leq i<j\leq n, \val_F(t_1)\geq\cdots\geq\val_F(t_n)\}.
  \end{align*}
  Then we have,
  \begin{equation*}
    G = K_0\cdot M_0^+ \cdot K_0.
  \end{equation*}
\end{proposition}

Note that we have 
\begin{equation*}
  M_0^\circ\subseteq M_0^+.
\end{equation*}

We also note that $M_0^\circ = K_0\cap M_0$ by the definition of $K_0$. Thus a double coset of $K_0 M_0^+K_0$ is determined by $m\in M_0^+/(M_0)^\circ$.
From this we obtain another decomposition.

\begin{proposition}[Cartan Decomposition]
  \begin{equation*}
    G=K_0M_0^+K_0  =\bigcup_{m\in M_0^+/(M_0)^\circ}K_0 mK_0 = \coprod_{m\in M_0^+/(M_0)^\circ}K_0 mK_0.
  \end{equation*}
\end{proposition}

\begin{example}[$\G=\GL_n$]
  Recall we have the following identification.
  \begin{align*}
    M_0^+ & = \{\diag(t_1,\cdots,t_n)\in A_0\sep \val_F(t_1)\geq\cdots\geq \val_F(t_n)\}.\\
    (M_0)^\circ & = \{\diag(t_1,\cdots,t_n)\in A_0\sep \val_F(t_1)=\cdots= \val_F(t_n)=0\}.
  \end{align*}
  Thus easy to see $(M_0)^\circ\subseteq M_0^+$. Also from these, we can identify,
  \begin{equation*}
    M_0^+/(M_0)^\circ \cong \{(a_1,\cdots,a_n)\in\Z^n\sep a_1\geq\cdots\geq a_n\},
  \end{equation*}
  under the identification,
  \begin{equation*}
    [\diag(\varpi^{a_1},\cdots,\varpi^{a_n})]\mapsfrom(a_1,\cdots,a_n)\in\Z^n.
  \end{equation*}
  We now show that 
  \begin{equation*}
    \GL_n(F) = \coprod_{a_1\geq\cdots\geq a_n}\GL_n(\mathcal{O})\{\diag(\varpi^{a_1},\cdots,\varpi^{a_n})\}\GL_n(\mathcal{O}).
  \end{equation*}
\end{example}

\begin{definition}
  Fix a minimal parabolic subgroup $\Para_0$ of $\G$. A parabolic subgroup $\Para$ of $\G$ is said to be standard if it contains $\Para_0$.
\end{definition}

\begin{definition}
  Let $H\subseteq G$ be a subgroup and $g\in G$. We denote the conjugation of $H$ by $g$ as,
  \begin{equation*}
    \prescript{g}{}{H} = gHg^{-1}.
  \end{equation*}
\end{definition}

\begin{proposition}[Iwahori Decomposition]
  There existsa a neighborhood basis of the identity $1\in G$ consisting of compact open subgrups $K\subseteq K_0$ such that for any standard parabolic subgroup $\Para = \M\ltimes\N$, we have,
  \begin{equation*}
    K = (K\cap N)(K\cap M)(K\cap\overline{N}).
  \end{equation*}
  For $m\in M_0^+$, we have,
  \begin{align*}
    \prescript{m}{}{(K\cap N_0)}&\subseteq K\cap N_0,\\
     \prescript{m}{}{(K\cap M_0)}&= K\cap M_0,\\
      \prescript{m}{}{(K\cap \overline{N}_0)}&\supseteq K\cap \overline{N}_0.
  \end{align*}
\end{proposition}

\begin{example}[$\G=\GL_n$]
  In this case, we can choose a neighborhood basis $(K_r)_{r\in\N}$ to be such that 
  \begin{equation*}
    K_r = \{g\in\GL_n(\mathcal{O})\sep g\equiv 1+\mathfrak{p}^r\},
  \end{equation*}
  where 
  \begin{equation*}
    \mathfrak{p}^r = \varpi^r\Mat_{n\times n}(\mathcal{O}_F).
  \end{equation*}
\end{example}


\begin{example}[$\G=\GL_2$]
  For $\Para = \Para_0$, and $K=K_r$ for some $r\in\N$, observe that 
  \begin{align*}
    K\cap N &= \left\{\begin{pmatrix}
      1 & x\\
      0& 1
    \end{pmatrix}
    \:\Bigg{|}\: x\in\mathfrak{p}^r\right\},\\
    K\cap M_0 & = \left\{\begin{pmatrix}
      t_1 & 0\\
      0& t_2
    \end{pmatrix}
    \:\Bigg{|}\: t_1,t_2\in1+\mathfrak{p}^r\right\},\\
    K\cap \overline{N} &= \left\{\begin{pmatrix}
      1 & 0\\
      y& 1
    \end{pmatrix}
    \:\Bigg{|}\: y\in\mathfrak{p}^r\right\}.
  \end{align*}
  We want to show that 
  \begin{equation*}
    \begin{pmatrix}
      t_1+xt_2y & xt_2\\
      t_2y & t_2
    \end{pmatrix}
    \in I_2+\mathfrak{p}^r.
  \end{equation*}
  But this is obvious as $x,y\in\mathfrak{p}^r$.
\end{example}

\begin{remark}[Warning]
  \begin{equation*}
    \GL_n(\mathcal{O}_F)\supsetneqq (\GL_n(\mathcal{O}_F)\cap N_0)(\GL_n(\mathcal{O}_F)\cap M_0)(\GL_n(\mathcal{O}_F)\cap \overline{N}_0).
  \end{equation*}
\end{remark}

\begin{example}
  For $\G=\GL_2$ and for $a\geq b$, let
  \begin{equation*}
    m = \begin{pmatrix}
      \varpi^a & 0\\
      0&\varpi^b
    \end{pmatrix}.
  \end{equation*}
  Then 
  \begin{align*}
    \prescript{m}{}{(K\cap N_0)}&= \left\{\begin{pmatrix}
      1 & \varpi^{a-b}x\\
      0& 1
    \end{pmatrix}
    \:\Bigg{|}\: x\in\mathfrak{p}^r\right\},\\
     \prescript{m}{}{(K\cap M_0)}& = \left\{\begin{pmatrix}
      t_1 & 0\\
      0& t_2
    \end{pmatrix}
    \:\Bigg{|}\: t_1,t_2\in1+\mathfrak{p}^r\right\},\\
      \prescript{m}{}{(K\cap \overline{N}_0)}&\supseteq K\cap \overline{N}_0.
  \end{align*}
\end{example}

\subsection{Structure of Hecke Algebra $\hecke(G,K)\supseteq\hecke(G^\circ,K)$}

\begin{definition}
  Let $K_0$ be the maximal compact open subgroup defined in Remark \ref{remark_maximal_compact_open_subgroup} satisfying all Iwasawa, Cartan, and Iwahori decompositions.
  Let $K\triangleleft K_0$ be a normal subgroup. We define,
  \begin{equation*}
    \hecke_0\defeq\{f\in\hecke(G,K)\sep \supp(f)\subseteq K_0\}.
  \end{equation*}
\end{definition}

\begin{remark}
  Any element $f\in\hecke_0$ is determined by its values on the double cosets $K\backslash K_0/ K$. Furthermore, for this reason, $\hecke_0$ is finite dimensional as $K_0$ is compact.
\end{remark}

\begin{definition}
  Let $x\in G$ and we define,$\alpha_x\in\hecke(G,K)$ to be such that
  \begin{enumerate}[i).]
    \item $\supp(\alpha_x)=KxK$,
    \item $\int_G\alpha_x(g)dg = 1$.
  \end{enumerate}
\end{definition}

By definition $\alpha_x$ only depends on the double coset $KxK$. Recall that as a $\C$-vector space, we have,
\begin{equation*}
  \hecke(G,K) = \bigoplus_{x\in K\backslash G/K}\C\alpha_x.
\end{equation*}

From this observation we see,

\begin{equation*}
  \{\alpha_k\}_{k\in K\backslash K_0/K},
\end{equation*}

forms a basis of $\hecke_0$.

\begin{lemma}$\:$
  \label{lemma_35}
  \begin{enumerate}[1).]
    \item For $f_1,f_2\in\hecke$, we have $\supp(f_1\ast f_2)\subseteq \supp(f_1)\supp(f_2)$.
    \item Let $x\in G$ and $k\in K_0$, we have $a_x\ast a_k = a_{xk},a_k\ast a_x = a_{kx}$.
    \item Let $m_1,m_2\in M_0^+$ then we have $a_{m_1}\ast a_{m_2} = a_{m_1m_2}$.
  \end{enumerate}
\end{lemma}
\begin{proof}
  \begin{equation*}
    (f_1\ast f_2)(x) = \int_G f(g)f(g^{-1}x)dg = \int_{\supp f_1}f_1(g)f_2(g^{-1}x)dg.
  \end{equation*}
  Hence, $(f_1\ast f_2)(x)\not=0$ then there is $g\in \supp(f_1)$ and $g^{-1}x\in\supp(f_2)$. Thus $x\in\supp(f_1)\supp(f_2)$.
  \par For the second statement, note that $K_0$ normalizes $K$. From the previous part, we have, 
  \begin{equation*}
    \supp(\alpha_x\ast\alpha_k) \supp(a_x)\supp(a_k) = KxKkK = KxkK.
  \end{equation*}
  Moreover, 
  \begin{align*}
    \int_G(a_x\ast a_k)(g)dg & = \int_G\int_Ga_x(h)a_k(h^{-1}g)dhdg,\\
    & = \int_G\int_Ga(h)a_k(h^{-1}g)dgdh,\\
    & = \int_Ga_x(h)\int_Ga_k(h^{-1}g)dgdh,\\
    & \stackrel{g\to hg}{=}\int_Ga_x(h)\int_Ga_k(g)dgdh,\\
    & = 1.
  \end{align*}
  Similarly to the previous part, we have $\int_G(a_{m_1}\ast a_{m_2})(g)dg = 1$. Remains to examine its support. That is to show that 
  \begin{equation*}
    \supp(a_{m_1}\ast a_{m_2}) \subseteq Km_1m_2K.
  \end{equation*}
  Combining the first part and Iwahori decomposition, we have,
  \begin{align*}
    \supp(a_{m_1}\ast a_{m_2}) & \subseteq \supp(a_{m_1})\supp(a_{m_2}),\\
    & = Km_1Km_2K,\\
    & = Km_1(K\cap N_0)(K\cap M_0)(K\cap \overline{N}_0)m_2K,\\
    & = K(m_1(K\cap N_0)m_1^{-1})(m_1 (K\cap M_0)m_1^{-1})m_1m_2(m_2^{-1}(K\cap\overline{N_0})m_2)K,\\
    & = K\prescript{m_1}{}{(K\cap N_0)}\prescript{m_1}{}{(K\cap M_0)}m_1m_2\prescript{m_2}{}{(K\cap \overline{N}_0)}K.
  \end{align*}
\end{proof}

\begin{definition}
  We now define subspaces of $\hecke(G,K)$ and $\hecke(G^\circ,K)$ by,
  \begin{align*}
    \mathcal{M} & \defeq \bigoplus_{m\in M_0^+/(M_0)^\circ}\C a_m\subseteq\hecke(G,K),\\
    \mathcal{M}^\circ & \defeq \bigoplus_{m\in (M_0^+\cap G^\circ)/(M_0)^\circ}\C a_m\subseteq\hecke(G^\circ,K).
  \end{align*}
\end{definition}

\begin{remark}
  In general, we have $(M_0)^\circ\subsetneqq M_0\cap G^\circ$.
\end{remark}

\begin{example}
  For $G=\GL_2(F)\supseteq M_0 = \{\diag(t_1,t_2)\}$. We have,
  \begin{align*}
    (M_0)^\circ &= \{\diag(t_1,t_2)\sep \val_F(t_1)=\val_F(t_2) = 0\},\\
    M_0\cap G^\circ & \{g=diag(t_1,t_2)\sep \val_F(\det(g)) = \val_F(t_1t_2) = 0\}.
  \end{align*}
\end{example}

\begin{lemma}[Gordon]
  Let $X_1,\cdots,X_k\in\Z^n$. Then the monoid 
  \begin{equation*}
    C = \{Y\in\Z^n\sep i=1,\cdots,n \langle X_i,y\rangle\geq0\},
  \end{equation*}
  is finitely generated where the inner product is defined in the regular sense.
\end{lemma}

\begin{corollary}
  \label{corollary_36}
  $\mathcal{M}$ and $\mathcal{M}^\circ$ are commutative subalgebras of $\hecke(G,K),\hecke(G^\circ,K)$, respectively. Moreover, they are both finitely generated.
\end{corollary}

\begin{proof}
  We will show the statement for $\mathcal{M}$. Since $M_0^+/(M_0)^\circ\subseteq M_0/(M_0)^\circ$ and $M_0/(M_0)^\circ$ is free abelian group of finite type by Lemma \ref{lemma_31}, we conclude 
  $M_0/(M_0)^\circ$ is a commutative monoid. Using the second statement of Lemma \ref{lemma_35} and Gordon's lemma, we obtain the statement.
\end{proof}

\begin{proposition}
  \label{proposition_37}
  We have,
  \begin{equation*}
    \hecke(G,K) = \hecke_0\ast\mathcal{M}\ast\hecke_0,\hecke(G^\circ,K) = \hecke_0\ast\mathcal{M}^\circ\ast\hecke_0.
  \end{equation*}
  In particular, $\hecke(G,K),\hecke(G^\circ,K)$ are both finitely generated.
\end{proposition}

\begin{proof}
  We wil prove the claim for $G$. By Cartan decomposition, we have,
  \begin{equation*}
    G = K_0M_0^+K_0 = \bigcup_{k_1,k_2\in K_0/K}Kk_1M_0^+k_2K.
  \end{equation*}
  Hence, we have,
  \begin{equation*}
    \hecke(G,K) = \bigoplus_{x\in K\backslash G/K}\C a_x = \Span_\C\{a_{k_1mk_1}\sep k_1,k_2\in K_0,m\in M_0^+\}.
  \end{equation*}
  By Lemma \ref{lemma_35}, we have 
  \begin{equation*}
    a_{k_1mk_2} = a_{k_1}\ast a_M \ast a_{k_2}\in\hecke_0\ast\mathcal{M}\ast\hecke_0.
  \end{equation*}
\end{proof}

\begin{theorem}[Burnside's Theorem]
  \label{theorem_burnsides}
  Let $S$ be a finite dimensional $\C$-vectorspace and $A\subseteq \End_\C(S)$ be a subalgebra. Suppose that $S$ is simple as an $A$-module.
  Then, we have 
  \begin{equation*}
    A = \End_\C(S).
  \end{equation*}
\end{theorem}

\begin{theorem}[Uniform Admissibility]
  \label{theorem_38}
  Let $K\subseteq G^\circ \subseteq G$ be a compact open subgroup. Recall that any compact open subgroup is contained in $G^\circ$.
  Then there is a positive constant $c=C(G,K)$ such that for any irreducible admissible representation $(\pi,V)$ of $G$ or $G^\circ$, we have,
  \begin{equation*}
    \dim V^K\leq C.
  \end{equation*}
\end{theorem}
\begin{proof}
  We prove it only for $G$ but the same argument applies to $G^\circ$. We may also assume that $K$ is contained in $K_0$ and is normalized by it.
  \par Recall Proposition \ref{proposition_34}. From this, it suffices to show that for any finite dimensional simple $\hecke(G,K)$-module $S$, we have $\dim S\leq C$. Let $\dim S = k$ and consider the structure map,
  \begin{equation*}
    \rho_S = [\hecke(G,K)\ni f\mapsto [s\mapsto f\cdot s]\in\End_\C(S)].
  \end{equation*}
  Let $A=\rho_S(\hecke(G,K))$ and apply Burnside's theorem to this, as $S$ is already simple $\hecke(G,K)$-module, we see 
  \begin{equation*}
    \rho_S(\hecke(G,K)) = \End_\C(S).
  \end{equation*}
  Recall $\hecke(G,K) = \hecke_0\ast\mathcal{M}\ast\hecke_0$. Thus 
  \begin{equation*}
    k^2 = \dim(\End_\C(S)) = \dim(\rho_S(\hecke(G,K))) \leq \dim(\hecke_0)^2\cdot\dim(\rho(\mathcal{M})).
  \end{equation*}
  It suffices to show that there is some $\varepsilon>0$ such that 
  \begin{equation*}
    \dim(\rho_S(\mathcal{M}))\leq k^{2-\varepsilon},
  \end{equation*}
  which only depends on $G$ and $K$. Suppose we have such $\varepsilon$ then 
  \begin{equation*}
    k^2\leq k^{2-\varepsilon}(\dim\hecke_0)^2\Rightarrow k^{{\frac \varepsilon 2}}\leq \dim\hecke_0.
  \end{equation*}
  Therefore, we have,
 %Ended here 1/3
\end{proof}

%11/3
\begin{fact}
  Let $\Para_0 = \M_0\N_0$ be the minimal parabolic subgroups of $\G$ and denote $\A_0$ as the maximal split torus of $\M_0$.
  \par The opposite parabolic subgroup is denoted by 
  \begin{equation*}
    \overline{\Para}_0 = \M_0\overline{\N}_0.
  \end{equation*}
  Note that $\A_0$ is a maximal split torus of $\G$ and $\M_0 = \Cent_\G(\A)$.
\end{fact}

\begin{example}
  For $\G=\GL_n$, we can take,
  \begin{equation*}
    \Para_0 = \left\{\begin{pmatrix}
      \ast & \cdots & \ast\\
      \: & \ddots & \vdots\\
      \: & \: & \ast
    \end{pmatrix}\right\},
  \end{equation*}
  ie a set of upper triangular matrix. Furthermore, for,
  \begin{equation*}
    \M_0 = \left\{\begin{pmatrix}
      \ast &\: & \:&\:\\
      \: &\ast & \: &\:\\
      \: & \:& \ddots && \:\\
      \: & \:&\: & \ast
    \end{pmatrix}\right\},\N_0=\left\{\begin{pmatrix}
      1 &\ast & \cdots&\ast\\
      \: &1 & \ddots &\vdots\\
      \: & \:& \ddots & \ast\\
      \: & \:&\: & 1
    \end{pmatrix}\right\},
  \end{equation*}
  then we have,
  \begin{equation*}
    \Para_0 = \M_0\ltimes\N_0.
  \end{equation*}
  In this case, we have, $\M_0 = \A_0$.
  \par The opposite parablic groups are, 
  \begin{equation*}
    \overline{\Para_0}=\left\{\begin{pmatrix}
      \ast &\: & \:&\:\\
      \vdots &\ddots & \: &\:\\
      \vdots & \:& \ddots & \:\\
      \ast & \cdots&\cdots & \ast
    \end{pmatrix}\right\},
    \overline{\N_0}=\left\{\begin{pmatrix}
      1 &\: & \:&\:\\
      \ast &1 & \: &\:\\
      \vdots & \ddots& \ddots& \:\\
      \ast & \cdots&\ast & 1
    \end{pmatrix}\right\}.
  \end{equation*}
\end{example}

\begin{remark}
  \label{remark_maximal_compact_open_subgroup}
  There exists a maximal compact open subgroup $K_0\subseteq G$ such that 
    \begin{equation*}
    K_0\cap M_0 = (M_0)^\circ = \{m\in\M_0\sep\forall \chi\in\Hom_F(\M_0,\G_m), \val_F(\chi(m)) = 0\}.
    \end{equation*}
\end{remark}

\begin{example}
  \begin{equation*}
    \GL_n(\mathcal{O})\cap \left\{\begin{pmatrix}
      \ast & \: & \:\\
      \: & \ddots & \:\\
      \: & \: & \ast
    \end{pmatrix}\right\}
    =\left\{
      \begin{pmatrix}
      t_1 & \: & \:\\
      \: & \ddots & \:\\
      \: & \: & t_n
    \end{pmatrix}
    \Bigg{|}t_1,\cdots,t_n\in\mathcal{O}^\times
    \right\}.
  \end{equation*}
  We also have an identification, $\Hom_F(\M_0,\G_m)\cong \Z^n$, which is given by,
  \begin{equation*}
    \Hom_F(\M_0,\G_m)\ni
    \left[\begin{pmatrix}
      t_1 & \: & \:\\
      \: & \ddots & \:\\
      \: & \: & t_n
    \end{pmatrix}\mapsto\prod_{i=1}^n t_i^{a_i}\right]\longmapsfrom (a_1,\cdots,a_n)\in \Z^n
  \end{equation*}
  With this, we get,
  \begin{equation*}
    (\M_0)^\circ = \left\{\begin{pmatrix}
      t_1 & \: & \:\\
      \: & \ddots & \:\\
      \: & \: & t_n
    \end{pmatrix}\Bigg{|}\val_F\left(\prod_{i=1}^nt_i^{a_i}\right) = 0\right\}.
  \end{equation*}
\end{example}


\begin{corollary}
  \label{corollary_40}
  \begin{equation*}
    \Rep(G^\circ)\cong\Rep^f(G^\circ)\times\Rep^{\nf}(G^\circ).
  \end{equation*}
\end{corollary}

\begin{proof}
  Due to Theorem \ref{theorem_32} and Proposition \ref{proposition_33}.
\end{proof}

%11/10

Out next goal is to show that 

\begin{equation*}
  \Rep(G) \cong \Rep^{\supcus}(G)\times \Rep^{\nsc}(G).
\end{equation*}

where $\Rep^{\supcus}(G)$ amounts to the supercuspidal representations and $\Rep^{\nsc}(G)$ for the nonsupercuspidal representations.

\begin{notation}[Warning for the notation]
\end{notation}

As an intermediate goal, we will show the following.
\begin{proposition}
  Irreducible representations of $G$ is supercuspidal if and only if it's restriction to $G^\circ$ is finite.
\end{proposition} 

\begin{lemma}[Frobenius reciprocity]
  \label{lemma_41}
  \label{lemma_regular_frobenius}
  Let $H'\subset H$ be a closed subgroup. Let $(\sigma,W)$ be a smooth representation of $H'$ and $(\pi,V)$ be a smooth represenattion for $H$. \\
  \par Then we have,
  \begin{equation*}
  \Hom_H(V,\Ind_{H'}^H(W)) \cong \Hom_{H'}(V|_{H'},W),
  \end{equation*} 
  $V|_{H'}$ is the restriction of $\pi:H\to\End(V)$ to $H'$.
\end{lemma}

\begin{proof}
  Consider
  \begin{equation*}
  \Hom_H(V,|\Ind_{H'}^H(W))\ni\alpha\mapsto\beta_\alpha\in\Hom_{H'}(V|_{H'},W),
  \end{equation*}
  such that 
  \begin{equation*}
  \forall v\in V, \beta_\alpha(v) = \alpha(v)(\id_{H'}).
  \end{equation*}
  For the other way, we consider,
  \begin{equation*}
  \Hom_H(V,\Ind_{H'}^H(W))\ni\alpha_\beta\mapsfrom\beta\in\Hom_{H'}(V|_{H'},W),
  \end{equation*}
  where 
  \begin{equation*}
    \forall v\in V, \alpha_\beta(v) = [H\ni g\mapsto \beta(\pi(g)v)].
  \end{equation*}
  It is an exercise to check that these two maps are well-defined and are inverses of each other.
\end{proof}

Consider $\Para = \M\rtimes \N\subseteq\G$ be a parabolic subgroup with Levi subgroup $\M$ and the unipotent radical $\N$. From now on whenever we write $(\pi,V)$, we mean a smooth representation of $G$.

\begin{definition}
  \begin{equation*}
    V(N) \defeq \Span_\C\{\pi(n)(v) - v\sep n\in N,v\in V\}.
  \end{equation*}
\end{definition}

\begin{remark}
  $V(N)$ is a $P$-subrepresentation of $V|_P$ since $P$ normalizes $N$.
\end{remark}

\begin{definition}
  \begin{equation*}
    V_N\defeq V/V(N).
  \end{equation*}
\end{definition}

\begin{remark}
  $V_N$ is the maximal quotient of $V$ on which $N$ acts trivially on it and $V_N$ is a representation of $P=MN$.
\end{remark}

\begin{remark}
  We now obtain a map 
  \begin{equation*}
    j_N:\Rep(G)\to\Rep(M),V\stackrel{j_N}{\mapsto} V_N,
  \end{equation*}
  and $j_N$ is a functor. (Exercise).
\end{remark}

\begin{definition}
  Such functor $j_N:\Rep(G)\to\Rep(M)$ is called the (not-normalizes) Jacquet functor. And $M$-representation $V_N$ is called the (not-normalized) Jacquet module of the representation $V$ with respect to $P=MN$.
\end{definition}

\begin{notation}
  We denote this action of $M$ on $V_N$ induced by $(\pi,V)$ by $\pi_N$. Thus we have $(\pi_N,V_N)\in\Rep(M)$.
\end{notation}

\begin{lemma}[Frobenius reciprocity]
  \label{lemma_42}
  Let $(\sigma,W)\in\Rep(M)$, then 
  \begin{equation*}
    \Hom_G(V,\Ind_P^GW) \cong \Hom_M(V_N,W).
  \end{equation*}
\end{lemma}

\begin{proof}
  By Lemma \ref{lemma_41}, we get 
  \begin{equation*}
    \Hom_P(V|_P,W) = \Hom_G(V,\Ind_P^GW).
  \end{equation*}
  Note that $W$ is an representation on $P$ which $P$ acts $P\twoheadrightarrow M$ with kernel $N$. Thus we conclude,
  \begin{equation*}
    \Hom_P(V|_P,W) \cong \Hom_M(V_N,W).
  \end{equation*}
  Since $N$ acts trivially on them, they must factor through $V_N$.
\end{proof}

\subsection{Properties of the Jacquet Functors (Debacker $\S6.3$)}

\begin{lemma}
  \label{lemma_43}
  Let $(\pi,V)$ be a finitely generated $G$-representation then $(\pi_N,V_N)$ is a finitely generated $M$-representation.
\end{lemma}

\begin{proof}
  Let $v_1,\cdots,v_n\in V$ be generators of $V$ (ie. $V= \Span_\C\{\pi(g)(v_i)\sep g\in G,i=1,\cdots,n\}$). By the finitely generatedness, we pick $K\subset G$ a comapct open subgroup such that 
  \begin{equation*}
    \forall i=1,\cdots,n, v_i\in V^K.
  \end{equation*}
  Note that $P\backslash G$ is compact(eg. since $G = P_0K_0$ for $P_0\subseteq P$ and $K_0$ for some compact subgroup by Iwasawa decomposition rr by the definition of parabolic subgroups).\\
  \par Hence we can choose a finite set $\{g_1,\cdots,g_l\}$ of representatives for the $P\backslash G/K$. Then we can write
  \begin{equation*}
    V = \Span_\C\{\pi(p)\pi(g_i)v_j\sep 1\leq i\leq l, 1\leq j\leq l,p\in P = MN\}.
  \end{equation*}
  This implies that 
  \begin{equation*}
    V_N = \Span_\C\{\pi_N(m)\pi_N(n)j_N(\pi(g_i)(v_j))\sep 1\leq i\leq l, 1\leq j\leq n,p\in P = MN\}.
  \end{equation*}
  Since $n$ acts trivially, we get,
  \begin{equation*}
    V_N = \Span_\C\{\pi_N(m)j_N(\pi(g_i)(v_j))\sep 1\leq i\leq l, 1\leq j\leq n,p\in P = MN\},
  \end{equation*}
  where $\pi_N = j_N\circ\pi$. Thus $V_N$ is generated by finitely many elements.
\end{proof}

\begin{lemma}
  \label{lemma_44}
  Let 
  \begin{equation*}
  V(N) = \{v\in V\sep e_{K_N}v = 0\text{ for some compact open subgoup }K_N\subseteq N\}.
  \end{equation*}
\end{lemma}

\begin{proof}
Exercise.
\end{proof}

\begin{proposition}
  \label{proposition_45}
  The functor $j_N:\Rep(G)\ni V\to V_N\in\Rep(M)$ is exact.
\end{proposition}

\begin{proof}
  Consider a short exact sequence
  \begin{center}
    \begin{tikzcd}
0 \arrow[r] & W \arrow[r] & V \arrow[r] & W' \arrow[r] & 0
\end{tikzcd}
  \end{center}
  of smooth representation of $G$. Then 
  \begin{center}
    \begin{tikzcd}
W_N \arrow[r] & V_N \arrow[r] & W'_N \arrow[r] & 0
\end{tikzcd}
  \end{center}
  is exact where $W'_N = W'/W'(N)$. Thus it suffices to show that 
  \begin{equation*}
    W_N\stackrel{f_N}{\to} V_N,
  \end{equation*}
  is also injective.\\
  \par Let $\overline{w}\in\Ker(f_N)$ and $w\in W$ be such that $j_N(w) = \overline{w}$. Since $f_N(\overline{w}) = 0$, by Lemma \ref{lemma_44}, there exists some open compact subgropu $K_N\subseteq N$ such that 
  \begin{equation*}
    f(e_{K_N}w) = e_{K_N}f(w) = 0.
  \end{equation*}
  Since $f$ is injective. $e_{K_N} = 0$, then $w\in W(N)$. Thus $\overline{w}=0$.
\end{proof}

\subsection{Refresher on Parabolic Subgroups of $G$}

Reference is DeBacker $\S4$, Casselman 'Introduction to the theory of admissible representations of $p$-adic groups'\\
\par Recall out notations, 
\begin{enumerate}
  \item $P_0=M_0N_0\subset G$ is the minimal parabolic subgroup. For example in $\GL_n$, $P_0$ can be taken as a subgroup of upper triangular matrices.
  \item $\overline{\Para}_0 = \M_0\overline{\N_0}$ is the opposite parabolic subgroup of $\Para_0$ with respect to $\N_0$ (ie. $\overline{\Para_0}\cap\Para_0 = \M_0$). Forexample $\overline{\Para_0}$ can be taken as a lower triangular matrices when $\Para_0$ is a group of upper triangular matrices.
\end{enumerate}

\begin{notation}
  $A_0$-maximal split torus ($\cong \G_m^{\dim A_0}$ where $\G_m = \GL_1$), in the cneter of $\M_0$, which is amaximal split torus of $\G$. In $\GL_n$, our example is the group of diagonal matrices.
\end{notation}

\begin{definition}
  We call a parabolic subgroup $\Para$, a standard parabolic subgroup if $\Para\supseteq\Para_0$.
\end{definition}

\begin{remark}
  A Zariski closed subgroup of $G$ containing $\Para_0$ is a parabolic subgroup.
\end{remark}

\begin{fact}
  \label{fact_46}
  Every parabolic subgroup is conjugate to a standard parabolic subgroup.
\end{fact}

Recall, $\stackrel{\stackrel{\text{algebraic action}}{\curvearrowright}}{\Lie(\G)}$ where $\G$ acts via the adjoint action (ie. the derivative of conjugation action).
We have 
\begin{equation*}
  \Lie(\G) \cong \bigoplus_{\chi\in\Hom_F(\A_0,\G_m)}\Lie(\G)\chi,
\end{equation*}
where $\Lie(\G)$ is a subspace on which $\A_0\subset\G$ acts on the character $\chi$. We write 
\begin{equation*}
  \Phi(\G,\A_0) = \{\chi\in\Hom_F(\A_0,\G_m)\backslash\{0\}\}.
\end{equation*}
Note that $\Hom_F(\A_0,\G_m)\cong \Z^{\dim \A_0}$, and $A_0 = \A_0(F)\ni a\mapsto 1\in F$, the trivial characters. \\
\par For the set of (relative) roots of $\G$ with respect to $\A_0$. We denote the positive (relative) roots by
\begin{equation*}
  \Phi(\G,\A_0)^+ = \{\alpha\in\Phi(\G,\A_0)\sep\Lie(\G)_\alpha \subset\Lie(\Para_0)\},
\end{equation*}
with basis $\Delta \defeq \Delta(\G,\A_0)\subseteq\Phi(\G,\A_0)^+$.
\begin{remark}[Exercise]
  Make sure to compute some examples of these objects above in the case of $\GL_n$.
\end{remark}
\begin{fact}
  \label{fact_47}
  There is a one-to-one correspondence between the following,
  \begin{equation*}
    \{\text{Standard parabolic subgroups of a reductive group $G$}\}\leftrightarrow\{\text{subsets of $\Delta$}\}
  \end{equation*}
  where $\Para_0\leftrightarrow\emptyset\subset\Delta$ and $\G\leftrightarrow\Delta\subseteq\Delta$. Furthermore,
  \begin{equation*}
    \Para\mapsto\{\alpha\in\Delta\sep\Lie(G)_{-\alpha}\subseteq\Lie(\Para)\}=\Delta\backslash\{\alpha\in\Delta\sep \Lie(\G)_\alpha\subseteq \Lie(\N)\}.
  \end{equation*}
  If we have $\theta\subset\Delta$, then we denote 
  \begin{equation*}
    \M_\theta\N_\theta = \Para_\theta \mapsfrom\theta\subset\Delta,
  \end{equation*}
  given by 
  \begin{equation*}
    \A_0\supset\A_\theta\defeq \left(\bigcap_{\alpha\in\theta}\alpha\right)^\circ,
  \end{equation*}
  where the symbol $(\cdot)^\circ$ denotes the Zariski-connected component. In this case, we have,
  \begin{equation*}
    \M_\theta = Z_G(\A_\theta),\Para_\theta = \M_\theta\Para_0,\N_\theta = \mathcal{R}_u\Para_\theta.
  \end{equation*}
  and in particular, $\A_\theta$ is the center of $\M_\theta$. \\
  \par It is an exercise to work this out for $\GL_3$.
\end{fact}
\begin{proposition}[Jacquet's Lemma]
  \label{proposition_48}
  If $(\pi,V)$ be an admissible representation and $K\subseteq G$ be a compact open subgroup with the property that
  \begin{equation*}
     K = (K\cap N)(K\cap M)(K\cap\overline{N}),
  \end{equation*}
  then 
  \begin{equation*}
    j_N(V^K) = (V_N)^{K\cap M}.
  \end{equation*}
\end{proposition}

\begin{proof}
If $v\in V^K$, and $k\in K\cap M$, then 
\begin{equation*}
  \pi_N(k)j_N(v) = j_N(\pi(k)(v)) = j_N(v),
\end{equation*}
hence the image $j_N(V^K)\subseteq(V_N)^{K\cap M}$.\\
\par To prove the other direction, (ie. $j_N(V^K)\supset(V_N)^{K\cap M}$). We assuem without the loss of generality that $\Para = \Para_\theta$ for some $\theta\subset\Delta$. Let $t\in A_\theta$ such that 
\begin{equation*}
  \forall \alpha\in\Delta\backslash\theta, \val_F(\alpha(t)) >0 .
\end{equation*}
Note that $\forall \alpha\in\theta,\alpha(t) = 1$, so $t\in\A_0^+=\{a\in A_0\sep \val_F(\alpha(a)) \geq 0\forall \alpha\in\Delta\}$.\\
\par Such an element exists for example. Consider $\G=\GL_n$ and 
% \begin{equation*}
% P = \left\{
% \begin{pNiceMatrix}[margin]
% \Block[borders={left,bottom,right,top}]{1-1}{A} & \ast & \ast & \ast & \ast\\
% 0 & \Block[borders={left,bottom,right,top}]{3-3}{B} &&& \ast\\
% 0 & &&&\ast\\
% 0 & &&&\ast\\
% 0 & 0 & 0 & 0& \Block[borders={left,bottom,right,top}]{1-1}{C} 
% \end{pNiceMatrix}
% \:\Bigg{|} A\in\GL_{n_1}(F),B\in\GL_{n_2}(F),C\in\GL_{n_3}(F),n_1+n_2+n_3=n\right\}.
% \end{equation*}
\par Also from the observation we have, $t$ is in the center of $M=M_\theta$ (where $M_\theta$ centralizes $A_\theta$) and the set 
\begin{equation*}
  \{t^{-m}(K\cap\overline{N})t^m,m\in\N\}
\end{equation*}
forms a basis of open neighborhoods of the identity in $\overline{N}$.
\begin{claim}
\begin{equation*}
  \pi_N(t)j_V(V^K) = j_N(V^K).
\end{equation*}
\end{claim}
\begin{claimproof}
  Let $v\in V^K$, since $K\cap\overline{N} \subseteq t(K\cap \overline{N})T^{-1}$. In the previous part, we have denoted 
  \begin{equation*}
    \prescript{t}{}{(K\cap\overline{N})}\defeq t(K\cap\overline{N})t^{-1}.
  \end{equation*}
  We also have $t(K\cap M)t^{-1} =K\cap M$, we then have,
  \begin{equation*}
    e_{K\cap\overline{N}}\underbrace{\pi(t)v}_{\text{fixed under $tKt^{-1}$}} = \pi(t)v = e_{K\cap M}\pi(t)v.
  \end{equation*}
  Hence,
  \begin{align*}
    j_N(\underbrace{e_K}_{\text{projection onto $V^K=((V^{K\cap \overline{N}})^{K\cap M})^{K\cap N}$}}\pi(t)v) &= j_N(e_{K\cap N}e_{K\cap M}e_{K\cap \overline{N}}\pi(t)v),\\
    & = j_N(e_{K\cap N}\pi(t)v),\\
    &=e_{K\cap N}\pi_N(t)j_N(v),\\
    & = \pi_N(t)j_N(v).\\
    \Rightarrow& \pi_N(t)j_N(v)\subseteq j_N(V^K),\\
    \Rightarrow & \pi_N(t)j_N(V^K) \subseteq j_N(V^K).
  \end{align*}
  Since $\pi_N(t)\in\End(V_N)$ is invertible,
  \begin{equation*}
  \dim(\pi_N(t)j_N(V^K)) = \dim (j_N(V^K))\overbrace{<}^{\text{admissibility}}\infty,
  \end{equation*}
  hence $\pi_N(t)j_N(V^K) = j_N(V^K)$.\\
\end{claimproof}
\par Let $\overline{v}\in (V_N)^{K\cap M}$, let $v'\in V$ such that $\overline{v} = j_N(v')$. Set $v = e_{K\cap M}v'$, then $j_N(v)=\overline{v}$ and $v\in V^{K\cap M}$.\\
\par Let $m>0$ be such that $v\in V^{t^{-m}(K\cap \overline{N})t^m}$. In other words, $\pi(t^m)v\in V^{K\cap \overline{N}}$. Since $t^{-m}(K\cap M)t^m = K\cap M$, we have,
\begin{equation*}
  j_N(\underbrace{e_K\pi(t^m)v}_{\in V^K})  = j_N(e_{K\cap N}\pi(t^m)v) = e_{K\cap N}\pi_N(t^m)j_N(v) = \pi_N(t^m)\overline{v}.
\end{equation*}
Thus we obtained that $\overline{v}\in\pi_N(t^{-m})j_N(V^K)=j_N(V^K)$.
\end{proof}
\begin{remark}
  Proposition \ref{proposition_48} holds for all smooth representations. (See "Le centre de Bernstein").
\end{remark}
\begin{corollary}
  \label{corollary_49}
  If $(\pi,V)$ is admissible, then $(\pi_N,V_N)$ is an admissible representation of $M$.
\end{corollary}
\begin{proof}
Since $G$ has a basis of neighborhoods of the identity consisitng of $K$ with $K= (K\cap N)(K\cap M)(K\cap \overline{N})$, this follows from Proposition \ref{proposition_48} immediately.
\end{proof}

\begin{remark}
  Tori do not contain any parabolic subgruops, so do isotropic groups.
\end{remark}

\begin{lemma}
  \label{lemma_50}
  An irreducible representation $(\pi,V)$ of $G$ is supercuspidal if and only if for every proper parabolic subgroup $P=MN\subsetneqq G$ we have, $V_N=\{0\}$. Equivalently,
  if and only if every standard parabolic subgruop $V_N=\{0\}$.
\end{lemma}
\begin{proof}
  If $V_N=\{0\}$, then 
  \begin{equation*}
    \Hom_G(V,\Ind_P^G(V_N))\stackrel{\text{Frobenius Reciprocity}}{\cong} \Hom_M(V_N,V_N),
  \end{equation*}
  contains a non-trivial element $V\to\Ind_P^GV_N$ which is injective as $V$ is irreducible. Thus $(\pi,V)$ is not supercuspidal.\\
  \par If $(\pi,V)$ is not supercuspidal, then there is $P=MN\subsetneqq G$ such that $V\hookrightarrow \Ind_P^G(W)$ for some $W$. Then again using Frobenius Reciprocity,
  \begin{equation*}
    \Hom_M(V_N,W)\cong \Hom_G(V,\Ind_P^G(W)),
  \end{equation*}
  contains a non-trivial element, thus $V_N\not=\{0\}$.\\
  \par if $P$ is not standard, then there exists some element $g\in G$, such that $gPg^{-1}$ is standard. Then by definition, 
  \begin{equation*}
    V(gNg^{-1}) = \pi(g)V(N),
  \end{equation*}
  hence 
  \begin{equation*}
  V_{gNg^{-1}}\stackrel{\text{as $\C$-vectorspaces}}{\cong} V_N.
  \end{equation*}
\end{proof}
\begin{definition}[Equivalent Definition of Supercuspidal Representation]
  We call $(\pi,V)\in\Rep(G)$ supercuspidal if for all proper parabolic subgroups $P=MN\subsetneq G$, we have 
  \begin{equation*}
    V_N = \{0\}.
  \end{equation*}
\end{definition}
\begin{theorem}[Jacuqet, Harish-Chandra, Debacker 75, Renard 5.21]
  \label{theorem_51}
  \par The following are equivalent for a representation $(\pi,V)$.
  \begin{enumerate}[1).]
    \item $(\pi,V)$ is supercuspidal.
    \item For every $v\in V$ $\lambda\in\tilde{V}$, the image of the support of the matrix coefficient $m_{\lambda,v}$ in $G/Z(G)$ is compact. (we say that "the support of $m_{\lambda,v}$ is compact modulo center").
    \item The restriction of $(\pi,V)$ to $G^\circ$ is finite.
  \end{enumerate}
\end{theorem}

%From Nov 3 lecture, cite the statement explicitly,
Recall that for $\Para_0=\M_0\N_0 \subseteq G$ be minimal parabolic, then there exists a maximal compact open subgroup $K_0\subseteq G$ such that some properties hold, including Cartan Decomposition, 
\begin{equation*}
G = \coprod_{m\in M_0^+/M_0^\circ} K_0mK_0.
\end{equation*}
Also recall that for the maximal split torus $\A_0$ in the center of $\M_0$,
\begin{fact}
  We have,
  \begin{equation*}
    G=\coprod_{\substack{w\in\mathcal{W}\\t\in A_0^+/A_0^\circ}}K_0wtK_0=\coprod_{\substack{w\in\mathcal{W}\\t\in A_0^+/A_0^\circ}}K_0twK_0.
  \end{equation*}
  where $\mathcal{W}$ is a finite subset of $M_0^+$ and $A_0^+ = \{t\in A_0\sep \forall \alpha\in\Delta\val_F(\alpha(t))\geq 0\}$, and 
  $A_0^\circ=\{t\in A_0\sep\forall \chi\in \Hom_F(\A_0,\G_m)\val_F(\chi(t))=0\}$.
\end{fact}
\begin{example}
  For $G=\GL_n$, we have,
  \begin{equation*}
    A_0^+ = \left\{\begin{pmatrix}
    t_1 & \: & \:\\
    \: & \ddots & \:\\
    \: & \: & t_n
    \end{pmatrix}\bigg{|} \val_F(t_1)\geq\cdots\geq\val_F(t_n)\right\},
    A_0^\circ = \left\{\begin{pmatrix}
    t_1 & \: & \:\\
    \: & \ddots & \:\\
    \: & \: & t_n
    \end{pmatrix}\bigg{|} \val_F(t_1)=\cdots=\val_F(t_n)=0\right\}
  \end{equation*}
\end{example}
\begin{fact}
  If $t\in A_0$ satisfies $\alpha(t)=1$ for all $\alpha\in \Delta$, then $t\in Z(G)$.
\end{fact}
\begin{proof}$\:$\\
  \par $1)\Rightarrow 2)$. For this direction, we prove the contrapositive. Assume there exits some element $v\in V,\lambda\in\tilde{V}$ such that $m_{\lambda,v}$ is not compactly supported modulo center.
  \par Let $A_{0,\lambda,v}\subset A_0^+$ be the subset such that ,
  \begin{equation*}
    \supp(m_{\lambda,v})\cap K_0twK_0 \not=\emptyset \Leftrightarrow t\in A_{0,\lambda,v},
  \end{equation*}
  for some $w\in\mathcal{W}$. Then $A_{0,\lambda,v}/(Z(G)\cap A_0)A_0^\circ$ is infinite, hence by the fact above (exercise to complete this part), there exists $w\in\mathcal{W}$ such that $\alpha\in\Delta$ and a sequence $\{t_m\}_{n\in\N}$ in $A_{0,\lambda,v}$ such that 
  \begin{equation*}
    \supp(m_{\lambda,v})\cap K_0t_mwK_0\not=\emptyset,
  \end{equation*}
  and 
  \begin{equation*}
    \val_F(\alpha(t_m))\stackrel{m\to\infty}{\to}\infty.
  \end{equation*}
  Consider the standard parabolic subgroup $\M\rtimes \N=\Para = \Para_{\Delta-\{\alpha\}}\subsetneqq \G$. Then with all $\chi$ of the form 
  \begin{equation*}
    \chi = \alpha +\sum_{\alpha_i\in\Delta,c_i\geq 0}c_i\alpha_i,
  \end{equation*}
  \begin{example}
    For $\G=\GL_3$, we have,
    \begin{equation*}
      A_0 = \left\{\begin{pmatrix}
      t_1 & 0 & 0 \\
      0 & t_2 & 0 \\
      0 & 0 & t_3
      \end{pmatrix}\Bigg{|}\alpha\left(\begin{pmatrix}
      t_1 & 0 & 0 \\
      0 & t_2 & 0 \\
      0 & 0 & t_3
      \end{pmatrix}\right)\mapsto t_2t_3^{-1}\right\}.
    \end{equation*}
    Then we have,
    \begin{equation*}
      A_0 = \left\{\begin{pmatrix}
      t_1 & 0 & 0 \\
      0 & t_2 & 0 \\
      0 & 0 & t_3
      \end{pmatrix}\Bigg{|}\alpha\left(\begin{pmatrix}
      t_1 & 0 & 0 \\
      0 & t_2 & 0 \\
      0 & 0 & t_3
      \end{pmatrix}\right)\mapsto t_1t_2^{-1}\right\}.
    \end{equation*}
  We have,
  \begin{equation*}
    P_{\Delta-\{\alpha\}} = \begin{pmatrix}
      \ast & \ast & \ast \\
      \ast & \ast & \ast \\
      0 & 0 & \ast
    \end{pmatrix}.
  \end{equation*}
  To see this $P_{\Delta-\{\alpha\}}$ contains the minimal parabolic subgroup thus all the upper triangular matrices. As the Lie algebra acts on matrices by adjoints, we have,
  \begin{equation*}
    (\Lie\GL_3)_{-\gamma} = \begin{pmatrix}
      0 & 0& 0\\
      \ast & 0 & 0\\
      0 & 0& 0
    \end{pmatrix},
    -\gamma\begin{pmatrix}
      t_1 & \: & \:\\
      \: & t_2 & \:\\
      \: & \: &t_3
    \end{pmatrix}
    \mapsto t_1^{-1}t_2.
  \end{equation*}
  Thus 
  \begin{equation*}
    N= N_{\Delta-\{\alpha\}} = \begin{pmatrix}
      1 & 0 & \ast \\
      0 & 1 & \ast \\
      0 & 0& 1
    \end{pmatrix}.
  \end{equation*}
  Let 
  \begin{equation*}
    \N = \begin{pmatrix}
      0 & 0& \ast\\
      0 & 0& \ast\\
      0 & 0& 0
    \end{pmatrix}
    = \begin{pmatrix}
      0 & 0& \ast\\
      0 & 0& 0\\
      0 & 0& 0
    \end{pmatrix}\oplus
    \begin{pmatrix}
      0 & 0& 0\\
      0 & 0& \ast\\
      0 & 0& 0
    \end{pmatrix}
  \end{equation*}
    \end{example}
    Let $K\triangleleft K_0$ open compact normal subgroup such that 
    \begin{equation*}
      K = (K\cap N)(K\cap M)(K\cap\overline{N}).
    \end{equation*}
    (ie. $K$ has an Iwahori decomposition).\\
    \par Also consider $v\in V^K,\lambda\in\tilde{V}^K$. Then we will show that 
    \begin{equation*}
      \bigcup_{m\in\N} t_m^{-1}(K\cap N)t_m = N,
    \end{equation*}
    because $\val_F(\alpha(t_m^{-1}))\stackrel{m\to\infty}{\to} 0$ or equivalently $\vert \alpha(t_m^{-1})\stackrel{m\to\infty}{\to}\infty$. (Think about them by next time).
\end{proof}
\begin{fact}
  Let $P=MN,P'=M'N'\subseteq G$ be two standard parabolic subgroups such that $P'\subseteq P$ and $M'\subseteq M$. Then $P'\cap M$ is a parabolic subgoupr of $M$ with Levi decomposition,
  \begin{equation*}
    P'\cap M = M'(N'\cap M).
  \end{equation*}
  Moreover, we have,
  \begin{equation*}
    N(M\cap N')(N\cap N') = (M\cap N')N.
  \end{equation*}
  This inequality is due to that $P$ is bigger than $P'$ thus $N\subseteq N'$.
  Moreover, every standard parabolic subgroup of $M$ is of this form that is %later add matrix diagram.
\end{fact}
\begin{corollary}
  \label{corollary_53}
  Let $(\pi,V)$ be an irreducible representation. Then there is a parabolic subgoup $P=MN\subseteq G$ and an irreducible supercuspidal representation $(\sigma,V_\sigma)$ of $M$ such that 
  \begin{equation*}
    \pi\hookrightarrow\Ind_P^G\sigma.
  \end{equation*}
  In other words, the supercuspidal representations are building blocks.
\end{corollary}
  \setcounter{claim}{0}
\begin{proof}
  Without loss of generality, we assume $P=MN\subseteq G$ is a standard parabolic subgroup that is minimal among those for which $V_N\not=\{0\}$.
  \begin{claim}
    $(\pi_N,V_N)$ is supercuspidal. 
  \end{claim}
  \begin{claimproof}
    Suppose it is not supercuspidal. Let $P'_M=M'_MN_M'\subsetneq M$ be a proper standard parabolic subgoup such that $(V_N)_{N_M'}\not=\{0\}$. Let $P'\subseteq P$ be a standard parabolic subgroup of $G$ such that 
    \begin{equation*}
      P'_M=P'\cap M.
    \end{equation*}
    Then 
    \begin{align*}
      \{0\}\not=(V_N)_{N'_M} & = (V_N)_{N'\cap M},\\
      & = V_{N(N'\cap M)},\\
      & = V_{N'} =\{0\},
    \end{align*}
    by the minimality of $P$. Thus $(\pi_N,V_N)$ is supsercuspidal. %later explain the argument by yourself
  \end{claimproof}
  Using Lemma \ref{lemma_43}, as $(\pi,V)$ is finitely generated so is $(\pi_N,V_N)$. Let $\{v_1,\cdots,v_n\}$ be the generators of $(\pi_N,V_N)$. Let $\mathscr{I}$ be the set of proper $M$ subrepresentations of $V_N$. Then $\{0\}\in\mathscr{I}$. Consider $(W_i)_{i\in \N}=\mathscr{W}\subseteq\mathscr{I}$ which is an ascending sequence of proper $M$-subrepresentations. Thus there is $v_i$ such that for any $j\in I$, $v_i\not\in W_j$.
  Hence $\cup_{j\in I}W_j\in\mathscr{I}$. This doesn't contain $v_i$ and thus an upper bound for this chain. By Zorn's lemma, there is a maximal element $W\in\mathscr{I}$. Note that $V_\sigma\defeq V_N/W$ is irreducible representation of $M$. 
  \par $V_\sigma$ is supercuspidal. Suppose it is not, then there is $P_M'=M_M'N_M'\subseteq M$ such that 
  \begin{equation*}
    j_{N_M'}(V_\sigma)=(V_\sigma)_{N'_M}\not=\{0\}.
  \end{equation*}
  As Jacquet functors are exact, we have,
  \begin{center}
    \begin{tikzcd}
0 \arrow[r] & W \arrow[r] \arrow[d] & V_N \arrow[r] \arrow[d] & V_\sigma \arrow[r] \arrow[d] & 0 \\
0 \arrow[r] & W_{N_M'} \arrow[r]    & (V_N)_{N_M'} \arrow[r]  & (V_\sigma)_{N_M'} \arrow[r]  & 0
\end{tikzcd}
  \end{center}
  As $(V_\sigma)_{N'_M}\not=\{0\}$, so is $(V_N)_{N'_M}\not=\{0\}$. This is a contradiction. Using Lemma \ref{lemma_41}, we have,
  \begin{equation*}
    \Hom_G(V,\Ind_P^G V_\sigma)=\Hom_M(V_N,V_\sigma)\not=\{0\}.
  \end{equation*}
  Since $V$ is irreducible, we obtain an injection $V\hookrightarrow\Ind_P^G V_\sigma$.
\end{proof}
\begin{lemma}
  Let $N$ be a subgroup of a closed group $H$ and $(\pi,W)$ be an representation of $N$. If $W$ is admissible and $N\backslash H$ is compact then $\cind_N^H\pi$ is also admissible.
  \label{lemma_hoemework_4_6_b}
\end{lemma}
\begin{corollary}
  \label{corollary_54}
  Irreducible representations are admissible.
\end{corollary}
\begin{proof}
  Let $(\pi,V)$ be an irreducible representation. By Corollary \ref{corollary_53}, there is an irreducible supercuspidal representation $(\sigma,V_\sigma)$ such that $V\hookrightarrow\Ind_P^G V_\sigma$ exists. Therefore, for any compact open subgroup $K\subseteq G$,
  \begin{equation*}
    \dim V^K \leq \dim(\Ind_P^GV_\sigma)^K.
  \end{equation*}
  Therefore, it suffices to show that $\Ind_P^G\sigma$ is admissible. We can now assume $(\pi,V)$ to be supercuspidal as well. Recall that $P\backslash G$ is compact, thus by Lemma \ref{lemma_hoemework_4_6_b}, it suffices to show $V_\sigma$ is admissible. 
  Since $V$ is irreducible and by Lemma \ref{lemma_3}, $Z(G)$ acts on $V$ via a central character. For $v\in V\backslash\{0\}$, set,
  \begin{equation*}
    \{\pi(g)v\sep [g]\in G/(Z(G)G^\circ)\}.
  \end{equation*}
  Then this is a set of generators for the restriction $V|_{G^\circ}$ of $V$ to $G^\circ$. Hence $V|_{G^\circ}$ is finitely generated and by Lemma \ref{theorem_51} this is finite. Therefore by Lemma \ref{lemma_9}, $V|_{G^\circ}$ is admissible. Since all compact open subgroups are contained in $G^\circ$, we have 
  \begin{equation*}
    \dim V^K=\dim (V|_{G^\circ})^K<\infty.
  \end{equation*}
  This complete the proof.
\end{proof}
We will now follow $\S1.4$ of \cite{Roche}.
\begin{notation}
  \begin{enumerate}[1).]
    \item $\Rep^{\supcus}(G)$ is the full subcategory of $\Rep(G)$ whose objects are supercuspidal representations of $G$.
    \item $\Rep^{\nsc}(G)$ is the full subcategory of $\Rep(G)$ whose objects are those representations all of whose subquotients are not supercuspidal.
  \end{enumerate}
\end{notation}
\begin{corollary}
  \label{corollary_55}
  \begin{equation*}
    \Rep(G)\cong\Rep^{\supcus}(G)\times\Rep^{\nsc}(G).
  \end{equation*}
\end{corollary}
\begin{proof}
  By Corollary \ref{corollary_40}, we have,
  \begin{equation*}
    \Rep(G^\circ)\cong\Rep^f(G^\circ)\times\Rep^{\nf}(G).
  \end{equation*}
  Let $(\pi,V)\in\Rep(G)$ and consider,
  \begin{equation*}
    V|_{G^\circ} = \underbrace{V^f}_{\in \Rep^f(G^\circ)}\oplus \underbrace{V^{\nf}}_{\in \Rep^{\nf}(G^\circ)}.
  \end{equation*}
  Let $g\in G$ then $\pi(g)V^f\subseteq V|_{G^\circ}$ is a $G^\circ$-representation sicne $G^\circ\triangleleft G$ and it is finite as the support of the matric coefficient of $\pi(g)V^f$ is the conjugate of $G^\circ$.
  By the uniqueness of the decomposition, $\pi(g)V^f\subseteq V^f$. Now consider $\pi(g)V^{\nf}$. Suppose we have a decomposition,
  \begin{equation*}
    \pi(g)V^{\nf}=W^f\oplus W^{\nf}.
  \end{equation*}
  Then 
  \begin{equation*}
    V^{\nf}=\pi(g^{-1})(W^f\oplus W^{\nf}) = \underbrace{\pi(g^{-1})W^f}_{\in\Rep^{f}(G^\circ)}\oplus\pi(g^{-1})W^{\nf}.
  \end{equation*}
  Therefore, $W^f=0$. We conclude $\pi(g)V^{\nf}\subseteq V^{\nf}$. Hence $V=V^{f}\oplus V^{\nf}$ as a $G$-representation. By Theorem \ref{theorem_51}, we have $V^f\in\Rep^{\supcus}(G)$ and $V^{\nf}\in\Rep^{\nsc}(G)$. %otherwise V^{\nf} contsin a finite summand?
  Moreover 
  \begin{equation*}
    \Hom_G(V^f,V^{\nf})\subseteq\Hom_{G^\circ}(V^f|_{G^\circ},V^{\nf}|_{G^\circ}) = \{0\}.
  \end{equation*}
\end{proof}
\section{Decomposing Supercuspidals}
\subsection{Unramified Twists}
\begin{definition}
  A character $\chi:G\to\C^\times$ is said to be unramified if it is trivial on $G^\circ$. 
  \par The set of all unramified characters of $G$ is dentoed as $X_{\nr}(G)$.
\end{definition}
\begin{definition}
  Let $(\pi,V)\in\Rep(G)$ and $\chi\in X_{\nr}(G)$. The unramified twist of $\pi$ by $\chi$ is the representation $\pi\chi$ such that 
  \begin{equation*}
    \pi\chi = [G\ni g\mapsto \pi(g)\chi(g)\in \End_\C(V)].
  \end{equation*}
\end{definition}
\begin{lemma}
  \label{lemma_56}
  Let $H_2\subseteq H_1\subseteq G$ be open subgroups with $\vert H_1/H_2\vert<\infty$. Then $(\pi,V)\in\Rep(H_1)$ is semisimple if and only if it is semisimple as $H_2$-representation.
\end{lemma}
\begin{proof}$\:$
  \par \textbf{$(\Leftarrow)$} Suppose $(\pi,V)$ is $H_2$-semisimple. Let $U\subseteq V$ be a $H_1$-subspace then there is $W\subseteq U$ such that $W$ is a $H_2$-subspace such that $V=U\oplus W$. Let $f:V\to U$ be $f(u+w) = u$. We define,
  \begin{equation*}
    f^{H_1}:V\to U, v\mapsto{\frac 1{\vert H_1/H_2\vert}}\sum_{g\in H_1/H_2}\pi(g)f(\pi(g^{-1})v).
  \end{equation*}
  Then $f^{H_1}\in\Hom_{H_1}(V,U)$. We have $V=U\oplus\ker f^{H_1}$. Thus $V$ is $H_1$-semisimple.
  \par \textbf{$(\Rightarrow)$}Suppose $V$ is $H_1$-semisimple. Then $V=\bigoplus_{i\in I}V_i$ for some irreducible $H_1$-subspaces $\{H_i\}_{i\in I}$. Hence, it suffices to consider irreducible $V$. 
  Then $V$ is finitely generated as a $H_1$-representation with $\vert H_1/H_2\vert$ finite. $V$ is thus finitely generated as a $H_2$-representation. Hence it admits an irreducible $H_2$ quotient $U$.
  \par Suppose $H_2$ is normal in $H_1$. By Lemma \ref{lemma_regular_frobenius},
\begin{equation*}
  \{0\}\not=\Hom_{H_2}(V,U)\cong\Hom_{H_1}(V,\Ind_{H_2}^{H_1}U).
\end{equation*}
Since $V$ is irreducible, there is an injection $V\hookrightarrow\Ind_{H_2}^{H_1}U$. As $U$ is irreducible, $\Ind_{H_2}^{H_1}U$ is a direct sum of $H_1$-conjugates of $U$, hence it is also $H_2$-semisimple. %later
\par If $H_2$ is normal in $H_1$, consider,
\begin{equation*}
  H_3\bigcap_{g\in H_1/H_2}gH_2g^{-1}.
\end{equation*}
Then $H_3\triangleleft H_1$ with finite index. Hence by the previous argument, $V$ is $H_3$-semisimple. By the argument of the other direction, we see it is $H_2$-semisimple.
\end{proof}
\begin{lemma}
  \label{lemma_57}
 The following statements are equivalent for two irreducible representations $(\pi,V),(\pi',V')$ of $G$.
 \begin{enumerate}[1).]
  \item There is $\chi\in X_{\nr}(G)$ such that $\pi\simeq \pi'\chi$.
  \item $\pi|_{G^\circ}\simeq\pi'|_{G^\circ}$.
  \item $\Hom_{G^\circ}(V|_{G^\circ},V'|_{G^\circ})\not=\{0\}$.
 \end{enumerate}
\end{lemma}

\begin{proof}\textbf{1)$\Rightarrow$2)} is trivial as $\chi$ is unramified. Also for \textbf{2)$\Rightarrow$3)}, we have,
  \begin{equation*}
    \varphi:\pi|_{\Gc}\stackrel{\sim}{\to}\pi|_\Gc\in\Hom_\Gc(V|_\Gc,V'|_\Gc).
  \end{equation*}
  For \textbf{3)$\Rightarrow$1)}, recall that $Z(G)G^\circ$ has a finite index in $G$. Thus together with $(\pi,V)$ is irreducible, we can use Lemma \ref{lemma_56} and have,
  \begin{equation*}
    V|_{Z(G)G^\circ}=V_1\oplus\cdots\oplus V_n
  \end{equation*}
  for irreducible $Z(G)G^\circ$-representations $V_1,\cdots,V_n$. Since $Z(G)$ acts on $V_i$ via central characters, $V_i$ is an irreducible $G^\circ$-representation. Consider,
  \begin{equation*}
    W\defeq\Hom_{G^\circ}(W'|_{\Gc},V|_{\Gc}),
  \end{equation*}
  which is a non-trivial finite dimensional complex vector space by the assumption and Lemma \ref{schur_lemma}. We can further consider a $G/\Gc$-representation of $W$ such that for $f\in W$ and $g\in G/\Gc$, set,
  \begin{equation*}
    g\cdot f = \pi(g)\circ f\circ \pi'(g)^{-1}.
  \end{equation*}
  Since $\dim W<\infty$, it admits an irreducible $G/\Gc$-subrepresentation $W'$. As $G/G^\circ$ is abelian we conclude $\dim W'=1$. (Using Schur's lemma, each element acts by a scalar).
  \par Observe by definition, we have,
  \begin{equation*}
    \Hom(G/G^\circ,\C^\times)\cong X_{\nr}(G).
  \end{equation*}
  Let $f\in W'\backslash\{0\}$, and $\chi\in\Hom(G/G^\circ,\C^\times)$ be such that 
  \begin{equation*}
    \forall g\in G/G^\circ, g\cdot f = \chi(g) f.
  \end{equation*}
  By assumption, we have,
  \begin{equation*}
    \pi(g)\circ f\circ\pi'(g^{-1}) = \chi(g)f \Rightarrow \pi(g)\circ f=. f\circ (\pi'(g)\chi(g)).
  \end{equation*}
  Thus $f\in \Hom_G(\pi'\chi,\pi)$. As $\pi,\pi'$ are irreducible thus using Lemma \ref{schur_lemma}, we conclude $\pi'\chi\cong\pi$. 
\end{proof}
\begin{notation}
  We will denote a category of irreducible supercuspidal representations of $G$ as,
  \begin{equation*}
    \Irr_{\supcus}(G).
  \end{equation*}
\end{notation}
\begin{definition}
  Let $(\pi,V)\in\Irr_{\supcus}(G)$, we fefine, 
  \begin{equation*}
    \Rep^{[\pi]}(G) = \Rep(G)^{[G,\pi]}
  \end{equation*}
  to be the full subcategory of $\Rep(G)$ whose objects are those representations all of whose irreducible subquotients are isomorphic to unramified twists of $\pi$. 
\end{definition}\begin{notation}
  Let $\pi_1,\pi_2\in\Irr_{\supcus}(G)$. We write $\pi_1\sim\pi_2$ or say they are innertially equivalent (generalized notion will be introduced later) if there is $\chi\in X_{\nr}(G)$ such that 
  \begin{equation*}
    \pi_1\cong\pi_2\chi.
  \end{equation*}
  We denote the whole innertial equivalence class of $\pi_1$ as $[\pi_1]$. The collection of all such innertial equivalence classes is denoted by 
  \begin{equation*}
    \mathcal{I}^{\supcus}(G)\defeq \Irr_{\supcus}(G)/\sim.
  \end{equation*}
\end{notation}
\begin{corollary}
  \label{corollary_58}
  \begin{equation*}
    \Rep^{\supcus}(G)\simeq\prod_{[\pi]\in\mathcal{I}^{\supcus}(G)}\Rep^{[\pi]}(G).
  \end{equation*}
\end{corollary}
\begin{proof}
  Recall from Corollary \ref{corollary_30} that 
  \begin{equation*}
    \Rep^f(G^\circ) \simeq \prod_{\pi^\circ\in\Irr_f(G^\circ)}\Rep^{\pi^\circ}(G^\circ).
  \end{equation*}
  Let $[\pi]\in\mathcal{I}^{\supcus}(G)$. Then $\pi|_{G^\circ}$ is a direct sum of copies of $\pi_1^\circ,\cdots,\pi_n^\circ$. If $\tau\in\Rep^{[\pi]}(G)$ then by definition all irreducible subquotients of $\tau$ are isomorphic to $\pi\chi$ for some $\chi\in X_{\nr}(G)$.
  Using Lemma \ref{lemma_57}, this is equivalent to say that for any irreducible subquotient $\tau'$ of $\tau$, $\tau|_{G^\circ}$ is a direct sum of copies of $\pi_1^\circ,\cdots,\pi_n^\circ$. Thus we conclude $\tau|_{G^\circ}$ is a direct sum of copies of $\pi_1^circ|_{G^\circ},\cdots,\pi_n^circ|_{G^\circ}$. 
  \par From the observation above we conclude,
  \begin{equation*}
  \tau|_{G^\circ}\in\Rep^{\pi_1^\circ}(G^\circ)\times\cdots\times\Rep^{\pi_n^\circ}(G^\circ).
  \end{equation*}
  The right hand side is a direct summand of $\Rep^f(G^\circ)$. Thus $\Rep^[\pi](G)$ is a direct summand of $\Rep^{\supcus}(G)$.
\end{proof}
\begin{remark}
  $\Rep^{[\pi]}(G)$ is not necessarily semisimple. For example, we can show that $\G=\GL_1$ and $\pi=\triv$. We have $\Rep^{[\triv]}(F^\times)$ is not semisimple. Details are left to the readers to show but it can also be found in \cite{Renard} VI 3.6.
\end{remark}
\begin{lemma}
  Let $(\tau,V_\tau)\in\Rep^{[\tau]}(G)$ be admissible and $(\pi,V)$ be a subquotient of it. Then $(\pi,V)$ is a subrepresentation and a quotient of $(\tau,V_\tau)$. In other words, we have 
  \begin{equation*}
    \Hom_G(\pi,\tau)\not=\{0\}\text{ and }\Hom_G(\tau,\pi)\not=\{0\}.
  \end{equation*}
\end{lemma}
\begin{proof}
  Let $K\subseteq G$ be a compact open subgroup such that $V^K\not=\{0\}$. Since $K\subseteq G^\circ$, and every irreducible subquotient of $V_\tau$ contains an unramified twist of a representation of $\pi$, we we can see that 
  \begin{equation*}
    V_\tau|_{G^\circ} = \underbrace{V|_{G^\circ}\oplus\cdots\oplus V|_{G^\circ}}_{\text{finitely many summand}}.
  \end{equation*}
  Moreover, each $V|_{G^\circ}$ is a finite direct sum of irreducible $G^\circ$-representation. Since $G^\circ Z(G)$ is of finite index in $G$, hence 
  \begin{equation*}
    0<\dim\Hom_{G^\circ}(\pi,\tau)<\infty.
  \end{equation*}
  As in the proof of Lemma \ref{lemma_57}, $G/G^\circ$ acts on $f\in\Hom_{G^\circ}(\pi,\tau)$ via 
  \begin{equation*}
    g\cdot f = \tau(g)\circ f\circ \pi(g)^{-1}.
  \end{equation*}
  By our assumption, there exists $V_1\subseteq V_2\subseteq V_\tau$ such that $V_2/V_1\cong V$ as a $G$-representation. Since $V_2|_{G^\circ}$ is semisimple ,we have the following exact sequence,
  \begin{center}
    \begin{tikzcd}
0 \arrow[r] & {\Hom_{G^\circ}(V|_{G^\circ},V_1|_{G^\circ})} \arrow[r] & {\Hom_{G^\circ}(V|_{G^\circ},V_2|_{G^\circ})} \arrow[r] & {\Hom_{G^\circ}(V|_{G^\circ},V_2/V_1|_{G^\circ})} \arrow[r] & 0
\end{tikzcd}
  \end{center}
  We also have an injection,
  \begin{center}
    \begin{tikzcd}
{\Hom_{G^\circ}(V|_{G^\circ},V_2|_{G^\circ})} \arrow[r, hook] & {\Hom_{G^\circ}(V|_{G^\circ},V_\tau|_{G^\circ})}
\end{tikzcd}
  \end{center}
  Both sequences are equipped by $G/G^\circ$-actions. We also know that 
  \begin{equation*}
    (\Hom_{G^\circ}(V,V_2/V_1))^G = \Hom_G(V,V_2/V_1)\not\cong\{0\}.
  \end{equation*}
  Set 
  \begin{equation}
    S = \Hom_{G^\circ}(V|_{G^\circ},V_\tau|_{G^\circ}),
  \end{equation}
  we see $S$ has a non-trivial subquotient as a $G/G^\circ$-representation where the action is defined trivially. As $\Hom_{G^\circ}(V,V_2/V_1)$ is a quotient of 
\end{proof}
  %11/26
  \begin{theorem}
    The functors,
    \begin{equation*}
      F_\Pi:\Rep^{[\pi]}(G)\ni (\tau,V_\tau)\stackrel{F_\Pi}{\mapsto}\Hom_G(\Pi,\tau)\in\Mod-\End(\Pi),
    \end{equation*}
    and 
    \begin{equation*}
      F_\Pi:\Rep^{[\pi]}(G)\ni M\tens{\End_G(\Pi)}\stackrel{G_\Pi}{\mapsfrom} M\in\Mod-\End(\Pi),
    \end{equation*}
    are quasi-inverse of equivalence of categories.
    \label{theorem_66}
  \end{theorem}

  \begin{proof}
    %Refer to the note from 11/24
    Let $F = G_\Pi\circ F_\Pi$. Then we see,
    \begin{equation*}
      \ev_{V_\tau}:(\tau,V_\tau)
    \end{equation*}
    \begin{claim}
      $\ev:F\to\id_{\Rep^[\pi](G)}$ is an isomorphism.
    \end{claim}
    \begin{proof}
      We have 
      
    \end{proof}
  \end{proof}

  %12/1
\begin{lemma}
  \label{lemma_71}
  \begin{equation*}
    \supp(\varphi_i\ast\varphi_j) = x_ix_jG^\circ.
  \end{equation*}
\end{lemma}

\begin{proof}
  By definition, the convolution evaluated at $x$ is ,
  \begin{equation*}
    (\varphi_i\ast\varphi_j)(x) = \sum_{g\in G/G^\circ}\varphi_i(g)\varphi_j(g^{-1}x) = \varphi_i(x_i)\varphi_j(x_i^{-1}x).
  \end{equation*}
  The last equality is that $\varphi_i$ is supported on $x_iG^\circ$, thus the only element which is mapped to non-zero element is $x_iG^\circ$. Similarly, $\varphi_j(x_i^{-1}x)\not=0$ if and only if $x_i^{-1}x\in x_jG^\circ$. Thus $x \in x_ix_jG^\circ$. 
\end{proof}
\begin{corollary}
  \label{corollary_72}
  \begin{equation*}
    \varphi_i\ast\varphi_j = c_{ij}\varphi_j\ast\varphi_i,
  \end{equation*}
  where $c_{ij}\in\C^\times$. 
\end{corollary}
\begin{proof}
  We have seen,
  \begin{equation*}
    \dim\hecke(x_ix_jG^\circ,\pi^\circ) = 1.
  \end{equation*}
  Since $x_iG^\circ,x_jG^\circ$ commute we have $\varphi_i\ast\varphi_j,\varphi_j\ast\varphi_i\in\hecke(x_ix_jG^\circ,\pi^\circ)$.
\end{proof}
\begin{corollary}
  \label{corollary_73}
  $\hecke(G,\pi^\circ)$ has a $\C$-vectorspace basis, given by,
  \begin{equation*}
    \{\phi_1^{e_1}\ast\cdots\ast\phi_n^{e_n}\sep e_1,\cdots,e_n\in\Z\}.
  \end{equation*}
  Furthermore, 
  \begin{equation*}
  \hecke(G,\pi^\circ)^\times  = \{c\phi_1^{e_1}\ast\cdots\ast\phi_n^{e_n}\sep c\in\C^\times,e_1,\cdots,e_n\in\Z\}. 
  \end{equation*}
\end{corollary}

\begin{remark}
  Make sure to calculate, 
  \begin{equation*}
  \hecke(G,\pi^\circ)^\times  = \{c\phi_1^{e_1}\ast\cdots\ast\phi_n^{e_n}\sep c\in\C^\times,e_1,\cdots,e_n\in\Z\}. 
  \end{equation*}
  indeed holds. Note that 
  \begin{equation*}
    1 = \phi_1^0\ast\cdots\ast\phi_n^0.
  \end{equation*}
  It is also an exercise that
  \begin{equation*}
    \sum c_i\phi_1^{e_{1,i}}\ast\cdots\ast\phi_n^{e_{n,i}}
  \end{equation*}
  is invertible if and only if all $c_i$ are zero except for a single $c_i$.
\end{remark}

\begin{notation}
  We write,
  \begin{equation*}
    \hecke(G,\pi^\circ)\to\prod_{\substack{M\in\Mod-\hecke(G,\pi^\circ)/\simeq\\ \text{$M$ is simple}}}\End_\C M, \phi\mapsto(\phi_M),
  \end{equation*}
  for the product of the structure map.
\end{notation}

\begin{definition}
  The radical $\rad\hecke(G,\pi^\circ)$ is the kernel of the map $\phi\mapsto(\phi_M)$. 
\end{definition}

\begin{lemma}
  For any $\phi\in\rad\hecke(G,\pi^\circ)$. Denote 
  \begin{equation*}
    1\defeq\left[G\ni g\mapsto\begin{cases}
      \pi^{-1}(g)\quad g\in G^\circ,\\
      0\quad \text{otherwise}.
    \end{cases}
    \right]
  \end{equation*}
  We have,
  \begin{equation*}
    1+\phi\in\hecke(G,\pi^\circ)^\times.
  \end{equation*}
  Then the map,
  \begin{equation*}
    \hecke(G,\pi^\circ)\ni\psi\mapsto(1+\phi)\ast\psi\in\hecke(G,\pi^\circ)\in\hecke(G,\pi^\circ),
  \end{equation*}
  is surjective.
  \label{lemma_74_claim_1}
\end{lemma}
\begin{proof}
  Suppose not, hence,
  \begin{equation*}
    (1+\phi)\ast\hecke(G,\pi^\circ)\backslash\hecke(G,\pi^\circ),
  \end{equation*}
  is not trivial and we let $M$ be a simple quotient of this quotient, where the above expression denotes a right $\hecke(G,\pi^\circ)$-module.
  That is $M = I\backslash\hecke(G,\pi^\circ)$ for some $I\supseteq (1+\phi)\ast\hecke(G,\pi^\circ)$.
  \par If $\phi\in I$, then $1=(1+\phi)-\phi\in I$, thence $I = \hecke(G,\pi^\circ)$. Thus this contradicts that $M\not=0$. Hence $\phi$ acts non-trivially on $M$. In other words, $\phi_M\not=0$. This contradicts
  that $\phi\in\rad\hecke(G,\pi^\circ)$ and the structure of $\hecke(G,\pi^\circ)$.
\end{proof}

\begin{lemma}
  \label{lemma_74_claim_2}
  Then
\end{lemma}

\begin{lemma}
  \label{lemma_74}
  The map $\phi\mapsto(\phi_M)$ is injective, or equivalently $\rad\hecke(G,\pi^\circ)=\{0\}$. 
\end{lemma}
\begin{proof}
  Let $\phi\in\rad\hecke(G,\pi^\circ)$, then $\phi^2\in\rad\hecke(G,\pi^\circ)$, hence $1+\phi^2$ is also invertible. Let $x,y\in G$ be such that 
  \begin{equation*}
    \supp(1+\phi)=xG^\circ,\supp(1+\phi^2)=yG^\circ.
  \end{equation*}
  Therefore,
  \begin{align*}
    \underbrace{(1+\phi^2)-1}_{\text{the support is $yG^\circ\cup G^\circ$.}} = \phi^2 = (1+\phi-1)^2 = \underbrace{(1+\phi)^2-2(1+\phi)+1}_{\text{the support is $x^2G^\circ\cup xG^\circ\cup G^\circ$.}}.
  \end{align*}
  It follows that $x\in G^\circ$ thus $1+\phi = c\cdot 1$ for some $c\in\C^\times$. That means $\phi = (c-1)1$. In other words, $\phi_M$ is a multiplication by $c-1$, for all simple module $M$. 
  Finally, we see,
  \begin{equation*}
    c-1=0\Rightarrow \phi=0.
  \end{equation*}
\end{proof}

\begin{proposition}
  \label{proposition_75}
  $\hecke(G,\pi^\circ)$ is commutative if and only if $\pi\vert_{G^\circ}$ is multiplicity free(ie. $\pi|_{G_0}=\pi_1^\circ\oplus\cdots\oplus\pi_{n'}^\circ$ with $\pi^\circ_i\not\simeq\pi_j^\circ$ for $i\not=j$).
\end{proposition}
\begin{proof}
  Let $m$ be the multiplicity of of $\pi^\circ$ in $\pi|_{G^\circ}$. (ie. $\vert\{i\sep \pi_i^\circ\simeq\pi^\circ\}\vert$).
  Then the multiplicity of $\pi_i^\circ$ in $\pi|_{G^\circ}$ is also $m$ for all $i$(recall that $\pi^\circ=\prescript{g}{}{\pi}^\circ, \pi\simeq\prescript{g}{}{\pi}$). 
  Since the irreducible representations in $\Rep^[\pi](G)$ are of the form $\pi\chi$ for some unramified character $\phi\in X_{\text{nr}}(G)$. Therefore, the simple $\hecke(G,\pi^\circ)$-moduels are of the form $\Hom_{G^\circ}(\pi^\circ,\pi\chi|_{G^\circ})$ which is by Corollary \ref{corollary_69}.
  Furthermore, we have,
  \begin{equation*}
    \dim_\C\Hom_{G^\circ}(\pi^\circ,\pi\chi|_{G^\circ}) = \dim_\C\Hom_{G^\circ}(\pi^\circ,\pi) = m.
  \end{equation*}
  If $m=1$,then 
  \begin{equation*}
    \hecke(G,\pi^\circ)\stackrel{\text{Lemma \ref{lemma_74}}}{\hookrightarrow}\prod_{\substack{M\in\hecke(G,\pi^\circ)-\Mod/\simeq\\\text{$M$ is simple.}}}\End_\C(M) \stackrel{\text{$M$ is simple}}{=} \prod\C.
  \end{equation*}
  Note that for the last equality, we have $M=\Hom_{G^\circ}(\pi^\circ,\pi\chi_{G^\circ})$. And we know this is $1$-dimensional(Need to cite the proposition from that this follows).
  If $\hecke(G,\pi^\circ)$ is commutative, then all its simple modules have dimension $1$(since $\hecke(G,\pi^\circ)$ is finitely generated), hence $m=1$.
\end{proof}
\begin{definition}
  Let $\cat$ be an abelian category with a small skelton (ie. the equivalence classes of objects who are isomorphic to one another in the same class forms a set.)
  Then we define the center $\mathcal{Z}(\cat)$ of the category $\cat$ is $\End_\cat(id_\cat)$, that is the set of natural transformations of the identity functor which has a ring structure arising from the addition structure of homomorphisms together with compositions.
\end{definition}
\begin{remark}
  Making the above notion more explicit, each element $z\in\mathcal{Z}(\cat)$ is a collection, 
  \begin{equation*}
    \{z_A:A\to A\}_{A\in\cat}\st \forall f\in\Hom(B,C),
  \end{equation*}
  the diagram, 
  \begin{center}
    \begin{tikzcd}
B \arrow[d, "z_B"'] \arrow[r, "f"] & C \arrow[d, "z_C"] \\
B \arrow[r, "f"']                  & C                 
\end{tikzcd}
  \end{center}
  is commutative. 
\end{remark}
\begin{definition}
  The center $\mathcal{Z}(\Rep(G))$ of $\Rep(G)$ is called the Bernstein center.
\end{definition}
\begin{lemma}
  \label{lemma_76}
  If $R$ is a ring, then 
  \begin{equation*}
  \underline{Z(R)}_{\text{center of the ring $R$}}\ni c\mapsto \mu_c\in\mathcal{Z}(\Mod(R)), 
  \end{equation*}
  with $(\mu_c)_M:M\to M$ such that 
  \begin{equation*}
    \mu(m) = mc,
  \end{equation*}
  is a ring isomorphism.
\end{lemma}
\begin{proof}
  It is an injective ring homomorphism. It remains to prove 
\end{proof}

\begin{lemma}
  $\hecke(G,G^\circ,\pi^\circ)$ has no non-zero zero divisors. 
  \label{lemma_77_1}
\end{lemma}

\begin{proof}
  Homework problem.
\end{proof}
\begin{theorem}
  \label{theorem_77}
  The category $\Rep^{[\pi]}(G)$ is indecomposable.
\end{theorem}
%12/8

\begin{proof}
Suppose $\Rep^[\pi](G) = \cat_1\times\cat_2$, ie it is decomposable. Then for $V\in\Rep^[\pi](G)$, we have,
\begin{equation*}
  V= \underbrace{V_1}_{\in\cat_1}\oplus \underbrace{V_2}_{\in\cat_1}. 
\end{equation*}
and we set $z_{V}:V\mapsto V$ define by $z_{V}(v_1,v_2) = (v_1)$. Then 
\begin{equation*}
  z = \{z_{V}\}\in Z(\Rep^{[\pi]}(G)).
\end{equation*}
Similarly, we have,
\begin{equation*}
  z'=\id-z\in  Z(\Rep^{[\pi]}(G)).
\end{equation*}
Note that we have $zz' = 0$. On the other hand, 
\begin{equation*}
  Z(\Rep^{[\pi]}(G)) \simeq Z(\Mod(\hecke(G,G^\circ,\pi^\circ)))\stackrel{\text{Lemma\ref{lemma_76}}}\simeq Z(\hecke(G,G^\circ,\pi^\circ)).
\end{equation*}
Recall we have shown previously that,
\begin{equation*}
  \Rep(G) \stackrel{\text{Cor. \ref{corrolary_55}}}{=} \Rep^{\supcus}(G)\times\Rep^{\nsc}(G) \stackrel{\text{Cor\ref{corollary_58}}}{=}\left(\prod_{[\pi]\in\Irr^{\supcus}(G)}\Rep^{[\pi]}(G)\right)\times\Rep^{\nsc}(G)
\end{equation*}
\end{proof}

\section{The Bernstein decomposition}

The main reference of this section is Roche's note $\S$ 1.7.\\
\par Obseve that for an irreducible representation $(\pi,V)$ of $G$, there is $P=MN\subseteq G$ a parabolic subgroup such that 
there is a irreducible supercuspidal representation $(\sigma,V_\sigma)$ of $M$ such that
\begin{equation*}
  \pi\hookrightarrow\Ind_P^G(\sigma)\quad(\text{By Cor.\ref{corollary_53}}).
\end{equation*}
The sketch of how we study non-supercuspidal part of $\Rep^{\nsc}(G)$ is that we decompose $\Rep^{\nsc}(G)$ with respect to such $P=MN$ and $\sigma$.
\par To do this, we want to know when two parabolic inductions $\Ind_P^G(\sigma)$ and $\Ind_{P'}^G(\sigma')$ have a common irreducible subquotient. In particular, we want to know when we have,
\begin{equation*}
  0\not=\Hom_G(\Ind_P^G(\sigma),\Ind_P^G(\sigma'))=\Hom_{M'}(j_{N'}(\Ind_P^G(\sigma)), \sigma').
\end{equation*}
We will define later what $j_{N'}$. For now let us see the motivation behind the concept. 
\par For a parabolic subgroup $P=MN$, we normalize $\Ind_{P}^G$ and $j_N$ using a character
\begin{equation*}
  \delta_P:P\to\R^+
\end{equation*}
as follows. For $\sigma\in\Rep(M)$ and $\pi\in\Rep(G)$, we set
\begin{align*}
  i_P^G(\sigma) &= \Ind_P^G(\sigma\otimes\delta_{P}^{{\frac 1 2}}),\\
  \rho_N(\pi)& = j_N(\pi)\otimes \delta_P^{-{\frac 1 2}}
\end{align*}
They satisfy the following properties for $\sigma\in\Rep(M)$.
\begin{enumerate}[1).]
  \item $i_P^G(\sigma^\vee)\simeq i_P^G(\sigma)^\vee$
  \item $i_P^G$ preserves unitary representations.
  \item $\rho_{N'}(i_P^G(\sigma))$ has a better description than $j_{N'}(\Ind_P^G(\sigma))$.
  \item $\Hom_G(i_P^G(\sigma),i_{P'}^G(\sigma'))\not=0$ for any parabolic subgroup $P=MN$ and $P' = MN'$ with Levi factor $M$.
  \item The set of irreducible subquotinets of $i_P^G(\sigma)$ only depends on $M$ not on $P$.
\end{enumerate}
\begin{definition}
  Let $P=MN$ be a parabolic subgroup of $G$. The modulus character is a character $\delta_P:P\to\R^+$ is defined as for a fixed compact open subgroup $K$ of $P$,
  \begin{equation*}
    \delta_P(p) = \left\vert{\frac {pKp^{-1}} {pKp^{-1}\cap K}}\right\vert\left\vert{\frac {K} {pKp^{-1}\cap K}}\right\vert^{-1}.
  \end{equation*}
  Note that we cannot simply defne $\left\vert{\frac {pKp^{-1}} {K}}\right\vert$ as there necessarily doesn't exist an inclusion relation between two.
\end{definition}
In addition to the \cite{Roche}, we also refer to \cite[Vigneras] (I,2). Although we warn the readers that in \cite{Debacker} and \cite{Renard}, they use different conventions, that is they use $\delta_P(g)$ for $\delta_P(g^{-1})$.
\begin{lemma}
  \label{lemma_78}
  Let $P=MN$ be a parabolic subgroup. We have the following statements about $\delta_P$.
  \begin{enumerate}
    \item $\delta_P$ does not depend on the choice of a compact open subgroup $K$.
    \item $\delta_P$ is trivial on a compact open subgroups of $P$.
    \item $\delta_P$ is a character of $P$.
  \end{enumerate}
\end{lemma}
\begin{proof}
  Homework problems.
\end{proof}
\begin{lemma}
  \label{lemma_79}
  Let $\mu_P$ be a left Haar measure on $P$. Then $\delta_P\cdot\mu_P$ which by definition,
  \begin{equation*}
    \int_Pf(g)d(\delta_P\cdot\mu_P)(g) = \int_Pf(g)\delta_P(g)\mu_P(g),
  \end{equation*}
  is a right Haar measure on $P$
\end{lemma}
\begin{proof}
  Homework problems.
\end{proof}
\begin{definition}
  For a parabolic subgroup $P=MN\subseteq G$, $\sigma\in\Rep(M)$ and $\pi\in\Rep(G)$, we define,
  \begin{enumerate}[1).]
    \item The normalized parabolic induction, $i_P^G(\sigma) = \Ind_P^G(\sigma\otimes\delta_P^{{\frac 1 2}})$,
    \item The normalized Jacquet functor, $\rho_N(\pi) = j_N(\pi)\otimes \delta_P^{-{\frac 1 2}}$. 
  \end{enumerate}
\end{definition}
\begin{lemma}[Frobenius Reciprocity]
  \label{lemma_80}
  For $\sigma\in\Rep(M)$ and $\pi\in\Rep(G)$, we have,
  \begin{equation*}
    \Hom_G(\pi,i_P^G(\sigma)) = \Hom_M(\rho_N(\pi),\sigma).
  \end{equation*}
\end{lemma}
\begin{proof}
  Follows from Lemma \ref{lemma_42} and definitions of $i_P^G,\rho_N$.
\end{proof}
We will shortly follow \cite{Renard} VI 5.1.
\begin{lemma}[Mackey's formula(finite group analogue)]
  Let $G$ be a finite group and $P,Q$ be subgroups of $G$. Then for a representation $\tau$ of $P$, we have
  \begin{equation*}
    \Res_Q^G(\Ind_P^G(\tau)) = \bigoplus_{g\in Q\backslash  G/P}\Ind_{Q\cap gPg^{-1}}^Q(\Res_{Q\cap gPg^{-1}}(\prescript{g}{}{\tau})).
  \end{equation*}
  Note that $\tau\in\Rep(gPg^{-1})$.
\end{lemma}
\begin{proof}[Sketch of the proof]
  Consider,
  \begin{equation*}
    \Ind_P^G(\tau) = \{f:G\to\tau\sep f(pg) = \tau(p)(f(g))\} = \bigcup_{g\in Q\backslash G/P}\{f\in\Ind_P^G(\tau)\sep\supp(f)\subseteq Pg^{-1}Q\}.
  \end{equation*}
  Note that we have 
  \begin{equation*}
    G = \bigsqcup_{g\in Q\backslash G/ P}Pg^{-1}Q.
  \end{equation*}
  As $Q$-representation, we have,
  \begin{equation*}
    \{f\in\Ind_P^G(\tau)\sep \supp(f)\subseteq Pg^{-1}Q\} \ni f\stackrel{\sim}{\longmapsto}[q\mapsto f(g^{-1}q)]\in \Ind_{Q\cap gPg^{-1}}^Q(\prescript{g}{}{\tau})
  \end{equation*}
\end{proof}
\begin{remark}
  [The problem for the case of $p$-adic groups.]
  The subset $Pg^{-1}Q$ is not necessarily open in $G$, but for $f\in i_P^G(\tau)$, $\supp(f)$ should be open. 
\end{remark}
\begin{fact}[Renard, V 4.7, VI, 5.1]
  Let $P=LU,Q=MN$ be parabolic subgroups of $G$ then, we can take a set of representatives $\{w_1,\cdots,w_k\}$ for $Q\backslash G/P$, such that 
  for each $i$,
  \begin{equation*}
    Q\cap \prescript{w_i}{}{L} = \underbrace{(M\cap \prescript{w_i}{}{L})}_{\text{Levi factor}}\underbrace{(N\cap \prescript{w_i}{}{L})}_{\text{Unipotent part}},
  \end{equation*}
  is a parabolic subgroup of $\prescript{w_i}{}{L}$ and 
  \begin{equation*}
    M\cap \prescript{w_i}{}{P} = \underbrace{(M\cap \prescript{w_i}{}{L})}_{\text{Levi factor}}\underbrace{(M\cap \prescript{w_i}{}{U})}_{\text{Unipotent part}},
  \end{equation*}
  is a parabolic subgroup of $M$.
  \par The subsets 
  \begin{equation*}
    Pw_1^{-1}Q, Pw_1^{-1}Q\sqcup Pw_2^{-1}Q,\cdots,Pw_1^{-1}Q\sqcup\cdots\sqcup Pw_{k-1}^{-1}Q,Pw_1^{-1}Q\sqcup\cdots\sqcup Pw_{k}^{-1}Q=G
  \end{equation*}
  are all open in $G$.
\end{fact}

Note that for parabolic subgroups $P=LU, Q=MN\subset G$, we have 
\begin{center}
  \begin{tikzcd}
\Rep(L) \arrow[r, "i_P^G"] & \Rep(G) \arrow[r, "\rho_N"] & \Rep(M)
\end{tikzcd}
\end{center}

\begin{theorem}[Geometric lemma, cf \cite{Renard} VI 5.1 or \cite{Roche} section 1.71]
  \label{geometric_lemma}
  Let $P=LU,Q=MN$ be parabolic subgroups of $G$. Let $\tau\in\Rep(L)$. Then the representation
  \begin{equation*}
  (\rho_N\circ i_P^G)(\tau)\in\Rep(M),
  \end{equation*}
  has a filtration,
  \begin{equation*}
    0=\tau_0\subseteq \tau_1\subseteq\cdots\subseteq \tau_k = (\rho_N\circ i_P^G)(\tau),
  \end{equation*}
  such that 
  \begin{equation*}
    \tau_i/\tau_{i-1}\simeq i_{M\cap\prescript{w_i}{}{P}}^M(\rho_{N\cap \prescript{w_i}{}{L}}(\prescript{w_i}{}{\tau})).
  \end{equation*}
  Recall that $\prescript{w_i}{}{\tau}(x) = \tau(w_i^{-1}xw_i)$ and $\prescript{w_i}{}{L} = w_iLw_i^{-1}$. Also under these maps, we have 
  \begin{equation*}
    \Rep(\prescript{w_i}{}{L})\ni\tau\mapsto \underbrace{\rho_{N\cap \prescript{w_i}{}{L}}(\prescript{w_i}{}{\tau})}_{\in\Rep(M\cap \prescript{w_i}{}{L})}\mapsto i_{M\cap\prescript{w_i}{}{P}}^M(\rho_{N\cap \prescript{w_i}{}{L}}(\prescript{w_i}{}{\tau}))\in\Rep(M)
  \end{equation*}
  We shorten the notation by 
  \begin{equation*}
  \gr(\rho_N\circ i_P^G(\tau))\simeq \bigoplus_{i=1}^ki_{M\cap \prescript{w_i}{}{P}}^M(\rho_{N\cap\prescript{w_i}{}{L}}(\prescript{w_i}{}{\tau}))
  \end{equation*}
\end{theorem}
\begin{proof}[Sketch of the proof]
  Define a subrepresentation 
  \begin{equation*}
  \tau_i = \rho_N\left(\left\{f\in\Ind_P^G(\tau)\sep \supp(f)\subseteq\bigsqcup_{j=1}^iPw_j^{-1}Q\right\}\right).
  \end{equation*}
  Then we can prove that 
  \begin{equation*}
    \tau_i/\tau_{i-1}\simeq i_{M\cap\prescript{w_i}{}{P}}^M(\rho_{N\cap\prescript{w_i}{}{L}}(\prescript{w_i}{}{\tau})).
  \end{equation*}
  We construct the above isomorphism by integrating functions in parabolic inductions. At this point modulus characters are involved.
\end{proof}

%12/10

\begin{lemma}
  \label{lemma_82}
  Let $P=LU\subsetneqq G$ be a proper parabolic subgroup. For any $\tau\in\Rep(G)$, we have 
  \begin{equation*}
    i_P^G(\tau)\in\Rep^{\nsc}(G).
  \end{equation*}
\end{lemma}

\begin{proof}
Decompose $i_P^G(\tau)$ as 
\begin{equation*}
  i_P^G(\tau) = i_P^G(\tau)^{\supcus}\otimes i_P^G(\tau)^{\nsc},
\end{equation*}
where $i_P^G(\tau)^{\supcus}\in\Rep^{\supcus}(G)$ and $ i_P^G(\tau)^{\nsc}\in\Rep^{\nsc}(G)$. We will show that $i_P^G(\tau)^{\supcus}$ is trivial. Then we have 
\begin{equation*}
  \rho_U(i_P^G(\tau)^{\supcus}) = j_U(\underbrace{i_P^G(\tau)^{\supcus}\otimes\delta_P^{-{\frac 1 2}}}_{\text{supercuspidal}}) = 0.
\end{equation*}
Hence, 
\begin{equation*}
  0 = \Hom_L(\rho_U(i_P^G(\tau)^{\supcus}), \tau) \stackrel{\text{Frobenius Reciprocity}}{=}\Hom_G(i_P^G(\tau)^{\supcus},i_P^G(\tau)).
\end{equation*}
Therefore, we have $i_P^G(\tau)^{\supcus} = 0$.
\end{proof}

\begin{lemma}
  \label{lemma_83}
  Let $P=LU,Q=MN\subseteq G$ be parabolic subgroups. Take $\tau\in\Irr_{\supcus}(L)$. Decompose 
\begin{equation*}
  \rho_N(i_P^G(\tau)) =\rho_N(i_P^G(\tau))^{\supcus}\oplus \rho_N(i_P^G(\tau))^{\nsc},
\end{equation*}
  where $i_P^G(\tau)^{\supcus}\in\Rep^{\supcus}(G)$ and $ i_P^G(\tau)^{\nsc}\in\Rep^{\nsc}(G)$. Suppose that $\rho_N(i_P^G(\tau))^{\supcus}\not=0$, then $M$ is $G$ conjugate to $L$. Furthermore,
  \begin{equation*}
    \rho_N(i_P^G(\tau)) = \rho_N(i_P^G(\tau))^{\supcus}.
  \end{equation*}
  Moreover, we have, 
  \begin{equation*}
    \gr(\rho_N(i_P^G(\tau)))\cong\bigoplus_{{i\sep \prescript{w_i}{}{L} = M}}\underbrace{\prescript{w_i}{}{\tau}}_{\in\Rep(\prescript{w_i}{}{L}) = \Rep(M)}
  \end{equation*}
\end{lemma}

\begin{proof}
  By Theorem \ref{geometric_lemma}, we have,
  \begin{equation*}
    \gr(\rho_N(i_P^G(\tau))) \cong \bigoplus_{i=1}^k i_{M\cap\prescript{w_i}{}{P}}^M(\rho_{N\cap\prescript{w_i}{}{L}}(\prescript{w_i}{}{\tau})).
  \end{equation*}
  Since $\tau$ is supercuspidal, $\prescript{w_i}{}{\tau}$ is also supercuspidal. Hence, 
  \begin{equation*}
    \rho_{N\cap\prescript{w_i}{}{L}}(\prescript{w_i}{}{\tau}) = 0,
  \end{equation*}
  unless $M\cap \prescript{w_i}{}{L}=\prescript{w_i}{}{L}$. That is $\prescript{w_i}{}{L}\subseteq M$. Therefore,
  \begin{equation*}
    \gr(\rho_N(i_P^G(\tau)))\cong\bigoplus_{\{i\sep \prescript{w_i}{}{L}\subseteq M\}}i_{M\cap \prescript{w_i}{}{P}}^M(\prescript{w_i}{}{\tau}).
  \end{equation*}
  By Lemma \ref{lemma_82}, 
  \begin{equation*}
    i_{M\cap\prescript{w_i}{}{\tau}}^M(\prescript{w_i}{}{\tau}) = 0,
  \end{equation*}
  where $\prescript{w_i}{}{L}=M$. Hence the assumption that $\rho_N(i_P^G(\tau))^{\supcus}\not=0$ implies that there is an element $i$ such that $\prescript{w_i}{}{L}=M$.
  In particular, $L$ and $M$ have the same semisimple rank. Then we have 
  \begin{equation*}
    \gr(\rho_N(i_P^G(\tau))) \cong \bigoplus_{i\sep \prescript{w_i}{}{L}\subseteq M}i_{M\cap\prescript{w_i}{}{P}}^M(\prescript{w_i}{}{\tau}).
  \end{equation*}
  Note that the condition $\prescript{w_i}{}{L}\subseteq M$ is equivalent to $\prescript{w_i}{}{L}= M$. Therefore, we conclude,
  \begin{equation*}
    \gr(\rho_N(i_P^G(\tau))) \cong \bigoplus_{i\sep \prescript{w_i}{}{L}= M}\prescript{w_i}{}{\tau}.
  \end{equation*}
\end{proof}

\begin{proposition}
  \label{proposition_84}
  Let $\pi$ be an irreducible representation of $G$. We have the following statements.
  \begin{enumerate}[1).]
    \item There exists a parabolic subgroup $P=LU$ of $G$ and an irreducible supercuspidal representation $\sigma$ of $L$ such that $\pi$ is isomorphic to an irreducible subquotient of $i_P^G|_\sigma$.
    \item The pair $(L,\sigma)$ in the first assertion is uniquely determined by $\pi$ up to $G$-conjugation.
    \item Let $P=LU$ and $P' = LU'$ be parabolic subgroups with the same Levi factor. Let $\sigma$ be an irreducible supercuspidal representation of $L$. Then we have $\Hom_G(i_P^G(\sigma),i_{P'}^G(\sigma))\not=0$. 
  \end{enumerate}
  In particular, $i_P^G(\sigma)$ and $i_{P'}^G(\sigma)$ share an irreducible subquotient.
\end{proposition}

\begin{proof}
  According to Corollary \ref{corollary_53}, there exists a parabolic subgroup $P=LU$ and $\sigma'\in\Irr_{\supcus}(L)$ such that $\pi\hookrightarrow\Ind_P^G(\sigma') = i_P^G(\underbrace{\sigma'\otimes\delta_P^{-{\frac 1 2}}}_{\defqe \sigma})$. This proves the first assertion.
  \par For the second assertion, suppose that $\pi$ is also a subquotient of $i_Q^G(\tau)$ for some $Q=MN\subseteq G$ a parabolic subgroup and $\tau\in\Irr_{\supcus}(M)$. By the exactness of Jacquet functor which is proven in Proposition \ref{proposition_45}, 
  $\rho_U(\pi)$ is a subquotient of $\rho_U(i_Q^G(\tau))$. Since $\Hom_L(\rho_U(\pi),\sigma) = \Hom_G(\pi,i_P^G(\sigma))\not=0$, $\sigma$ is a subquotient of $\rho_U(\pi)$.
  Hence, $\sigma$ is a subquotient of $\rho_U(i_Q^G(\tau))$. Thus, we have $\rho_U(i_Q^G(\tau))^{\supcus}\not=0$. Applying Lemma \ref{lemma_83}, we have,
  \begin{equation*}
    \gr(\rho_U(i_Q^G(\tau))) = \bigoplus_{\prescript{w_i}{}{M} = L}\prescript{w_i}{}{\tau}.
  \end{equation*}
  Thus there is $w_i$ such that 
  \begin{equation*}
    \prescript{w_i}{}{M} = L,\prescript{w_i}{}{\tau}\cong\sigma.
  \end{equation*}
  According to Lemma \ref{lemma_83}, we have,
  \begin{equation*}
    \gr(\rho_U(i_P^G(\sigma))) \cong \bigoplus_{\prescript{w_i}{}{L} = L}\prescript{w_i}{}{\sigma}.
  \end{equation*}
  Recall that $\{w_i\}$, the set of representatives for $Q\backslash G/P = P'\backslash G/P$. Thus there is $w_i$ corresponding to $P'1_GP$, thus we take $w_i=1_G$.
  The precise treatment can be seen in \cite{Renard} V.4.7 and VI.5.1. Obvisouly $\prescript{1}{}{L} = L$. Thus $\sigma$ is an irreducible subquotient of $\rho_{U'}(i_P^G(\sigma))$.
  Using Lemma \ref{lemma_59}, $\sigma$ is quotient of $\rho_{U'}(i_P^G(\sigma))$, that is 
  \begin{equation*}
    0 \not= \Hom_L(\rho_{U'}(i_P^G(\sigma)),\sigma) = \Hom_G(i_P^G(\sigma),i_{P'}^G(\sigma))
  \end{equation*}
\end{proof}

\begin{definition}
  A cuspidal pair for $G$ is a pair $(L,\sigma)$ where $l$ is a Levi factor of some parabolic subgroup $P$ of $G$ and $\sigma\in\Irr_{\supcus}(L)$.
\end{definition}

\begin{definition}
  By Proposition \ref{proposition_84}, for any irreducible representation $\pi$ of $G$, there exists a unique cuspidal pair $(L,\sigma)$ up to $G$-conjugacy, such that $\pi$ is an irreducible subquotient of $i_P^G(\sigma)$ for some parabolic subgroup $P=LU$.
  THe $G$-conjugacy class of $(L,\sigma)$ above is called the cuspidal support of $\pi$.
\end{definition}

\begin{definition}
  Let $(L,\sigma)$ and $(M,\tau)$ be the cuspidal pairs. We say they are innertially equivalent nad we write, 
  \begin{equation*}
    (L,\sigma)\sim(M,\tau),
  \end{equation*}
  if there exists $g\in G$ and $\nu\in X_{\nr}(M)$ such that 
  \begin{equation*}
    \prescript{g}{}{L} = M, \prescript{g}{}{\sigma} = \tau\nu.
  \end{equation*}
  An equivalence class of such relation represented by $(L,\sigma)$ is denoted by $[L,\sigma]_G$. We further denote,
  \begin{equation*}
    \mathcal{I}(G) = \{[L,\sigma]_G\sep (L,\sigma)\text{ is a cuspidal pair}\}.
  \end{equation*}
  as the set of all innertial equivalence classes of cuspidal pairs. 
\end{definition}

\begin{definition}
  Let $\pi$ be an irreducible representation of $G$ with its cuspidal support $(L,\sigma)$. We call $[L,\sigma]$ the innertial support of $\pi$.
  We define $J:\Irr(G)\ni\pi\mapsto [L,\sigma]_G\in\mathcal{I}(G)$, where it sends $\pi$ to its innertial support.
\end{definition}

\begin{remark}
  $J(\pi) = [L,\sigma]_G\Leftrightarrow \exists \nu\in X_{\nr}(L), P= LU$ such that $\pi$ is a subquotient of $i_P^G(\sigma\nu)$.
\end{remark}

\begin{definition}
  Let $s=[L,\sigma]_G\in\mathcal{I}(G)$ and denote $\Rep^s(G)$ to be the full subcategory of $\Rep(G)$ such that each irreducible subquotient of $\pi$ has innertial support $s$.
\end{definition}

\begin{remark}
  $\pi\in\Irr_{\supcus}(G)\Rightarrow [G,\pi]_G\in\mathcal{I}(G)$, then we have,
  \begin{equation*}
    \Rep^{[G,\pi]_G}=\Rep^{[\pi]}(G),
  \end{equation*}
  which the right hand side is defined previously. Namely, our definition is compatible with the previous ones.
\end{remark}

\begin{theorem}[Bernstein Decomposition]
  \label{theorem_85}
  We have the following decomposition,
  \begin{equation*}
    \Rep(G) = \prod_{s\in\mathcal{I}(G)}\Rep^s(G).
  \end{equation*}
  Furthermore, for each $s\in \mathcal{I}(G),\Rep^s(G)$ is indecomposable which is called a Bernstein block/component.
\end{theorem}

\begin{definition}
  For $(\pi,V)\in\Rep(G)$ and $s\in\mathcal{I}(G)$, we define, $V^s$ to be the sum of all $G$-subrepresentation of $V$ in $\Rep^s(G)$.
\end{definition}

\begin{lemma}
  $V^s\in\Rep^s(G)$.
\end{lemma}
\begin{proof}
  Exercise. See \cite{Roche} Lemma 1.7.3.2.
\end{proof}

To prove Theorem \ref{theorem_85}, we need to show that for each $V\in\Rep(G)$, we have,
\begin{equation*}
  V= \bigoplus_{s\in\mathcal{I}(G)}V^s.
\end{equation*}

\begin{definition}
  A smooth representation $(\pi,V)$ of $G$ is caleld split if we have the decomposition,
  \begin{equation*}
  V= \bigoplus_{s\in\mathcal{I}(G)}V^s.
\end{equation*}
\end{definition}

Thus our goal is to prove that any smooth representation is split. To do so, we follow two steps. 
\begin{enumerate}[Step 1.]
  \item We prove that each representation $(\pi,V)$ of $G$ embeds in a split representation.
  \item We prove that subrepresentations of split representations are again split.
\end{enumerate}

\begin{definition}
  Recall also that a parabolic subgropu is called standard if $P_0\subseteq P$. Fora standard parabolic subgroup $P$, there is a unique Levi factor $L= L_P$ ie $L_0\subseteq L$. We call such Levi subgroups standard Levi subgroups.
\end{definition}

\begin{remark}
  In the case of $\GL_n$, we can take 
  \begin{equation*}
    P_0 = \begin{pmatrix}
      \ast & \cdots & \ast \\
      \: & \ddots & \vdots\\
      \: & \: & \ast
    \end{pmatrix}
    = \begin{pmatrix}
      \ast & \: & \: \\
      \: & \ddots & \:\\
      \: & \: & \ast
    \end{pmatrix}
    \ltimes\begin{pmatrix}
      1 & \cdots & \ast \\
      \: & \ddots& \vdots\\
      \: & \: & 1
    \end{pmatrix}
  \end{equation*}
\end{remark}

\begin{remark}
  We havea bijection,
  \begin{equation*}
    \{\text{standard parabolic subgroups}\}\stackrel{P\mapsto L_P}{\stackrel{\leftrightarrows}{LP_0\mapsfrom L}}\{\text{standard Levi subgroups}\}
  \end{equation*}
\end{remark}

\begin{definition}
  We definea category $\cat(G)$ as 
  \begin{equation*}
    \cat(G) = \prod_{\substack{P=LU\\ P_0\subseteq P}}\Rep^{\supcus}(L).
  \end{equation*}
  We define functors, $\cat(G)\substack{i\\\rightleftarrows\\\rho}\Rep(G)$,
  \begin{equation*}
    \begin{cases}
      i((\rho_L,W_L)_L) = \bigoplus_{\substack{P=LU\\ P_0\subseteq P}}i_P^G(\rho_L),\\
      \rho(\pi) = (\underbrace{\rho_U(\pi)^{\supcus}}_{\in\Rep^{\supcus}(L)})_{\substack{P=LU\\ P_0\subseteq P}}.
    \end{cases}
  \end{equation*}
\end{definition}

\begin{lemma}
  Let $P_0$ be a minimal parabolic subgroup of $G$, we have,
  \begin{enumerate}
    \item For $V\in\Rep(G)$ and $W = (W_L)\in\cat(G)$, we have 
    \begin{equation*}
      \Hom_{\cat(G)}(\rho(V),W)\cong \Hom_G(V,i(W)).
    \end{equation*}
    \item The functors $i$ and $\rho$ are exact.
    \item The functor $\rho$ is faithful, tat is for any $0\not=V\in\Rep(G), \rho(V)\not=0$.
  \end{enumerate}
  \label{lemma_86}
\end{lemma}

\begin{proof}
  The first assertion follows from Frobenius reciprocity. We can compute 
  \begin{align*}
        \Hom_{\cat(G)}(\rho(V),W) & = \prod_{\substack{P=LU\\ P_0\subseteq P}}\Hom_L(\rho_U(V)^{\supcus},\underbrace{W_L}_{\in\Rep^{\supcus}(L)}),\\
    & = \prod_{\substack{P=LU\\ P_0\subseteq P}}\Hom_L(\underbrace{\rho_U(V)}_{=\rho_U(V)^{\supcus}\otimes\rho_U(V)^{\nsc}},W_L),\\
    &\cong\prod_{\substack{P=LU\\ P_0\subseteq P}}\Hom_G(V,i(W_L)),\\
    & = \Hom_G\left(V,\bigoplus_{\substack{P=LU\\ P_0\subseteq P}}i(W_L)\right),\\
    & = \Hom_G(V,i(W)).
  \end{align*}
  For the second assertion, according to Proposition \ref{proposition_45}, $\rho_U$ is exact for all $P=LU$. Since $i_P^G$ is the right adjoint of $\rho_U$, $i_P^G$ is left exact. It is assigned to the readers to prove that $i_P^G$ is exact using Lemma \ref{lemma_7} ie the local property of representations. 
  Since the projection functor $\Rep(L)\to\Rep^{\supcus}(L)$ is also exact, we obtain the statement.
  \par For the third, let $0\not=V\in\Rep(G)$ and $P=LU$ be a standard parabolic subgroup that is minimal among the parabolic subgroup for which 
  \begin{equation*}
    \rho_U(V)\not=0.
  \end{equation*}
  Then by the transitivity of the Jacquet functor, we have 
  \begin{equation*}
    \rho_U(V)\in\Rep^{\supcus}(L).
  \end{equation*}
  This follows from the same argument as the proof of Corollary \ref{corollary_53}. Hence 
  \begin{equation*}
    \rho_U(V)^{\supcus}=\rho_U(V)\not=0,\text{ and }\rho(V)\not=0.
  \end{equation*}
\end{proof}

\begin{lemma}
  Let $(\pi,V)\in\Rep(G)$ then $\rho(V)\in\cat(G)$. 
\begin{equation*}
  \Hom_{\cat(G)}(\rho(V),\rho(V)) \cong\Hom_G(V,i(\rho(V))).
\end{equation*}
Since the left hand side contains an identity $\id_{\rho(V)}$, there is an element $\alpha_V$ corresponds to the identity.
We have $\alpha_V$ is injective.
\label{lemma_87}
\end{lemma}

\begin{proof}
  Let $K = \ker(\alpha_V)$, Applying the functor $i\circ\rho$ to the inclusion $K\stackrel{\iota}{\hookrightarrow} V$, we get 
  \begin{center}
    \begin{tikzcd}
K \arrow[d, "\alpha_K"'] \arrow[r, "\iota", hook] & V \arrow[d, "\alpha_V"] \\
(i\circ\rho)(K) \arrow[r, "(i\circ\rho)(\iota)"'] & (i\circ\rho)(V)        
\end{tikzcd}
  \end{center}
  It is an exercise for the readers to check the above diagram is commutative.
  \par We then have 
  \begin{equation*}
    (i\circ\rho)(\iota)\circ\alpha_K = \alpha_V\circ\iota\stackrel{K = \ker\alpha_V}{=}0.
  \end{equation*}
  Furthermore, $(i\circ\rho)(\iota)$ is an inclusion (ie. it is injective), we obtain that $\alpha_K=0$. However, $\alpha_K$ corresponds to $\id_{\rho(K)}$ in 
  \begin{equation*}
    \Hom_{\cat(G)}(\rho(K),\rho(K)) \cong\Hom_G(K,i(\rho(K))),
  \end{equation*}
  we conclude $\rho(K)=0$. Since $\rho$ is faithful, we obtain $K=0$.
\end{proof}

For each standard parabolic subgroup, $P=LU$, we apply Corollary \ref{corollary_58}, we have,
\begin{equation*}
  \rho_U(V)^{\supcus} = \bigoplus_{[\sigma]\in I(L)^{\supcus}}\underbrace{\rho_U(V)^{[\sigma]}}_{\in\Rep^{[\sigma]}(L) = \Rep^{[L,\sigma]}(L)}.
\end{equation*}

Hence,
\begin{equation*}
  i(\rho(V)) = \bigoplus_{\substack{P=LU\\ P_0\subseteq P}}i_P^G(\rho_U(V)^{\supcus}) = \bigoplus_{\substack{P=LU\\ P_0\subseteq P}}i_P^G\left(\bigoplus_{[\sigma]\in I(L)^{\supcus}}\rho_U(V)^{[\sigma]}\right).
\end{equation*}

\begin{lemma}
  \label{lemma_88}
  For any direct sum $\bigoplus_{i\in I}W_i\in\Rep(L)$, we have,
  \begin{equation*}
    i_P^G\left(\bigoplus_{i\in I}W_i\right) \cong\bigoplus_{i\in I}i_P^G(W_i).
  \end{equation*}
\end{lemma}

\begin{proof}
  We have an injective $G$-homomorphism,
  \begin{equation*}
    \bigoplus_{i\in I}i_P^G(W_i)\ni(f_i)_{i\in I}\mapsto [g\mapsto (f_j(g))_{i\in I}]\in i_P^G\left(\bigoplus W_i\right).
  \end{equation*}
  We will prove that this is surjective. Let $f\in i_P^G\left(\bigoplus_{i\in I}W_i\right)$, then there exists a compact, open subgroup $K\subseteq G$ such that 
  \begin{equation*}
    \forall g\in G,k\in K, f(gk) = f(g).
  \end{equation*}
  Hence, $f$ is determined by its values on a set of representatives for the double cosets $P\backslash G/K$ which is a finite set. To see it is finite, recall the Iwasawa decomposition
  \begin{equation*}
    G = P K_0.
  \end{equation*}
  Thus the set of representatives of double coset $P\backslash G/K$ is contained in $K_0/(K\cap K_0)$. Hence, there is a finite set $\{i_1,\cdots,i_k\}$ such that 
  \begin{equation*}
    f\in i_P^G\left(\bigoplus_{j=1}^k W_{i_j}\right) = \bigoplus_{j=1}^k i_P^G(W_{i_j}).
  \end{equation*}
\end{proof}

\begin{corollary}
  
\end{corollary}

\begin{proof}
  \begin{align*}
    \Ind_P^G\left(\bigoplus_{i\in I}W_i\right) & = i_P^G\left(\left(\bigoplus_{i\in I}W_i\right)\otimes\delta_P^{-{\frac 1 2}}\right),\\
    & = i_P^G\left(\bigoplus(W_i\otimes\delta_P^{-{\frac 1 2}})\right).
  \end{align*}
\end{proof}

Using Lemma \ref{lemma_88}, we have,
\begin{align*}
  i(\rho(V)) & = \bigoplus_{P=LU}i_P^G\left(\bigoplus_{[\sigma]}\rho_U(V)^{[\sigma]}\right),\\
  & = \bigoplus_{P=LU}\bigoplus_{[\sigma]\in I(L)^{\supcus}}i_P^G(\rho_U(V)^{[\sigma]}).
\end{align*}
We want to show that $i_P^G(\rho_U(V)^{[\sigma]})\in\Rep^{[L,\sigma]_G}(G)$. 

\begin{lemma}
  \label{lemma_89}
  For $W\in\Rep^{[L,\sigma]_L}(L)$, we have, 
  \begin{equation*}
    i_P^G(W)\in\Rep^{[L,\sigma]_G}(G).
  \end{equation*}
\end{lemma}
\begin{proof}
  Let $\pi$ be an irreducible subquotient of $i_P^G(W)$. There is $Q=MN$ and $\tau\in\Irr_{\supcus}(M)$ such that 
  \begin{equation*}
    \Hom_G(\pi,i_Q^G(\tau))\not=0.
  \end{equation*}
  What we want is to show that $(M,\tau)\sim (L,\sigma)$. But 
  \begin{equation*}
    \Hom_G(\pi,i_Q^G(\tau)) \cong \Hom_M(\rho_N(\pi),\tau).
  \end{equation*}
  On the other hand, $\rho_N(\pi)$ is a subquotient of $\rho_N(i_P^G(\sigma))$. Hence $\tau$ is an irreducible subquotient of $\rho_N(i_P^G(\sigma))$. By Lemma \ref{geometric_lemma}, we have,
  \begin{equation*}
    \gr(\rho_N(i_P^G(\sigma))) = \bigoplus_{\prescript{w}{}{L}=M}\prescript{w}{}{W}.
  \end{equation*}
  Thus there is $g\in G$ such that $\prescript{g}{}{L} = M$, and $\tau$ is an irreducible subquotient of $\prescript{w}{}{W}$. Since $W\in\Rep^{[L,\sigma]_L}(L)$, that is there exists an unramified character $\nu\in X_{\nr}(L)$ such that 
  \begin{equation*}
    (\prescript{g}{}{L},\prescript{g}{}{(\sigma\nu)}) = (M,\tau).
  \end{equation*}
  Thus $(L,\sigma)\sim (M,\tau)$, thus $\tau(\pi)=[M,\tau]_G=[L,\sigma]_G$.
\end{proof}

\begin{corollary}
  \label{corollary_90}
  The representation $i(\rho(V))$ is split. 
\end{corollary}

\begin{proof}
  
  \begin{align*}
    i(\rho(V)) & = \bigoplus_{\substack{P=LU\\ P_0\subseteq P}}\bigoplus_{[\sigma]\in\Irr^{\supcus}}i_P^G(\rho(V)^{[\sigma]})\\
    & = \bigoplus_{[L,\sigma]_G\in I(G)}\left(\underbrace{\bigoplus_{\substack{P'=L'U'\\ P_0\subseteq P'\\ [\sigma']\in\Irr(L')^{\supcus}\\ [L',\sigma']_G=[L,\sigma]_G}}\underbrace{i_{P'}^G(\rho_{U'}(V)^{[\sigma']})}_{\in\Rep^{[L',\sigma']_G}(G)=\Rep^{[L,\sigma]_G}(G)}}_{\in\Rep^{[L,\sigma]_G}(G)}\right).
  \end{align*}
\end{proof}

\begin{lemma}
  \label{lemma_91}
  Let $\{V_i\}_{i\in I}$ be a family of representations of $G$ such that no irreducible subquotient of $V_i$ is isomorphic to irreducible subquotient of $V_j$ for $i\not=j$. Let $W$ be a subrepresentation of $\bigoplus_{i\in I}V_i$.
  Then $W$ can be written as 
  \begin{equation*}
    W = \bigoplus_{i\in I}(W\cap V_i).
  \end{equation*}
  In particular, this implies that a subrepresentation of a split representation is also split.
\end{lemma}

\begin{proof}
  Suppose that $\bigoplus_{i\in I}(W\cap V_i)\subsetneq W$. Let $w\in W\backslash \bigoplus_{i\in I}(W\cap V_i)$. We write,
  \begin{equation*}
    w = v_1+\cdots+v_k
  \end{equation*}
  where $v_i\in V_i$ for each $i=1,\cdots,k$ with appropriate renumbering. We may suppose that $v_1\not\in W\cap V_1$, since the assumption, at least one of them is not contained in $W$. Let 
  \begin{equation*}
    v = v_2+\cdots+v_k.
  \end{equation*}
  Then we have,
  \begin{equation*}
    v+w = v_1.
  \end{equation*}
  Using $v$, we define $G\cdot v$ which is a $G$-subrepresentation of $\bigoplus_{i\in I}V_i$ generated by $v$. Denote $V = \bigoplus_{i\in I}V_i$, we consider
  \begin{center}
    \begin{tikzcd}
G\cdot v \arrow[d, hook] \arrow[rd, "v\mapsto v_1+W"] \arrow[rdd, "\exists", dashed, bend left=67] &                                                                  \\
V \arrow[r, two heads]                                                                             & V/W                                                              \\
V_1 \arrow[r, two heads] \arrow[u, hook]                                                           & V_1/(W\cap V_1) \arrow[u, "v_1+(W\cap V_1)\mapsto v_1+W"', hook]
\end{tikzcd}
  \end{center}
  We havea non-zero $G$-homomorphism $G\cdot v\to V_1/(W\cap V_1)$. We now have,
  \begin{center}
    \begin{tikzcd}
G\cdot v \arrow[d, hook] \arrow[r] & V_1/(W\cap V_1)          \\
V_2\oplus\cdots\oplus V_k          & V_1 \arrow[u, two heads]
\end{tikzcd}
  \end{center}
  Thus, there exists a common irreducible subquotient $V_2\oplus\cdots\oplus V_k$ and $V_1$ which is a contradiction. As irreducible subquotient of $V_2\oplus\cdots\oplus V_k$ is an irreducible subquotient of some $V_j$ for $j\in\{2,\cdots,k\}$.
\end{proof}

\begin{remark}
  The above setting is equivalent to say that $\gr V_i$ and $\gr V_j$ have no common factor unless $i=j$.
\end{remark}

\begin{theorem}[Bernstein Decomposition]
  For any reductive group $G$, we have,
  \begin{equation*}
    \Rep(G)  = \prod_{s\in I_(G)}\Rep^s(G).
  \end{equation*}
  That is any smooth representation is split.
\end{theorem}

\begin{proof}
  Follows from Corollary \ref{corollary_90} and Lemma \ref{lemma_91}.
\end{proof}

Now we move on to show that $\Rep^s(G)$ indecomposable. Our strategy is to study Bernstein center $Z(\Rep^s(G))$. We will describe $Z(\Rep^{[L,\sigma]_G}(G))$ using the center $Z(\Rep^{[L,\sigma]_L}(L))$. Recall that $L\subseteq G$ and $\sigma\in\Irr^{\supcus}(L)$. Thus 
$\Rep^{[L,\sigma]_L}(L)$ is a supercuspidal Bernstein block.

\section{More about supercuspidal representations}

The main reference of this section is from \cite{Roche} $\S1.6$.

We fix $(\pi,V)\in\Irr_{\supcus}(G)$ and $(\pi^\circ,W)$ which is an irreducible subrepresentation of $\pi|_{G^\circ}$. Recall that we have define the Hecke algebra,
\begin{equation*}
  \hecke(G,G^\circ,\pi^\circ) = \left\{\phi:G\to\End_G(W)\sep \substack{\forall g,g'\in G^\circ, x\in G,\phi(gxg') = \pi^\circ(g)\circ\phi(x)\circ\pi^\circ(g)\\ \supp(\phi) \text{ is a finite union of $G^\circ$-cosets}}\right\},
\end{equation*}
with the product defined as the convolution. 
\par Let 
\begin{equation*}
  N = N_{G^\circ}(\pi^\circ) = \{n\in G\sep \prescript{n}{}{\pi^\circ} \cong \pi^\circ\}.
\end{equation*}

Using Lemma \ref{lemma_70}, we see that $\supp(\hecke(G,G^\circ,\pi^\circ)) = N$ and for $n\in N/G^\circ$,
\begin{equation*}
  \dim(\{\phi\in\hecke(G,G^\circ,\pi^\circ)\sep \supp(\phi)\subseteq n\cdot G^\circ\}) = 1.
\end{equation*}

These imply that 
\begin{equation*}
  \hecke(G,G^\circ,\pi^\circ)\stackrel{\text{as vectorspaces}}{\cong}\bigoplus_{n\in N/G^\circ}\C\phi_n,
\end{equation*}
cf. Corollary \ref{corollary_73}. ALso observe that $N/G^\circ\cong Z^r$ and $\phi_n\ast\phi_m = c\phi_m\ast\phi_n$ for some appropriate $c\in \C$.
\par We will study the center $Z(\hecke(G,G^\circ,\pi^\circ))$ in more details.

\begin{definition}
  Let $H$ be such that 
  \begin{equation*}
    H = \Stab_G(W) = \{g\in G\sep \pi(g)(W) = W\}.
  \end{equation*}
\end{definition}
\begin{definition}
  Observe that an action $G\stackrel{\pi}{\curvearrowright}V$ restricts to $H\stackrel{\pi_H}{\curvearrowright}W$. 
\end{definition}
\begin{remark}
  \begin{equation*}
    \pi_H|_{G^\circ} = \pi^\circ.
  \end{equation*}
\end{remark}
Note that this is stronger than the condition of normalizers. Thus we have, 
  \begin{equation*}
    H\subseteq N.
  \end{equation*}
  Furthermore, we have $H=N$ if $\pi|_{G^\circ}$ is of multiplicity $1$ that is $\dim\Hom_{G^\circ}(\pi^\circ,\pi|_{G^\circ})=1$. But of course these two notions differ in general. Clearly $Z(G)\cdot G^\circ\subseteq H$.
  Since $\pi$ is irreducible, in particular, we have 
  \begin{equation*}
    V = \sum_{g\in G/H}\pi(g)(W).
  \end{equation*}
  Replacing $(\pi^\circ,W)$ with another irreducible subrepresentation of $\pi|_{G^\circ}$, we assume that $H$ is maximal among 
  \begin{equation*}
    \{\Stab_G(W')\sep ((\pi^\circ)',W')\text{ an irreducible subrepresentation of $\pi|_{G^\circ}$}\}.
  \end{equation*}
  This is indeed possible since $Z(G)\cdot G^\circ\subseteq\Stab_G(W')$. On the other hand, $G/Z(G)\cdot G^\circ$ is finite which was proved in Lemma \ref{lemma_31}. From now on, we will assume $H$ is maximal.
  \begin{lemma}
    Suppose $(\pi_H,V)$ be a representation of $H$ and $H\subseteq H_1$ be such that $H_1/H$ is cyclic. Then we have,
    \begin{enumerate}[1).]
      \item $\pi_H$ extends to $H_1$.
      \item Any irreducible representation $\pi_1$ of $H_1$ such that $\Hom_H(\pi_H,\pi_1|H)\not=0$ is an extension of $\pi_H$.
    \end{enumerate}
    \label{lemma_92_1}
  \end{lemma}
  \begin{proof}
    Exercise.
  \end{proof}
  \begin{lemma}
    \label{lemma_92}
    Assume that $H$ is maximal. Then the sum 
    \begin{equation*}
      V=. \sum_{g\in G/H}\pi(g)(W),
    \end{equation*}
    is a direct sum.
  \end{lemma}
  \begin{proof}
    Note that $[G,G]\subseteq G^\circ\subseteq H\subseteq G$. Thus $H$ is normal in $G$.
    Since $H$ is normal in $G$, $\pi(g)(W)$ is $H$-stable. Moreover, we have an isomorphism of $H$-representations,
    \begin{equation*}
      (\prescript{g}{}{\pi_H},W)\stackrel{\sim, w\mapsto\pi(g)(w)}{\longrightarrow}(\pi|_H,\pi(g)(W)).
    \end{equation*}
    In particular, each term of the sum is isomorphic to $\prescript{g}{}{\pi_H}$.
    \par Hence, to prove the lemma, it suffices to show that 
    \begin{equation*}
      \prescript{g}{}{\pi_H},(g\in G/H)
    \end{equation*}
    are distinct that is 
    \begin{equation*}
      \tilde{H}\defeq \{g\in G\sep \prescript{g}{}{\pi_H}\cong \pi_H\},
    \end{equation*}
    agrees with $H$. Suppose $H\subsetneq \tilde{H}$, since $\tilde{H}/H\subseteq G/H \twoheadleftarrow G/Z(G)\cdot G^\circ$ which is a finite gruop. We can take $H \subseteq H_1\subseteq \tilde{H}$ such that $H_1/H$ is cyclic. Note that $H_1$ of course normalizes $\pi_H$.
    \par Let $W_1$ be a $H_1$-subrepresentation of $V$ such that $\Hom_H(\pi_H,W_1|_H)\not=0$. By Lemma \ref{lemma_92_1}, $W_1$ is an extension of $\pi_H$. Note that 
    \begin{equation*}
      W_1|_{G^\circ} = \pi_H|_{G^\circ}=\pi^\circ,
    \end{equation*}
    is an irreducible subrepresentation of $\pi|_{G^\circ}$. Hence $(\pi_{G^\circ},W_1)$ is an irreducible subrepresentation of $\pi|_{G^\circ}$.
    \par To conclude note that $\Stab_G(W_1)\supseteq H_1\supsetneq H$ which contradicts to the maximality of $H$. 
  \end{proof}
  \begin{proposition}
    \begin{equation*}
      V = \bigoplus_{g\in G/H}\pi(g)(W)\cong \cind_H^G(\pi_H).
    \end{equation*}
  \end{proposition}
  \begin{proof}
    Exercise.
  \end{proof}
  \begin{corollary}
    \label{corollary_93}
    Let $n\defeq \{\text{multiplicity of $\pi^\circ$ in $\pi|_{G^\circ}$}\} = [N:H]$. 
  \end{corollary}

  \begin{proof}
    By Lemma \ref{lemma_92}, we have,
    \begin{equation*}
      \pi|_H\cong\bigoplus_{g\in G/H}\prescript{g}{}{\pi_H}\Rightarrow\pi|_{G^\circ}\cong\bigoplus_{g\in G/H}\prescript{g}{}{\pi^\circ}.
    \end{equation*}
    Note that $\prescript{g}{}{\pi^\circ} = \pi^\circ$ if and only if $g\in N$. Thus $\pi^\circ$ appears $[N:H]$ times in this direct sum.
  \end{proof}
  \begin{definition}
    We define 
    \begin{equation*}
      X_{\nr}(G)_\pi = \{\nu\in X_{\nr}(G)\sep \pi\nu\cong\pi\}.
    \end{equation*}
    This notion is somewhat an analogonous of stabilizer. Furthermore, we define,
    \begin{equation*}
      T = \bigcap_{\nu\in X_{\nr}(G)_\pi}\ker\nu \supseteq G^\circ, (\nu|_{G^\circ}=\id).
    \end{equation*}
  \end{definition}
  \begin{lemma}$\:$
    \label{lemma_94}
    \begin{enumerate}
      \item We have $Z(G)\cdot G^\circ\subseteq T$,
      \item $X_{\nr}(G)_{\pi} = \{\nu\in X_{\nr}(G)\sep \nu|_T = \id\}$. 
      \item For any subgroup $S$ of $G$ containing $G^\circ$, we have 
      \begin{equation*}
        S = \bigcup_{\{v\in X_{\nr}(G)\sep \nu|_S=\id\}}\ker\nu.
      \end{equation*}
    \end{enumerate}
  \end{lemma}

  %12/22

  %1/7
  We have proved that 
  \begin{equation*}
    \Rep(G) = \prod_{[L,\sigma]_G\in I(G)}\Rep^{[L,\sigma]_G}(G).
  \end{equation*}
  Our goal was to show that each Bernstein block $\Rep^{[L,\sigma]_G}(G)$ is indecomposable.
  \par The strategy we use is to use the center $z(\Rep^{[L,\sigma]_G}(G))$ has no non-zero zero divisors. Let us recall that, in the supercuspidal case $L=G$ and $\sigma=\pi\in\Irr_{\supcus}(G)$, we have,
  \begin{equation*}
    \Rep^{[G,\pi]_G}(G)\cong \Mod(\End_G(\cind^G_{G^\circ}(\pi^\circ))), \pi^\circ\hookrightarrow \pi|_{G^\circ},
  \end{equation*}
  which has been proven in Theorem \ref{theorem_66}. The key statement is that $\cind_{G^\circ}^G(\pi^\circ)$ is progenerator, that is it is projective, finitely generated operator of $\Rep^{[G,\pi]_G}(G)$ which is from Proposition \ref{proposition_64}. We also proved that 
  \begin{equation*}
    \Mod(\End_\C(\cind_{G^\circ}^G(\pi^\circ))) \cong \Mod(\hecke(G,G^\circ,\pi^\circ)).
  \end{equation*}
  And in the last lecture, we have seen that 
  \begin{equation*}
    Z(\Rep^{[G,\pi]_G}(G))\cong Z(\hecke(G,G^\circ,\pi^\circ))\stackrel{\text{Proposition \ref{proposition_97}}}{\cong} \hecke(T,G^\circ,\pi^\circ).
  \end{equation*}
  To generalize this, we construct a progenerator of a general Bernstein block $\Rep^{[L,\sigma]_G}(G)$. Let $\sigma^\circ$ be an irreducible subrpresentation of $\sigma|_{L^\circ}$. Define $\Sigma = \cind_{L^\circ}^L(\sigma^\circ)$.
  \begin{remark}
    We already know that $\Sigma$ is a progenerator of $\Rep^{[L,\sigma]_L}(L)$.
  \end{remark}
  Let $P$ be a parabolic subgroup of $G$ with Levi factor $L$. Then by Lemma \ref{lemma_89}, if
  \begin{equation*}
    i_P^G(\Sigma)\in\Rep^{[L,\sigma]_G}(G).
  \end{equation*}
  We will show that this turns out to be a progenerator.
  \begin{theorem}
    \label{theorem_second_adjunction}
    Let $P=LU$ be a parabolic subgroup and $\overline{P}=L\overline{U}$ be the opposite subgroup of $P$. Then the normalized Jacquet functor $\rho_{\overline{U}}$ is right adjoint fuctor to $i_P^G$.
  \end{theorem}
  \begin{proof}
    \cite{Renard} VI.9.6. Theorem
  \end{proof}
  \begin{remark}
    The above Theorem can be rephrased as follows. Let $V\in\Rep(G)$ and $W\in\Rep(L)$. Then by Frobenius reciprocity, we have,
    \begin{equation*}
      \Hom_G(V,i_P^G(W)) \cong \Hom_L(\rho_U(V),W).
    \end{equation*}
    By above Theorem, we get,
    \begin{equation*}
      \Hom_G(i_P^G(W),V) \cong \Hom_L(W,\rho_{\overline{U}}(V)).
    \end{equation*}
  \end{remark}
  \begin{definition}
    A representation is called of finite length if there exists a filtration satisfying conditions of Geometric lemma.
  \end{definition}
  \begin{remark}
    The Jacquet functor preserves representations of finite length. The proof can be seen in \cite{Renard} VI 6.4.
  \end{remark}
  \setcounter{claim}{0}
  \begin{theorem}
    \label{theorem_99}
    Let $L$ be a %missing,
    then, we have the following statements.
    \begin{enumerate}[1).]
      \item The isomorphism classes of $i_P^G(\Sigma)$ is independent of choice of parabolic subgroup $P$. That is, such representation depends on $L$ but not on $P$.
      \item $i_P^G(\Sigma)$ is a progenerator of $\Rep^{[L,\sigma]_G}(G)$.
    \end{enumerate}
  \end{theorem}
  \begin{proof}
  For the independence on $P$, we will refer to \cite{Renard} VI.10.1, Corollary. We also leave a comment that, in the note, we first reduce to the case where $L$ is maximal. We then use Geometric lemma to prove the general case.
  \par For the second part, we leave the general case as an exercise. However, we will leave some steps for proving this theorem. Consider the case when $P$ is a projective object of $P\in\Rep(G)$. First we show that the following statements are equivalent.
  \begin{enumerate}
    \item $P$ is finitely generated.
    \item $P$ is small that is for any family $\{W_i\}_{i\in I}$ in $\Rep(G)$, the natural map,
    \begin{equation*}
      \bigoplus_{i\in I}\Hom_G(P,W_i) \to \Hom_G\left(P,\bigoplus_{i\in I}W_i\right),
    \end{equation*}
    is an isomorphism.
  \end{enumerate}
  For the next step, prove that if $W\in\Rep(L)$ is a projective and small then $i_P^G(W)$ is projective and small.
  The crucial part is to use the second adjunction theorem and the facts that $\rho_{\overline{U}}$ is exact which is proven in Proposition \ref{proposition_45} and commutes with direct sums.
  \par Suppose we have proven these two steps above, we obtain that $i_P^G(\Sigma)$ is projective and finitely generated. We will prove that $i_P^G(\Sigma)$ is a generator of $\Rep^{[L,\sigma]_G}(G)$. That is for any non-zero representation $\pi\in\Rep^{[L,\sigma]_G}(G)$ we have,
  \begin{equation*}
    \Hom_G(i_P^G(\Sigma),\pi) \not=0.
  \end{equation*}
  Let $\pi'$ be an irreducible subquotient of $\pi$. This means there is a subrepresentation $\pi_1\subseteq\pi$ with a surjective morphism $\pi_1\twoheadrightarrow\pi'$. Since $i_P^G(\Sigma)$ is projective, we have, in the following diagram,
  \begin{center}
    \begin{tikzcd}
{\Hom_G(i_P^G(\Sigma),\pi_1)} \arrow[d, two heads] \arrow[r, hook] & {\Hom_G(i_P^G(\Sigma),\pi)} \\
{\Hom_G(i_P^G(\Sigma),\pi')}                                       &                            
\end{tikzcd}
  \end{center}
  if $\Hom_G(i_P^G(\Sigma),\pi')$ is non-zero then $\Hom_G(i_P^G(\Sigma),\pi_1)$ is non-zero. And $\Hom_G(i_P^G(\Sigma),\pi_1)$ is non-zero then so is $\Hom_G(i_P^G(\Sigma),\pi)$ is non-zero.
  Hence, it suffices to show that $\Hom_G(i_P^G(\Sigma),\pi')\not=0$. Thus, we may assume that $\pi$ is irreducible.
  \par Let $\pi\in\Rep^{[L,\sigma]_G}(G)$ be irreducible.
  \begin{claim}
    There is a parabolic subgroup $P'=L'U'$ and a representation $\sigma'\in\Irr_{\supcus}(L')$ such that 
    \begin{equation*}
      \Hom_G(i_{P'}^G(\sigma'),\pi)\not=0.
    \end{equation*}
    Note that this resembles to Corollary \ref{corollary_53}. Recall that to prove the corollary, we used Frobenius reciprocity. Thus the intuition is to use the similar machinery which is Theorem \ref{theorem_second_adjunction}.
  \end{claim}
  \begin{claimproof}
    Let $P'=L'U'$ be a parabolic subgroup of $G$ that is minimal among those for which $\rho_{\overline{U'}}(\pi)\not=0$.
    As in the proof of Corollary \ref{corollary_53}, that is by the traisitivity of Jacquet functor, $\rho_{\overline{U'}}(\pi)$ is supercuspidal. (This part is exactly the same as in the corollary). We will prove that $\rho_{\overline{U'}}(\pi)$ is of finite length. In particular, it has some irreducible subrepresentation.
    \par By Corollary \ref{corollary_53}, there is $Q=MN$ and $\tau\in \Irr_{\supcus}(M)$ such that $\pi\hookrightarrow i_Q^G(\tau)$. Since $\rho_{\overline{U'}}$ is exact, we have $\rho_{\overline{U'}}(\pi)$ is a subrepresentation of $\rho_{\overline{U'}}(i_Q^G(\tau))$.
    By geometric lemma (Lemma \ref{lemma_83}), we have 
    \begin{equation*}
      \gr(\rho_{\overline{U'}(i_Q^G(\tau))}) = \bigoplus_{\{w_i\in P'\backslash G/Q\sep\prescript{w_i}{}{M}=L'\}}\prescript{w_i}{}{\tau},
    \end{equation*}
    and $P'\backslash G/Q$ is finite. Hence $\rho_{\overline{U'}}(i_Q^G(\tau))$ is of finite length. Thus $\rho_{\overline{U'}}(\pi)$ is of finite length.
    \par Since $\rho_{\overline{U'}}(\pi)$ is of finite length, there is $\sigma'\in\Irr_{\supcus}(L')$ such that $\Hom_{L'}(\sigma',\rho_{\overline{U'}(\pi)})\not=0$. Using Theorem \ref{theorem_second_adjunction}, we have,
    \begin{equation*}
      \Hom_{L'}(\sigma',\rho_{\overline{U'}}(\pi)) \cong \Hom_G(i_{P'}^G(\sigma'),\pi).
    \end{equation*}
  \end{claimproof}
  For $\pi\in\Rep^{[L,\sigma]_G}(G)$ irreducible, we want $i_P^G(\Sigma)\twoheadrightarrow\pi$. We have proved,
  \begin{equation*}
    i_{P'}^G(\sigma')\twoheadrightarrow\pi.
  \end{equation*}
  Since $\pi\in\Rep^{[L,\sigma]_G}(G)$, there is $g\in G$ and $\nu\in X_{\nr}(L)$ such that 
  \begin{equation*}
    \prescript{g}{}{(L',\sigma')}=(L,\sigma\nu).
  \end{equation*}
  Since 
  \begin{equation*}
    i_{\prescript{g}{}{P'}}^G(\prescript{g}{}{\sigma'}) \ni f\mapsto [x\to f(gx)]\in i_{P'}^G(\sigma'),
  \end{equation*}
  we may assume that $g=1$. Since $\Sigma$ is a generator fo $\Rep^{[L,\sigma]_G}(L)$, we have a surjection $\Sigma\twoheadrightarrow\sigma\nu\cong\sigma'$. Since $i_{P'}^G$ is exact, we have 
  \begin{equation*}
    i_{P'}^G(\Sigma)\twoheadrightarrow i_{P'}^G(\sigma').
  \end{equation*}
  Combining this with $i_{P'}^G\twoheadrightarrow\pi$, we obtain that 
  \begin{equation*}
    \Hom_G(i_{P'}^G(\Sigma),\pi)\not=0.
  \end{equation*}
  Also we have,
  \begin{equation*}
    i_{P'}^G(\Sigma)\cong i_P^G(\Sigma).
  \end{equation*}
  Thus we have proven the statement. 
  \end{proof}
  \begin{definition}
    We define functor
    \begin{center}
      \begin{tikzcd}
{\Rep^{[L,\sigma]_G}(G)} \arrow[r, "F_{i_P^G(\Sigma)}", shift left] & \Mod(\End_G(i_P^G(\Sigma))) \arrow[l, "G_{i_P^G(\Sigma)}", shift left]
\end{tikzcd}
    \end{center}
    such that 
    \begin{align*}
      F_{i_P^G(\Sigma)}(V) & = \Hom_G(i_P^G(\Sigma),V),\\
      G_{i_P^G(\Sigma)}(M) & = M\tens{\End_G(i_P^G(\Sigma))}i_P^G(\Sigma).
    \end{align*}
  \end{definition}
  \begin{theorem}
    \label{theorem_100}
    $F_{i_P^G(\Sigma)}$ is an equivalence of categories with quasi-inverse $G_{i_P^G(\Sigma)}$.
  \end{theorem}
  \begin{proof}
    Same as Theorem \ref{theorem_66}. This works for categories with progenerators if $F$ and $G$ are constructed in the similar manner..
  \end{proof}
  \begin{notation}
    From now on, we will denote $\mathcal{A}=\End_G(i_P^G(\Sigma))$ and $\mathcal{B} = \End_L(\Sigma)$.
  \end{notation}
  To study $\mathcal{A}$ we will study $\mathcal{B}$ and transfer the result under the following algebra morphism,
  \begin{equation*}
    t_P:\mathcal{B}\to\mathcal{A}, t_P(b) = i_P^G(b).
  \end{equation*}
  This is well-defined as $i_P^G$ is a functor.
  \begin{proposition}
    $t_P$ is injective. That is $i_P^G$ is faithful.
  \end{proposition}
  \begin{proof}
    Exericise.
  \end{proof}
  Thus $\mathcal{B}$ can be viewed as a subalgebra of $\mathcal{A}$.
  %1/12
  We will show that 
  \begin{equation*}
    Z(\cA)\subseteq Z(\cB).
  \end{equation*}
  Furthermore, we will even show that $Z(\cB)$ turns out to be an integral domain therefore so is $Z(\cA)$.
  \begin{remark}
    $\cA$ is not an integral domain in general.
  \end{remark}
  \subsection{Structure of $\cA$ as a right $\cB$-module}
  \begin{definition}
    \begin{equation*}
      N^{[L,\sigma]_L}=\{n\in N_G(L)\sep \exists \nu\in X_{\nr}(L),\prescript{n}{}{\sigma}\simeq \sigma\nu\}.
    \end{equation*}
  \end{definition}
  \begin{remark}
    $N^{[L,\sigma]_L}$ is a stabilizer of 
    \begin{equation*}
      [L,\sigma]_L=\{\sigma\nu\sep \nu\in X_{\nr}(L)\}.
    \end{equation*}
  \end{remark}
  \begin{definition}
    \begin{equation*}
      W^{[L,\sigma]_L} = N^{[L,\sigma]_\sigma}/L.
    \end{equation*}
  \end{definition}
  \begin{definition}
    \begin{equation*}
      \rho_{\overline{U}}^{[L,\sigma]_L}=\Rep^{[L,\sigma]_G}(G)\stackrel{\rho_{\overline{U}}}{\to}\Rep(L)\stackrel{\text{proj.}}{\to}\Rep^{[L,\sigma]_L}(L).
    \end{equation*}
  \end{definition}
  \begin{lemma}
    $\rho_{\overline{U}}^{[L,\sigma]_L}$ is a right adjoint to $i_P^G:\Rep^{[L,\sigma]_L}(L)\to\Rep^{[L,\sigma]_G}(G)$. We can also prove that $\rho_U^{[L,\sigma]_L}$ is left adjoint to $i_P^G$.
  \end{lemma}
  \begin{proof}
    Exercise. Use Theorem \ref{theorem_second_adjunction}.
  \end{proof}
  \begin{proposition}
    \label{proposition_102}
    As a right $\cB$-module, $\cA$ is free of rank $\vert W^{[L,\sigma]_L}\vert$. 
  \end{proposition}
  \begin{proof}
    By Lemma \ref{lemma_101}, we have a natural isomorphism, 
    \begin{equation*}
      \cA = \Hom_G(i_P^G(\Sigma),i_P^G(\Sigma))\simeq\Hom_L(\Sigma,\rho_{\overline{U}}^{[L,\sigma]_L}(i_P^G(\Sigma))).
    \end{equation*}
    By the naturality and the definition of $t_P$, this is an isomorphism as right $\cB$-module. By Theorem \ref{geometric_lemma}, more specifically by Lemma \ref{lemma_83}, we have,
    \begin{equation*}
      \gr(\rho_{\overline{U}}(i_P^G(\Sigma)))\simeq\bigoplus_{\{w_i\in \overline{P}\backslash G/P\sep\prescript{w_i}{}{L}=L\}}\prescript{w_i}{}{\Sigma}.
    \end{equation*}
    Observe that we have,
    \begin{equation*}
      \overline{P}\backslash G/P\supseteq \underbrace{(\overline{P}\cap N_G(L))}_{=L}\backslash N_G(L)/\underbrace{(P\cap N_G(L))}_{=L}=N_G(L)/L.
    \end{equation*}
    Indeed, suppose $e'\in L$ and $p = ue\in P$ be such that 
    \begin{equation*}
      uee'e^{-1}u^{-1}\in L.
    \end{equation*}
    Then we also have,
    \begin{equation*}
      u(ee'e^{-1}u^{-1}(ee'e^{-1})^{-1})ee'e^{-1}\in L. 
    \end{equation*}
    Note that 
    \begin{equation*}
      u = ee'e^{-1}u(ee'e^{-1})^{-1}.
    \end{equation*}
    That is $u$ commutes with any elements in $L$. Also observe that we have a torus $T\subseteq L$ and $\Cent_G(T)=T$. Thus we conclude $u=1$.
    Since $\Sigma$ is projective, we have 
    \begin{equation*}
      \rho_{\overline{U}}(i_P^G(\Sigma)) \simeq \bigoplus_{n\in N_G(L)/L}\prescript{n}{}{\Sigma}.
    \end{equation*}
    \begin{claim}
      FOr $n\in N_G(L)$, the following three conditions are equivalent.
      \begin{enumerate}
        \item $n\in N_G(L)^{[L,\sigma]_L}$.
        \item $\prescript{n}{}{\Sigma}\simeq\Sigma$.
        \item $\prescript{n}{}{\Sigma}\in\Rep^{[L,\sigma]_L}(L)$.
      \end{enumerate}
    \end{claim}
    \begin{claimproof}$\:$
      \par $a)\Rightarrow b)$. we have,
      \begin{align*}
        \prescript{n}{}{\Sigma}& = \prescript{n}{}{\cind_{L^\circ}^L\sigma^\circ},\\
        & = \cind_{L^\circ}^L(\prescript{n}{}{\sigma^\circ}).\\
        \prescript{n}{}{\sigma^\circ}&\hookrightarrow\prescript{n}{}{\sigma}|_{L^\circ}\simeq \sigma\nu|_{L^\circ}=\sigma|_{L^\circ}.(\nu\in X_{\nr}(L)).
      \end{align*}
    Since the isomorphism class of $\Sigma$ does not depend on the choice of subrepresentation of $\sigma|_{L^\circ}$. We obtain,
    \begin{equation*}
      \Sigma\simeq\prescript{n}{}{\Sigma}.
    \end{equation*}
    $b)\Rightarrow c)$ is obvious.
    \par $c)\Rightarrow a)$, suppose that $n\not\in N_G(L)^{[L,\sigma]_L}$. Then, 
    \begin{align*}
      \prescript{n}{}{\Sigma}\in\Rep^{[L,\prescript{n}{}{\sigma}]_L}\not=\Rep^{[L,\sigma]_L}(L).
    \end{align*}
    \end{claimproof}
    \begin{align*}
      \rho_{\overline{U}}^{[L,\sigma]_L}(i_P^G(\Sigma))\simeq \bigoplus_{w\in W^{[L,\pi]_L}}\underbrace{\prescript{w}{}{\Sigma}}_{\simeq\Sigma}\simeq\Sigma^{\vert W^{[L,\sigma]_L}\vert}.
    \end{align*}
    Furthermore, 
    \begin{equation*}
      \cA\simeq \Hom_{L}(\Sigma,\rho_{\overline{U}}^{[L,\sigma]_L}(i_P^G(\Sigma)))\simeq\Hom_L(\Sigma,\Sigma^{\vert W^{[L,\sigma]_L\vert}})\sim\cB^{\vert W^{[L,\sigma]_L\vert}}.
    \end{equation*}
  \end{proof}
  \subsection{Restriction with Parabolic induction}
  \begin{proposition}
    \label{proposition_103}
    The following diagram commutes up to natural equivalence.
    \begin{center}
      \begin{tikzcd}
{\Rep^{[L,\sigma]_G}(G)} \arrow[r, "\sim"]                  & \Mod(\cA)                                  \\
{\Rep^{[L,\sigma]_L}(L)} \arrow[u, "t_P"] \arrow[r, "\sim"] & \Mod(\cB) \arrow[u, "\cdot\tens{\cB}\cA"']
\end{tikzcd}
    \end{center}
  \end{proposition}
  \begin{proof}
    For $V\in\Rep^{[L,\sigma]_G}(G)$, we havea natural isomorphism,
    \begin{equation*}
      \Hom_G(i_P^G(\Sigma),V)\simeq\Hom_L(\Sigma,\rho_{\overline{U}}^{[L,\sigma]_L}(V)).
    \end{equation*}
    By the naturality, this is an isomorphism of right $\cB$-modules. In other words, the following diagram is commutative.
    \begin{center}
      \begin{tikzcd}
{\Rep^{[L,\sigma]_G}(G)} \arrow[r, "{\substack{V\mapsto\Hom_G(i_P^G(\Sigma),V)\\\\}}"] \arrow[d, "{\rho_{\overline{U}}^{[L,\sigma]_L}}"'] & \Mod(\cA) \arrow[d, "\res_\cB^\cA"] \\
{\Rep^{[L,\sigma]_L}(L)} \arrow[r, "{\substack{\\\\ \rho_{\overline{U}}^{[L,\sigma]_L}(V)\mapsto\Hom_L(\Sigma,\rho_{\overline{U}}^{[L,\sigma]_L}(V))}}"']                     & \Mod(\cB)                          
\end{tikzcd}
    \end{center}
    Since $i_P^G$ is a left adjoint to $\rho_{\overline{U}}^{[L,\sigma]_L}$ and $\cdot\tens{\cB}\cA$ is a left adjoint to $\res_\cB^\cA$, the proposition follows by uniqueness of left adjoints.
  \end{proof}

\subsection{Generic irreducibility}
\begin{definition}
  $\Irr^{[L,\sigma]_L}(L)$ is the set of irreducible representations in $\Rep^{[L,\sigma]_L}(L)$.
\end{definition}
\begin{remark}
  $W^{[L,\sigma]_L}$ acts on $\Irr^{[L,\sigma]_L}(L)$.
\end{remark}
\begin{definition}
  \begin{equation*}
    \chi_{\reg,\nn} = \left\{\tau\in\Irr^{[L,\sigma]_L}(L)\sep\substack{\text{$\tau$ is unitary}\\\text{$\Stab_{W^{[L,\sigma]_L}}(\tau)=1$}}\right\}.
  \end{equation*}
\end{definition}
\begin{lemma}
  \label{lemma_104}
  For $\tau\in\chi_{\reg,\nn}$, $i_P^G(\tau)$ is irreducible.
\end{lemma}
\begin{proof}
  Since the normalized parabolic induction $i_P^G$ preserves the unitaricity which we have seen in Homework 9, $i_P^G(\tau)$ is unitary. 
  In particular, $i_P^G(\tau)$ is semi-simple which is by Corollary \ref{corollary_13}.
  \par SO it suffices to show that $\End_\G(i_p^G(\tau))=\C$. By the same argument as Proposition \ref{proposition_102}, we have,
  \begin{align*}
    \End_G(i_P^G(\tau)) & = \Hom_G(i_P^G(\tau),i_P^G(\tau)),\\
    & = \Hom_L(\tau,\rho_{\overline{U}}^{[L,\sigma]_L}(i_P^G(\tau))),\\
    &=\Hom_L\left(\tau,\bigoplus_{w\in W^{[L,\sigma]_L}}\prescript{w}{}{\tau}\right).
  \end{align*}
  Since $\Stab_{W^{[L,\sigma]_L}}(\tau)=1$, we conclude,
  \begin{equation*}
    \Hom_L\left(\tau,\bigoplus_{w\in W^{[L,\sigma]_L}}\prescript{w}{}{\tau}\right)\simeq\C.
  \end{equation*}
\end{proof}
\begin{lemma}
  An irreducible supercuspidal representation with a unitary central character is unitary.
  \label{lemma_105_1}
\end{lemma}
\begin{proof}
  Exercise. Hint use \ref{lemma_11} that is use finiteness and irreducibility implies unitaricity.
\end{proof}
\begin{lemma}
  \label{lemma_105}
  There exists a unitary representation $\sigma_0\in\Irr^{[L,\sigma]_L}(L)$.
\end{lemma}
\begin{proof}
  Obviously we have $\sigma\in [L,\sigma]_L$. We would twist this by an unramified character. Note that we have a central character 
  \begin{equation*}
    \chi_{\sigma}:Z(L)\to\C^\times.
  \end{equation*}
  Taking the absolute value we have,
  \begin{equation*}
    \vert\chi_{\sigma}\vert:Z(L)\to\R_{>0}.
  \end{equation*}
  By the construction of $L^\circ$, we have $Z(L)\cap L^\circ\subseteq Z(L)$ which is compact. Therefore, we have a factorization,
  \begin{equation*}
    \vert \chi_{\sigma}\vert:Z(L)/(Z(L)\cap L^\circ)\to\R_{>0}.
  \end{equation*}
  Note that $L/L^\circ \cong \Z^r$ and $Z(L)/(Z(L)\cap L^0)$ is of finite index. Thus we can define,
  \begin{equation*}
    \tilde{\chi}:L/L^\circ\cong\Z^r\to\R_{>0}.
  \end{equation*}
  Let $\sigma_0 = \sigma\otimes \tilde{\chi}^{-1}\in\Irr^{[L,\sigma]_L}(L)$. Then by definition, $\sigma_0$ is a supercuspidal representation with a unitary central character.
  Conclude the proof with Lemma \ref{lemma_105_1}.
\end{proof}
\begin{definition}
  \begin{equation*}
    X_{\text{unitary}}(L) = \{\nu\in X_{\nr}(L)\sep \text{$\nu$ is unitary}\}.
  \end{equation*}
\end{definition}
\begin{remark}
  Recall that 
  \begin{equation*}
    \Hom_{\gr}(\underbrace{L/L^\circ}_{\simeq \Z^r},\C^\times)\simeq(\C^\times)^r.
  \end{equation*}
  Therefore, we can view this as an algebraic torus over $\C$ with the ring of regular functions $\C[L/L^\circ]$(group algebra oc $L/L^\circ$).
  \par Using this we obtain,
  \begin{equation*}
    X_{\text{unitary}}(L) = \Hom_{\gr}(L/L^\circ,S^1)\simeq (S^1)^r(\text{algebraic torus over $\R$}).
  \end{equation*}
  where, 
  \begin{equation*}
  S^1 = \Ker(\res_{\C/\R}(\G_m)\stackrel{N_n}{\G_m})(\R).
  \end{equation*}We further remark that $X_{\unitary}(L)$ is Zariski dense in $X_{\nr(L)}$.
\end{remark}

Using Lemma \ref{lemma_104}, we have,

\begin{equation*}
\gr(\underbrace{\rho_{\overline{U}}^{[L,\sigma]_L}(i_P^G(\tau))}_{\text{admissible}}) \cong \bigoplus_{w\in W^{[L,\sigma]_L}}\prescript{w}{}{\tau}.
\end{equation*}
By Corollary \ref{corollary_54}, we have,
\begin{equation*}
  \forall K\leq G,\text{$K$ is compact open}, \dim(\prescript{w}{}{\tau}^K)<\infty.
\end{equation*}
Thus we derive,
\begin{equation*}
  \rho_{\overline{U}}^{[L,\sigma]_L}(i_P^G(\tau)) \cong \bigoplus_{w\in W^{[L,\sigma]_L}}\prescript{w}{}{\tau}.
\end{equation*}
Also recall that 
\begin{equation*}
\Hom_G(i_P^G(\tau),i_P^G(\tau)) \cong \Hom_L(\tau,\rho_{\overline{U}}^{[L,\sigma]_L}(i_P^G(\tau))).
\end{equation*}

By Lemma \ref{lemma_59}, $\prescript{w}{}{\tau}$ is a subrepresentation of $\rho_{\overline{U}}^{[L,\sigma]_L}(i_P^G(\tau))$. Since 
\begin{equation*}
\{\prescript{w}{}{\tau}\sep w\in W^{[L,\sigma]_L}\}
\end{equation*}
is distinct that is 
\begin{equation*}
  w\not=w' \Rightarrow \prescript{w}{}{\tau}\not\cong\prescript{w'}{}{\tau}.
\end{equation*}
We conclude that we have an equality,
\begin{equation*}
  \rho_{\overline{U}}^{[L,\sigma]_L}(i_P^G(\tau))=\bigoplus_{w\in W^{[L,\sigma]_L}}\prescript{w}{}{\tau}.
\end{equation*}
\begin{lemma}
  \label{lemma_106}
  The set,
  \begin{equation*}
    \{\nu\in X_{\unitary}(L)\sep \Stab_{W^{[L,\sigma]_L}}(\sigma_0\nu)=1\},
  \end{equation*}
  is a Zariski dense subset of $X_{\nr}(L)$.
\end{lemma}

  \setcounter{claim}{0}

\begin{proof}
  Since $X_{\unitary}(L)$ is dense in $X_{\nr}(L)$. Thus it suffices to show that the set,
  \begin{equation*}
    \{\nu\in X_{\unitary}(L)\sep \Stab_{W^{[L,\sigma]_L}}(\sigma_0\nu)=1\}
  \end{equation*}
  is dense in $X_{\unitary}(L)$. Note that 
  \begin{equation*}
    \{\nu\in X_{\unitary}(L)\sep \Stab_{W^{[L,\sigma]_L}}(\sigma_0\nu)=1\} = X_{\unitary}\backslash\left(\bigcup_{1\not=w\in W^{[L,\sigma]_L}}\{\nu\sep \prescript{w}{}{(\sigma_0\nu)}\cong \sigma_0\nu\}\right).
  \end{equation*}
  Let $\chi_0$ be the central character of $\sigma_0$. Isomorphic representations have the same central character, we have,
  \begin{equation*}
    \{\nu\sep \prescript{w}{}{(\sigma_0\nu)}\cong \sigma_0\nu\}\subseteq\{\nu\sep \prescript{w}{}{(\chi_0\nu)}=\chi_0\nu\}.
  \end{equation*}
  \begin{claim}
      \begin{equation*}
        \phi_w:X_{\unitary}(L)\to X_{\unitary}(L), \phi(\nu) \defeq (\prescript{w}{}{\nu})\nu^{-1},
      \end{equation*}
      is non-trivial.
    \end{claim}
    \begin{claimproof}
      Suppose it is trivial. Then $w$ acts on $X_{\unitary}(L) = \Hom_{\gr}(L/L^\circ,S^1)$ as the identity.
      Then $w$ acts on $L/L^\circ$ as the identity. 
      \par Let $A_L$ denote the maximal split subtorus of $Z(L)$. Thenwe have 
      \begin{equation*}
        \Cent_G(A_L)=L.
      \end{equation*}
      For example, for $G=\GL_4\supseteq L$ such that 
      \begin{equation*}
      L = \begin{pmatrix}
        \GL_2&O_{2\times 2}\\
        O_{2\times 2} & \GL_2
      \end{pmatrix}.
      \end{equation*}
      and then we have,
      \begin{equation*}
        Z(L) = \left\{\begin{pmatrix}
          t& \: & \: &\: \\
          \: & t & \: & \:\\
          \: & \: & s & \:\\
          \: & \: & \: & s
        \end{pmatrix}
        \Bigg{|}t,s\in F^\times\right\} = A_L.
      \end{equation*}
      Then $w$ acts trivially on $A_L/\underbrace{(A_L\cap L^\circ)}$ where $A_L\cap L^\circ$ is a maximal compact subgroup of $A_L$. Since $A_L$ is a split torus, we have,
      \begin{equation*}
        A_L/(A_L\cap L^\circ)\tens{\Z}\R\cong X_k(A_L)\tens{\Z}\R, a\mapsto [X^*(A_L)\ni\chi\mapsto \val_F(\chi(a))].
      \end{equation*}
      where 
      \begin{equation*}
        X_k(A_L) \defeq \Hom_\Z(X^*(A_L),\Z), X^*(A_L) \defeq \Hom_{\text{alg-grp}}(A_L,\G_m).
      \end{equation*}
      Note that $N_G(L)$ acts on both sides of the map and the morphism is compatible with this action. Since $w$ acts on $A_L/(A_L\cap L^\circ)$, and $w$ acts tirivially on $X_k(A_L)$, and $X^*(A_L)$. Hence, $w$ acts on $A_L$ as the identity.
      Furthermore, $\Cent_G(A_L)=L,w=1$ in $N_G(L)/L\supseteq W^{[L,\sigma]_L}$.
    \end{claimproof}
  \begin{claim}
    \begin{equation*}
      \{\nu\sep \prescript{w}{}{(\chi_0\nu)}=\chi_0\nu\}
    \end{equation*}
    is a proper Zariski closed subset of $X_{\unitary}(L)$.
  \end{claim}
  \begin{claimproof}
    We consider an algebraic morphism 
    \begin{equation*}
      \phi_w:X_{\unitary}(L)\to X_{\unitary}(L), \phi(\nu) = (\prescript{w}{}{\nu})\nu^{-1}.
    \end{equation*}
    THen the set above is inverse image of $\chi_0\cdot(\prescript{w}{}{\chi_0})^{-1}$, hence is Zariski closed. 
  \end{claimproof}
\end{proof}

\begin{remark}
  From Lemma \ref{lemma_106}, we have,
  \begin{equation*}
    \{\nu\in X_{\nr}(L)\sep i_P^G(\sigma_0\nu)\text{ is irreducible}\} \supseteq \{\nu\in X_{\unitary}(L)\sep \Stab_{W^{[L,\sigma]_L}}(\sigma_0\nu)=1\}.
  \end{equation*}
  And $ \{\nu\in X_{\nr}(L)\sep i_P^G(\sigma_0\nu)\text{ is irreducible}\} $ is Zariski open in $X_{\nr}(L)$. For more details, see \cite{Renard} VI, 6.4,6.5.
\end{remark}

\subsection{Another Interpretation of Generic Irreducibility}

\begin{definition}
  We denote
  \begin{equation*}
    \Max Z(\cB) \defeq \{\text{maximal ideals of $Z(\cB)$}\},
  \end{equation*}
  and 
  \begin{equation*}
    \Irr(\cB) \defeq \{\text{simple right $\cB$-modules}\}.
  \end{equation*}
\end{definition}

The goal of this subsection is to show that 
\begin{equation*}
  \Max Z(\cB)\leftrightarrow \Irr(\cB)\leftrightarrow \Irr^{[L,\sigma]_L}(L).
\end{equation*}
Let us fix a simple right $\cB$-module $M$, which corresponds to $\sigma'\in\Irr^{[L,\sigma]_L}(L)$. That is 
\begin{equation*}
  M = \Hom_L(\cind_{L^\circ}^L(\sigma^\circ),\sigma').
\end{equation*}
By Schur's lemma, we have, $Z(\cB)$ acts on $M$ via a $\C$-algebra homomorphism,
\begin{equation*}
  \chi_M:Z(\cB)\to \C.
\end{equation*}
We set,
\begin{equation*}
  I = I_M\defeq \ker(\chi_M)\in\Max Z(\cB).
\end{equation*}

Consider a structure map $\phi_M:\cB\to\End_\C(M)$ that is 
\begin{equation*}
  \cB\ni b\mapsto [m\mapsto m\cdot b].
\end{equation*}
We know that this map factors through $\cB/I\cB$ thus we consider $\phi_M:\cB/I\cB\to \End_\C(M)$. Using Bernstein's theorem, this map is a surjection.
\par Since $\cB$ is free of rank $m^2$ over $Z(\cB)$, where $m$ is the multiplicity of $\sigma^\circ$ in $\sigma|_{L^\circ}$ by Corollary \ref{corollary_98}. We have,
\begin{equation*}
  \dim_\C\cB/I\cB = m^2.
\end{equation*}
On the other hand,
\begin{equation*}
  M = \Hom_G(\cind_{L^\circ}^L(\sigma^\circ),\sigma') \cong \Hom_{L^\circ}(\sigma^\circ,\underbrace{\sigma'|_{L^\circ}}_{=\sigma|_{L^\circ}}).
\end{equation*}
Hence, $\dim_\C M = m$ and $\dim_\C\End_\C(M)=m^2$. Thus 
\begin{equation*}
  \phi_M:\cB/I\cB \stackrel{\sim}{\to}\End_{\C}(M).
\end{equation*}
Moreover, 
\begin{equation*}
  \End_{\C}(M) = M\otimes \Hom_{\C}(M,\C).
\end{equation*}
As $\cB$ acts trivially on $\Hom_{\C}(M,\C)$ and to the right on $M$, we have,
\begin{equation*}
  \Hom_{\C}(M,\C)\stackrel{\Mod-\cB}{\cong} \M^{\oplus m}
\end{equation*}
\begin{lemma}
  \label{lemma_107}
  We have a bijection,
  \begin{center}
\begin{tikzcd}
\Max Z(\cB) \arrow[r, "\substack{IM\mapsfrom M\\}", shift left] & \Irr(\cB) \arrow[l, "\substack{\\I\mapsto M_I}", shift left] \arrow[r, shift left] & {\Irr^{[L,\sigma]_L}(L)} \arrow[l, shift left]
\end{tikzcd}
  \end{center}
  where $M_I$ is unique simple module such that 
  \begin{equation*}
    \cB/I\cB \simeq M_I^{\oplus m}.
  \end{equation*}
\end{lemma}
\begin{proof}
Let $I\in\Max(Z(\cB))$ and $M$ be a simple quotient of $\cB/I\cB$. Then we have $I_M = I$. Hence, the argument above, we have
\begin{equation*}
  \cB/I\cB\simeq M^{\oplus m}.
\end{equation*}
\end{proof}
Using the bijections, $\Max Z(\cB)\cong \Irr\cB\cong \Irr^{[L,\sigma]_L}(L)$, we regard $\Irr\cB$ and $\Irr^{[L,\sigma]_L}(L)$ as complex algebraic varieties over $\C$
with the ring of regular functions $Z(\cB)$.
\begin{lemma}
  \label{lemma_108}
  The map 
  \begin{equation*}
    X_{\nr}(L)\stackrel{\nu\mapsto\sigma_0\nu}{\twoheadrightarrow}\Irr^{[L,\sigma]_L}(L),
  \end{equation*}
  is algebraic.
\end{lemma}
\begin{corollary}
  \label{corollary_109}
  The set $\chi_{\reg,\nn}$ is Zariski dense in $\Irr^{[L,\sigma]_L}(L)$.
\end{corollary}
\begin{proof}$\:$
  \begin{center}
    \begin{tikzcd}
X_{\nr}(L) \arrow[r, two heads]                                                                                                 & {\Irr^{[L,\sigma]_L}(L)}          \\
{\{\nu\in X_{\unitary}(L)\sep \Stab_{W^{[L,\sigma]_L}}(\sigma_0\nu)=1\}} \arrow[u, "\text{Zariski dense}"] \arrow[r, two heads] & {\chi_{\reg,\nn}} \arrow[u, hook]
\end{tikzcd}
  \end{center}
\end{proof}

\begin{align*}
  X_{\nr}(L) & = \Hom_{\text{grp}}(L/L^\circ,\C^\times),\\
  & = \Hom_{\C-\text{alg}}(\C[L/L^\circ],\C),\\
  &\cong \Max(\C[L/L^\circ]).
\end{align*}

That is algbraic variety over $\C$ with ring of regular functions $\C[L/L^\circ]$. 

\begin{equation*}
  \Irr^{[L,\sigma]_L}(L)\cong \Irr\cB \cong \Max(Z(\cB)),
\end{equation*}
ie an algebraic variety over $\C$ with ring of regular functions $Z(\cB)$. 

\begin{lemma}
  \label{lemma_108}
  The map,
  \begin{equation*}
    X_{\nr}(L)\to\Irr^{[L,\sigma]_L}(L), v\mapsto \sigma\circ\nu.
  \end{equation*}
  where the image $\sigma$ is fixed unitary irreducible representation in $\Irr^{[L,\sigma]_L}(L)$, is algebraic.
\end{lemma}
\begin{proof}
  \begin{align*}
    \cB & = \End_L(\cind_{L^\circ}^L(\sigma^\circ)),\\
    & = \hecke(L,L^\circ,\sigma^\circ),\\
    & = \left\{\varphi:L\to\End_\C(V_{\sigma^\circ})\:\bigg{|}\:\begin{cases}
    \varphi(e_1ee_2) = \sigma^\circ(e_1)\circ\varphi(e)\circ\sigma^\circ(ee_2)\quad (e_1,e_2\in L^\circ,e\in L),\\
    \supp\varphi\text{ is a finite union of double cosets $L^\circ eL^\circ$}.
    \end{cases}\right\}
  \end{align*}
  Note that $L^\circ\subseteq T\subseteq L,T=\bigcap_{\nu\in X_{\nr}(L)_\sigma}\ker\nu$. Recall that 
  \begin{equation*}
    X_{\nr}(L)_\sigma = \{\nu\in X_{\nr}(L)\sep \sigma\nu\cong \sigma\}.
  \end{equation*}
  Using Proposition \ref{proposition_97},
  \begin{equation*}
    Z(\cB) \cong\hecke(T,L^\circ,\sigma^\circ)\subseteq\hecke(L,L^\circ,\sigma^\circ)\subseteq\hecke(L,L^\circ,\sigma^\circ).
  \end{equation*}
\end{proof}

\begin{lemma}
  We have a non-trivial (ie. depends on the choise of base point $\sigma_0$) isomorphism,
  \begin{equation*}
    \hecke(T,L^\circ,\sigma^\circ) \cong\C[\underbrace{T/L^\circ}_{\subseteq L/L^\circ, \text{free abelian group of finite rank}}].
  \end{equation*}
  \label{lemma_110}
\end{lemma}

\begin{proof}
  According to Lemma \ref{lemma_95}, up to isomorphism, we have, a unique subrepresentation, $\sigma_T$, of $\sigma_0|_{T}$ such that $\sigma_T|_{L^\circ} = \sigma^\circ$. We have,
  \begin{align*}
    L^\circ\subseteq & T\subseteq L,\\
    \sigma^\circ \to \sigma_T\hookrightarrow \sigma_0|_T.
  \end{align*}
  For $[t]\in T/L^\circ$, let $\varphi_{[t]}\in\hecke(T,L^\circ,\sigma^\circ)$ be the element such that 
  \begin{enumerate}[i).]
    \item $\supp\varphi_{[t]}=[t]=tL^\circ$,
    \item $\varphi_{[t]}(x) = \underbrace{\sigma_T(x)}_{\in\End_\C(V_{\sigma_T})=\End_\C(V_{\sigma^\circ})}$ for $x\in[t]$
  \end{enumerate}
  Then with respect to an appropriate Haar measure, we have a $\C$-algebra isomorphism,
  \begin{equation*}
    \hecke(T,L^\circ,\sigma^\circ)\cong \C[T/L^\circ], \varphi_{[t^{-1}]}\mapsfrom b_{[t]}.
  \end{equation*}
\end{proof}

\begin{remark}
  \begin{equation*}
    \hecke(T,L^\circ,\sigma^\circ)\cong\C[T/L^\circ], \varphi_{[t^{-1}]}\mapsfrom b_{[t]},
  \end{equation*}
  is also an $\C$-algebra isomorphism.
\end{remark}
 The reason why we use $\varphi_{[t^{-1}]}$ here is the following lemma.
\begin{lemma}
  For $\nu\in X_{\nr}(L)$, the element $\varphi_{[t]},[t]\in T/L^\circ$ of $[t]$ as the right $\cB$-module, 
  \begin{equation*}
  \Hom_L(\cind_{L^\circ}^L(\sigma^\circ),\sigma_0\nu),
  \end{equation*}
  by multiplication by $\nu^{-1}(t)$. 
  \label{lemma_111}
\end{lemma}

\begin{proof}
  Follows from the claim in the proof of Proposition \ref{proposition_97}.
\end{proof}

\begin{proof}[Proof of Lemma \ref{lemma_108}]
  \begin{center}
\begin{tikzcd}
X_{\nr}(L) \arrow[r, "\substack{\nu\mapsto\sigma_0\nu\\\:}"] & {\Irr^{[L,\sigma]_L}(L)} \arrow[r, "{\substack{\sigma_0\nu\mapsto\Hom_L(\cind_{L^\circ}^L(\sigma^\circ),\sigma_0\nu)\\\:}}"] & \Irr(\cB) &  \Max(Z(\cB)) & {\Max(\C[T/L^\circ])}
\end{tikzcd}
  \end{center}
  Using Lemma \ref{lemma_111}, $b_{[t]}\in\C[T/L^\circ]\cong Z(\cB)$ acts on 
  \begin{equation*}
    \Hom_L(\cind_{L^\circ}^L(\sigma^\circ),\sigma_0\nu),
  \end{equation*}
  by $\nu(t)$. Hence the map,
  \begin{center}
    \begin{tikzcd}
X_{\nr}(L) \arrow[r] \arrow[d, "\cong"']                     & {\Irr^{[L,\sigma]_L}(L)} \arrow[d, "\cong"] \\
{\Hom_{\C-\text{alg}}(\C[L/L^\circ],\C)} \arrow[d, "\cong"'] & {\Max(\C[T/L^\circ])}                       \\
{\Max(\C[L/L^\circ])}                                        &                                            
\end{tikzcd}
  \end{center}
  in induced by the natural inclusion,
  \begin{equation*}
    \C[T/L^\circ]\hookrightarrow\C[L/L^\circ].
  \end{equation*}
\end{proof}

\begin{corollary}
  \label{corollary_109}
  The set 
  \begin{equation*}
    \chi_{\reg,\nn}=\left\{\sigma\in\Irr^{[L,\sigma]_L}(L)\:\bigg{|}\: \begin{cases}
      \sigma\text{ is unitary},\\
      \Stab_{W^{[L,\sigma]_L}}(\sigma)=1.
    \end{cases}
    \right\}
  \end{equation*}
\end{corollary}

\begin{corollary}
  \label{corollary_112}
  Let $\mathfrak{J}_{\reg,\nn}$ denote the image of $\chi_{\reg,\nn}$ via the bijection,
  \begin{equation*}
    \Irr^{[L,\sigma]_L}(L)\cong\Max Z(\cB).
  \end{equation*}
  Then we have,
  \begin{equation*}
    \bigcap_{I\in\mathfrak{J}}I=0.
  \end{equation*}
\end{corollary}
\begin{theorem}
  \label{theorem_113}
  As subalgebras of $\cA=\End_G(i_P^G(\Sigma))$ where $\Sigma=\cind_{L^\circ}^L(\sigma^\circ)$. We have,
  \begin{equation*}
    Z(\cA)\subseteq Z(\cB).
  \end{equation*}
  Note that 
  \begin{equation*}
    \cA = \End(i_P^G(\Sigma))\supseteq\cB = \End_L(\Sigma).
  \end{equation*}
  In particular $Z(\cA)$ is a 
\end{theorem}
\begin{proof}
  By Corollary \ref{corollary_98}, $\cB$ is free of finite rank over $Z(\cB)$. Moreover, $\cA$ is a free rank $\cB$-module of finite rank,
  \begin{equation*}
    \cA\supseteq\cB\supseteq Z(\cB).
  \end{equation*}
  Combining these, $\cA$ is free right $Z(\cB)$ module of finite rank. Now fix a $Z(\cB)$-basis of $\cA$.
  Let $z\in Z(\cA)$. With respect to the $Z(\cB)$-basis of $\cA$, the multiplication of (missing) on $\cA$, is representted b a matrix $(z_{ij})_{ij}$ for $z_{ij}\in Z(\cB)$, It suffices to show that $(z_{ij})_{ij}$ is a scalar matrix.
  \par By Corollary \ref{corollary_112}, it suffices to show that $(z_{ij \mod I})_{ij}$ is a scalar matrix for all $I\in\mathfrak{J}_{\reg,\nn}$. Observe that the matrix $(z_{ij}\mod I)$ represents an action of $Z(\cA)$ on $\cA/I\cA$.
  \par For $I\in\mathfrak{J}_{\reg,\nn}$, let $\M_I\in\Irr(\cB)$ and $\sigma_I\in\chi_{\reg,\nn}\subseteq \Irr^{[L,\sigma]}(L)$ denote the corresponding elements under the bijections,
  \begin{equation*}
    \underbrace{\Irr^{[L,\sigma]_L}(L)}_{\supseteq \chi_{\reg,\nn}}\leftrightarrow\Irr(\cB)\leftrightarrow\underbrace{\Max(Z(\cB))}_{\supseteq \mathfrak{J}_{\reg,\nn}}
  \end{equation*}
  By Lemma \ref{lemma_107}, we have 
  \begin{equation*}
    \cB/I\cB\cong M_I^{\oplus m}.
  \end{equation*}
  By Proposition \ref{proposition_103}, we have the following commutative diagram,
  \begin{center}
    % https://tikzcd.yichuanshen.de/#N4Igdg9gJgpgziAXAbVABwnAlgFyxMJZABgBpiBdUkANwEMAbAVxiRAB12AlGNAPWDIAMqU7YA5gFs6FAPoBxAL4AKeQEoQi0uky58hFAEZyVWoxZtOAWWgBaTgGMAgpu0gM2PASJlDp+sysiBzcvALCouwS0nJCKkIaWjqe+kTGftQBFsHWdo4AQpqmMFDi8ESgAGYAThCSSGQgOBBIxmaBluwOBOKuVbX1iABM1M1IAMzUDHQARjAMAAq6XgYg1VjiABY4IJnmQSHdYL1JIDV1E6MtiG1ZB45QEDicMxtPWJLwssAFio4uU1m8yWKW8wXWWx2p3OgxGTWujTubCwsgWfHkRUUQA
\begin{tikzcd}
{\Rep^{[L,\sigma]_G}(G)} \arrow[r, "\cong"]                     & \Mod-\cA                                        \\
{\Rep^{[L,\sigma]_L}(L)} \arrow[r, "\cong"'] \arrow[u, "i_P^G"] & \Mod-\cB \arrow[u, "\cdot\tens{\cB}\cA"']
\end{tikzcd}
  \end{center}
  More explicitly,
  \begin{center}
    \begin{tikzcd}
i_P^G(\sigma_I^{\oplus m})\cong i_P^G(\sigma_I)^{\oplus m} & \cB/I\cB\tens{\cB}\cA\cong \cA/I\cA \\
\sigma^{\oplus m}_I \arrow[r] \arrow[u]                    & M_I^{\oplus m} \arrow[l] \arrow[u] 
\end{tikzcd}
  \end{center}
  According to Lemma \ref{lemma_104}, $i_P^G(\sigma_I)$ is irreducible. Hence the center $Z(\Rep^{[L,\sigma]_G}(G))\cong Z(\cA)$ acts on $i_P^G(\sigma_I)$ and $i_P^G(\sigma_I)$ and $i_P^G(\sigma_I)^{\oplus m}$ by a certain scalar.
  \par Thus $Z(\cA)$ acts on $\cA/\cA$ by the some scalar. THence, the matrix $(z_{ij}\mod I)_{ij}$ is a scalar matrix.
\end{proof}
\begin{corollary}
  \label{corollary_114}The Bernstein block $\Rep^{[L,\sigma]_G}(G)$ is indecomposable.
\end{corollary}
\begin{proof}
Same as Theorem \ref{theorem_77}.
\end{proof}
\subsection{Description of Bernstein Center}
We have seen 
\begin{equation*}
  Z(\cA)\subseteq Z(\cB).
\end{equation*}
Furthermore, we have,
\begin{equation*}
  \Irr^{[L,\sigma]_L}(L)\cong \Max(Z(\cB)).
\end{equation*}
Therefore, we can regard,
\begin{equation*}
  Z(\cB)\cong \underbrace{\Ouv(\Irr^{[L,\sigma]_L}(L))}_{\text{ring of regular functions on $\Irr^{[L,\sigma]_L}(L)$}}.
\end{equation*}
Explicitly, this is expressed as,
\begin{equation*}
  z\mapsto[\sigma'\mapsto z(\sigma')\in\C],
\end{equation*}
the scalar of the action of $z$ on $\sigma'$.
\par We will describe $Z(\cA)$ inside $Z(\cB)$.
\begin{equation*}
  W^{[L,\sigma]_L}=\{n\in N_G(L)\sep
    \prescript{n}{}{\sigma}\sigma\nu\text{ for some $\nu\in X_{\nr}(L)$}\}/L.
\end{equation*}
With the algebraic action on $\Irr^{[L,\sigma]_L}(L)$. Therefore,
\begin{equation*}
  W^{[L,\sigma]_L}\curvearrowright\Ouv(\Irr^{[L,\sigma]_L}(L))\cong Z(\cB).
\end{equation*}
\begin{theorem}
  We have,
  \begin{equation*}
    Z(\cA) = Z(\cB)^{W^{[L,\sigma]_L}}\cong\Ouv(\Irr^{[L,\sigma]_L}(L)/W^{[L,\sigma]_L}).
  \end{equation*}
  For each $z\in Z(\cA)$, $z$ acts on $i_P^G(\tau)$ for $\tau\in\Irr^{[L,\sigma]_L}(L)$ exactly as $z$ viewed as an element of $Z(\cB)$ acts on $\tau$.
  (For $X=\Spec(R)$ and $W$ acting on $X$, we denote $X\defeq X/W = \Spec R^W$).
  \label{theorem_115}
  $z\in \cB$ acts on $\tau\in\Rep^{[L,\sigma]_L}(L)$ as a scalar. Thus $z$ acts on $i_P^G(\tau)$ as a scalar with the same value as the element of $\cB$.
\end{theorem}
  \setcounter{claim}{0}
\begin{proof}
  We first prove the second claim. Let $z\in \cA\subseteq\cB$ and $\tau\in\Irr^{[L,\sigma]_L}(L)$. We write $M_\tau=\Hom_L(\Sigma,\tau)\in\Irr\cB$. By Proposition \ref{proposition_103}, we have the following commutative diagram.
  \begin{center}
    \begin{tikzcd}
{\Rep^{[L,\sigma]_G}(G)} \arrow[r]                    & \Mod-\cA                             \\
{\Rep^{[L,\sigma]_L}(L)} \arrow[u, "i_P^G"] \arrow[r] & \Mod-\cB \arrow[u, "\tens{\cB}\cA"']
\end{tikzcd}
  \end{center}
  More explicitly,
  \begin{center}
    \begin{tikzcd}
i_P^G(\tau) \arrow[r]    & M_\tau\tens{\cB}\cA \\
\tau \arrow[u] \arrow[r] & M_\tau \arrow[u]   
\end{tikzcd}
  \end{center}
  Since $z$ acts on $M_\tau$ and $M_\tau\tens{\cB}\cA$ by the same scalar, we also obtain that $z$ acts on $\tau$ and $i_P^G(\tau)$ by the same scalar.
  \par We now prove the first claim. That is $Z(\cA)\subset Z(\cB)^{W^{[L,\sigma]_L}}$. Let $z\in Z(\cA),\tau\in\Irr^{[L,\sigma]_L}(L)$, and $w\in W^{[L,\sigma]_L}$.
  Then we have $i_P^G(\tau)\cong i_P^G(\prescript{w}{}{\tau})$. By Proposition \ref{proposition_84},$i_{\prescript{w}{}{P}}^G(\prescript{w}{}{\tau}),i_P^G(\prescript{w}{}{\tau})$ share the same subquotient.
  Hence, $z$ acts on $i_P^G(\tau)\cong i_{\prescript{w}{}{\tau}}^G$ and $i_P^G(\prescript{w}{}{\tau})$ by the same scalar. Combining these observation with the first part, we obtain that regarding $z\in Z(\cA)\subseteq Z(\cB)\cong \Ouv(\Irr^{[L,\sigma]_L}(L))$,
  \begin{equation*}
    z(\tau) =z(\prescript{w}{}{\tau}).
  \end{equation*}
  Hence $z\in Z(\cB)^{W^[L,\sigma]_L}$.
  \par We now prove the other inclusion. That is 
  \begin{equation*}
    Z(\cB)^{W^{[L,\sigma]_L}}\subseteq Z(\cA)\curvearrowright\Rep^{[L,\sigma]_G}(G).
  \end{equation*}
  Let $z\in Z(\cB)^{W^{[L,\sigma]_L}}$. Consider the action of $z$ on the right $\cA-\Mod,\Hom_G(i_P^G(\Sigma),i_P^G(\tau))$ for $\tau\in X_{\reg,\nn}\subseteq\Irr^{[L,\sigma]_L}(L)$. By Theorem \ref{theorem_second_adjunction}, we have
  \begin{equation*}
    \Hom_G(i_P^G(\Sigma),i_P^G(\tau))\cong \Hom_L(\Sigma,\rho_{\overline{U}}^{[L,\sigma]_L}(i_p^G(\tau))),
  \end{equation*}
  as right $\cB$-modules. By Theorem \ref{geometric_lemma} and the fact $\tau\in\chi_{\reg,\nn}$, we have,
  \begin{equation*}
    \rho_{\overline{U}}^{[L,\sigma]_L}(i_P^G(\tau))\cong\gr(\rho_{\overline{U}}^{[L,\sigma]_L}(i_P^G(\tau)))\cong\bigoplus_{w\in W^{[L,\sigma]_L}}\prescript{w}{}{\tau},
  \end{equation*}
  as in the proof of Lemma \ref{lemma_104}. That is as $\tau$ is regular, $\prescript{w}{}{\tau}$ is distinct and conclude similarly as in the proof. Hence, 
  \begin{equation*}
    \Hom_G(i_P^G(\Sigma),i_P^G(\tau)) \cong \bigoplus_{w\in W^{[L,\sigma]_L}}\Hom_L(\Sigma,\prescript{w}{}{\tau}).
  \end{equation*}
  Note that 
  \begin{align*}
    \Irr\cB&\leftrightarrow\Irr^{[L,\sigma]_L}(L)\curvearrowleft Z(\cB),\\
    \Hom_L(\Sigma,\prescript{w}{}{\tau})&\leftrightarrow\prescript{w}{}{\tau}.
  \end{align*}
  Since $z\in Z(\cB)^{W^{[L,\sigma]_L}}$, $z$ acts on $\prescript{w}{}{\tau}$ and $\Hom_L(\Sigma,\prescript{w}{}{\tau})$ by a scalar that is independent of $w\in W^{[L,\sigma]_L}$.
  Hence, $z$ acts on $\Hom_G(i_P^G(\Sigma),i_P^G(\tau))$ by this scalar.
  \par We suppose that $\tau$ corresponds to $I\in\mathfrak{J}_{\reg,\nn}\subseteq\Max Z(\cB)$ and write
  \begin{equation*}
    \tau=\sigma_I, M_I = \Hom_L(\Sigma,\tau)\in\Irr(\cB),
  \end{equation*}
  and,
  \begin{equation*}
    \Ind  M_I = M_I\tens{\cB}\cA\cong\Hom_G(i_P^G(\Sigma),i_P^G(\tau)).
  \end{equation*}
  For $w\in W^{[L,\sigma]_L}$, let $\prescript{w}{}{I}$ denote the element of $\mathfrak{J}_{\reg,\nn}$ corresponding to $\prescript{w}{}{\tau}=\prescript{w}{}{\sigma}_I$.
  \par By definition,
  \begin{equation*}
    I\subseteq\Ann(M_I)=\{b\in\cB\sep \forall m\in M_I, mb=0\}.
  \end{equation*}
  Hence 
  \begin{align*}
    \bigcap_{w\in W^{[L,\sigma]_L}}\prescript{I}{}{I}&\subseteq\Ann\left(\bigoplus_{w\in W^{[L,\sigma]_L}}M_{\prescript{w}{}{I}}\right)\\
    &=\bigoplus_{w\in W^{[L,\sigma]_L}}\Hom_L(\Sigma,\prescript{w}{}{\sigma}_I),\\
    &\cong \Hom_G(i_P^G(\Sigma),i_P^G(\sigma_I)),\\
    & = \Ind M_I.
  \end{align*}
  By the generic irreducibility (ie. Lemma \ref{lemma_104}), $i_P^G(\sigma-I)$ is irreducible. Hence, $\Ind M_I$ is a simple right $\cA$-module. Thus, by Bernstein's theorem, the structure map,
  \begin{equation*}
    \phi_I:\cA\to \End_\C(\Ind M_I),
  \end{equation*}
  is surjective. Now, we have,
  \begin{equation*}
    \overline{\phi}_I:\cA/\cA\left(\bigcap_{w\in W^{[L,\sigma]_L}}\prescript{w}{}{I}\right)\twoheadrightarrow\cA/\Ann(\Ind M_I)\twoheadrightarrow\End_\C(\Ind M_I).
  \end{equation*}
  \begin{claim}
    $\overline{\phi}_I$ is an isomorphism hence,
    \begin{equation*}
      \cA\left(\bigcap_{w\in W^{[L,\sigma]_L}}\prescript{w}{}{L}\right)=\Ann(\Ind M_I).
    \end{equation*}
  \end{claim}
  \begin{claimproof}
    We will compare the dimensions. Since 
    \begin{align*}
      \Ind M_I=\Hom_G(i_P^G(\Sigma),i_P^G(\tau))\cong \bigoplus_{w\in W^{[L,\sigma]_L}}\Hom_L(\Sigma,\prescript{w}{}{\sigma}_I),
    \end{align*}
    and by Frobenius reciprocity,
    \begin{equation*}
      \Hom_L(\underbrace{\Sigma}_{\cind_{L^\circ}^L(\sigma^\circ)},\prescript{w}{}{\sigma}_I)\cong \Hom_{L^\circ}(\sigma^\circ,\underbrace{\prescript{w}{}{\sigma}_I|_{L^\circ}}_{=\sigma|_{L^\circ}}).
    \end{equation*}
    Thus,
    \begin{equation*}
      \dim_\C(\Ind M_I)=m\cdot\vert W^{[L,\sigma]_L}\vert,
    \end{equation*}
    where $m$ is the multiplicity of $\sigma^\circ$ in $\sigma|_{L^\circ}$.
    \begin{equation*}
      \dim_\C(\End_\C(\Ind M_I)) = m^2\cdot\vert W^{[L,\sigma]_L}\vert^2.
    \end{equation*}
    On the other hand, by Corollary \ref{corollary_98} and Proposition \ref{proposition_102}, $\cA$ is free $Z(\cB)$-module of rank $m^2\cdot\vert W^{[L,\sigma]_L}\vert$. 
    \begin{equation*}
      Z(\cB)\overbrace{\underbrace{\subseteq}_{m^2}}^{\text{Cor. \ref{corollary_98}}}\cB\overbrace{\underbrace{\subseteq}_{\vert W^{[L,\sigma]_L}\vert}}^{\text{Prop. \ref{proposition_102}}}\cA.
    \end{equation*}
    Hence,
    \begin{equation*}
      \cA/\cA\left(\bigcap{w\in W^{[L,\sigma]_L}}\prescript{w}{}{I}\right)\cong \left(Z(\cB)/\bigcap_{w\in W^{[L,\sigma]_L}}\prescript{w}{}{I}\right)^{m^2\vert W^{[L,\sigma]_L}\vert}.
    \end{equation*}
    Since $I\in\mathfrak{J}_{\reg,\nn}$, for each $w\in W^{[L,\sigma]_L}$, $\prescript{w}{}{I}$ is distinct. Thus by Chinese remainder theorem, we have,
    \begin{equation*}
      Z(\cB)/\bigcap_{w\in W^{[L,\sigma]_L}}\prescript{w}{}{I}\cong \prod_{w\in W^{[L,\sigma]_L}}\underbrace{Z(\cB)/\prescript{w}{}{I}}_{\cong \C}.
    \end{equation*}
    THus,
    \begin{equation*}
      \dim_\C\left(\cA/\cA\left(\bigcap_{w\in W^{[L,\sigma]_L}}\prescript{w}{}{I}\right)\right) = m^2\cdot\vert W^{[L,\sigma]_L}\dim_C\left(Z(\cB)/\bigcap_{w}\prescript{w}{}{I}\right)=m^2\vert W^{[L,\sigma]_L}\vert^2.
    \end{equation*}
  \end{claimproof}
  Now, we claim that the product of structure maps,
  \begin{equation*}
    \cA\to\prod_{I\in\mathfrak{J}_{\reg,\nn}}\End_\C(\Ind M_I)
  \end{equation*}
  is injective. By the claim, its kernel is 
  \begin{equation*}
    \bigcap_{I\in\mathfrak{J}_{\reg,\nn}}\Ann(\Ind M_I)=\bigcap_{I\in\mathfrak{J}_{\reg,\nn}}\cA\left(\bigcap_{w\in W^{[L,\sigma]_L}}\prescript{w}{}{I}\right).
  \end{equation*}
  Since $\cA$ is a finite $Z(\cB)$-module, we have,
  %detailed explanation missing in the argument below. See pictures.
  \begin{align*}
    \bigcap_{I\in\mathfrak{J}_{\reg,\nn}}\cA\left(\bigcap_{w\in W^{[L,\sigma]_L}}\prescript{w}{}{I}\right) &\cong \cA\left(\bigcap_{I\in \mathfrak{J}_{\reg,\nn}}\bigcap_{w\in W^{[L,\sigma]_L}}\prescript{w}{}{I}\right),\\
    & \overbrace{\cong}^{\text{Cor. \ref{corollary_112}}}\cA\left(\bigcap_{I\in\mathfrak{J}_{\reg,\nn}}I\right),\\
    & = 0.
  \end{align*}
  For $z\in Z(\cB)^{W^{[L,\sigma]_L}}$, we proved thta $z$ acts on $\Ind M_I$ by a scalar for each $I\in \mathfrak{J}_{\reg,\nn}$. Hence the image of $z$ in $\prod_{I\in\mathfrak{J}_{\reg,\nn}}\End_\C(\Ind M_I)$ via 
  \begin{equation*}
    \cA\hookrightarrow\prod_{I\in\mathfrak{J}_{\reg,\nn}}\End_\C(\Ind M_I),
  \end{equation*}
  is contained in the center of $\prod_{I\in\mathfrak{J}_{\reg,\nn}}\End_\C(\Ind M_I)$. Since the map $\cA\hookrightarrow\End_\C(\Ind M_I)$ is injective, we conclude that $z\in Z(\cA)$.
  That is $f=(f_I)_{I}\in\prod_{I}\End_\C(\Ind M_I),z=(\underbrace{c_I}_{\C})_I$, 
  \begin{equation*}
    f\cdot z = (f_I\cdot c_I)_{I}=(c_I f_I)_I = z\cdot f.
  \end{equation*}
\end{proof}

\subsection{The Structure of $\Rep^{[L,\sigma]_G}(G)$ and Theory of Types}

\begin{example}
Consider $G=\SL_2(F)$. Take $L=G$ and $\sigma=\pi\in\Irr_{\supcus}(\SL_2)$. Then,
\begin{center}
  \begin{tikzcd}
{\Rep^{[\SL_2,\pi]_{\SL_2}}(\SL_2)} \arrow[r, "\pi\mapsto\C"] & \Vect_\C
\end{tikzcd}
\end{center}
That is any object in $\Rep^{[\SL_2,\pi]_{\SL_2}}(\SL_2)$ is of the form,
\begin{equation*}
  \pi\oplus\cdots\oplus\pi.
\end{equation*}
\end{example}

\begin{example}
  Consider 
  \begin{equation*}
    L = T\defeq\{\diag(t,t^{-1})\sep t\in F^\times\}, B = \left\{\begin{pmatrix}
      t & x\\
      0 & t^{-1}
    \end{pmatrix}\:\bigg{|}\: t\in F^\times,x\in F\right\}.
  \end{equation*}
  Then principal block is 
  \begin{equation*}
    \Rep^{[T,\triv_T]_{\SL_2}}(\SL_2)\ni\Ind_B^{\SL_2}(\triv_T).
  \end{equation*}
  Then we have a short exact sequence,
  \begin{center}
    \begin{tikzcd}
1 \arrow[r] & \triv_{\SL_2} \arrow[r] & \Ind_B^{\SL_2}(\triv_T) \arrow[r] & \underline{\St} \arrow[r] & 1
\end{tikzcd}
  \end{center}
  where $\underline{\St}$ is the Steinberg representation. This sequence however, does not split.
\end{example}

We will now examine the theory of types. The main reference of this part is \cite{Buschnell}.

\begin{definition}[Type]
  Let $K\subseteq G$ be a compact open subgroup and $\rho$ is an irreducible smooth representation of $K$. A pair $(K,\rho)$ is a type associated with a Bernstein block $\Rep^{[L,\sigma]_G}(G)$ or an $[L,\sigma]_G$-type if for any irreducible representation $(\pi,V)$ of $G$, the following conditions are equivalent.
  \begin{enumerate}[i).]
    \item $(\pi,V)\in\Rep^{[L,\sigma]_G}(G)$.
    \item $\Hom_K(\rho,\pi|_K)\not=0$.
  \end{enumerate}
\end{definition}

\begin{example}
  \label{example_first_example_type}
  Let $G=\SL_2(F)$. Suppose that $\pi_\rho\defeq\cind_K^{\SL_2(F)}(\rho)$ is irreducible and supercuspidal. Then $(K,\rho)$ is an $[\SL_2,\pi_\rho]_{\SL_2}$-type. 
\end{example}

\begin{remark}
  Every irreducible and supercuspidal representation of $\SL_2(F)$ is obtained by a compact induction.
\end{remark}

\begin{example}
  Consider,
  \begin{equation*}
    I=I_w\defeq\left\{\begin{pmatrix}
      a & b\\
      c & d 
    \end{pmatrix}
    \in\SL_2(\Ouv_F)\:\bigg{|}\:c\in\mathfrak{p}_F\right\}=\left\{\begin{pmatrix}
      \Ouv_F^\times & \Ouv_F\\
      \mathfrak{p}_F & \Ouv_F^\times
    \end{pmatrix}
    \right\}.
  \end{equation*}
  This is called a Iwahori subgroup. Then $(I,\triv_I)$ is a type associated with $\Rep^{[T,\triv_T]_{\SL_2}}(\SL_2)$.
\end{example}
The following theorem is a generalization of Example \ref{example_first_example_type}
\begin{theorem}
  Let $Z(G)\subseteq\tilde{K}\subseteq G$ be a compact subgroup such that $\tilde{K}/Z(G)$ is compact. Let $\tilde{\rho}$ be an irreducible representation of $\tilde{K}$ and $\pi = \cind_{\tilde{K}}^G(\tilde{\rho})$ and suppose $\pi$ is irreducible and supercuspidal.
  Let $K=\tilde{K}\cap G^\circ$ be the maximal compact subgroup of $\tilde{K}$ and $\rho\hookrightarrow\tilde{\rho}|_K$ be an irreducible subrepresentation. Then $(K,\rho)$ is a $[G,\pi]_G$-type.
\end{theorem}

\begin{proof}
  Exercise. The proof can also be found in \cite{Buschnell} 5.4.
\end{proof}

\begin{definition}[G-covers]
  Let $L\subseteq G$ be a Levi subgroup and $K_L\subseteq L,K\subseteq G$ be compact open subgroups. Let $\rho$ be an irreducible represenattion of $K$ and $\rho_L$ be an irreducible represenattion of $K_L$. We say $(K,\rho)$ is a $G$-cover of $(K_L,\rho_L)$, if for any parabolic subgroup $P=LU$ and its opposite parabolic subgroup $\overline{P}=L\overline{U}$.
  \begin{enumerate}[i).]
    \item $K=(K\cap U)(K\cap L)(K\cap\overline{U})$ and $K\cap L = K_L$.
    \item $\rho|_{K_L}=\rho_L$ and $\rho|_{K\cap U}$ and $\rho_{K\cap U}$ and $\rho|_{K\cap\overline{U}}$ are $\id$-isotypic. That is for $k\in K\cap U$ we have $\rho(k):V_\rho\to V_\rho$ is an identity.
    \item For any irreducible represenattion $(\pi,V)$ of $G$, the restriction of the upper row to the lower row in 
    \begin{center}
      \begin{tikzcd}
V \arrow[r, two heads]                   & V_U \\
{V^{(K,\rho)}} \arrow[u, hook] \arrow[r] & V_U
\end{tikzcd}
    \end{center}
    where
    \begin{equation*}
      V_U \defeq V/\langle v-\pi(u)v\sep v\in V,u\in U\rangle.
    \end{equation*}
    and $V^{(K,\rho)}$ is the maximal subspace of $V$ on which $\pi|_K$ acts $\rho^{\oplus n}$,(ie $(K,\rho)$-isotypic part) is injective. 
  \end{enumerate}
  \label{definition_g_cover}
\end{definition}

\begin{remark}
  In the setting of Definition \ref{definition_g_cover}, if $L\subsetneq G$ then
  \begin{equation*}
    UL\overline{U}\subsetneq G,
  \end{equation*}
  and it is a Zariski open set.
\end{remark}

\begin{example}
  Let 
  \begin{equation*}
    L=T\defeq\left\{\begin{pmatrix}
      t & 0 \\
      0 & t^{-1}
    \end{pmatrix}
    \:\bigg{|}\:t\in F^\times\right\},K_L = T_0\defeq \left\{\begin{pmatrix}
      t & 0 \\
      0 & t^{-1}
    \end{pmatrix}
    \:\bigg{|}\:t\in \Ouv_F^\times\right\},
  \end{equation*}
  Then $(I,\triv_I)$ is a $SL_2$-cover of $(T_0,\triv_{T_0})$ where,
  \begin{equation*}
    I = \left\{\begin{pmatrix}
      \Ouv_F^\times & \Ouv_F\\
      \mathfrak{p}_F & \Ouv_F^\times
    \end{pmatrix}
    \right\} = \left\{\begin{pmatrix}
      1 & \Ouv_F\\
      0 & 1
    \end{pmatrix}
    \right\}\left\{\begin{pmatrix}
      \Ouv_F^\times & \:\\
      \: & \Ouv_F^\times
    \end{pmatrix}
    \right\}\left\{\begin{pmatrix}
      1 & \:\\
      \mathfrak{p}_F & 1
    \end{pmatrix}
    \right\}.
  \end{equation*}
\end{example}

\begin{example}[Counter Example without i) in Definition \ref{definition_g_cover}]
  Consider
  \begin{equation*}
    \GL_2(\Z_p)\supsetneq \left\{\begin{pmatrix}
      1 & \Z_p\\
      0 & 1
    \end{pmatrix}
    \right\}\left\{\begin{pmatrix}
      \Z_p^\times & 0\\
      0 & \Z_p^\times
    \end{pmatrix}
    \right\}\left\{\begin{pmatrix}
      1 & 0\\
      \Z_p & 1
    \end{pmatrix}
    \right\}
  \end{equation*}
  Note that $\left\{\begin{pmatrix}
      0 & 1\\
      1 & 0
    \end{pmatrix}
    \right\}$ is not contained in the right hand side.
\end{example}

\begin{theorem}
  \label{theorem_117}
  Let $(K_L,\rho_L)$ be an $[L,\sigma]_L$-type in $L$ and $(K,\rho)$ be a $G$-cover of $(K_L,\rho_L)$. Then $(K.\rho)$ is a $[L,\sigma]_G$-type in $G$.
  \begin{center}
    \begin{tikzcd}
{\Rep^{[L,\sigma]_G}(G)}                                                                                                                              & {(K,\rho)} \arrow[d, "\text{$G$-cover}"] \\
{\Rep^{[L,\sigma]_L}(L)} \arrow[r, "{\substack{\\\: \sigma=\cind_{\tilde{K}_L}^L\stackrel{\text{Theorem \ref{theorem_116}}}{\mapsto}(K_L,\rho_L)}}"'] & {(K_L,\rho_L)}                          
\end{tikzcd}
  \end{center}
\end{theorem}

Types are explicitly constructed through $G$-covers. The following are the know constructions,
\begin{enumerate}
  \item Bushnell-Kutzko showed for $\GL_n$.
  \item Moy and Prasad showed for the depth zero types. Roughly speaking depth zero types are related to the representation of finite gruops.
  \item Kim-Yu types are showed by collaborations of many mathematicians. First proposed by Ju-Lee Kim and Jiu-Kang Yu and later fixed and modified by Prof. Jessica Fintzen, Prof. Tasho Kaletha, and Loren Spice.
\end{enumerate}

\begin{theorem}[Prof. Fintzen]
  If $G$ splits over a tamely ramified extension of $F$ and the residue characteristic $p$ of $F$ does not divide the order of the Weyl group $W_G$ of $G$, then for any Bernstein block $\Rep^{[L,\sigma]_G}(G)$, we have a Kim-Yu type associated with its Bernstein block $\Rep^{[L,\sigma]_G}(G)$.
\end{theorem}

\begin{proof}
  See \cite{Fintzen}.
\end{proof}

\begin{theorem}
  \label{theorem_119}
  Let $(K,\rho)$ be atype associated with $\Rep^{[L,\sigma]_G}(G)$ then the functor,
  \begin{center}
    \begin{tikzcd}
{\Rep^{[L,\sigma]_G}(G)} \arrow[r, "{\substack{(\pi,V)\mapsto\Hom_G(\cind_K^G(\rho),\pi)\\\:}}"] & \Mod-\End_G(\cind_K^G\rho)
\end{tikzcd}
  \end{center}
  gives an equivalence of categories.
\end{theorem}

\begin{proof}
  We will prove that $\cind_K^G\rho$ is a progenerator of $\Rep^{[L,\sigma]_G}(G)$ (cf. Theorem \ref{theorem_66}). It is an exercise to prove that $\cind_k^G(\rho)$ is finitely generated and projective. We don't use the properties of types. 
  \par Now we prove that this is actually a generator of the Bernstain block $\Rep^{[L,\sigma]_G}(G)$. To prove the statement, it suffices to show that for any irreducible representation $(\pi,V)$ of $G$,
  \begin{equation*}
    \Hom_G(\cind_K^G(\rho),\pi)\not=0
  \end{equation*}
  if and only if $(\pi,V)\in\Rep^{[L,\sigma]_G}(G)$.  (cf. proof of Theorem \ref{theorem_99}). Using Frobenius reciprocity, we have,
  \begin{equation*}
    \Hom_G(\cind_K^G(\rho),\pi)\simeq\Hom_K(\rho,\pi|_K).
  \end{equation*}
  And the claim follows from the defintion of types.
\end{proof}

\begin{proposition}[Relations of Two Progenerator]
  For a Bernstein block $\Rep^{[L,\sigma]_G}(G)$,
  \begin{equation*}
    \sigma=\cind_{\tilde{K}_L}^L(\tilde{\rho}_L)\to (K_L,\rho_L)
  \end{equation*}
  is an $[L,\sigma]_L$-type and $(K,\rho)$ is a $G$-cover of $(K_L,\rho_L)$, then $(K,\rho)$ is a $[L,\sigma]_G$-type.
\end{proposition}

Recall we have two progenerators.
\begin{enumerate}[1).]
  \item $i_P^G(\cind_{L^\circ}^L(\sigma^\circ))$ from Theorem \ref{theorem_99},
  \item $\cind_K^G(\rho)$ from Theorem \ref{theorem_119}.
\end{enumerate}

\begin{theorem}
  \label{theorem_120}
  We have explicit isomorphisms 
  \begin{equation*}
  \cind_{L^\circ}^L(\sigma^\circ)\simeq\cind_{K_L}^L(\rho_L),
  \end{equation*}
  and,
  \begin{equation*}
  i_P^G(\cind_{L^\circ}^L(\sigma^\circ))\simeq\cind_K^G(\rho).
  \end{equation*}
\end{theorem}

\begin{proof}
  The proof can be found \cite{Ohara}, \cite{Blondel}, and \cite{Dat}.
\end{proof}

\subsection{Hecke Algebras and Their Structures}

\begin{definition}
  Let $K\subseteq G$ be a compact open subgroup. The Hecke algebra $\hecke(G,K,\rho)$ attached to $(K,\rho)$ is 
  \begin{equation*}
    \hecke(G,K,\rho)\defeq\left\{\varphi:G\to\End_\C(V_\rho)\:\bigg{|}\:\substack{\text{$\varphi$ is compactly supported}\\\forall k_1,k_2\in K,g\in G,\varphi(k_1gk_2)=\rho(k_1)\circ\varphi(g)\circ\rho(k_2)}\right\}.
  \end{equation*}
  We define the multiplication by convolutions. That is for $\varphi_1,\varphi_2\in\hecke(G,K,\rho)$, we set,
  \begin{equation*}
    (\varphi_1\ast\varphi_2)(g)=\int_G\varphi_1(x)\varphi_2(x^{-1}g)dx.
  \end{equation*}
\end{definition}

\begin{lemma}
  \label{lemma_121}
  We have an isomorphism of $\C$-algebras,
  \begin{equation*}
    \hecke(G,K,\rho)\stackrel{\substack{\varphi\mapsto t_\varphi = [f\mapsto\varphi\ast f],\sim}}{\to}\End_G(\cind_K^G\rho).
  \end{equation*}
  Note that
  \begin{equation*}
    (\varphi\ast )f(g) = \int_G\underbrace{\overbrace{\varphi(x)}^{\in\End_\C(V_\rho)}\overbrace{(f(x^{-1}g))}^{\in V_\rho}}_{\in V_\rho}dx
  \end{equation*}
\end{lemma}

\begin{proof}
  Same as Lemma \ref{lemma_67}.
\end{proof}

\begin{corollary}
  \label{corollary_122}
  If $(K,\rho)$ is a $[L,\sigma]_G$-type, we have,
  \begin{equation*}
    \Rep^{[L,\sigma]_G}(G)\cong\Mod-\hecke(G,K,\rho).
  \end{equation*}
\end{corollary}

\begin{example}
  Suppose $Z(G)$ is finite (eg. $G=\SL_2(F)$) and $\pi=\cind_K^G$ is irreducible and supercuspidal, then 
  \begin{equation*}
    \hecke(G,K,\rho)\cong\End_G(\cind_K^G\rho)\cong\C.
  \end{equation*}
\end{example}

\begin{example}
  \begin{equation*}
    \Rep^{[G,\pi]_G}(G)\cong\Mod-\hecke(G,K,\rho)\cong\Vect_\C.
  \end{equation*}
\end{example}

\begin{example}
  Consider $G=\SL_2(\Q_p)$ and 
  \begin{equation*}
    T = \left\{\begin{pmatrix}
    t & \:\\
    \: & t^{-1}
    \end{pmatrix}\:\Bigg{|}\: t\in\Q^\times_p\right\}, I = \left\{\begin{pmatrix}
      \Z_p^\times & \Z_p\\
      p\Z_p & \Z_p^\times
    \end{pmatrix}
    \right\}.
  \end{equation*}
  Then we set,
  \begin{equation*}
    T_0\defeq T\cap I = \left\{\begin{pmatrix}
      t & 0\\
      0 & t^{-1}
    \end{pmatrix}\:\bigg{|}\: t\in\Z_p^\times\right\}.
  \end{equation*}
  Then we have,
  \begin{equation*}
    \hecke(G,I,\triv_I)=\mathcal{C}_C(I\backslash\SL_2(F)/I).
  \end{equation*}
  Furthermore, we have a bijection,
  \begin{equation*}
    N_{\SL_2(F)}(T)/T\stackrel{\sim}{\to}I\backslash\SL_2(F)/I.
  \end{equation*}
  Note that the right hand side is a set but the left hand side has a group structure. In particular,
  \begin{equation*}
    N_{SL_2(F)}(T)/T_0=\left\langle\underbrace{\overline{\begin{pmatrix}0 & 1\\ -1 & 0 \end{pmatrix}}}_{\defqe s_0},\underbrace{\overline{\begin{pmatrix}0 & p^{-1}\\ -p & 0 \end{pmatrix}}}_{\defqe s_1}\right\rangle.
  \end{equation*}
  Then we observe that,
  \begin{equation*}
    s_0^2=s_1^2 = \overline{\begin{pmatrix}
      -1 & 0\\
      0 & \underbrace{-1}_{\in\Z_p^\times}
    \end{pmatrix}} = 1.
  \end{equation*}
  We then have,
  \begin{equation*}
    N_{\SL_2(F)}(T)/T_0\cong\langle s_0,s_1\sep s_0^2=s_1^2=1\rangle \defqe W_{\text{aff}}(\tilde{A}_1),
  \end{equation*}
  where $W_{\text{aff}}(\tilde{A}_1)$ is the affine Weyl group of type $\tilde{A}_1$. This is an infinite group.
  \par For $w\in N_{\SL_2(F)}(T)/T_0$, consider,
  \begin{equation*}
    \varphi_w\in\hecke(G,I,\triv_I)=\mathcal{C}_C(I\backslash \SL_2(F)/I),
  \end{equation*}
  such that $\supp(\varphi_{w})=IwI$. Then 
  \begin{equation*}
    \hecke(G,I,\triv_I)=\bigoplus_{w\in N_{\SL_2(F)}(T)/T_0\cong W_{\text{aff}}(\tilde{A}_1)}\C\varphi_w.
  \end{equation*}
  We can prove the existence of $\varphi_{s_i}\in\hecke(G,I,\triv_I)$ such that 
  \begin{equation*}
    (\varphi_{s_i}-p1)(\varphi_{s_i}+1)=0,
  \end{equation*}
  where $1$ is the unit of the group. Moreover, we can prove that 
  \begin{equation*}
    \hecke(G,I,\triv_I)\cong\langle\varphi_{s_0},\varphi_{s_1}\sep i=0,1,(\varphi_{s_i}-p1)(\varphi_{s_i}+1)=0\rangle\defqe \hecke_{\text{aff}}(W_{\text{aff}}(\tilde{A}_1),p),
  \end{equation*}
  and it can be even shown that there exists no other trivial relation. $\hecke_{\text{aff}}(W_{\text{aff}}(\tilde{A}_1),p)$ is called the affine Hecke algebra of type $\tilde{A}_1$ and parameter $p$. If we consider,
  \begin{equation*}
    \hecke_{\text{aff}}(W_{\text{aff}}(\tilde{A}_1),1) = \langle\varphi_{s_0},\varphi_{s_1}\sep i=0,1,(\varphi_{s_i}-1)(\varphi_{s_i}+1)=0\rangle = \C[W_{\text{aff}}(\tilde{A}_1)].
  \end{equation*}
\end{example}

\begin{remark}
  We have a non-canonical isomorphism of $\C$-algebras between 
  \begin{equation*}
    \hecke_{\text{aff}}(W_{\text{aff}}(\tilde{A}_1),p)\simeq\C[W_{\text{aff}}(\tilde{A}_1)].
  \end{equation*}
\end{remark}

More about Affine Hecke algebras can be found in \cite{Soll}.
\par We now refer to \cite{Adl_1},\cite{Adl_2}. 
\begin{center}[]
\centering
\begin{tabular}{c|c}
$\SL_2$                                                                                          & $G$                                                                                                                                                                                                                                                                                                            \\ \hline
$\Rep^{[T,\triv_T]}(\SL_2)$                                                                      & $\Rep^{[L,\sigma]_G}(G)$                                                                                                                                                                                                                                                                                       \\ \hline
$(I,\triv_I)$                                                                                    & $(K,\rho)$:Kim-Yu type                                                                                                                                                                                                                                                                                         \\ \hline
$\hecke(\SL_2,I,\triv_I)=\mathcal{C}_C(I\backslash\SL_2/I)\cong\bigoplus_{I\backslash\SL_2/I}\C$ & $\substack{\hecke(G,K,\rho)\simeq\bigoplus_{g\in K\backslash I_G(\rho)/K}\hecke(G,K,\rho)_g\simeq\bigoplus_{K\backslash I_G(\rho)/K}\C\\\text{where }I_G(\rho)=\{x\in G\sep\Hom_{K\cap gKg^{-1}}(\prescript{g}{}{\rho},\rho)\not=0\}\\\hecke(G,K,\rho)_g = \{\varphi\sep\supp(\varphi)\subseteq KgK\}\cong\C}$
\end{tabular}
\end{center}

Note that 
\begin{equation*}
  I\backslash \SL_2(F)/I\cong N_{\SL_2}(T)/T\cap I = \langle s_0,s_1\sep s_i^2=1\rangle = W_{\text{aff}}(\tilde{A}_1).
\end{equation*}

There exists $N^\heartsuit\subseteq N_G(L,\underbrace{K_L}_{\defqe K\cap L})$ such that 
\begin{equation*}
  K\backslash I_G(\rho)/K\simeq N^\heartsuit/(N^\heartsuit \cap K_L)\simeq W_{\text{aff}}(\rho)\ltimes\Omega(\rho),
\end{equation*}
where 
\begin{equation*}
  W_{\text{aff}}(\rho) = \left\langle s_1(s_i\in \underbrace{S(\rho)}_{\text{finite sequence}})\:\Bigg{|}\:\substack{s_i^2=1\\ \underbrace{s_is_js_is_j \cdots}_{\text{$m_{ij}$-terms}} = \underbrace{s_js_is_js_i\cdots}_{\text{$m_{ij}$-terms}}}\right\rangle,
\end{equation*}
where $m_{ij}\in \Z_{\geq 2}\cup\{\infty\}$ and $m_{ij}=\infty$ means that there is no relation and $\Omega(\rho)$ is a complement.
\begin{equation*}
  \hecke(G,K,\rho)\simeq\hecke_{\text{aff}}(W_{\text{aff}}(\rho),\{q_{s_i}\}_{s_i\in S(\rho)})\rtimes\C[\Omega(\rho),\mu],
\end{equation*}
where,
\begin{equation*}
  \hecke_{\text{aff}}(W_{\text{aff}}(\rho,\{q_{s_i}\}_{s_i\in S(\rho)}))\simeq\left\langle\varphi_{s_i},(s_i\in S(\rho))\:\Bigg{|}\substack{(\varphi_{s_i}-q_{s_i})(\varphi_{s_i}+1)=0\\ \varphi_{s_i}\varphi_{s_j}\varphi_{s_i}\cdots = \varphi_{s_j}\varphi_{s_i}\varphi_{s_j}\cdots}\right\rangle.
\end{equation*}

And $\mu:\Omega(\rho)\times\Omega(\rho)\to\C^\times$ is a $2$-cocycle (note that this depends on $\rho$),
\begin{equation*}
  \C[\Omega(\rho),\mu]=\bigoplus_{t\in\Omega(\rho)}\C b_t\st b_{t_1},b_{t_2}=\mu(t_1t_2)b_{t_1t_2}.
\end{equation*}
This is called the twisted group algebra. 
This can be used to show that 
\begin{align*}
\Rep^{[L,\sigma]_G}(G)&\simeq\Mod-\hecke(G,K,\rho),\\
\Rep^{[L^0,\sigma^0]_{G^0}}(G^0) & \simeq\Mod-\hecke(G^0,K^0,\rho^0).
\end{align*}
These are somewhat easier blocks to study. We have a correspondence between,
\begin{equation*}
  \text{depth-zero block}\leftrightarrow\text{representation of finite groups of Lie type}.
\end{equation*}

And we have the following implication,
\begin{equation*}
  \hecke(G,K,\rho)\simeq\hecke(G^0,K^0,\rho^0)\Rightarrow\Rep^{[L,\sigma]_G}(G)\simeq\Rep^{[L^0,\sigma^0]_{L^0}}(G^0).
\end{equation*}

\begin{theorem}[Adler-Fintzen-Mishra-Ohara]
  If $G$ splits over a finitely ramified extension of $F$ and the residue characteristic of $p$ of $F$ does not divide the order of $W_G$ then every Bernstein block $\Rep^{[L,\sigma]_G}(G)$ is equivalent to a depth-zero Bernstein block of some reductive subgroup $G^0$ of $G$. ($G^0$ is called the twisted Levi subgruop. That is After a finite transcendental extension and by base changing, $G_E^0\subseteq G_E$ is the Levi subgruop.)
\end{theorem}

\begin{thebibliography}{9}
\bibitem{Vigneras} Vignéras, Marie-France Représentations $l$-modulaires d'un groupe réductif p-adique avec $l\not=p$
\bibitem{Roche}Roche, Alan, The Bernstein Decomposition and the Bernstein Centre.
\bibitem{Debacker} Debacker, Stefan, Lecture notes https://websites.umich.edu/~smdbackr/MATH/notes.pdf.
\bibitem{Renard} Renard, David (2010), Representations des groupes reductifs p-adiques Séminaire Bourbaki page 193-219, Société Mathématique de France.
\bibitem{Buschnell} Buschnell, Colin, Kutzko, Phillip (1998),  Smooth representations of reductive p-adic groups: structure theory via types, Proceedings of the London Mathematical Society.
\bibitem{May-Prasad} Moy, A., Prasad, G. Jacquet functors and unrefined minimal K-types. Commentarii Mathematici Helvetici 71, 98–121 (1996).
\bibitem{Fintzen} Fintzen, Jessica. (2020), Types for tame p-adic groups. Annals of Mathematics.
\bibitem{Ohara} Ohara, Kazuma. A comparison of endomorphism algebras.
\bibitem{Blondel} Blondel, Corinne. Quelques propríetés des paires couvrantes.
\bibitem{Dat} Dat, Jean-Francois. Finitude pour les representations lisses de groupes p-adiques.
\bibitem{Soll} Solleveld, Maarten. Affine Hecke algebras and their representations.
\bibitem{Adl_!} Jeffrey D. Adler, Jessica Fintzen, Manish Mishra, Kazuma Ohara. Structure of Hecke algebras arising from types.
\bibitem{Adl_1} Jeffrey D. Adler, Jessica Fintzen, Manish Mishra, Kazuma Ohara. Reduction to depth zero for tame p-adic groups via Hecke algebra isomorphisms.
\end{thebibliography}
\end{document}
